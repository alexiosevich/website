\documentstyle{amsppt}
\tolerance 3000
\pagewidth{5.5in}
\vsize7.0in
\magnification=\magstep1
\widestnumber \key{AAAAAAAAAAAAA}
\topmatter
\author Alex Iosevich, Nets Katz, and Terry Tao
\endauthor
\thanks Research supported in part by the NSF grants
\endthanks
\title Fuglede conjecture holds for convex planar domains
\endtitle
\thanks The research of Alex Iosevich is partially supported by
the NSF Grant DMS00-87339. The research of Nets Katz is
partially supported by the NSF Grant DMS98-01410. Terry Tao is a
Clay prize fellow and is supported by grants from the Packard
and Sloan foundations. \endthanks
\abstract Let $\Omega$ be a
compact convex domain in the plane. We prove that $L^2(\Omega)$
has an orthogonal basis of exponentials if and only if $\Omega$
tiles the plane by translation.
\endabstract
\endtopmatter

\def\dist{\hbox{{\rm dist}}}
\def\diam{\hbox{{\rm diam}}}
\def\hull{\hbox{{\rm hull}}}
\def\R{\hbox{{\bf R}}}
\def\Z{\hbox{{\bf Z}}}
\def\eps{\varepsilon}


\head Section 0: Introduction \endhead  Let $\Omega$ be a domain
in ${\Bbb R}^d$, i.e., $\Omega$ is a Lebesgue measurable subset
of $\Bbb{R}^d$ with finite non-zero Lebesgue measure. We say
that a set $\Lambda \subset {\Bbb R}^d$ is a {\it spectrum} of
$\Omega$ if ${\{e^{2\pi i x \cdot \lambda} \}}_{\lambda \in
\Lambda}$ is an orthogonal basis of $L^2(\Omega)$.

\example{Fuglede Conjecture}(\cite{Fug74}) A domain $\Omega$
admits a spectrum if and only if it is possible to tile ${\Bbb
R}^d$ by a family of translates of $\Omega$. \endexample

If a tiling set or a spectrum set is assumed to be a lattice,
then the Fuglede Conjecture follows easily by the Poisson
summation formula. In general, this conjecture is nowhere near
resolution, even in dimension one. However, there is some recent
progress under an additional assumption that $\Omega$ is convex.
In \cite{IKP99}, the authors prove that the ball does not admit
a spectrum in any dimension greater than one. In \cite{Kol99},
Kolountzakis proves that a non-symmetric convex body does not
admit a spectrum. In \cite{IKT00}, the authors prove that any
convex body in ${\Bbb R}^d$, $d>1$, with a smooth boundary, does
not admit a spectrum. In two dimensions, the same conclusion
holds if the boundary is piece-wise smooth and has at least one
point of non-vanishing curvature. The main result of this paper
is the following:

\proclaim{Theorem 0.1} Let $\Omega$ be a convex compact set in
the plane. The Fuglede conjecture holds. More precisely,
$\Omega$ admits a spectrum if and only if $\Omega$ is either a
quadrilateral or a hexagon.
\endproclaim

Our task is simplified by the following result due to
Kolountzakis. See \cite{Kol99}.

\proclaim{Theorem 0.2} Convex non-symmetric subsets of ${\Bbb
R}^d$ do not admit a spectrum.
\endproclaim

Thus, it suffices to prove Theorem 0.1 for symmetric sets.
Recall that a set $\Omega$ is symmetric with respect to the
origin when $x \in \partial \Omega$ if and only if $-x \in
\partial \Omega$.

This paper is organized as follows. The first section deals with
basic properties of spectra. The second section is dedicated to
the properties of the Fourier transform of the characteristic
function of a convex set. In the third section we prove Theorem
0.1 for polygons, and in the fourth section we prove that any
convex set which is not a polygon does not admit a spectrum,
thus completing the proof of Theorem 0.1.

\head Section 1: Basic properties of spectra \endhead

Let
$$Z_{\Omega}= \left\{\xi \in \Bbb R^d: \hat \chi_{\Omega}(\xi)=
\int_\Omega e^{-2\pi i \xi \cdot x}\ dx=0 \right\}. \tag1.1$$
The orthogonality of a spectrum $\Lambda$ means precisely that
$$ \lambda - \lambda' \in Z_\Omega \hbox{ for all } \lambda,
\lambda' \in \Lambda, \lambda \neq \lambda'.  \tag1.2$$

It follows that the points of a spectrum $\Lambda$ are separated
in the sense that $$ |\lambda-\lambda'| \gtrsim 1 \ \text{for
all } \lambda \not=\lambda', \ \ \lambda, \lambda' \in \Lambda.
\tag1.3$$ Here, and throughout the paper, $a \lesssim b$ means
that there exists a positive constant $C$ such that $a \leq Cb$.
We say that $a \approx b$ if $a \lesssim b$ and $a \gtrsim b$.

The following result is due to Landau. See \cite{Lan67}. Let
$$ D_R^{+}= \max_{x \in {\Bbb R}^n} \# \{ \Lambda \cap Q_R(x)\}, \tag1.4$$
where $Q_R(x)$ is a cube of sidelength $2R$ centered at $x$, and
let $$ D_R^{-}= \min_{x \in {\Bbb R}^n}\#\{ \Lambda \cap
Q_R(x)\}. \tag1.5$$

Then
$$ \limsup_{R \rightarrow \infty} \frac{D_R^{\pm}}{{(2R)}^n} =
|\Omega|. \tag1.6$$

It is at times convenient to use the following related result.
We only state the special case we need for the proof of Theorem
0.1. For a more general version see \cite{IosPed99}.
\proclaim{Theorem 1.1} Let $\Omega$ be a convex domain in ${\Bbb
R}^2$. Then there exists a universal constant $C$ such that if
$$ R \ge C
\left(\frac{{|\partial \Omega|}}{|\Omega|}\right), \tag1.7$$ then
$$ \Lambda \cap Q_R(\mu) \not= \emptyset \tag1.8$$ for every
$\mu \in {\Bbb R}^2$, and any set $\Lambda$ such that
$E_{\Lambda}$ is an exponential basis for $L^2(\Omega)$, where
$Q_R(\mu)$ denotes the cube of side-length $2R$ centered at
$\mu$. \endproclaim

The proof of Theorem 1.1, and the preceding result due to
Landau, are not difficult. Both proofs follow, with some work,
from the fact that $\Omega$ admits a spectrum $\Lambda$ if and
only if $$ \sum_{\Lambda}
{|\widehat{\chi}_{\Omega}(x-\lambda)|}^2 \equiv 1, \tag1.9$$ and
some averaging arguments. To say that $\Omega$ admits a spectrum
$\Lambda$ means that the Bessel formula
${||f||}_{L^2(\Omega)}^2=\sum_{\Lambda} {|\hat{f}(\lambda)|}^2$
holds. Since the exponentials are dense, it is enough to
establish such a formula with $f=e^{2 \pi i x \cdot\xi}$, which
is precisely the formula $(1.9)$.

\head Section 2: Basic properties of $\widehat{\chi}_{\Omega}$
and related properties of convex sets \endhead

Throughout this section, and the rest of the paper, $\Omega$
denotes a convex compact planar domain. The first two results in
this section are standard and can be found in many books on
harmonic analysis or convex geometry.

\proclaim{Lemma 2.1} $|\widehat{\chi}_{\Omega}(\xi)| \lesssim
\frac{diam{\Omega}}{|\xi|}$. Moreover, if $\Omega$ is contained
in a ball of radius $r$ centered at the origin, then $|\nabla
\widehat{\chi}_{\Omega}(\xi)| \lesssim \frac{r^2}{|\xi|}$.
\endproclaim

The lemma follows from the divergence theorem which reduces the
integral over $\Omega$ to the integral over $\partial \Omega$
with a factor of $\frac{1}{|\xi|}$, and the fact that convexity
implies that the measure of the boundary $\partial \Omega$ is
bounded by a constant multiple of the diameter. The second
assertion follows similarly.

\proclaim{Lemma 2.2} Suppose that $\xi$ makes an angle of at least
$\theta$ with every vector normal to the boundary of $\Omega$. Then
$$ |\widehat{\chi}_{\Omega}(\xi)| \lesssim \frac{1}{| \theta {|\xi|}^2|}.
\tag2.1$$

Moreover, if $\Omega$ is contained in a ball of radius $r$,
then $|\nabla \widehat{\chi}_{\Omega}(\xi)|
\lesssim \frac{r}{| \theta {|\xi|}^2|}$.
\endproclaim

To prove this, one can again reduce the integral to the boundary
while gaining a factor $\frac{1}{|\xi|}$. We may parametrize a
piece of the boundary in the form $\{(s, -\gamma(s)+c): a \leq s
\leq b\}$, where  $\gamma$ is a convex function, and, without
loss of generality, $c=0$, $a=0$, $b=1$, and
$\gamma(0)=\gamma'(0)=0$. We are left to compute
$$ \int_{0}^{1} e^{i(s \xi_1-\gamma(s) \xi_2)} J(s)ds, \tag2.2$$
where $J(s)$ is a nice bounded function that arises in the
application of the divergence theorem. The gradient of the phase
function $s \xi_1-\gamma(s) \xi_2$ is $\xi_2 \left(
\frac{\xi_1}{\xi_2}-\gamma'(s) \right)$, and our assumption that
$\xi$ makes an angle of at least $\theta$ with every vector
normal to the boundary of $\Omega$ means that the absolute value
of this expression is bounded from below by $|\xi_2| \theta$.
Integrating by parts once we complete the proof in the case
$|\xi_1| \lesssim |\xi_2|$. If $|\xi_1|>>|\xi_2|$, the absolute
value of the derivative of $s \xi_1-\gamma(s) \xi_2$ is bounded
below by $|\xi_1|$, so integration by parts completes the proof.
The second assertion follows similarly.

\proclaim{Lemma 2.3} Let $f$ be a non-negative concave function on an interval
$[-1/2,1/2]$. Then, for every $0<\delta \lesssim 1$, there exists
$R \approx \frac{1}{\delta}$ such that $|\hat{f}(R)|
\gtrsim \delta f \left( \frac{1}{2}-\delta \right)$.
\endproclaim

To see this, let $\phi$ be a positive function such that
$\phi(x) \lesssim {(1+|x|)}^{-2}$,
$\widehat{\phi}$ is compactly supported,
and $\phi(0)=1$ in a small neighborhood of the origin. Consider
$$ \int f \left( \frac{1}{2}-\delta t \right)(\phi(t+1)-K\phi(K(t+1)))dt,
\tag2.3$$ where
$f$ is defined to be $0$ outside of
$[a,b]$ and $K$ is a large positive number. If $K$ is sufficiently large,
$(\phi(t+1)-K\phi(K(t+1)))$ is positive for $t>0$,
and $\approx 1$ on $[\frac{1}{2},1]$. It follows that
$$ \int f \left( \frac{1}{2}-\delta t \right)(\phi(t+1)-K\phi(K(t+1)))dt
\gtrsim f \left( \frac{1}{2}-\delta \right). \tag2.4$$

Taking Fourier transforms, we see that
$$ \int \frac{1}{\delta} \hat{f} \left( \frac{r}{\delta} \right)
e^{i \pi r}\left(\hat{\phi}(r)-\hat{\phi} \left( \frac{r}{K}
\right) \right) dr \gtrsim  f \left( \frac{1}{2}-\delta \right).
\tag2.5$$

Multiplying both sides by $\delta$ and using the compact support of
$\hat{\phi}(r)-\hat{\phi} \left( \frac{r}{K} \right)$, we complete the proof.

\proclaim{Corollary 2.4} Let $\Omega$ be a convex body of the form
$$ \Omega=\{(x,y): a \leq x \leq b, \ -g(x) \leq y \leq f(x)\}, \tag2.6$$
where $f$ and $g$ are non-negative concave functions on $[a,b]$.
Then for every $0<\delta \lesssim b-a$, there exists $R \approx
\frac{1}{\delta}$ such that $$ |\widehat{\chi}_{\Omega}(\xi)|
\gtrsim \delta \left(f \left( \frac{1}{2}-\delta \right)+g
\left( \frac{1}{2}-\delta \right) \right). \tag2.6$$
\endproclaim

\head Section III: Lattice properties of spectra \endhead
\vskip.125in

Let $\Omega$ be a compact convex body in $\R^2$ which is
symmetric around the origin, but is not a quadrilateral.  Let
$\Lambda$ be a spectrum of $\Omega$ which contains the origin.
The aim of this section is to prove the following two
propositions which show that if a spectrum exists, it must be
very lattice-like in the following sense.

\proclaim{Proposition 3.1}  Let $I$ be a maximal closed interval in
$\partial \Omega$ with midpoint $x$.  Then
$$ \xi \cdot 2x \in \Z \tag3.1$$ for all $\xi \in \Lambda$.
\endproclaim

\proclaim{Proposition 3.2}  Let $x$ be an element of $\partial
\Omega$ which has a unit normal $n$ and which is not contained
in any closed interval in $\Omega$. Then
$$ \xi \cdot 2x \in \Z \tag3.2$$ for all $\xi \in \Lambda$.
\endproclaim

In the next Section we shall show how these facts can be used to
show that the only convex bodies which admit spectra are
quadrilaterals and hexagons.

\subhead Proof of Proposition 3.1 \endsubhead We may re-scale so
that $x = e_1$, the coordinate direction $(1, 0 \dots, 0)$, and
$I$ is the interval from $(e_1 - e_2)/2$ to $(e_1 + e_2)/2$.
Thus, we must show that
$$ \Lambda \subset \Z \times \R. \tag 3.3$$

The set $\Omega$ thus contains the unit square $Q :=
[-1/2,1/2]^2$. Since we are assuming $\Omega$ is not a
quadrilateral, we therefore have $|\Omega| > 1$. In particular,
$\Lambda$ has asymptotic density strictly greater than $1$, i.e
the expression $(1.6)$ is strictly greater than $1$.

A direct computation shows that
$$ \hat \chi_Q(\xi_1, \xi_2) = \frac{ \sin(\pi \xi_1)
\sin(\pi \xi_2) }{\pi^2 \xi_1 \xi_2}. \tag 3.4$$
The zero set of this is
$$ Z_Q := \{ (\xi_1, \xi_2): \xi_1 \in \Z - \{0\} \hbox{ or }
\xi_2 \in \Z -
\{0\} \}. \tag3.5$$
Note that $Z_Q \subset G$, where $G$ is the Cartesian grid
$$ G := (\Z \times \R) \cup (\R \times \Z). \tag3.6$$

Heuristically, we expect the zero set $Z_\Omega$ of $\hat
\chi_{\Omega}$ to approximate $Z_Q$ in the region $|\xi_1| \gg
|\xi_2|$. The following result shows that this indeed the case.
\proclaim{Lemma 3.3} For every $A \gg 1$ and $0 < \eps \ll 1$,
there exists an $R \gg A$ depending on $A$, $\eps$, $\Omega$,
such that $Z_\Omega \cap S_{A,R}$ lies within a $O(\sqrt{\eps})$
neighborhood of $Z_Q$, where $S_{A,R}$ is the slab
$$ S_{A,R} := \{ (\xi_1, \xi_2): |\xi_1| \geq R; |\xi_2| \leq A \}. \tag3.7$$
\endproclaim

\demo{Proof}
Fix $A$, $\eps$. We may write
$$ \hat \chi_\Omega = \hat \chi_{\Omega_-} + \hat \chi_Q + \hat \chi_{\Omega_+} \tag3.8$$
where $\Omega_-$ is the portion of $\Omega$ below $x_2 = -1/2$, and $\Omega_+$
is the portion above $x_2 = 1/2$.  In light of (3.4), it thus suffices to show that
$$ |\hat \chi_{\Omega_\pm}(\xi_1, \xi_2)| \lesssim \eps / |\xi_1| \tag 3.9$$
on $S_{A,R}$. By symmetry it suffices to do this for $\Omega_+$.

We may write $\Omega_+$ as
$$ \Omega_+ = \{ (x,y): -1/2 \leq x \leq 1/2; 1/2 \leq y \leq 1/2 + f(x) \} \tag3.10$$
where $f$ is a concave function on $[-1/2,1/2]$ such that $f(\pm 1/2) = 0$.

By continuity of $f$, we can find a $0 < \delta \ll \eps$ such that
$$ f(1/2 - \delta), f(\delta - 1/2) \leq \eps. \tag3.11$$
Draw the line segment from $(1/2,1/2)$ to $(1/2-\delta,
1/2+f(1/2-\delta))$, and the line segment from $(-1/2,1/2)$ to
$(-1/2+\delta, 1/2+f(-1/2+\delta)$. This divides $\Omega_+$ into

two small convex bodies and one large convex body. The diameter
of the small convex bodies is $O(\eps)$, and so their
contribution to (3.9) is acceptable by Lemma 2.1. If $R$ is
sufficiently large depending on $A$, then $(\xi_1, \xi_2)$ will
always make an angle of $\gtrsim \delta/\eps$ with the normals
of the large convex body. By Lemma 2.2, the contribution of this
large body is therefore $O(\delta/\eps |\xi|^2)$, which is
acceptable if $R$ is sufficiently large.
\enddemo

Let $A \gg 1$ and $0 < \eps \ll 1$, and let $R$ be as in Lemma
3.3. Since $\Lambda - \Lambda \subset Z_\Omega$, then by Lemma
3.3 we see that $\Lambda \cap (\xi + S)$ lies in an
$O(\sqrt{\eps})$ neighborhood of $Z_Q + \xi$ for all $\xi \in
\Lambda$.

Suppose that we could find $\xi, \xi' \in \Lambda$
such that $|\xi - \xi'| \ll A$ and
$$ \dist(\xi - \xi', G) \gg \sqrt{\eps}. \tag3.12$$

It follows that
$$ \Lambda \cap (\xi + S_{A,R}) \cap (\xi' + S_{A,R}) \tag3.13$$
lies in an $O(\sqrt{\eps})$ neighborhood of $G + \xi$ and in an
$O(\sqrt{\eps})$ neighborhood of $G + \xi'$.  Since $\Lambda$
has separation $\gtrsim 1$, it follows that $\Lambda$ has
density at most $1 + O(1/A)$ in the set $(\xi + S_{A,R}) \cap
(\xi' + S_{A,R})$.  However, this is a contradiction for $A$
large enough since $\Lambda$ needs to have asymptotic density
$1/|\Omega|<1/|Q|=1$.

By letting $\eps \to 0$ and $A \to \infty$ we see that
$$ \xi - \xi' \subset G \tag 3.14$$
for all $\xi, \xi' \in \Lambda$.  In particular, $\Lambda \subset G$
since $(0,0) \in \Lambda$.

Now suppose for contradiction that (3.3) failed.
Then there exists $(\xi_1, \xi_2) \in
\Lambda$ such that $\xi_1 \not \in \Z$. Since $\Lambda \subset G$,
we thus have that
$\xi_2 \in \Z$.  From $(3.14)$ we thus see that
$$ \Lambda \subset \R \times \Z. \tag3.15$$
For each integer $k$, let $R_k$ denote the intersection of
$\Lambda$ with
$\R \times \{k\}$.

Let $A \gg 1$ and $0<\eps \ll 1$, and let $R$ be as in Lemma
3.3. If $\xi, \xi' \in R_k$ and $|\xi - \xi'| \gg R$, then by
Lemma 3.3 we see that $\xi - \xi'$ lies in a $O(\sqrt{\eps})$
neighborhood of $\Z$.  From this and the separation of $\Lambda$
we see that one has
$$ \# \{ (\xi_1, k) \in R_k: |\xi_1| \leq M \} \lesssim M+R \tag3.16$$
for all $k$ and $M$. Summing this for $-M<k<M$ and then letting
$M \to \infty$ we see that $\Lambda$ has asymptotic density at
most $1$, a contradiction. This proves $(3.3)$, and Proposition
3.1 is proved.

\subhead Proof of Proposition 3.2 \endsubhead By an affine
re-scaling we may assume that $x = e_1/2$ and $n = e_1$, so that
our task is again to show $(3.3)$. We shall prove the following
analogue of Lemma 3.3. \proclaim{Lemma 3.4} For all $A \gg 1$,
$0 < \eps \ll 1$ there exists an $R \gg 1$ depending on $A$,
$\eps$, $\Omega$ such that $Z_\Omega \cap B(Re_1, A)$ lies
within $O(\eps)$ of $\Z \times \R$. \endproclaim

\demo{Proof} Fix $A, \eps$.  We can write $\Omega$ as
$$ \Omega= \{ (x,y): -1/2 \leq x \leq 1/2; -f(-x) \leq y \leq f(x) \}
\tag3.17$$
where $f(x)$ is a concave function on $[-1/2,1/2]$ which vanishes
at the endpoints
of this interval but is positive on the interior.

For each $0<\delta \ll 1$, define
$$ S(\delta) := \frac{f(1/2 - \delta) + f(-1/2 + \delta)}{\delta}.
\tag3.18$$
The function $\delta S(\delta)$ is decreasing to 0 as $\delta
\to 0$. Thus we may find a $0 <  \delta_0 \ll \eps/A$ such that
$\delta_0 S(\delta_0) \lesssim \eps/A$.

Fix $\delta_0$, and let $l_+$, $l_-$ be the line segments from
$(1/2 -2\delta_0,0)$ to $(1/2-\delta_0, f(1/2-\delta_0))$ and
$(1/2-\delta_0, -f(-1/2+\delta_0))$ respectively, and let
$-l_+$, $-l_-$ be the reflections of these line segments through
the origin.

By symmetry we have
$$ \hat \chi_\Omega = 2 \Re ( \widehat{\chi}_{\Omega_+} + \widehat{\chi}_{\Gamma_0} )
\tag 3.19$$ where $\Omega_+$ is the portion of $\Omega$ above
$l_+$, $-l_-$, and the $e_1$ axis, and $\Gamma_0$ is the small
portion of $\Omega$ between $l_+$ and $l_-$.

Since we are assuming $\Omega$ to have normal $e_1$ at $e_1/2$, we see that
$S(\delta) \to \infty$ as $\delta \to 0$. Thus we may find a
$0 < \delta \ll \delta_0$ such that
$$ S(\delta) \gg 1 + \frac{1}{\eps} S(\delta_0). \tag3.20$$

Fix this $\delta$. By Corollary 2.4 we may find an
$R \sim 1/\delta$ such that
$$ |\hat \chi_{\Gamma_0}(R e_1)| \gtrsim (f(1/2 - \delta) +
f(-1/2 + \delta)) \delta = \delta^2 S(\delta). \tag 3.21$$

Fix this $R$. Let $m_+$, $m_-$ be the line segments from
$(1/2 - 2\delta, 0)$ to $(1/2 - \delta, f(1/2 - \delta))$
and $(1/2 - \delta_0, -f(-1/2 + \delta))$ respectively. We can partition
$$ \hat \chi_{\Gamma_0} = \hat \chi_{\Gamma_+} + \hat \chi_{\Gamma_-} +
\hat \chi_\Gamma \tag3.22$$ where $\Gamma_+$ is the portion of $\Gamma$ above
$m_+$ and the $e_1$ axis, $\Gamma_-$ is the portion below $m_-$ and the $e_1$
axis, and $\Gamma$ is the portion between $m_+$ and $m_-$.

The convex body $\Gamma - e_1/2$ is contained inside a ball of radius
$O(S(\delta) \delta)$, hence by (0.2) we have
$$ |\nabla \hat \chi_{\Gamma - e_1/2}(\xi)| \lesssim (\delta S(\delta))^2 / R
\lesssim (\delta_0 S(\delta_0)) \delta^2 S(\delta) \lesssim \frac{\eps}{A}
\delta^2 S(\delta) \tag3.23$$ for $\xi \in B(Re_1, A)$.

If $\xi \in B(Re_1, A)$, then $\xi$ makes an angle of
$$O(A/R)=O(A\delta) \ll O(\delta/(\delta_0 S(\delta_0)))=
O(\delta/(\delta S(\delta))) = O(1/S(\delta)) \tag3.24$$ with the $e_1$ axis,
and hence makes an angle of $\gtrsim 1/S(\delta)$ with the convex bodies
$\Gamma_+ - e_1/2$, $\Gamma_- - e_1/2$. Since these bodies are in a ball of
radius $O(S(\delta_0) \delta_0)=O(\eps/A)$, we see from Lemma 2.2 that
$$ |\nabla \hat \chi_{\Gamma_\pm - e_1/2}(\xi)| \lesssim
\frac{\eps}{A} S(\delta) / R^2 \sim \frac{\eps}{A} \delta^2 S(\delta). \tag3.25$$

Summing, we obtain $$ |\nabla (e^{\pi i \xi_1} \hat
\chi_{\Gamma_0}(\xi))| \lesssim \frac{\eps}{A} \delta^2
S(\delta). \tag3.26$$

Integrating this and $(3.21)$ we get $$ \hat \chi_\Gamma(\xi) =
\hat \chi_{\Gamma_0}(R e_1)( e^{\pi i (R-\xi_1)}+O(\eps) ).\tag
3.27$$

If $\xi \in B(Re_1, A)$, then $\xi$ makes an angle of $\gtrsim
1/S(\delta_0)$ with every normal of $\Omega_+$. From Lemma 2.2
we get $$ |\hat \chi_{\Omega_+}(\xi)| \lesssim  S(\delta_0) /
R^2 \sim S(\delta_0) \delta^2 \ll \eps \delta^2 S(\delta)
\tag3.28$$ on $B(Re_1, A)$.  From this, $(3.20)$, $(3.22)$, and
$(3.23)$ we obtain $$ \hat \chi_\Omega(\xi) = 2 \Re( \hat
\chi_\Gamma(R e_1)( e^{\pi i(R - \xi_1)}+O(\eps) )) \tag3.29$$
on $B(Re_1, A)$, and the Lemma follows. \enddemo

Let $A \gg 1$, $0 < \eps \ll 1$, and let $R$ be as in Lemma 3.4.
If $\xi \in \Lambda$ are such that $|\xi| \ll A$, then from
Lemma 3.4 we see that
$$ Z_\Omega \cap B(Re_1 + \xi, A) \cap B(Re_1, A) \tag3.30$$
lies within $O(\eps)$
of $(\Z \times \R)$, and within $O(\eps)$ of $(\Z \times \R) +
\xi$. Since $Z_\Omega$ has asymptotic density $1/|\Omega|$, it
has a non-empty intersection with $B(Re_1, A) \cap B(Re_1 + \xi,
A)$, and thus $\xi$ must lie within $O(\eps)$ neighborhood of
$\Z \times \R$. Taking $\eps \to 0$ and then $A \to \infty$ we
obtain $(3.3)$, and Proposition 3.2 is proved.

\head Conclusion of the argument \endhead
\vskip.125in

We now use Proposition 3.1 and 3.2 to show that the only convex
symmetric bodies with spectra are the quadrilaterals and
hexagons. We may assume of course that $\Omega$ is not a
quadrilateral or a hexagon.

Suppose that there are two points $x$, $x'$ in $\partial \Omega$
for which either Proposition 3.1 or Proposition 3.2 applies.
From elementary geometry we thus see that $\Lambda$ must live in
a lattice of density $|2x \wedge 2x'|$. It follows that $$ 4|x
\wedge x'| \geq |\Omega| \tag4.1$$ for all such $x, x'$.  Since
$|x| \sim 1$ on $\partial \Omega$, this implies that there are
only a finite number of $x$ for which Proposition 3.1 and
Proposition 3.2 applies. Since almost every point in $\partial
\Omega$ has a unit normal, the only possibility left is that
$\Omega$ is a polygon.

Label the vertices of $\Omega$ cyclically by $x_1, \ldots,
x_{2n}$. Since $\Omega$ is not a quadrilateral or a hexagon, we
have $n \geq 4$. By symmetry we have $x_{n+i} = -x_i$ for all
$i$ (here we use the convention that $x_{2n+i}=x_i$).

From Proposition 3.1 we have $$ \xi \cdot (x_i - x_{n+i-1}) \in
\Z \tag4.2$$ for all $\xi \in \Lambda$. First suppose that $n$
is even. Then $n-1$ is coprime to $2n$, and by repeated
application of $(4.2)$ we see that
$$ \xi \cdot (x_i - x_j) \in \Z \tag4.3$$
for all $i, j$.  Arguing as in the derivation of $(4.1)$ we thus
see that $$ |(x_i - x_j) \cdot (x_i - x_k)| \geq |\Omega|
\tag4.4$$ for all $i, j, k$.  In other words, the triangle with
vertices $x_i$, $x_j$, $x_k$ has area at least $|\Omega|/2$ for
all $i, j, k$.  But $\Omega$ can be decomposed into $2n-2$ such
triangles, a contradiction since $n \geq 4$. Now suppose that
$n$ is odd, so that $n \geq 5$. Then $n-1$ and $2n$ have the
common factor of 2. Arguing as before we see that $(4.3)$ holds
for all $i,j,k$ of the same parity. But $\Omega$ contains the
three disjoint triangles with vertices $(x_1, x_3, x_5)$, $(x_1,
x_5, x_7)$, and $(x_1, x_7, x_9)$ respectively, and we have a
contradiction.

\newpage
\vskip.25in
\head References \endhead

\ref \key Fug74 \by B. Fuglede \paper Commuting self-adjoint partial
differential operators and a group theoretic problem \jour J. Funct.
Anal. \yr 1974 \vol 16 \pages 101-121 \endref

\ref \key Herz64 \by C. S. Herz \paper Fourier transforms related to
convex sets \jour Ann. of Math. \vol 75 \yr 1962 \pages 81-92
\endref

\ref \key IKP99 \by A. Iosevich, N. Katz, and S. Pedersen
\paper Fourier basis and the Erd\H os distance problem
\jour Math. Research Letter \yr 1999 \vol 6 \pages \endref

\ref \key IKT00 \by A. Iosevich, N. Katz, and T. Tao
\paper Convex bodies with a point of curvature do not have Fourier bases
\jour Amer. Jour. Math. (accepted for publication) \yr 2000 \endref

\ref \key IosPed99 \by A. Iosevich and S. Pedersen
\paper How wide are the spectral gaps? \jour Pacific J. Math.
\vol 192 \yr 1998 \pages 307-314 \endref

\ref \key Kol99 \by M. Kolountzakis \paper Non-symmetric convex
domains have no basis of exponentials \jour (preprint) \vol
\pages \yr 1999 \endref

\ref \key Lan67 \by H. Landau \paper Necessary density conditions for
sampling and interpolation of certain entire functions \jour Acta Math.
\vol 117 \pages 37-52 \yr 1967 \endref


\enddocument






\enddocument
