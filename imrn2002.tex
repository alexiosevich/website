\input amstex
\documentstyle{amsppt}
\tolerance 3000
\hcorrection{-5mm}
\pagewidth{5.7in} \vsize7.0in
\magnification=\magstep1 \widestnumber \key{AAAAAAAAAAAAA}
\NoRunningHeads \topmatter

\title A combinatorial approach to orthogonal exponentials
\endtitle


\author A. Iosevich and M. Rudnev
\endauthor
\abstract We prove that a symmetric strictly convex set with a
smooth boundary in ${\Bbb R}^d$ can possess no more than
finitely many orthogonal exponentials, unless $d=1 \mod(4)$. In
the latter case the non-existence theorem is true for a large
class of bodies, including a $d$-disk and its perturbations.
Otherwise,  any infinite set of the corresponding exponents
necessarily turns out to be a subset of some one-dimensional
lattice. We provide examples of convex bodies of revolution in
the above dimensions, for which infinite sets of orthogonal
exponentials exist.

The analysis is reduced to one dimension by studying the
distance set of the putative set of exponents with respect to an
appropriate metric. A combinatorial principle due to Erd\"{o}s
lies at the heart of the investigation. According to this
principle, if the distance set of an infinite set in ${\Bbb
R}^d$ is a subset of the integers, then the set itself is a
subset of some one-dimensional lattice.

We also provide a physical interpretation for the above
phenomena.
\endabstract

\date June 17, 2002
\enddate

\address
Alex Iosevich. Department of Mathematics. University of Missouri,
Columbia, MO 65211 U.S.A.
\endaddress
\address
http://www.math.missouri.edu/$\thicksim$iosevich/
\endaddress
\email iosevich\@math.missouri.edu \endemail

\address
Michael Rudnev. School of Mathematics. University of Bristol,
University Walk, Bristol BS8 1TW, UK
\endaddress

\email m.rudnev\@bris.ac.uk \endemail



\subjclass Primary 11H06, 42B10, Secondary 42B25
\endsubjclass

\keywords Orthogonal bases, Fourier transform, convex sets, integer distances
\endkeywords


\thanks Research supported in part by the NSF grants DMS00--87339
and DMS00-72153.
\endthanks




\endtopmatter
\document

\head Introduction \endhead

Physicists have long believed that plane waves are not
sufficient to completely describe the quantum billiard problem.
However, Berry in \cite{Berry94} argues that despite this
insufficiency to  represent a general  solution for an
eigenfunction of the Laplacian in the billiard domain $K$ due to
the appearance of evanescent waves, the latter can be still
arbitrarily well approximated by superpositions of exponentials.

In the language of analysis, the "insufficiency" of plane waves
in this context means that $L^2(K)$ does not possess an
orthogonal basis of exponentials. Indeed, in {\cite{Kol00}}
Kolountzakis proves that a non-symmetric convex domain does not
possess an orthogonal basis of exponentials. It was proved in
{\cite{IKT01}} that symmetric convex domains with a point of
curvature on the boundary do not possess orthogonal bases of
exponentials either. Both results are motivated by a paper of
Fuglede \cite{Fuglede74} who conjectured that $L^2(K)$ has an
orthogonal basis of exponentials if and only if $K$ tiles ${\Bbb
R}^d$ by translation. In the context of convex planar domains,
this conjecture is proved in \cite{IKT02}.

While at the first glance a problem in analysis, the question of
existence of orthogonal exponential bases or sub-bases has a
distinctive combinatorial flavor. The definition of
orthogonality, combined with the asymptotics of the Fourier
transform of the characteristic function of $K$ strongly suggest
that the existence of a putative set $A$ such that ${\{e^{2 \pi
i x \cdot a}\}}_{a \in A}$ are pairwise orthogonal in $L^2(K)$
is closely tied to the properties of the distance set
$$\Delta(A)=\{\rho_{*}(a-a'): a,a' \in A,\,a\neq a'\},$$ where $\rho_{*}$
is the Legendre transform of the Minkowski functional of $K$.
This brings the study of orthogonal exponentials into the  realm
of combinatorial distance problems pioneered by Erd\"{o}s. See
the treatise \cite{AP} and the references contained therein. See
also \cite{Falc87}, \cite{Wolff99} and references contained
therein where a closely related continuous analog known as the
Falconer conjecture is treated.

In this paper we study the case when $K$ is symmetric, has a
smooth boundary and is {\it strictly convex}. By strict
convexity we mean that the boundary $\partial K$ is smooth and
has everywhere non-vanishing Gaussian curvature. Then for $d\neq
1 \mod(4)$ we prove non-existence in $L^2(K)$ of
infinite-dimensional subspaces allowing orthogonal exponential
bases for any of the above $K$. However, in dimensions
$5,9,13,\ldots$ such sub-spaces may exist for certain $K$, as
the examples that we provide below show. If this is the case,
all the corresponding exponents must necessarily be contained in
some one-dimensional lattice. In the context of quantum
billiards, this means that either one can have only finitely
many mutually orthogonal plane wave solutions of the Dirichlet
problem for the Helmholtz equation in the interior of $K$, or
(for certain $K$ in the exceptional dimensions above) any
infinite set of the corresponding wave vectors must be
one-dimensional. Our proof is based on a principle that goes
back to Erd\"{o}s, which says that if a set of Euclidean
distances of a planar set is a subset of the integers, then the
set itself is contained in a straight line.

From now on we fix $K$ and without loss of generality assume
that it has a unit volume with respect to the Lebesgue measure.

\definition{Definition} Let $A$ be a subset of ${\Bbb R}^d$ such
that the orthogonality relation
$$ \widehat{\chi}_K(a-a')=\int_K e^{2 \pi i x \cdot (a-a')} dx=
\delta_{aa'} \tag0.1$$ holds whenever $a,a' \in A$, the right
hand side being the Kronecker delta. Then $A=A(K)$ is called a
set of {\it orthogonal exponents} for $K$. \enddefinition

\definition{Definition} $A$ is maximal if for any
$a'\in {\Bbb R}^d\setminus A$
there exists $a\in A$ such that
$ \widehat{\chi}_K(a-a')\neq0$. \enddefinition

By continuity of $\widehat{\chi}_K$, the set $A(K)$ is
separated.  Namely, there exists a uniform constant $c=c(K)>0$
such that $|a-a'| \ge c$, for all non-equal $a,a' \in A$. The
notation $|\cdot|$ stands for the Euclidean distance. Indeed,
since $K$ is symmetric (0.1) is a cosine transform. Then the
constant $c$ is bounded from below in inverse proportionality to
the length of the longest diagonal of $\partial K$, passing
through the center. So, if $A$ is a set of orthogonal exponents,
it can be sequentially completed to a maximal set, and we will
further assume that this is the case. Besides, any translation
of $A$ is also a set of orthogonal exponents, so in order to fix
$A$ in a certain sense, let's assume that $0\in A$.



Our main result is the following.

\proclaim{Theorem 0.1} Let $K \subset {\Bbb R}^d$ be a closed
symmetric convex domain of unit volume with a smooth  boundary
$\partial K$, such that the Gaussian curvature does not vanish
anywhere on $\partial K$. Then for $d \neq 1 \mod(4)$, any
maximal set $A(K)$ of orthogonal exponents for $K$ is finite.
Otherwise, either $A(K)$ is finite, or it is a subset of some
one-dimensional lattice.
\endproclaim

Fuglede \cite{Fug74} proved non-existence of an infinite set
$A(K)$ for a disk in ${\Bbb R}^2$. In the process of revising
this manuscript, we became aware of the recent paper
\cite{Fug01} extending the result to a disk in ${\Bbb R}^d$ for
any $d\geq2$.

Theorem 0.1 is driven by strict convexity of $K$ which is
heavily used throughout the proof. Otherwise, a unit cube in
$R^d$ has a basis of orthogonal exponentials, a cylinder has
infinitely many of them located on the symmetry axis. A somewhat
less obvious example of a convex set with an orthogonal
exponential basis is a hexagon in $R^2$.

As we mention above, the main tool in the forthcoming proof of
Theorem 0.1 is the following Erd\"{o}s combinatorial principle
on integer distances.

\proclaim{Lemma 0.2}  Suppose, $T$ is an infinite point set in
${\Bbb R}^d$ such that the the distance set of $T$  is a subset
of the set of positive integers ${\Bbb N}$.
Then $T$ is contained in a straight line.
\endproclaim

A two-dimensional version of this statement appeared in
\cite{Erd\"{o}s45} and is well known. The higher dimensional
version follows from the same proof. The higher-dimensional
generalization provided by Lemma 0.2 follows from a theorem of
Kuz'minyh \cite{Kuz77}, which allows any $d\geq2$ and the
distances being only asymptotically integer with
$\lim\sup_{n\rightarrow\infty}n\epsilon(n)=0$, where
$\epsilon(n)$ is the difference between the Euclidean distance
between a pair of points and an integer $n$. This theorem is
further generalized in Lemma 1.4  below. In particular, the
Euclidean structure of ${\Bbb R}^d$ is irrelevant for the
principle in question. Hence, it can be extended to the case
when the distance is defined in terms of the Legendre transform
of the Minkowski functional of $K$, which appears further in the
asymptotic formula (1.1) for $\widehat{\chi}_K$.

We remark that
Lemma 0.2 is certainly driven by curvature. It would not be true,
for instance, if $K$ were a unit square, i.e if the distance
were the "taxi-cab" distance. Indeed, the latter distance between
any two points of the integer lattice is an integer.



We start out by giving a brief outline of the proof.  Let $\rho$
denote the Minkowski functional associated with $K$. Namely,
$\rho(x)$ is a degree one homogeneous function\footnote{By
Euler's  homogeneity relation,  $\rho(x)$ is a solution of the
first order PDE boundary value problem
$x\cdot\nabla\rho=\rho,\,\rho_{\partial K}=1$, where
$x=(x_1,\ldots,x_d)$. Then $\|x\|_\rho=\rho(x)$ is a norm
equivalent to the Euclidean one $|\cdot|$. The latter
corresponds to the case when $K$ is a disc.}, such that
$K=\{x:\rho(x) \leq 1\}$. Let
$$ \rho_*(\xi)=\sup_{x \in \partial K}x \cdot \xi \tag0.2$$
be the norm, dual to $\rho$. It is also equivalent to the
Euclidean norm,  for the set $K^{*}=\{\xi: \rho_*(\xi)=1\}$ dual
to $K$ retains all the essential geometric properties of $K$. It
is a standard calculation [Herz64] to show that at every point
on the boundary of $K^{*}$ the Gaussian curvature is inversely
proportional to the Gaussian curvature at the corresponding
point on the boundary of $K$, with the proportionality
coefficient bounded in terms of $K$.

The exposition proceeds as follows. In Lemma 1.1 we argue
that if $A$ is a set of orthogonal exponents for $K$, then for
non-equal $a,a'\in A$
$$\rho_*(a-a')=\frac{k}{2}+{d-1\over8}+O
\left({|a-a'|}^{-1} \right), \ k \in {\Bbb N}. \tag0.3$$ The
error term can be expanded further to higher orders of
asymptotics. It follows that  given a fixed pair of $a_0,a_1\in
A$ (in the sequel we assume by default that $a_0\neq a_1$), for
$a\in A$ one has

$$|\rho_*(a_0-a)- \rho_*(a_1-a)|=\frac{k}{2}+O
\left({|a|}^{-2} \right), \ k \in {\Bbb Z}. \tag0.4$$

Having established the above two formulas, we proceed by {\it
reductio ad absurdum}.  Suppose, $A$ is maximal and  infinite.
Lemma 1.4 provides an asymptotic in the sense of the formula
(0.4) $\rho_*$-generalization of Lemma 0.2 in order to see that
all the members of $A$ live precisely on some straight line $L$.
This is in contradiction with the formula (0.3), unless the
phase shift ${d-1\over8}$ in the later formula is a half-integer
itself when $d=4k+1,\,k\in{\Bbb N}.$ In the latter case $A$
shall be a subset of a lattice supported on $L$.

Then we look at the asymptotics for the zeroes of the Fourier
transform of the characteristic function of $K$, restricted to
the line $L$ in order to argue that for a variety of $K$'s,
including the $d$-disk, the contradiction still can be obtained.
The relevant calculations constitute the scope of Lemma 1.6.
However, the non-existence result for all $K$'s is not true in
these dimensions, as is illustrated by examples.

\head Proof of Theorem 0.1 \endhead

The following lemma
gives an asymptotic expression for the Fourier transform
$ \widehat{\chi}_K$ of the characteristic function $\chi_K$ of $K$.

\proclaim{Lemma 1.1} Let $K$ be as in Theorem 0.1. Then for any
$N\in\Bbb N$,
$$ \matrix
\widehat{\chi}_K(\xi)&=&\sum_{\alpha=0}^N
C_\alpha\left(\xi\over|\xi|\right)J_{{d\over2}+\alpha}(2\pi\rho_*(\xi))
{|\xi|}^{-\frac{d}{2}-\alpha}&+& O\left(|\xi|^{-{d+3\over2}+N}\right) \\ \hfill \\
&=&\tilde C_0\left(\xi\over|\xi|\right)\sin\left(2\pi\rho_*(\xi)-\pi{d-1\over4}\right)
{|\xi|}^{-\frac{d+1}{2}}&+&O\left(|\xi|^{-{d+3\over2}}\right).
\endmatrix
\tag1.1$$  For $\alpha=0,\ldots,N$ the functions
$C_\alpha,\tilde{C}_0$ are  smooth functions of $K$ alone, with
$C_0, \tilde C_0$ being strictly positive.
\endproclaim
By the $O(\cdot)$ symbol, we mean that the constants buried in
it are functions of $K$ alone. Below we shall give the sketch of
the  proof which will be further revisited in Lemma 1.6. Details
can be found in \cite{Herz64} and \cite{Sogge93}.

Without loss of generality  assume that $K$ is centered at the
origin and a chosen $\xi$ is directed along the axis, dual to
the $x$-axis, namely $\xi=(z,0,\ldots,0)$.  Let $F(x)$ be the
cross-sectional area of $K$ along the $x$-axis. One can assume
that $F$ is supported on the interval $[-1,1]$: to make up for
this assumption the ensuing formulae should be amended by
writing $\rho_*(\xi)$ instead of $z$. From symmetry of $K$, the
function $F(x)$ is even. Let us compute its Fourier transform
$\widehat F(z)$. Locally near a point where the $x$-axis
intersects  $\partial K$, there exists a smooth function
$x=x(y)$, where $y=(y_1,\ldots,y_d)$ embraces the rest of the
coordinates. It can be rewritten as
$$1-x^2 = Qy\cdot y + O(|y|^3), \tag1.2$$
where $Q$ is a positive definite quadratic form with constant
coefficients and $(\cdot)$ - the Euclidean scalar product. By
the Morse lemma, the function in the right hand side can be
conjugated to $\sum_{i=1}^d y_i^2$ via a smooth change of the
$y$-variables. Thus locally near $x=\pm1,$ \vskip.125in

$$
F(x)=\chi_{[-1,1]}\int_{|y|\leq\sqrt{1-x^2}}|{\frak
J}(y)|dy=\chi_{[-1,1]} \sum_{\alpha=0}^N\left[C_\alpha
(1-x^2)^{{d-1\over2}+\alpha} + O \left(
(1-x^2)^{{d+1\over2}+N}\right)\right]. \tag1.3$$ Here
$\chi_{[-1,1]}$ is the characteristic function of the interval
$[-1,1]$ and ${\frak J}(y)=C_0+O(|y|)$ is the Jacobian of the
coordinate change.  The constant $C_0=(\det Q)^{-1/2}$ is
positive by the strict convexity assumption. The Jacobian can be
Taylor expanded up to order $2N+1$ in powers of $y$, and by
symmetry, odd powers of $y$ contribute zero in the above
integral. Away from $x=\pm1$, the function $F(x)$ is smooth.

Taking the Fourier transform of the above expansion  for $F$,
one obtains the following expansion in the Bessel functions:
$$
\widehat F(z)=\sqrt{\pi}\sum_{\alpha=0}^N C_\alpha\Gamma \left(
{d+1\over2}+\alpha\right)\left( {1\over \pi
z}\right)^{{d\over2}+\alpha} J_{{d\over2}+\alpha}(2\pi z) +
O\left(|z|^{{d+3\over2}+N} \right). \tag1.4$$ Changing the
support of the cross-section area function $F$ depending on  the
direction according to the Minkowski functional $\rho(x)$
results in the appearance of its dual $\rho_*(\xi)$ in the
formula (1.1) instead of $z$ above. Change of the the direction
of the $x$-axis makes the quantities $C_\alpha$ smooth functions
on $RP^{d-1}$, with $C_0$ bounded away from zero. These
functions also absorb the rest of the constants. The second line
of the formula (1.1) follows from the well known asymptotic
expansion for the Bessel function $J_{d\over2}$ in the principal
term of the sum. $\blacksquare$



Then (1.1) yields (0.3) and (0.4). The phase shift in the
formula (0.3)  will come into play later. The purpose of the
following argument is to give an asymptotic $\rho_*$-version of
the combinatorial principle, formulated as Lemma 0.2.

Let the set of the orthogonal exponents $A$ for $K$ be maximal
and  countably infinite. From (0.3) we notice that the pairwise
$\rho_*$-distances between the members of  $A$ cling to the
lattice ${1\over8}\Bbb Z$ as these points get farther from each
other. By changing the scale, there is no harm assuming that
$$\forall a,a'\in A,\,a\neq a',\,\,\rho_*(a-a')=k+
O(|a-a'|^{-1}), \,\, k \in {\Bbb N}. \tag1.5 $$


Choose a pair of points $a_0,a_1\in A$. For $a\in A$ consider
the  possible values for the limit
$\lim_{|a|\rightarrow\infty}|\rho_*(a_0-a)-\rho_*(a_1-a)|\equiv
l$. The asymptotic expansion (1.1) imposes stringent constraints
on possible location of the "infinitely distant" point $a$, i.e.
$$
|\rho_*(a_0-a)-\rho_*(a_1-a)|=l+O(|a|^{-2}),\qquad
l\in\{0,1,\ldots, [\rho_*(a_0-a_1)]\}\subset{\Bbb Z}. \tag1.6$$
Above $[\cdot]$ stands for the integer part.  The formula (1.5)
is a  somewhat coarser restatement of (0.3), with a change of
scale for convenience in order to prove Lemma 1.4. The proof is based on
exploiting the relation (1.6).

\proclaim{Lemma 1.2} As $|a|\rightarrow\infty$, a point $a\in A$
is either located within a tubular neighborhood of the line
connecting the points $a_0,a_1$ (which implies that
$\rho_*(a_0-a_1)\in {\Bbb N}$) or approaches asymptotically  one
of $\rho_*$-hyperboloids $\Gamma(a_0,a_1, \Delta)$ with  foci
$a_{0},a_1$ and an integer parameter
$\Delta,\,0\leq\Delta\leq[\rho_*(a_0-a_1)]$. A
$\rho_*$-hyperboloid is defined as follows:
$$ \Gamma(a_0,a_1,\Delta)\equiv\{x \in {\Bbb R}^d:
|\rho_*(x-a_0)-\rho_*(x-a_1)|= \Delta\}.\tag1.7 $$
\endproclaim

Before proving Lemma 1.2 we summarize the relevant properties of
$\rho_*$-hyperboloids. In the sequel, the midpoint $O_1$ of the
section $[a_0,a_1]$ will be referred to as a {\it vertex} of the
$\rho_*$-hyperboloid $\Gamma(a_0,a_1, \Delta)$ and the line
$a_0a_1$ connecting the foci as its {\it axis}.
\proclaim{Proposition 1.3} \roster
\item
$\Gamma(a_0,a_1,\Delta)$ is a straight
line coinciding with its axis iff $\Delta=\rho_*(a_0-a_1)$.
\item
Otherwise it is a smooth unbounded hypersurface,
symmetric with respect to the vertex.

(i) If $0<\Delta<\rho_*(a_0-a_1)$, this hypersurface has two
connected components. The intersection of a connected component
with any two-plane containing the foci is a smooth curve
intersecting the axis transversely at a point distinct from the
vertex and asymptotic to a pair of transverse rays emanating
from the vertex on both sides of the axis.


(ii) If $\Delta=0$, this hypersurface has one connected
component. Its intersection  with any two-plane containing the
foci is a  smooth curve passing through the  vertex and
asymptotic to a straight line passing through the vertex and
intersecting the axis transversely.
\endroster
\endproclaim
The first statement follows from the fact that
$\rho_* $ is a norm. The transversality and smoothness statements
follow from strict convexity and smoothness of the function
$\rho_*$ defined in terms of the body $K$ only.
The statement about asymptotics is proved as follows.


Fix $\Delta$ such that $0\leq\Delta<\rho_*(a_0-a_1)$ and  consider the
intersection ${\Cal H}={\Cal H}(\Pi)$ of
$\Gamma(a_0,a_1,\Delta)$ with a two-plane $\Pi$ containing the
axis $a_0a_1$. Then a branch of ${\Cal H}$ can be smoothly
parameterized
by $x(t)$, where $x\in \Pi$, $t\in{\Bbb R}$. Clearly,
$|x|\rightarrow\infty$ as $t\rightarrow\infty$.
Let $f(x)$ be the restriction of the norm $\rho_*$ to the plane $\Pi$.
Then one branch of ${\Cal H}$ is given by $f(x-a_0)-f(x-a_1)=\Delta$.
Without loss of generality, suppose that the vertex $O_1$
is the origin.

Suppose, $f(x)=\epsilon^{-1}$ for some small $\epsilon>0$. Since $f$
is a homogeneous function of degree one, one has
$f(\epsilon(x-a_0))-f(\epsilon(x-a_1))=\epsilon\Delta$. Then,
$\nabla f(\epsilon x)\cdot (a_1-a_0)=\Delta+O(\epsilon)$,
where $(\cdot)$ is the Euclidean scalar product on $\Pi$.

The gradient $\nabla f(\epsilon x)$ is evaluated on the (smooth)
boundary of a strictly convex body $K_2=K^*\cap\Pi$. It is a
homogeneous function of degree zero, thus $\nabla f(\epsilon
x)=\nabla f(x)$. Then as $\epsilon\rightarrow0$, there exists a
limit $x_\Delta\in K_2$ such that $\nabla f(x_\Delta)\cdot
(a_1-a_0)=\Delta$. In other words, as $t\rightarrow\infty$, the
intersection of the line connecting $x$ with the midpoint $O_1$
of the section $[a_0a_1]$ with $K_2$ limits at $x_\Delta$.

Furthermore, differentiating the equation
$f(x(t)-a_0)-f(x(t)-a_1)=\Delta$ with respect to $t$, and
considering the limit as $t\rightarrow\infty$, we see that
$\dot{x}(t)$ must be perpendicular to the vector $\nabla
f(x(t)-a_0) - \nabla f(x(t)-a_1)$. The latter vector  approaches
a vector tangent to  $K_2$ at the point $x_\Delta$. Hence, there
exists a limit for the direction of $\dot{x}(t)$ as
$t\rightarrow\infty$, so the straight line $O_1x_\Delta$ is the
asymptote for $x(t)$ as $t\rightarrow\infty$. In the same
fashion, there exists an asymptote as $t\rightarrow -\infty$.
From central symmetry, if $0<\Delta<\rho_*(a_0-a_1)$ the same
pair of lines are asymptotes for the second connected component
of ${\Cal H}$. $\blacksquare$



We now prove Lemma 1.2. Consider some $a\in A$ such that
$\rho_*(a-a_0) = \epsilon^{-1}$ for a small $\epsilon>0$.
Restrict the analysis to some plane $\Pi$ containing the points
$aa_0a_1$ (they can be on the same line). Let $f(x), \,x\in \Pi$
be the restriction of the distance $\rho_*$ to the plane $\Pi$.
The point $a$ lies on the $f$-circle $f(a-a_0) = \epsilon^{-1}$
in this plane. By (1.6), $f(a-a_1) =
\epsilon^{-1}-\Delta+O(\epsilon^2),$ where
$\Delta\in\{0,1,\ldots,[\rho_*(a_0-a_1)]\} \subset{\Bbb Z}$. We
shall consider two cases, $\Delta=[\rho_*(a_0-a_1)]=
\rho_*(a_0-a_1)$, and $0 \leq \Delta <\rho_*(a_0-a_1)$. In the
former case, the point $a$ lies at the intersection of the
$f$-circle $f(a-a_0) = \epsilon^{-1}$ and an $f$-annulus
centered at $a_1$, of radius $\epsilon^{-1}-\rho_*(a_0-a_1)$ and
width $O(\epsilon^2)$. The $f$-circles $f(a-a_0) =
\epsilon^{-1}$ and $f(a-a_1) = \epsilon^{-1}-\rho_*(a_0-a_1)$
are tangent to one another at a point $b$, which is located on
the line containing $a_0$ and $a_1$. Since an $f$-circle is
locally a parabola (by the strict convexity assumption) the
point $a$ is contained in a  const.$\times\epsilon^2$ rectangle
centered at $b$, where the constant depends only on
$\rho_*(a_0-a_1)$ and the bounds for the curvature on $\partial
K$.

In the case $0 \leq \Delta <\rho_*(a_0-a_1)$ the intersection
of the $f$-circle $f(a-a_0) = \epsilon^{-1}$ and the
$f$-annulus centered at $a_1$, of radius $\epsilon^{-1}-\Delta$
and width $O(\epsilon^2)$ is transverse with the angle of
$O(\epsilon)$, which implies the second claim. $\blacksquare$

The following statement is the key aspect of the proof.

\proclaim{Lemma 1.4} If $A$ is infinite, there exists a straight
line $L$ containing it.
\endproclaim
The lemma follows from the following claim.

\vskip.125in

\noindent {\bf Claim:} For any pair of points $a_0,a_1\in A$ it
is impossible to have infinitely many members of $A$ lying
outside some tubular  neighborhood of the straight line
connecting $a_0$ and $a_1$.

The lemma follows from the claim immediately. To see this,
assume the claim  and suppose that  some point $a_2$ lies
outside the line connecting $a_0$ and $a_1$. Then all but
finitely many points of $A$ lie in a tubular neighborhood of the
line connecting $a_0$ and $a_2$. In same fashion, all but
finitely many points of $A$ must lie in a tubular neighborhood
of the line connecting $a_0$ and $a_1.$ The intersection of
these two tubular neighborhoods is a bounded set which cannot
contain infinitely many members of $A$ due to the fact that $A$
is separated. This argument will be repeated throughout the rest
of the proof.

To prove the claim, suppose is not true. Then there exists a
pair of points $a_0,a_1\in A$, such that (by Lemma 1.2) there is
an infinite set $A_1\subset A$, such that the members $a\in A_1$
approach asymptotically some $\rho_*$-hyperboloid
$\Gamma(a_0,a_1, \Delta)$, for some integer $\Delta$,
$0\leq\Delta <\rho_*(a_0-a_1)$.

Consider $A_1\cup\{a_0,a_1\}$. Let $C_1$ be the asymptotic cone
for the above hyperboloid with  the vertex $O_1$. $A_1$ is an
infinite set of points located asymptotically close to $C_1$. By
Lemma 1.2 and Proposition 1.3, one can always find a point
$a_2\in A_1$ such that the midpoint $O_2$ of the segment
$[a_0a_2]$ lies outside the cone $C_1$.

Then, since the lines $a_0a_1$ and $a_0a_2$ are transverse,
there must exist an infinite subset $A_2\subseteq A_1$ of points
lying asymptotically close to some $\rho_*$-hyperboloid
$\Gamma(a_0,a_2,\Delta)$ with an integer
$\Delta,\,0\leq\Delta<\rho_*(a_0-a_2)$ (the inequality is
strict, because the line $a_0a_2$ is transverse to the cone
$C_1$), hence to the cone $C_2$ with the vertex $O_2$.  Thus,
the members of $A_2$ lie asymptotically close to the
intersection ${\Cal S}_2\equiv C_1\cap C_2$ (let also ${\Cal
S}_1=C_1$). If the ${\Cal S}_2$ is bounded, there is a
contradiction with the assumption that $A_2$ is infinite.
Moreover, the intersection of the two cones along ${\Cal S}_2$
is transverse, for any point thereon  is formed by the
intersection of  a pair of straight lines connecting it to the
vertices $O_1$ and  $O_2$, the latter vertex lying by
construction outside the cone $C_1$. Thus, ${\Cal S}_2$ is a
piecewise smooth unbounded surface of dimension $d-2$.

One can continue reducing the dimension by induction. Before the
$i$th step, $i\geq 1$, there will be cones $C_j,\,j=1,\ldots,i$,
with foci $a_0,a_j$ intersecting transversely at a piecewise
smooth unbounded surface ${\Cal S}_i\equiv \bigcap_{j=1}^{i}C_j$
of dimension $d-i$. There will be an infinite set
$A_{i}\subseteq A_1$ of points located asymptotically close to
${\Cal S}_i$. At this point again one can again pick a point
$a_{i+1}$ such that the midpoint $O_{i+1}$ of the segment
$[a_0a_{i+1}]$  lies outside $\bigcup_{j=1}^{i} C_j$, for $a_0$
is a focus for all the hyperboloids asymptotic to the cones
$C_1,\ldots C_{i}$. Furthermore  there is an infinite set
$A_{i+1}\subseteq A_i$ of points, asymptotic to some cone
$C_{i+1}$, whose vertex $O_{i+1}$ does not belong to any of the
cones $C_1,\ldots,C_i$. Hence, for any point of ${\Cal S}_i$,
the line connecting it with $O_{i+1}$ is transverse to each of
the cones $C_1,\ldots,C_i$, thus to their intersection.
Therefore, the intersection ${\Cal S}_{i+1}\equiv
C_{i+1}\cap{\Cal S}_i$ is transverse an should be an unbounded
piecewise smooth surface of dimension $d-i-1$.

It  becomes impossible to proceed with this construction for
$i\geq d$, without a contradiction with the fact that $A_1$
should be infinite.  This proves the claim and the lemma.
$\blacksquare$

\proclaim{Proposition 1.5} If $d\neq 1 \mod (4)$, the set $A$ is
finite. Otherwise, if $d=4k+1,\,k\in {\Bbb N}$, the set $A$ is
either finite or it is an infinite subset of a lattice supported
on some line $L$. \endproclaim

Indeed, if $A$ is  infinite, by Lemma 1.4 it is contained in
some line $L$.  By fixing $0\in A$, we ensure that $L$ passes
through the origin.

If $d\neq 1 \mod(4)$, take three different points
$a_0,a_1,a_2\in A$ such that $a_1$  lies between $a_0$ and
$a_2$. Then $\rho_*(a_2-a_0)=\rho_*(a_2-a_1)+\rho_*(a_1-a_0)$.
Since $A$ is separated, the points $a_0,a_1,a_2$ can be chosen
far enough from each other, so that the formula (0.3) cannot be
satisfied simultaneously for the quantities $\rho_*(a_2-a_0),
\rho_*(a_2-a_1)$ and $\rho_*(a_1-a_0)$, due to the constant
phase shift ${d-1\over8}$ in it.

If $d=1 \mod(4)$, then the phase shift ${d-1\over8}$ is a
half-integer itself.  In this case, having infinitely many
points on the line $L$ would imply that for any pair of points
$a_0,a_1$ there exists a distant point $a$ such that the
distances $\rho_*(a-a_0)$ as well as  $\rho_*(a-a_1)$ are
arbitrarily close to a half-integer. Thus $\rho_*(a_1-a_0)$ is a
half-integer itself. $\blacksquare$

This completes the proof of Theorem 0.1. Finally, let us
investigate  further the case $d=4k+1,\, k\in {\Bbb N}$. As in
the proof of Lemma 1.1, suppose $L$ is dual to the $x$-axis. Let
$F(x)$ be the cross-sectional area of $K$ along the $x$-axis.
Suppose $F$ is supported on the interval $[-1,1]$. At the points
$x=\pm1$ the function $F(x)$ has a smoothness defect of the same
type and order as the function $(1-x^2)^{2k},$ corresponding to
the case when $K$ is a $d$-disk. Consider the Fourier series
expansion
$$F(x)=\chi_{[-1,1]}\sum_{n=0}^\infty F_n\cos(\pi nx), \tag1.8$$
where
$$F_0=1,\qquad F_n=\int_{-1}^1 F(x)\cos(\pi n x)dx, \ \ \
n\in{\Bbb N}. \tag1.9$$

The existence of an infinite set of orthogonal exponents $A(K)$
with $0\in A$,  supported on the line $L$, is equivalent to the
existence of an infinite set $E\subset {\Bbb N}$ such that for
all $n\in {\Bbb N}\cap[E\cup (E-E)]$ the corresponding Fourier
coefficients $F_n=0$.

In general, this is possible. E.g. let
$$F(x)=\chi_{[-1,1]}c_k[1+\cos(\pi x)]^k,\ k\in {\Bbb N},
\tag1.10$$
where the constant $c_k>0$ is chosen to ensure $F_0=1$.  Then
for $n>k$ one has $F_n=0$ for the the Fourier coefficients $F_n$
above. In dimension $d=4k+1$ take  $K$ as a body of revolution
$$r(x)=\left[\tilde{c}_k(1+\cos(\pi x))\right]^{1\over4},
\tag1.11$$
where $r$ is the radius-vector in ${\Bbb R}^{4k}$. The constant
$\tilde{c}_k>0$  is to yield the above cross-section area
$F(x)$. The function $r(x)$ has a negative second derivative for
$x\in(-1,1)$, whereas locally near $x=\pm1$ it has the same type
of singularity as the function $\sqrt{1-x^2}$. Thus $K$ obtained
this way is strictly convex.

However, for a variety of $K$'s in ${\Bbb R}^{4k+1},\,k\in{\Bbb
N}$,  the non-existence of an infinite set $A(K)$ is still the
case.

\proclaim{Lemma 1.6} Suppose $K$ is such that the cross-section
area  $F(x)$ in any direction after a suitable dilation of $K$
can be represented as \vskip.125in
$$ F(x)=\chi_{[-1,1]}\left(\sum_{m\geq 2k}^S C_m(1-x^2)^m +\epsilon
R(x)\right),\qquad C_{2k}>0, \;C_S\neq0, \qquad S=2k+N, \;
N\geq0, \tag1.12$$ where the error term $R(x)$ is a smooth even
function of $x$, with  $R^{(\alpha)}(1)=0$ for
$\alpha=0,1,\ldots,S$. Then for small $\epsilon$ the maximal set
$A(K)$ is finite.
\endproclaim

The proof is a direct computation. Using the Bessel function
expansion of Lemma 1.1 along with the formula
$$
\matrix
J_{m+1/2}(z) &= &\sqrt{{2\over\pi z}} &\left\{  \sin\left( z-{\pi\over2}m \right) \;
\sum_{l=0}^{[m/2]} {(-1)^l (m+2l)! \over (2l)!(m-2l)!(2z)^{2l} } \right.
\\ \hfill \\
& &+ & \left.\cos\left( z-{\pi\over2}m\right)\; \sum_{l=0}^{[(m-1)/2]}
{(-1)^l (m+2l+1)!\over (2l+1)!(m-2l-1)!(2z)^{2l+1} } \right\}.
\endmatrix \tag1.13$$
we obtain (with the notation $[\cdot]$ for the integer part):
$$
\matrix
F_n &=&\left( {2\over\pi n}\right)^{4k} & \sum_{s=0}^{N}(-1)^{s+n}(2s+2k+1)!
\left( {2\over\pi n}\right)^{2s} \\ \hfill \\
& & \times & \sum_{l=0}^{s+1}(-1)^l4^{-l}C_{2s+2k+1-l}\left(\matrix2s+2k+1-l\\
l\endmatrix \right) \\ \hfill \\
&&+&\epsilon \;O\left( n^{-2\left[{S+1\over 2}\right]-2} \right).
\endmatrix
\tag1.14$$

The quantities, expressed by the sums in the second line of
(1.14)  bear responsibility for $F_n$ being nonzero. These
quantities are listed in the following table. Given $k\geq1$,
let $C_m=0$ for $m<2k$, whereas $C_{2k}>0$. Also $C_S\neq0$.
Take the scalar product of the first row of the table with each
subsequent $i$th row of the table, $i=2,\ldots$ in order to get
a coefficient, multiplying $n^{-2i-2}$ up to a nonzero factor,
coming from the first line of the formula (1.14).
\vskip.125in
$$
\matrix
-C_2&C_3&-C_4&C_5&-C_6&C_7&\ldots&(-1)^SC_{S-1}& (-1)^{S+1}C_{S} \\ \hfill \\
{1\over4} \left(\matrix 2\\ 1\endmatrix\right) &  \left(\matrix 3\\ 0\endmatrix\right)
&0&0&0&0&\ldots&0&0\\ \hfill \\
0& {1\over16} \left(\matrix 3\\ 2\endmatrix\right) & {1\over4} \left(\matrix 4\\
1\endmatrix\right) &
\left(\matrix 5\\ 0\endmatrix\right)&0&0&\ldots &0&0\\ \hfill \\
0& 0&{1\over64} \left(\matrix 4\\ 3\endmatrix\right) & {1\over16} \left(\matrix 5\\
2\endmatrix\right)
& {1\over4} \left(\matrix 6\\ 1\endmatrix\right)&
\left(\matrix 7\\ 0\endmatrix\right)&\ldots&0&0 \\
\hfill \\
&&&&\ldots\\ \hfill \\
0&0&0&0& 0 &0&\ldots &0&{1\over 4^{S-1}} \left(\matrix S\\ S-1\endmatrix\right)\\
\hfill \\
\endmatrix
$$ As $C_{2k}>0,\,C_S\neq0$ and $\epsilon$ is small
enough, it follows that  as $n\rightarrow\infty$, the absolute
value of $F_n$ is asymptotically bounded away from zero by a
positive constant times ${\displaystyle n^{-2\left[{S+1\over
2}\right]-2}.\;\blacksquare}$







\definition{Acknowledgment} We are grateful to J.P. Keating
for having pointed out the connection between the orthogonal
exponentials and problems arising in studies of quantum
billiards. We are also grateful to M. Kolountzakis for
several helpful suggestions. A.I. wishes to thank Professor
Fuglede for the offprint of his paper \cite{Fug01} and for his
interest in his work on related topics.
\enddefinition


\newpage
\head Bibliography \endhead

\ref \key AP \by P.  Agarwal and  J. Pach \paper Combinatorial geometry
 \yr 1995 \jour Wiley-Interscience Series
in Discrete Mathematics and Optimization, a Wiley-Interscience Publication.
John Wiley \& Sons, Inc., New York.
\endref

\ref\key Berry94 \by M.V. Berry \paper Evanescent and real waves in quantum
billiards and Gaussian beams \jour J. Phys. A: Math. Gen. \yr 1994  \vol 27
\pages L391-L398 \endref

\ref \key Erd\"{o}s45 \by P. Erd\"{o}s \paper Integral distances \yr
1945 \jour Bull. Am. Math. Soc. \vol 51 \pages 996 \endref

\ref \key Falc87 \by K.J. Falconer \paper On the Hausdorff dimensions of
distance sets \jour Mathematika \vol 32 \yr 1985  \pages 206--212
\endref

\ref \key Fug01 \by B. Fuglede \paper Orthogonal Exponentials on the Ball \jour
Expo. Math. \vol 19 \yr 2001\pages 267-272 \endref

\ref \key Fug74 \by B. Fuglede \paper Commuting self-adjoint
partial differential operators and a group theoretic problem \jour
J. Funct. Anal. \yr 1974 \vol 16 \pages 101-121 \endref


\ref \key Herz64 \by C.S. Herz \paper Fourier transforms related
to convex sets \jour Ann. of Math. \vol 75 \yr 1962 \pages 81-92
\endref

\ref \key IKT01 \by A. Iosevich, N. Katz, and T. Tao \paper Convex
bodies with a point of curvature do not have Fourier bases \jour
Amer. J. Math. \vol 123 \yr 2001 \pages 115-120 \endref

\ref \key IKT02 \by A. Iosevich, N. Katz, and T. Tao  \paper
Fuglede conjecture holds for convex planar domains
\jour (submitted for publication)
\yr 2001  \endref

\ref \key Kol00 \by M. Kolountzakis \paper Non-symmetric convex
domains have no basis of exponentials \jour Illinois J. Math. \vol
44 \yr 2000 \pages 542--550 \endref

\ref \key Kuz77 \by A.V. Kuz'minyh  \paper
Integer-valued distances (Russian)
\jour Dokl. Akad. Nauk SSSR \vol 232 \yr 1977 \pages 1008--1010 \endref


\ref \key Sogge \by C.D. Sogge \paper Fourier integrals in classical analysis
\jour  Cambridge Tracts in Mathematics, Cambridge University Press, Cambridge
\vol 105  \yr 1993 \endref

\ref \key Wolff99 \by T. Wolff \paper Decay of circular means of
Fourier transforms of measures \jour Internat. Math. Res.
Notices \yr 1999 \vol 10 \pages 547--567 \endref

\enddocument
