\documentstyle{amsppt}
\tolerance 3000 \pagewidth{5.5in} \vsize7.0in
\magnification=\magstep1 \widestnumber \key{AAAAAAAAAAA}
\NoRunningHeads \topmatter
\author
\endauthor
\title Distance sets
\endtitle
\date July 21, 2002
\enddate
\address A. Iosevich, University of Missouri
\ email: iosevich \@ math.missouri.edu
\endaddress
\thanks The research of the authors is partially supported by NSF
grant \endthanks
\endtopmatter
\document

\head Chapter I: Erdos distance problem for general metrics
\endhead

\vskip.125in

In this chapter we begin discussing a beautiful problem introduced
by Paul Erdos (\cite{Erdos45}). The basic idea is to determine the
smallest number of distinct distances determined by a set of
points in a Euclidean space. In order to state the problem
precisely, we need the following definitions.

\definition{Definition 1.1} We say that a set $K \subset {\Bbb R}^d$ is
bounded if it is contained in a ball of finite radius.
\enddefinition

\definition{Definition 1.2} We say that a set $K \subset {\Bbb R}^d$ is
symmetric with respect to the origin if $-x \in K$ whenever $x \in
K$.
\enddefinition

\definition{Definition 1.3} We say that a set $K \subset {\Bbb R}^d$ is
convex if the line segment connecting points $x$ and $y$ is
contained in $K$ whenever $x$ and $y$ are both contained in $K$.
\enddefinition

\definition{Definition 1.4} We say that ${||\cdot||}_K$ is the norm
induced by a bounded convex set $K$, symmetric with respect to the
origin, if ${||x||}_K=\inf \{t: tx \in K\}$. \enddefinition

\definition{Definition 1.5} The cardinality of a finite set $S \subset
{\Bbb R}^d$ is the number of points it contains. \enddefinition

We are now ready to formulate the Erdos Distance Problem (EDP).
Let $S$ be a finite subset of ${\Bbb R}^d$, $d \ge 2$. Let
$$\Delta_K(S)=\# \{{||x-y||}_K: x,y \in S\}, \tag1.1$$ where
${||\cdot||}_K$ is the norm induced by a bounded convex set $K$,
symmetric with respect to the origin. The question we ask is, what
is the smallest possible cardinality of $\Delta_K(S)$,
$$ g_K(n,d)=\inf_{\# S=n} \# \Delta_K(S)? \tag1.2$$

\proclaim{Lemma 1.1} There exists a bounded symmetric convex set
$K$ such that
$$ g_K(n,d) \leq n^{\frac{1}{d}}. \tag1.3$$
\endproclaim

Indeed, let $S={[0,n^{\frac{1}{d}}]}^d \cap {\Bbb Z}^d$, and there
is no harm in assuming that $n^{\frac{1}{d}}$ is an integer. Let
$K={[-1,1]}^d$. Since the difference of two vectors in ${\Bbb
Z}^d$ is a vector in ${\Bbb Z}^d$, $\# \Delta_K(S)=\# \{{||x||}_K:
x \in S\}$. Observe that ${||x||}_K=\max \{x_1, \dots, x_d\}$,
where $x_j$ is the $j$th component of the vector $x \in S$. It
follows that $\# \{{||x||}_K: x \in S\} \leq n^{\frac{1}{d}}$, as
claimed.

Next we shall see that without further assumptions on $K$, Lemma
1.3 gives the best possible estimate for $g_K(n,d)$. More
previsely, \proclaim{Theorem 1.2} (\cite{Erdos45}) Without any
further assumptions on $K$,
$$ g_K(n,d) \ge C{(n-1)}^{\frac{1}{d}}. \tag1.4$$
\endproclaim

The proof is by induction on the dimension. We first prove the
result in two dimensions. Fix any point of $S$ and observe that
the remaining $n-1$ points of $S$ are located on $t$ circles
centered at that point. What $t$ is a bit of a mystery, but it is
an integer between $1$ and $n-1$. Let $N$ denote the largest
number of points of $S$ among those circles. It follows that
$$ Nt \ge n-1. \tag1.5$$

Consider the circle containing $N$ points of $S$. At least
$\frac{N}{2}$ of these points must lie on the same semi-circle. If
we fix one of these points, and measure distances to the other
points on the same semi-circle, we see that the number of distinct
distances is at least $\frac{N}{2}$. It follows that in two
dimensions,
$$ \# \Delta_K(S) \ge \max \left\{\frac{N}{2}, \frac{n-1}{N} \right\}.
\tag1.6$$

It follows that
$$ \# \Delta_K(S) \ge \sqrt{\frac{n-1}{2}}. \tag1.7$$

This completes the proof of the two-dimensional case of Theorem
1.2. We now observe that the proof above works just as well if we
replace ${\Bbb R}^2$ by the boundary of a bounded convex set in
${\Bbb R}^3$. This allows us to proceed by induction. Suppose that
Theorem 0.2 holds on ${\Bbb R}^{d-1}$ and also on the boundary of
a bounded convex set in ${\Bbb R}^d$. We must now prove that
Theorem 0.2 holds in ${\Bbb R}^d$. Fix a point of $S$ and observe
that the remaining $n-1$ points are located on $t$ spheres
centered at that point. As before, the positive integer $t$ is a
mystery to be resolved. Let $N$ again denote the largest number of
points of $S$ among those spheres. As before, $(1.5)$ holds. By
the induction hypothesis, the sphere with $N$ points determines at
least $C_dN^{\frac{1}{d-1}}$ distances for some $C_d>0$. It
follows that
$$ \# \Delta_K(S) \ge \max \left\{C_dN^{\frac{1}{d-1}}, \frac{n-1}{N}
\right\}. \tag1.8$$

It follows that
$$ \# \Delta_K(S) \ge C_d^{\frac{d-1}{d}} {(n-1)}^{\frac{1}{d}},
\tag1.9$$ and the proof of Theorem 0.2 is complete.

Before concluding this chapter, let us summarize what happened. We
posed the Erdos Distance Problem for metrics that come from
bounded convex sets. We saw that the number of distances
determined by $n$ points in $d$ dimensions is at least a constant
multiple of $n^{\frac{1}{d}}$, and we saw that for at least one
metric, the one that comes from $K={[-1.1]}^d$, this result is
best possible.

Where do we go from here? In order to get a hint, let us
re-examine the proof of Lemma 1.1. We saw that if $K={[-1,1]}^d$,
then $\# \Delta_K({[0,n^{\frac{1}{d}}]}^d \cap {\Bbb
Z}^d)=n^{\frac{1}{d}}.$ On the other hand, suppose that $K=\{x \in
{\Bbb R}^d: |x| \leq 1\}$. Then $\#
\Delta_K({[0,n^{\frac{1}{d}}]}^d \cap {\Bbb Z}^d)=\# \{(x_1,
\dots, x_d): x_1^2+x_2^2+\dots+x_d^2 \leq m; m=1,2, \dots,
n^{\frac{2}{d}}\}$. We shall see that the size of this set is very
close $n^{\frac{2}{d}}$ in all dimensions. This leads one to the
following conjecture due to Paul Erdos:

\proclaim{Erdos Distance Conjecture} With the notation above, for
any $\epsilon>0$, there exists $C_{\epsilon}>0$ such that
$$ g_{B_d}(n,d)=\inf_{\# S=n } \Delta_{B_d}(S) \ge C_{\epsilon}
n^{\frac{2}{d}-\epsilon}, \tag1.10$$ where $B_d$ is the unit ball
in ${\Bbb R}^d$. \endproclaim

In plain language, Erdos' conjecture says that the number of
standard Euclidean distances determined by $n$ points is nearly a
constant multiple of $n^{\frac{2}{d}}$. If anything like this
conjecture is true, and we are to understand why, we will need to
grasp the essential geometric difference between the metric that
comes from the cube ${[-1,1]}^d$, and the one that comes from the
unit ball $B_d$.

\vskip.125in

\head Exercises \endhead

\vskip.125in

\proclaim{Exercise 1.1} Explicitly compute the constant in front
of $n^{\frac{1}{d}}$ in Theorem 0.2. Is the resulting constant
best possible in the sense that there exists a set of $n$ points
and a convex set $K$ such that this constant is achieved?
\endproclaim

\proclaim{Exercise 1.2} A general positive-definite metric on
${\Bbb R}^d$ is a function $\rho: {\Bbb R}^d \times {\Bbb R}^d \to
{\Bbb R}$ that satisfies the following aximoms: \roster \item
$\rho(x,y) \ge 0$ and $\rho(x,y)=0$ if and only if $x=y$. \item
$\rho(x,y)=\rho(y,x)$. \item $\rho(x,y) \leq \rho(x,z)+\rho(y,z)$.
\endroster

Does Theorem 0.2 hold for such general metrics? Prove it or give a
counter-example. \endproclaim

\proclaim{Exercise 1.3} Prove that $g_{B_d}(n,d) \leq
Cn^{\frac{2}{d}}$ for some $C>0$. In fact, prove that $g_{K}(n,d)
\leq Cn^{\frac{2}{d}}$ for any $K$. \endproclaim

\proclaim{Exercise 1.4} Give an alternative proof of Theorem 0.2
in the case $K=B_d$ as follows. Let $S$ be a set of cardinality
$n$ in ${\Bbb R}^2$. Number these points from $1$ to $n$. Draw a
circle of radius $R$ around each point and number these circles
from $1$ to $n$. Consider an $n$ by $n$ matrix with entries
$I_{ij}$ such that $I_{ij}=1$ if the $i$th points lies on the
$j$th sphere, and $0$ otherwise.

Our first step is to prove that $I=\sum_{i,j}I_{ij} \leq
Cn^{\frac{3}{2}}$ for some $C>0$. Apply Cauchy-Scwartz, expand the
$2$ond power, and think about what it means for $I_{ij}I_{ij'}=1$
for many values of $i$ given fixed $(j,j')$.

Now observe that you have shown that a single distance cannot
repeat more than $Cn^{\frac{3}{2}}$ times. Since the total number
of distances (possibly repeating) is $\frac{n(n-1)}{2}$, conclude
that the number of distinct distances is at least
$Cn^{\frac{1}{2}}$ for some $C>0$. Now apply induction on the
dimension as in the proof of Theorem 1.2.

Can you carry out the same argument with $B_d$ replaced by an
arbitrary bounded symmetric convex set? If so, carry out the
details, otherwise explain why the proof of Theorem 1.2 given
above seems to give us more flexibility.

\endproclaim

\vskip.125in

\head Chapter II: Incidence theorems and applications to distance
sets
\endhead

\vskip.125in

In Chapter I we saw that there exists a bounded symmetric convex
set $K$ such that $g_K(n,d) \leq Cn^{\frac{1}{d}}$. We also proved
that $g_K(n,d) \ge Cn^{\frac{1}{d}}$ for any bounded symmetric
convex set $K$. However, the dicussion at the end of that chapter
suggests that this estimate may improve if we make appropriate
assumption on $K$. In this chapter we shall see that this is
indeed the case. Before we proceed to state the precise version of
this idea, let us explore further some of the ideas introduced in
the previous chapter.

\definition{Definition 2.1} We say that the pair $(p,l)$ is an incidence
of the point $p$ and the curve $l$ if $p$ is contained in $l$.
\enddefinition

In Exercise 1.4 we gave an alternate proof of Theorem 1.2 in the
case $K=B_d$. We recall that the basic idea was to show that a
single distance cannot repeat more than $Cn^{\frac{1}{2}}$ times
in two dimensions and then use induction on the dimension. The
argument requires one to observe that the intersection of $2$
circles with the same radius and different centers contains at
most two points. The method can be restated in the language of
incidences as follows.

\proclaim{Lemma 2.1} Let $S$ denote a set of $n$ points in ${\Bbb
R}^2$. Let $L$ denote the set of $n$ curves in ${\Bbb R}^2$
satisfying the condition that the intersection of any $2$ of these
curves contains at most $M$ points. Then then number of incidences
among the points of $S$ and curves in $L$ is at most
$C_Mn^{\frac{3}{2}}$. \endproclaim

A reasonable question to ask at this point is whether the
conclusion of Lemma 2.1 is sharp. The answer turns out to be no in
two dimensions, as demonstrated by the following celebrated result
due to Szemeredi and Trotter.

\proclaim{Theorem 2.2} Let $S$ be a set of $n$ points on ${\Bbb
R}^2$ or the boundary of a convex bounded set in ${\Bbb R}^3$. Let
$L$ be a set of lines in ${\Bbb R}^2$, or curves on ${\Bbb R}^2$
(or the boundary of a convex bounded set in ${\Bbb R}^3$
satisfying the following axioms: \roster
\item The intersection of any two curves in $L$ consists of at most
$\alpha$ points. \item No more than $\beta$ curves in $L$ pass
through any pair of points of $S$. \endroster

Then the number of incidences among the points of $S$ and curves
in $L$ is at most $C(n+m+{(nm)}^{\frac{2}{3}}{(\alpha
\beta)}^{\frac{1}{3}})$.

Moreover, this result is sharp, at least in the case
$\alpha=\beta=1$, in the sense that for every $n,m$, there exists
a set $S$ with $n$ points, and the set $L$ with $m$ lines (or
curves) in ${\Bbb R}^2$ satisfying the conditions above such that
the number of incidences is comparable to
$n+m+{(nm)}^{\frac{2}{3}}$. (See Exercise 2.1 below).
\endproclaim

Using Theorem 2.1 and induction on the dimension, we shall prove
the following result.

\definition{Definition 2.2} A translate of a convex body $K$ in ${\Bbb
R}^d$ is the set $x+K$ for some $x \in {\Bbb R}^d$. A dilate of a
convex body $K$ is the set $\lambda K$ for some $\lambda \in {\Bbb
R}^{+}$.
\enddefinition

\proclaim{Theorem 2.3} Let $K$ be a bounded symmetric convex set
in ${\Bbb R}^d$ such that the intersection of any $d$ translates
of arbitrary dilates of $\partial K$ contains at most $2$ points.
Then
$$ g_K(n,d) \ge Cn^{\frac{1}{d-\frac{1}{2}}}. \tag2.1$$ \endproclaim

We assume Theorem 2.2 for a moment and prove Theorem 2.3 in the
two-dimensional case. The extension to higher dimensions is
outlined in Exercise 2.2.

Let $S$ be a set of $n$ points in ${\Bbb R}^2$ (or a boundary of a
convex set in ${\Bbb R}^3$. Let $L$ be the set $\{x+\partial RK: x
\in S\}$. By Theorem 2.2, the number of incidences between points
in $S$ and curves in $L$ is at most $Cn^{\frac{4}{3}}$. It follows
that a single distance cannot repeat more than $Cn^{\frac{4}{3}}$
times, so the number of distinct distances is at least
$Cn^{\frac{2}{3}}$.

Before we turn to the proof of Theorem 2.2, we observe that so far
our approach to estimating the number of distinct distances has
been to show that a single distance cannot repeat very many times.
In fact, the following stronger version of the Erdos Distance
Conjecture has been posed.

\definition{Definition 2.3} We say that a convex curve in ${\Bbb R}^2$ is
strictly convex if it does not contain any straight line segments.
\enddefinition

\proclaim{Single Distance Conjecture (SDC)} Let $S$ be a subset of
${\Bbb R}^2$ containing $n$ points. Then for every $\epsilon>0$
there exists $C_{\epsilon}>0$ such that
$$ \# \{(x,y) \in S \times S: |x-y|=R\}
\leq C_{\epsilon}n^{1+\epsilon}, \tag2.2$$ where $|\cdot|$ denotes
the standard Euclidean distance.
\endproclaim

We shall see that SDC is false if we replace the Euclidean
distance by an arbitrary distance that comes from a bounded
symmetric convex set whose boundary has everywhere non-vanishing
curvature. See Exercise 2.3.

The best known result in the direction of solving the Single
Distance Conjecture is the one implied by Theorem 2.2, namely that
the left hand side of $(2.2)$ is bounded by $Cn^{\frac{4}{3}}$.
This implies, as we note above, that the number of Euclidean
distances determined by $n$ points in the plane is at least
$Cn^{\frac{2}{3}}$. We shall see that this latter estimate has
been strengthened considerably, while the estimate on the number
of repetitions of a single distance has so far resisted efforts to
improve it.

We now turn to the proof of Theorem 2.2.

\definition{Definition 2.4} A graph $G$ with $n$ vertiices and $e$ edges
is a set $G$ of cardinality $n$ equipped with a map $E: G \times G
\to \{0,1\}$. If $E(x,y)=1$, we say that $x$ and $y$ are connected
by an edge. \enddefinition

\definition{Definition 2.5} A planar drawing of a graph $G$ is a drawing
in the plane where elements of $G$ are represented by points
(called vertices) in the plane, and edges, connecting pairs of
points, are represented by piece-wise $C^1$ curves. \enddefinition

\definition{Definition 2.6} A crossing of a planar drawing of a graph is
an intersection of two edges not at a vertex. The crossing number
of a graph $G$ (denoted by $cr(G)$) is the minimum of crossings
over all the planar drawings of $G$. \enddefinition

\definition{Definition 2.7} We say that a graph $G$ is planar if it can
be drawn in the plane without any crossings. \enddefinition

\definition{Definition 2.8} We say that a graph $G$ is connected if one
can reach every vertex from any other vertex by following edges in
some planar drawing of $G$. \enddefinition

\definition{Definition 2.9} We say that a graph $G$ is simple if
every pair of vertices is connected by at most $1$ edge.
\enddefinition

\definition{Definition 2.10} The maximum edge multiplicity of a graph
with finitely many vertices and edges is the positive integer $m$
such that every pair of vertices is connected by at most $m$
vertices. \enddefinition

We shall deduce Theorem 2.2 from the following graph theoretic
estimate. \proclaim{Lemma 2.4} Let $G$ be a connected graph with
$n$ vertices and $e$ edges. Suppose that $e \ge 4n$. Suppose that
the maximum edge multiplicity of $G$ is $M$. Then
$$ cr(G) \ge C\frac{e^3}{Mn^2}. \tag2.3$$ \endproclaim

We shall prove Lemma 2.4 in the case $M=1$. The reader is then
asked to recover the general case in Exercise 2.3 below.

We need the following fact whose proof is outlined in the Exercise
2.4. Let $H$ be a planar graph with $n$ vertices and $e$ edges.
Then
$$ e \leq 3n-6. \tag2.4$$

It follows that if $G$ is any connected graph, then
$$ cr(G) \ge e-3n. \tag2.5$$


If the last claim is not transparent, play the following game with
an "oracle". You give the oracle a connected graph and ask him if
it is planar. The oracle responds by asking you if $(2.4)$ holds
since it is after all a criterion for planarity. If the answer is
no, there must be a crossing which can only be removed by removing
an edge. This game must be played until the graph is planar, at
which point at least $e-3n$ edges have been removed.

Now let $H$ be a random graph constructed by choosing each vertex
of $G$ (here $G$ is from the statement of Lemma 2.4) with
probability $p$. Keep an edge if both vertices are kept. It
follows that the expected number of vertices is $np$, the expected
number of edges is $ep^2$, and the expected value of the crossing
number of $H$ is at most $cr(G)p^4$. Since expectation is linear,
it follows from $(2.5)$ that
$$ cr(G) \ge \frac{e}{p^2}-\frac{3n}{p^3}. \tag2.6$$

Lemma 2.4 follows by choosing $p=\frac{e}{4n}$, as we may since $e
\ge 4n$ by assumption. For those not familiar with expectations,
just think of the expected value of the number of vertices in $H$,
for instance, as the average number of vertices in $H$ over all
possible outcomes of the coin flip with probability $p$. More
precisely, for each possible subgraph of $G$, count the number of
vertices and multiply the resulting number by the probability that
this given subgraph arises if the vertices are kept with
probability $p$. The sum of all these numbers is the expected
value of the number of vertices in $H$. The expected number of
edges and the expected value of the crossing number is computed in
the same way.

We are now ready to prove Theorem 2.2, which we do in the case
$\alpha=\beta=1$. The reader is asked to carry out the details of
the proof of the general case in Exercise 2.3 below. Let the
elements of $S$ be the vertices of $G$. Connect two vertices by
and edge if the corresponding points are consecutive on some curve
in $L$. Then $e=I-m$ (think about it...). If $e<4n$, then $I<m+4n$
and we are done. If $e \ge 4n$, then Lemma 2.4 kicks in and we
have $cr(G) \ge C\frac{e^3}{n^2} \ge C' \frac{{(I-m)}^3}{n^2}$. On
the other hand, an incidence cannot in our graph unless two lines
from our collection intersect. It follows that $cr(G) \leq m^2$.
Combining the two estimates yields Theorem 2.2 in the case
$\alpha=\beta=1$.

\head Exercises \endhead

\vskip.125in

\proclaim{Exercise 2.1} Prove that Theorem 2.2 is sharp. In the
case $n=m$, suppose that $n=4k^3$ for a positive integer $k$. Let
$S=\{(i,j): 0 \leq i \leq k-1; 0 \leq j \leq 4k^2-1\}$. Let $L$ be
the set of lines with equation $y=ax+b$, $a=0,1, \dots, 2k-1$,
$b=0, 1, \dots, 2k^2-1$. Check that there is $n$ points and $n$
lines, and the number of incidences is $\approx n^{\frac{4}{3}}$.
Generalize this example to arbitrary $n$ and $m$. \endproclaim

\proclaim{Exercise 2.2} Carry out the details of the induction on
dimension in the proof of Theorem 2.3. You should end up with the
recursion $\frac{1}{\alpha_d}-\frac{1}{\alpha_{d-1}}=1$, where the
number of distinct distances in $d$ dimensions determined by $N$
points is at least $N^{\alpha_d}$. \endproclaim

\proclaim{Exercise 2.3} Carry out the details of proof of Lemma
2.4 for the general multiplicty $m$. Use this result to deduce the
general version of Theorem 2.2 \endproclaim

\proclaim{Exercise 2.4} Construct a convex curve $\Gamma$, of
diameter $\approx N^3$, symmetric with respect to the origin, with
everywhere non-vanishing curvature, such that $\# \Gamma \cap
{\Bbb Z}^2 \approx N^2$. Let $K$ denote the interior of $\Gamma$.
Show that there exists a set $S$ of cardinality $n$ such that $\#
\{(x,y) \in S \times S: {||x-y||}_K=1\} \approx n^{\frac{4}{3}}$.
Conclude that the Single Distance Conjecture is in general false
for convex sets $K$ whose boundary has everywhere non-vanishing
Gaussian curvature. \endproclaim

\proclaim{Exercise 2.5} Prove $(2.4)$, i.e that if $H$ is a
connected planar graph with $n$ vertices and $e$ edges, then $e
\leq 3n-6$. First prove by induction that $n-e+f=2$, where $f$ is
the number of faces. Here a face is defined as a connected
component of a drawing of $H$, viewed as a subset of the
two-dimensional sphere. Then prove that $3f \leq 2e$. Conclude
that $e \leq 3n-6$, as advertised. \endproclaim

\proclaim{Exercise 2.6} Let $K$ be a convex domain in ${\Bbb
R}^d$, $d \ge 2$. We say that $K$ is spectral if $L^2(K)$ has an
orthogonal basis of the form ${\{e(x \cdot a)\}}_{a \in A}$, with
$\int_K e(x \cdot (a-a')) dx=0$ for $a \not=a'$, where $A$ is a
subset of ${\Bbb R}^d$ and $e(t)=e^{2 \pi it}$.

Prove that $B_d$ is not spectral by following the following
outline. Let $A$ be the putative spectrum of $B_d$. First prove
that $A$ is separated, i.e there exists $c>0$ such that $|a-a'|
\ge c$ for all $a \not=a'$. Then prove that $A$ is
well-distributed in the sense that there exists $r_0>0$ such that
every cube of side-length $r_0$ contains at least one point from
$A$. Now consider a ball of radius $R$ in ${\Bbb R}^d$. By
well-distributivity this ball contains $\approx R^d$ points of
$A$. Denote this subset of $A$ by $A_R$. On the other hand
$\int_{B_d} e(x \cdot (a-a')) dx=C_d {|a-a'|}^{-\frac{d-2}{2}}
J_{\frac{d}{2}}(2 \pi |a-a'|)$, where $J_{\frac{d}{2}}$ denotes
the Bessel function of order $\frac{d}{2}$. Use properties of
Bessel functions and this formula to see that there exists an
absolute constant $C>0$ such that $\# \Delta_{B_d}(A_R) \leq CR$.
Use Theorem 2.3 to derive a contradiction. Can you derive a
contradiction by using Theorem 1.2 instead? Calculate all the
constants carefully before you answer.
\endproclaim

\proclaim{Exercise 2.7} Let $P$ be a convex polygon in the plane
whose vertices have integer coordinates. Use the Szemeredi-Trotter
incidence theorem to prove that the number of vertices of $P$ is
bounded from below by a constant times the cube root of the area
of $P$. \endproclaim

\vskip.125in

\head Chapter III: In the realm of the Euclidean metric? \endhead

\vskip.125in

Let us briefly summarize what happened in the previous two
chapters. We first proved that for any distance induced by a
bounded symmetric convex set $K$, $g_K(n,d) \ge Cn^{\frac{1}{d}}$
and one cannot in general do better. However, in the following
chapter we saw that if the boundary of $K$ is sufficiently well
"curved" in the sense of intersection properties, then the
estimate for $g_K(n,d)$ improves considerably.

In this chapter we shall obtain better estimates on $g_K(n,d)$
under the assumption that $K=B_d$. At the end of the chapter we
shall take a small step towards extending this result to other
metrics by introducing some basic technology from algebraic
geometry.





















\enddocument
