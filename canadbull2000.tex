\documentstyle{amsppt}
\tolerance 3000
\pagewidth{5.5in}
\vsize7.0in
\magnification=\magstep1
\widestnumber \key{AAAAAAA}
\topmatter
\author Alex Iosevich and Guozhen Lu    
\endauthor
\title Sharpness results and Knapp's homogeneity argument*  
\endtitle
\endtopmatter 

\head Introduction \endhead 
\vskip.125in 

Let $S$ be a smooth compact hypersurface in ${\Bbb R}^n$. Let 
$$ F_S(\xi)=\int_S e^{i\langle x,\xi \rangle} d\sigma(x) \tag1$$ denote the
Fourier transform of the surface measure carried by $S$. 

Let ${\Cal R}f=\hat{f}{|}_{S}$, the restriction operator. It is well known
(see \cite{T}, \cite{G}, \cite{S}) that if 
$$|F_S(\xi)| \leq C{(1+|\xi|)}^{-r}, \ \ r>0, \tag2$$ then 
$$ {||{\Cal R}f||}_2 \leq C_p {||f||}_p, \ \ f\in {\Cal S}({\Bbb R}^n), 
\ \ \text{for} \ \ p\leq p_0=\frac{2(r+1)}{r+2}, \tag3$$ where 
${\Cal S}({\Bbb R}^n)$ is the standard Schwartz class. 

However, it is not in general known whether this result is sharp. More 
precisely, it is natural to ask the following.  
\proclaim{Question A} Does the estimate $(3)$ imply the estimate $(2)$?  
\endproclaim 
\vskip.125in

Let   
$$Tf(x,x_n)=\int f(x-y,x_n-\Phi(y)) \psi(y)dy, \tag4$$ where 
$x,y \in {\Bbb R}^{n-1}$, $\psi$ is a smooth cutoff function, $\Phi$ is 
smooth, $\Phi(0,...,0)=0$, and $\bigtriangledown \Phi(0,...,0)=(0,...,0).$

It is well known (see \cite{Str}) that if the estimate $(2)$ holds, then
$$ {||Tf||}_{p^{\prime}} \leq C_p{||f||}_p, \ \ \text{where} \ \ 
\frac{1}{p}-\frac{1}{2}\leq \frac{r}{2(r+1)}, \tag5$$ where  
$p^{\prime}$ denotes the conjugate exponent of $p$. 

The key estimate here is 
$$ {||Tf||}_{2(r+1)} \leq C{||f||}_{\frac{2(r+1)}{2r+1}}; \tag6$$ the
rest follows by interpolation. It is then natural to ask the following. 
\proclaim{Question B} Does the estimate $(5)$ imply the estimate $(2)$? 
\endproclaim  

The purpose of this paper is to answer questions $(A)$ and $B$ affirmitively 
in the case of the optimal exponents. 
We shall employ a multiparameter version of Knapp's homogeneity argument. 
(See e.g \cite{C} for a similar argument). 

More precisely, we will show that if the  
the estimate $(3)$ holds with $p=\frac{2(n+1)}{n+3}$, then the hypersurface
has everywhere non-vanishing Gaussian curvature. Similarly, we will show that
if the estimate $(5)$ holds with $p=\frac{n+1}{n}$, then the hypersurface 
has non-vanishing Gaussian curvature. 

We remark here, on the other hand,  that  non-vanishing Gaussian curvature implies that the estimate
$(2)$ holds with $r=\frac{n-1}{2}$ (see e.g. \cite{S}). Thus Question (A) 
is answered  affirmatively in the case $r=\frac{n-1}{2}$. Since the estimate 
$(2)$ with $r=\frac{n-1}{2}$ implies the estimate $(3)$ with 
$p=\frac{2(n+1)}{n+3}$, Theorem 2 below shows that the optimal decay of 
the Fourier transform (i.e. $r=\frac{n-1}{2}$) implies that the hypersurface
has non-vanishing Gaussian curvature. 




We will also see that if a hypersurface has $\leq k$ non-vanishing 
principal curvatures at each point, then the exponent $p$ in the estimate
$(3)$ can never exceed $\frac{2n+k-2}{6}$. Consequently, the estimate 
$(3)$ with $p\ge \frac{2n+k-2}{6}$ implies that at least $k$ principal 
curvatures are non-zero at each point. (See Theorem 3 below). Similarly, we  
will show that if the estimate $(5)$ holds with 
$p\ge \frac{2n+k+4}{2n+k+1}$, then at least $k$ principal curvatures are
non-zero at each point.  


The sharpness of the estimate $(3)$ is known in some cases. For example, if
the hypersurface has non-vanishing Gaussian curvature, Knapp's homogeneity 
argument can be used to show that the exponent $p=\frac{2(n+1)}{n+3}$ is the 
best possible. Indeed, non-vanishing Gaussian curvature 
implies that the hypersurface has contact of order two with its 
tangent plane at every point. Let 
$f_{\delta}(x)=g({\delta}^{-1}x, {\delta}^{-2}x_n)$, where 
$x=(x_1,...,x_{n-1})$, and $g$ is the characteristic function 
of the rectangle with sides $(1,...,1,C)$, $C$ large, with the long side
normal to the hypersurface. 

It is not hard to check that 
${||f_{\delta}||}_p \approx {\delta}^{(1-\frac{1}{p})(n+1)}$, whereas 
${||{\Cal R}f_{\delta}||}_2 \approx {\delta}^{\frac{n-1}{2}}$. The    
comparison yields $p\leq \frac{2(n+1)}{n+3}$.  

It should be noted that the above example does not verify even a special  
case of question $(A)$. For example, the above argument does not prove that 
if the estimate $(3)$ holds with $p=\frac{2(n+1)}{n+3}$, then the estimate  
$(2)$ holds with $r=\frac{n-1}{2}$. We will show (see Theorem 2 below) that 
this is indeed the case. 

The sharpness of the estimate $(5)$ can also be verified in some cases. 
By testing $T$ against a characteristic function of a small ball it is not
hard to check that if $T$ is bounded from $L^p({\Bbb R}^n)$ to 
$L^q({\Bbb R}^n)$, then $(\frac{1}{p}, \frac{1}{q})$ must be contained in
the triangle with the endpoints $(0,0)$, $(1,1)$, and 
$(\frac{n}{n+1}, \frac{1}{n+1})$. However, as before this does not prove that
if the estimate $(5)$ holds with $p=\frac{n+1}{n}$, then the estimate 
$(2)$ holds with $r=\frac{n-1}{2}$. We will show (see Theorem 5 below) that
this is indeed the case. 

\head Statement of results \endhead 
\vskip.125in

\proclaim{Theorem 1} Let $S=\{(x,x_n) \in {\Bbb R}^n: x_n=\Phi(x)\}$, 
where $x=(x_1,...,x_{n-1})$, 
$\Phi$ is a smooth function which does not vanish on a set of 
positive measure, $\Phi(0,...,0)=0$, and 
$\bigtriangledown \Phi(0,...,0)=(0,...,0)$.  Suppose that the estimate $(3)$
holds. Let $G$ be any continuous function which does not vanish on a set of 
positive measure satisfying $G(0,...,0)=0$. Then  
$${(|G(\delta)|)}^r \ge CR(\delta)^{r+1} 
|{\delta}_1{\delta}_2...{\delta}_{n-1}|, \tag7$$ where 
$R(\delta)=|\{x \in {[-1,1]}^{n-1}: 
|\Phi({\delta}_1x_1,...,{\delta}_{n-1}x_{n-1})| \leq C|G(\delta)| \}|$. 
\endproclaim   

\remark{Remark} If $G(\delta)$ is chosen to be $\Phi(\delta)$, and $\Phi$ 
is increasing in each variable separately, Theorem 1 says that the estimate
$(3)$ implies that 
${(|\Phi(\delta)|)}^r \ge C{\delta}_1{\delta}_2...{\delta}_{n-1}.$ The same
estimate would be true, of course, if we just assume that 
$R(\delta)$ is bounded below, which is a much weaker assumption. To prove 
Theorem 2, Theorem 3, Theorem 5, and Theorem 6  below we shall use Theorem 1  
with $$G(\delta)=\sup_{\{x \in {[-1,1]}^{n-1}\}}   
|\Phi(x_1{\delta}_1,...,x_{n-1}{\delta}_{n-1})|.\tag8$$  
\endremark 

\proclaim{Theorem 2} Suppose that  the estimate $(3)$ holds with 
$p=\frac{2(n+1)}{n+3}$. Then the hypersurface $S$ has everywhere non-vanishing
Gaussian curvature. 
\endproclaim  

\proclaim{Theorem 3} Suppose that the estimate $(3)$ holds with 
$p\ge \frac{2n+k-2}{6}$. Then the hypersurface $S$ has at least $k$ 
non-vanishing principal curvatures at each point. 
\endproclaim 

\remark{Remark} The conclusion of Theorem 3 can be motivated as follows. If 
the hypersurface has exactly $k$ non-vanishing principal curvatures at a
point, then after perhaps applying a rotation we can write it as a 
graph of the function $x_1^2+...+x_k^2+A(x)$, where $A$ is a higher order
remainder. It is not hard to believe that the best possible estimate 
$(2)$ is obtained if $A(x)={|x^{\prime \prime}|}^3$, where 
$x^{\prime \prime}=(x_{k+1},...,x_{n-1})$. This gives us the estimate
$(2)$ with $r=\frac{k}{2}+\frac{n-1-k}{3}$. The conclusion of Theorem 3
is the consequence of the fact that $\frac{2(r+1)}{r+2}=
\frac{2n+k-2}{6}$. 

\proclaim{Theorem 4} Let  
$\delta y=({\delta}_1y_1,...,{\delta}_{n-1}y_{n-1})$ and 
$g_{\delta}(s)=|\{y \in supp(\psi): 
|s-|\Phi(\delta y)/ \Phi(\delta)||\leq C \}|$.  
Suppose that the estimate $(5)$ holds. Then for $|\delta|$ sufficiently small,
$$ {(|\Phi(\delta)|)}^r \ge CP_{\delta}{||g_{\delta}||}_{L^{p^{\prime}}(ds)}. 
\tag9$$ 
\endproclaim  

\proclaim{Theorem 5} Suppose that the estimate $(5)$ holds with 
$r=\frac{n-1}{2}$. Let $S=\{(x,x_n): x\in supp(\psi), x_n=\Phi(x)\}$. Then 
$S$ has everywhere non-vanishing Gaussian curvature. 
\endproclaim 

\proclaim{Theorem 6} Suppose that the estimate $(5)$ holds with 
$p\ge \frac{2n+k+4}{2n+k+1}$. Then the hypersurface has at least $k$ 
non-vanishing principal curvatures at each point. 
\endproclaim 

(See the remark after Theorem 3 for the motivation of the conclusion of
Theorem 6). 

\head Proof of Theorem 1 \endhead 
\vskip.125in

Let ${\delta}x=({\delta}_1x_1,...,{\delta}_{n-1}x_{n-1})$, and 
${\delta}^{-1}x=({\delta}^{-1}_1x_1,...,{\delta}^{-1}_{n-1} x_{n-1})$. Let 
$\hat{f}_{\delta}(x,x_n)=g({\delta}^{-1}x, \frac{x_n}{|G(\delta)|})$,
where $g$ is the characteristic function of a rectangle with sides of length
$(1,1,...,1,C)$. Let $P_{\delta}=|{\delta}_1{\delta}_2...{\delta}_{n-1}|$. 
It is not hard to see that 
$$ {||f_{\delta}||}_p \approx {(P_{\delta}|G(\delta)|)}^{(1-1/p)}. \tag10$$ 

On the other hand, 
$$ {||{\Cal R}f_{\delta}||}^2_2=\int 
{ \left|g\left({\delta}^{-1}x, 
\frac{\Phi(x)}{|G(\delta)|}\right)\right|}^2 dx=$$
$$ P_{\delta} \int 
{ \left|g\left(x,\frac{\Phi(\delta x)}{|G(\delta)|}\right) \right|}^2 dx
\approx CP_{\delta}R(\delta), \tag11$$ where 
$R(\delta)$ is defined in the statement of the theorem. 

Comparing the estimates $(10)$ and $(11)$ we see that $(3)$ can hold only if
$$ {(|G(\delta)|)}^r \ge CP_{\delta}R^{r+1}(\delta), \tag12$$ for 
$|\delta|$ sufficiently small. This completes the proof of Theorem 1.  

\head Proof of Theorem 2 \endhead 
\vskip.125in 

Let $G(\delta)=\sup_{ \{x \in {[-1,1]}^{n-1} \}} 
|\Phi({\delta}_1 x_1,...,{\delta}_{n-1} x_{n-1})|.$ It follows that 
$R(\delta)\equiv 1$, and so 
$$ {(G(\delta))}^r \ge CP_{\delta}, \tag13$$ where 
$r=\frac{n-1}{2}$ by assumption. 

After perhaps applying a rotation, we can use Taylor's theorem to write  
$$ \Phi(x)=a_1x_1^2+a_2x_2^2+...+a_kx_k^2+A(x), \tag14$$ where $A(x)$ is a
higher order remainder term, and $k \leq n-1$. If $k=n-1$, then in a 
sufficiently small neighborhood of the origin the determinant of the 
Hessian matrix of $\Phi$ never vanishes, which would verify the claim of 
Theorem 2. We shall henceforth assume that $k<n-1$. 

It is not hard to check that 
$$ {(G(\delta))}^{\frac{n-1}{2}} \leq 
{(a_1{\delta}_1^2+...+a_k{\delta}_k^2+C{|\delta|}^3)}^{\frac{n-1}{2}}, \tag15$$
$|\delta|$ small. 

We must show that the estimate $(13)$ cannot hold if $k<n-1$. It suffices to
show that the right hand side of $(15)$   
is not bounded below by $CP_{\delta}$. We may assume that $A(x)$ is not 
identically $0$, and that $A(x)$ depends on $x_{n-1}$, for otherwise the
contradiction is immediate. Let  
${\delta}_j={\delta}_{n-1}^{\frac{3}{2}}$. If the right hand side were 
bounded below by $CP_{\delta}$, we could use the fact that $A(x)$ is a higher
order remainder term to force an inequality 
$$ {|{\delta}_{n-1}|}^{\frac{3(n-1)}{2}} \ge 
C{|{\delta}_{n-1}|}^{\frac{3n-4}{2}}, 
\tag16$$ ${\delta}_{n-1}$ small, which is not true. This shows that the 
estimate $(8)$ cannot hold unless $k=n-1$. This implies that there exists 
a small neighborhood of the origin where $S$ has non-vanishing Gaussian 
curvature. This completes the proof. 
\vskip.125in 

\head Proof of Theorem 3 \endhead 
\vskip.125in 

We must show that if $\Phi$ is as in the estimate $(14)$ above, with 
$k$ denoting the number of non-vanishing principal curvatures, then the 
estimate 
$$ {(G(\delta))}^r \ge CP_{\delta} \tag17$$ can only hold if 
$r \leq \frac{k}{2}+\frac{n-1-k}{3}=\frac{2n+k-2}{6}$. 

Let ${\delta}=({\delta}^{\prime}, {\delta}^{\prime \prime})$, where
${\delta}^{\prime}=({\delta}_1,...,{\delta}_k)$, and 
${\delta}^{\prime \prime}=({\delta}_{k+1},...,{\delta}_{n-1})$. 

Let ${\delta}_j={|{\delta}^{\prime \prime}|}^{\frac{3}{2}}$.
The estimate $(16)$ cannot hold if the inequality 
$$ {|{\delta}^{\prime \prime}|}^{3r} \ge C  
{|{\delta}^{\prime \prime}|}^{(\frac{3k}{2}+(n-1-k))} \tag18$$ is not 
satisfied. However, the estimate $(17)$ can only hold if 
$r \leq \frac{k}{2}+\frac{n-1-k}{3}=\frac{2n+k-2}{6}$. This completes the
proof. 
\vskip.125in 

\head Proof of Theorem 4 \endhead 
\vskip.125in  

Let ${\delta}^{-1}y=({\delta}^{-1}_1y_1,...,{\delta}^{-1}_{n-1}y_{n-1})$. Let
$f$ denote the characteristic function of the rectangle with sides of 
length $(1,1,...,1,C)$, $C$ large. Let ${\tau}_{\delta}f(x,x_n)=
f({\delta}x, |\Phi(\delta)|x_n)$, and ${\tau}^{-1}_{\delta}f(x,x_n)=
f({\delta}^{-1}x, {|\Phi(\delta)|}^{-1}x_n)$. Let $f_{\delta}(x,x_n)=
{\tau}^{-1}_{\delta}f(x,x_n)$. Let  
$$ T_{\delta}f(x,x_n)=\int f(x-y,x_n-\Phi(y)) \psi({\delta}^{-1}y)dy. 
\tag19$$ 

After making a change of variables we see that 
$$ T_{\delta}f_{\delta}(x,x_n)=P_{\delta}{\tau}^{-1}_{\delta} 
T^{*}_{\delta}f(x,x_n), \tag20$$ where 
$$ T^{*}_{\delta}f(x,x_n)=\int f(x-y,x_n-\Phi(\delta y)/ |\Phi(\delta)|) 
\psi(y)dy. \tag21$$ 

It is not hard to see that 
$$ {||f_{\delta}||}_p \approx P^{\frac{1}{p}}_{\delta} 
{(|\Phi(\delta)|)}^{\frac{1}{p}}. \tag22$$ 

Also, 
$$ {||T_{\delta}f_{\delta}||}_{p^{\prime}}={||P_{\delta}{\tau}^{-1}_{\delta}
T^{*}_{\delta}f||}_{p^{\prime}}=
P_{\delta} P^{\frac{1}{p^{\prime}}}_{\delta} 
{(|\Phi(\delta)|)}^{\frac{1}{p^{\prime}}} 
{||T^{*}_{\delta}f||}_{p^{\prime}}\approx$$ 
$$ P_{\delta}P^{\frac{1}{p^{\prime}}}_{\delta} 
{(|\Phi(\delta)|)}^{\frac{1}{p^{\prime}}} 
{||g_{\delta}||}_{L^{p^{\prime}}(ds)}, \tag23$$
where $g_{\delta}$ is defined above. 

Comparing the estimates $(22)$ and $(23)$ yields the assertion of the theorem.
\vskip.125in 

\head Proof of Theorem 5 and Theorem 6 \endhead 

Let $G(\delta)=\sup_{x \in {[-1,1]}^{n-1}} |\Phi(\delta x)|$. The proof of
Theorem 4 shows that if the estimate $(5)$ holds then 
$$ {(G(\delta))}^r \ge CP_{\delta}. \tag24$$ 

The proof of Theorem 5 and Theorem 6 now follows in the same way as the 
proofs of Theorem 2 and Theorem 3. 
\vskip.125in 

\remark{Acknowledgments} This note was first written and circulated in February 
of 1996. We wish to thank Michael Christ for pointing out that the 
multiparameter version of Knapp's homogeneity argument was used in his thesis, as
well as other helpful suggestions. We also wish to  thank Tom Wolff, 
 Allan Greenleaf, Chris Sogge, Eric Sawyer and Jim Vance
for many helpful discussions and conversations.  
\endremark
\vskip.25in 

\newpage 

\head References \endhead
\vskip.125in 

\ref \key C \by M. Christ 
\paper Restriction of the Fourier transform
to submanifolds of low codimension \jour Ph.D thesis; U. of Chicago  
\yr 1982 \endref 

\ref \key G \by A. Greenleaf 
\paper Principal curvature in harmonic analysis 
\jour Indiana U. Math. J. \vol 30 \pages 519-537 \yr 1981 
\endref 

\ref \key I \by A. Iosevich \paper Sharpness of $(L^p,L^q)$ bounds 
of averaging operators (preprint) \yr 1994 \endref 

\ref \key IS \by A. Iosevich and E. Sawyer 
\paper Oscillatory integrals and maximal averages over homogeneous surfaces
\jour Duke Math. J. \yr 1996 \endref 

\ref \key S \by E. M. Stein \paper Harmonic Analysis 
\jour Princeton Univ. Press \yr 1993 \endref 

\ref \key Str \by R. Strichartz 
\paper Convolutions with kernels having singularities on the sphere 
\jour Trans. Amer. Math. Soc. \vol 148 \pages 461-471 \endref 

\ref \key T \by P. Tomas \paper A restriction theorem for the Fourier 
transform \jour Bull. Amer. Math. Soc. \vol 81 \pages 477-478 \yr 1975 
\endref 
\vskip.25in

----------------------------------- \newline
Department of Mathematics \newline
Wright State University \newline 
Dayton, Ohio 45435 \newline 
iosevich \@zara.math.wright.edu \newline 
gzlu \@euler.math.wright.edu 

\vskip.125in 

(*) Both authors were supported in part by NSF grants. 




\bye
\enddocument

