\documentstyle{amsppt}
\tolerance 3000
\pagewidth{5.5in}
\vsize7.0in
\magnification=\magstep1
\widestnumber
\key{AAAAAAAAAAAAAAA}
\NoRunningHeads
\topmatter
\title Lattice points inside random ellipses
\endtitle
\author Steve Hofmann and Alex Iosevich
\endauthor
\thanks Research supported in part by NSF grants
\endthanks
\abstract Let $N_a(t)=\# \{t \Omega_a \cap {\Bbb Z}^2\}$, where
$\Omega_a= \left\{x \in {\Bbb R}^2: \frac{x_1^2}{a_1}+
\frac{x_2^2}{a_2} \leq 1 \right\}$. We prove that ${\left(
\int_{\frac{1}{2}}^{2} \int_{\frac{1}{2}}^2 {|E_a(t)|}^2 da_1da_2
\right)}^{\frac{1}{2}}
\lessapprox t^{\frac{1}{2}}$, improving a corresponding $L^1$
estimate recently obtained by Toth and Petridis. The proof still works if
$\Omega_a$ is replaced by a suitable dilation of any convex domain with a
smooth boundary and everywhere non-vanishing curvature.
\endabstract
\endtopmatter



\head Introduction \endhead
\vskip.125in



Let
$$ N_a(t)=\# \{t \Omega_a \cap {\Bbb Z}^2\}, \tag0.1$$ where
$$ \Omega_a= \left\{x \in {\Bbb R}^2: \frac{x_1^2}{a_1}+
\frac{x_2^2}{a_2} \leq 1 \right\}, \tag0.2$$ with $\frac{1}{2}
\leq a_j \leq 2$.

Let
$$ N_a(t)=t^2 |\Omega_a|+E_a(t). \tag0.3$$

It is well-known that
$$ |E_a(t)| \lesssim t^{\frac{2}{3}}, \tag0.4$$
where here and throughout the paper, $A \lesssim B$ means that
there exists a positive constant $C$ such that $A \leq CB$.
Similarly, $A \lessapprox B$, with a parameter $t$, means that
given $\delta>0$ there exists $C_{\delta}>0$ such that $A \leq
C_{\delta}t^{\delta}B$.

A number of improvements over $(0.4)$ have been obtained over
the years culminating in Huxley's $|E_a(t)| \lessapprox
t^{\frac{46}{73}}$ estimate a few years ago. See \cite{Huxley96}
and the references contained therein. However, the conjectured
result, $|E_a(t)| \lessapprox t^{\frac{1}{2}}$ is nowhere near
resolution. In fact, it is not known if there exists a single
$a=(a_1, a_2)$ such that $|E_a(t)| \lessapprox t^{\frac{1}{2}}$.
The question of finding such an $a$ was posed by Sarnak a number
of years ago.

Toth and Petridis (\cite{TP02}) recently proved that
$$ \int_{\frac{1}{2}}^{2} \int_{\frac{1}{2}}^{2} |E_a(t)| da \lessapprox
t^{\frac{1}{2}}. \tag0.5$$

In this paper we prove that
$$ {\left( \int_{\frac{1}{2}}^{2} \int_{\frac{1}{2}}^{2} {|E_a(t)|}^2 da
\right)}^{\frac{1}{2}} \lessapprox t^{\frac{1}{2}}. \tag0.6$$

Our main result is the following. \proclaim{Theorem 0.1} Let
$N_a(t)$, $E_a(t)$, and $\Omega_a$ be  as above. Then $(0.6)$
holds. \endproclaim

\remark{Remark} The proof of Theorem 0.1 goes  without any siginificant
changes if tne ellipse $\Omega_a$ is replaced by
$\{(a_1^{-\frac{1}{2}}x_1, a_2^{-\frac{1}{2}}x_2): x \in \Omega\}$, where
$\Omega$ is a convex planar set with a smooth boundary and curvature
bounded from below. \endremark

Obseve that Sarnak's question would be asnwered by the following
strengthening of $(0.6)$.
\proclaim{Conjecture 0.2}  Given any $\delta>0$,
$$ \sup_{t \ge 1} t^{-\frac{1}{2}-\delta}|E_{(\cdot)}(t)| \in L^p
\left(\left[\frac{1}{2},2 \right] \times \left[\frac{1}{2},2 \right]
\right), \tag0.7$$ for some $p \ge 1$ with a constant depending on
$\delta$.
\endproclaim

In fact, $(0.7)$ would, of course, imply that the estimate $|E_a(t)|
\lessapprox t^{\frac{1}{2}}$ holds for almost every $a \in
\left(\left[\frac{1}{2},2 \right] \times \left[\frac{1}{2},2 \right]
\right)$. We hope to address this issue in a subsequent paper.

Other types of square averages of lattice point discrepancy functions
have been studied in the past and in recent years. For example, a
classical result due to Kendall says that
$$ \int_{{\Bbb T}^2} {|\# \{(t \Omega +\tau) \cap {\Bbb
Z}^2\}-t^2|\Omega||}^2 d\tau \lesssim t^{\frac{1}{2}}, \tag0.8$$
for every convex planar domain whose boundary has everywhere
non-vanishing Gaussian curvature.

Another type of average is studied in \cite{ISS02}. The authors prove
that
$$ {\left( \frac{1}{h} \int_R^{R+h} {|\# \{t \Omega \cap {\Bbb
Z}^2\}-t^2|\Omega||}^2 dt \right)}^{\frac{1}{2}} \lesssim R^{\frac{1}{2}}
\tag0.9$$ with $h \gtrsim \log(R)$ provided that $\Omega$ is convex and
has a smooth boundary with everywhere non-vanishing Gaussian curvature.
Similar results are known in higher dimensions, but we do not address
this issue here. See, for example, \cite{Huxley96} and \cite{ISS02} and
references contained therein.

The proof of $(0.8)$ uses only the decay of the Fourier transform of the
characteristic function of $\Omega$. The proof of $(0.9)$ uses not only
the decay, but also the asymptotic formula for $\widehat{\chi}_{\Omega}$
and the structure of $\Omega^{*}$, the convex body dual to $\Omega$. The
proof of $(0.6)$ given in the proof of Theorem 0.1 requires, in addition,
a precise analysis of an oscillatory integral which arises from the
asymptotic formula for $\widehat{\chi}_{\Omega}$ and integration with
respect to the accentricities.

\vskip.125in

\head Proof of Theorem 0.1 \endhead

\vskip.125in

We
start with the following standard reduction. Let $\rho \in C^{\infty}_0
\left( \frac{1}{4}, 4\right)$ such that $\rho \equiv 1$ on $[1,2]$ and
$\int \rho(x) dx=1$. Let

$\rho_{\epsilon}(x)=\epsilon^{-2} \rho \left( \frac{x}{\epsilon} \right)$.
Let

$$ N_a^{\epsilon}(t)=\sum_{k \in {\Bbb Z}^2} \chi_{t
\Omega_a}*\rho_{\epsilon}(k)=t^2|\Omega_a|+t^2 \sum_{k \not=(0,0)}
\widehat{\chi}_{\Omega_a}(tk) \widehat{\rho}(\epsilon k)=t^2|\Omega_a|+
E_a^{\epsilon}(t). \tag1.1$$



It is not hard to see that there exists $C>0$ such that
$$ N_a^{\epsilon}(t-C\epsilon) \leq N_a(t) \leq
N_a^{\epsilon}(t+C\epsilon). \tag1.2$$



It follows that
$$ |E_a(t)| \lesssim |E_a^{\epsilon}(t)|+t \epsilon. \tag1.3$$



We conclude that it suffices to establish estimates for
$E_a^{\epsilon}(t)$ with
$\epsilon=t^{-\frac{1}{2}}$.



Using  the standard asymptotic formula for the Fourier transform
of the characteristic function of a bounded smooth convex domain
where the Gaussian curvature of the boundary is non-vanishing,
(see e.g \cite{Hertz60}), we see that
$\widehat{\chi}_{\Omega_a}(tk)$ is a sum of two terms of the
form
$$ e^{2 \pi i t {|k|}_a} \ t^{-\frac{3}{2}} {|k|}_a^{-\frac{3}{2}}+
O({(t|k|)}^{-\frac{5}{2}}), \tag1.4$$ where
$$ {|k|}_a=\sqrt{a_1 k_1^2+a_2 k_2^2}. \tag1.5$$



It follows that
$$ E^{\epsilon}_a(t)=t^{\frac{1}{2}} \sum_{k \not=(0,0)} e^{2 \pi i
t{|k|}_a} {|k|}_a^{-\frac{3}{2}}\widehat{\rho}(\epsilon
k)\widehat{\rho}(\epsilon l)+t^2
\sum_{k \not=(0,0)} {(t{|k|}_a)}^{-\frac{5}{2}}=I+II. \tag1.6$$



Since we can easily handle $II$ point-wise, we turn our
attention to $I$.  Squaring, integrating in $a$, and replacing
the limits of integration in $a$ by a smooth cutoff function, we
get
$$ t \sum_{k,l \not=(0,0)} {|k|}^{-\frac{3}{2}} {|l|}^{-\frac{3}{2}}
\widehat{\rho}(\epsilon k) \widehat{\rho}(\epsilon l) \int e^{2
\pi i t({|k|}_a-{|l|}_a)} \psi_{k,l}(a) da, \tag1.7$$ where
$$\psi_{k,l}(a)={\left( \frac{|k|}{{|k|}_a}\right)}^{\frac{3}{2}}
{\left( \frac{|l|}{{|l|}_a}\right)}^{\frac{3}{2}} \psi(a).
\tag1.8$$



Observe that when $k \not=(0,0)$ and $l \not=(0,0)$, $\psi_{k,l}\in
C_0^{\infty}$ with constants uniform in $k$ and $l$. It suffices to show
that this expression is bounded above
$C_{\delta} t^{\delta}$ for any $\delta>0$. Consider
$$ I_{k,l}(t)=\int e^{2 \pi i t({|k|}_a-{|l|}_a)} \psi_{k,l}(a) da.
\tag1.9$$



Let $\Phi_{k,l}(a)={|k|}_a-{|l|}_a$. A calculation shows that
$$ \det H\Phi_{k,l}(a)=-\frac{1}{16}
\frac{{(k_1^2l_2^2-l_1^2k_2^2)}^2}{{|k|}_a^3 {|l|}_a^3}.
\tag1.10$$



\head Non-zero determinant \endhead



\vskip.125in



We first consider the case $|\frac{k_1}{k_2}|
\not=|\frac{l_1}{l_2}|$, $|k_j|, |l_j| \ge 1$. Let $\delta$ be a
small positive number. Since
$$|\widehat{\rho}(\epsilon k)| \leq C_N {(1+|\epsilon k|)}^{-N},
\tag2.1$$ and $|I_{k,l}(t)| \lesssim 1$, it follows that
$$ t \sum_{|k_j|,|l_j| \ge
\epsilon^{-1-\delta}}{|k|}^{-\frac{3}{2}} {|l|}^{-\frac{3}{2}}
\widehat{\rho}(\epsilon k)\widehat{\rho}(\epsilon l)I_{k,l}(t)
\lesssim t \tag2.2$$ as seen by choosing $N \approx
\frac{2}{\delta}$.



We now consider
$$ t \sum_{|k_j|,|l_j| \leq
\epsilon^{-1-\delta}; |\frac{k_1}{k_2}|
\not=|\frac{l_1}{l_2}|}{|k|}^{-\frac{3}{2}} {|l|}^{-\frac{3}{2}}
\widehat{\rho}(\epsilon k)\widehat{\rho}(\epsilon l)I_{k,l}(t).
\tag2.3$$



Using the method of stationary phase, (see e.g. \cite{Stein93} and the
appendix below) the right hand side of $(2.3)$ is bounded by
$$ \sum_{|k_j|,|l_j| \leq
\epsilon^{-1-\delta}; |\frac{k_1}{k_2}| \not=|\frac{l_1}{l_2}|}
\frac{1}{|k_1^2l_2^2-l_1^2k_2^2|}=\sum_{|k_j|,|l_j| \leq
\epsilon^{-1-\delta}; |\frac{k_1}{k_2}| \not=|\frac{l_1}{l_2}|}
\frac{1}{|k_1l_2-l_1k_2||k_1l_2+l_1k_2|}. \tag2.4$$



Either $sgn(k_1l_2)=sgn(l_1k_2)$ or $sgn(k_1l_2)=-sgn(l_1k_2)$.
In either case, the expression $(2.4)$ is bounded by the
expression of the form
$$ \epsilon^{-2-2\delta} \sum_{|k_1|,|l_2| \leq
\epsilon^{-1-\delta}} \frac{1}{k_1} \frac{1}{l_2} \lessapprox t
\tag2.5$$ since $\epsilon=t^{-\frac{1}{2}}$.



We now deal with the case when one of $k_1,k_2,l_1,l_2$ is $0$.
Using $(1.10)$ with, say, $k_2=0$, we see that
$$ |I_{k,l}(t)| \lesssim t^{-1}k_1^{-2}l_2^{-2}. \tag2.6$$



Using $(2.6)$ we get, in view of $(2.2)$, with $k_2=0$,
$$ \sum_{\epsilon^{-1-\delta} \ge |k_1|, |l_1|, |l_2| \ge 1}
k_1^{-2} l_2^{-2} \lesssim t^{\frac{1}{2}}. \tag2.7$$



If more than one of $|k_1|$, $|k_2|$, $|l_1|$, $|l_2|$, the
estimate is even easier as we shall see below.



\head Zero determinant \endhead



\vskip.125in



We now handle the case where $\frac{k_1}{k_2}= \pm
\frac{l_1}{l_2}$. In this case, if $k_1=0$, then $l_1=0$.
Similarly, if $k_2=0$, then $l_2=0$. Estimating $I_{k,l}(t)$
trivially by $1$, we get
$$ t \sum_{|k_1|, |l_1| \ge 1} {|k_1|}^{-\frac{3}{2}}
{|l_1|}^{-\frac{3}{2}}
{(1+\epsilon|k_1|)}^{-N}{(1+\epsilon|l_1|)}^{-N} \lesssim t,
\tag3.1$$ and similarly if $k_1=l_1=0$.





Thus we may assume that $|k_j| \ge 1$ and $|l_j| \ge 1$. Let
$\Phi_{k,l}(a)={|k|}_a-{|l|}_a$. Differentiating and using the
fact that $|\frac{k_1}{k_2}|=|\frac{l_1}{l_2}|$, we have
$$ \frac{\partial^2 \Phi}{\partial
a_1^2}=-\frac{1}{4}
\left(\frac{k_1^4}{{|k|}_a^3}-\frac{l_1^4}{{|l|}_a^3}\right)=
-\frac{1}{4} \frac{k_1^4}{{|k|}_a^3}
\left(\frac{|k_2|-|l_2|}{|k_2|} \right), \tag3.2$$ and
$$ \frac{\partial^2 \Phi}{\partial
a_2^2}=-\frac{1}{4} \left(
\frac{k_2^4}{{|k|}_a^3}-\frac{l_2^4}{{|l|}_a^3}\right)=
-\frac{1}{4} \frac{k_2^4}{{|k|}_a^3}
\left(\frac{|k_1|-|l_1|}{|k_1|} \right). \tag3.3$$



In the regime $|k_j|=|l_j|$, the estimate follows trivially. We
just dominate $I_{k,l}(t)$ by a constant and end up with
$$ t \sum_{|k_j| \ge 1} {|k|}^{-3} \lesssim t. \tag3.4$$



If $|k_j| \not=|l_j|$, then either $||k_1|-|l_1|| \ge 1$ or
$||k_2|-|l_2|| \ge 1$. In the former case we use van der Corput
lemma (see e.g. \cite{Stein93} and the appendix below) with $(3.3)$, and
in the latter case with $(3.2)$. The two cases are the same, so we
consider the former without loss of generality. Taking $(2.2)$ into
account, we end up with
$$ t \times t^{-\frac{1}{2}} \sum_{1 \leq |k_j|, |l_1| \leq
\epsilon^{-1-\delta}; ||k_1|-|l_1|| \ge 1} {|k|}^{-\frac{3}{2}}
{|l|}^{-\frac{3}{2}} {|k|}^{\frac{3}{2}} {|k_1|}^{\frac{1}{2}}
{|k_2|}^{-2}{||k_1|-|l_1||}^{-\frac{1}{2}}$$ $$=t^{\frac{1}{2}}
\sum_{1 \leq |k_j|, |l_1| \leq \epsilon^{-1-\delta};
||k_1|-|l_1|| \ge 1} {|l_1|}^{-\frac{3}{2}}
{|k_1|}^{\frac{1}{2}}
{|k_2|}^{-2}{||k_1|-|l_1||}^{-\frac{1}{2}}$$
$$ \lessapprox t^{\frac{1}{2}} \epsilon^{-\frac{1}{2}}
\sum_{1 \leq |k_1|, |l_1| \leq \epsilon^{-1-\delta};
||k_1|-|l_1|| \ge 1}
{|l_1|}^{-\frac{3}{2}}{||k_1|-|l_1||}^{-\frac{1}{2}} \lessapprox
t. \tag3.5$$

This completes the proof of Theorem 0.1.

\vskip.125in

\head Appendix: Oscillatory integrals of the first kind \endhead

\vskip.125in

In this paper we made use of the following basic facts about the
oscillatory integrals of the form
$$ I(t)=\int_{{\Bbb R}^d} e^{i t f(x)} \psi(x) dx, \tag4.1$$ where
$\psi$ is a smooth cutoff function and $f$ is smooth. The proofs can be
found in many books. See, for example \cite{Stein93}.

\proclaim{Theorem 4.1} Suppose that $f$ is convex and $\det(D^2f) \ge
c_0>0$, where $D^2f$ denotes the Hessian matrix of $f$. Then
$$ |I(t)| \lesssim t^{-\frac{d}{2}} c_0^{-\frac{1}{2}}. \tag4.2$$
\endproclaim

\proclaim{Theorem 4.2} Suppose that $|\frac{\partial^2 f}{\partial
x_j^2}| \ge c_0$. Then
$$ |I(t)| \lesssim t^{-\frac{1}{2}} c_0^{-\frac{1}{2}}. \tag4.3$$
\endproclaim

We note that in both theorems the constants may depend on the upper
bounds of derivatives of $f$.

\newpage

\head References \endhead

\vskip.125in

\ref \key Huxley96 \by M. N. Huxley \book Area, Lattice Points,
and Exponentials Sums \yr 1996 \bookinfo London Mathematical
Society Monographs New Series 13 \publ Oxford Univ. Press
\endref

\ref \key ISS02 \by A. Iosevich, E. Sawyer, and A. Seeger \paper Mean
square discrepancy bounds for the number of lattice points in large
convex bodies \jour Journal D'Analyse, Tom Wolff Memorial Issue (to
appear) \yr 2002 \endref

\ref \key Stein93 \by E. M. Stein \book Harmonic Analysis \yr
1993 \publ Princeton University Press \endref

\ref \key TP02 \by J. Toth and Y. Petridis \paper \jour
(preprint) \yr 2002 \endref

\enddocument
