\documentstyle{amsppt}
\tolerance 3000
\pagewidth{5.5in}
\vsize7.0in
\magnification=\magstep1
\widestnumber
\key{AAAAAAAAAAAAAAA}
\NoRunningHeads
\topmatter
\title Introduction to the Besicovitch/Kakeya conjecture: Part I
\endtitle
\author Alex Iosevich
\endauthor
\date August 1, 2004
\enddate
\address Department of Mathematics, University of Missouri-Columbia,
Columbia Missouri 65211 USA \endaddress
\email iosevich \@ math.missouri.edu
\endemail
\abstract This is a set of notes on the Besicovitch/Kakeya conjecture
written for the summer school for mathematically talented high-school
students to be held in August 2004. The notes are dedicated to Tom Wolff
whose legacy continues to tower over the study of the Besicovitch/Kakeya
conjecture and many other important problems in mathematics to this day.
\endabstract
\thanks The work was partly supported by a grant from the National
Science Foundation
\endthanks
\endtopmatter
\document

\head Introduction \endhead

\vskip.125in

This is a set of notes written for the summer program in combinatorial
mathematics to be held at the University of Missouri-Columbia in August
2004. The program will last a total of five days, with approximately 1.5
hours of lectures and 1.5 hours of problem solving each day. The purpose
of this program and these notes is to introduce a motivated high-school
students to one of the most far-reaching and beautifull problems of modern
mathematics- the Besicovitch/Kakeya conjecture which related the size of
sets in the Euclidean space with the number unit line segments in
different directions. Due to the technical nature of the full-fledged
Besicovitch/Kakeya conjecture and its connection to problems in analysis,
partial differential equations and analytic number theory, no effort is
made to present an exhaustive and rigorous view of the subject. Instead,
the students are given a glimpse of this sophisticated problem in a simple
"discrete" setting, where the most salient features of the Kakeya
conjecture can already be seen.

The program was initially designed for two weeks and consequently, these
notes contain way too much material to be covered in one week. It is my
hope that the participants will choose to continue reading and exploring
these notes in particular and the subject matter in general after the
program is over. I also plan to reintroduce some of the material in these
notes in the process of teaching Math 395, the Problem Solving class,
during the Fall semester. 

In the first part of these notes we present the basics of the
Besicovitch/Kakeya problem that does not require anything beyond the
Cauchy-Schwartz in equality (described and illustrated in Section 2
below) and basic counting. In the second part of the notes slicing
arguments, discrete Fourier transform, discrete restriction theory, and
more advanced arithmetic arguments are introduced.

The student is expected to work hard on these notes. This is not bed time
reading, nor is it a fantasy novel. You must have a pen and plenty of
paper handy, and expect to fill up about ten pages of calculations for
every page you read. Mathematics is not a spectator sport, so create in
addition to reading and computing. Every time you see a theorem or a
calculation, try to formulate a new one. Every time you see a proof, try
to find a better one. And most importantly, have fun!

\subhead Acknowledgements \endsubhead Much of the material in these notes
is either inspired by or taken directly from a set of notes, entitled "On
Besicovitch sets", by Ben Green, a survey article by Tom Wolff,
entitled "Recent work connected with the Kakeya problem" (\cite{W99}),
class notes by Terry Tao on the Besicovitch/Kakeya that can be found at
http://www.math.ucla.edu/
\~ \ tao, a survey of the Besicovitch/Kakeya problem by Izabella Laba,
which can be found on her website at http://www.math.ubc.ca/ \~ \ ilaba,
and a semi-expository article by A. Iosevich, entitled "Geometric measure
theory and Fourier analysis" (\cite{I04}). 

\head Basic setup \endhead Let $q$ be a positive integer, prime in
the sense that an integer $a$ divides $q$ if and only if $a=1$ or
$a=q$. Define ${\Bbb F}_q$ to be the set $\{0,1,2, \dots, q-1\}$ with the
rule that addition and multiplication is taken modulo $q$. What this
means is that if $a \in {\Bbb F}_q$ and $b \in {\Bbb F}_q$, $a+b$ (in the
world of ${\Bbb F}_q$) is obtained by adding $a$ and $b$ in the standard
way and computing the remainder after division by $q$. Similarly, to
compute $a \cdot b$, we multiply $a$ and $b$ in the standard way and
again compute the remainder after the division by $q$.

\proclaim{Example 1.1} Let $q=5$. Then ${\Bbb F}_5=\{0,1,2,3,4\}$. Suppose
that we want to multiply $2$ and $4$ in the world of ${\Bbb F}_5$. Well,
$2 \cdot 4=8$ in the sense of regular multiplication. If we divide $8$ by
$5$, the remainder is $3$. Thus $2 \cdot 4=3$ in the world of ${\Bbb
F}_5$.

Now let's compute $2+4$. In the sense of regular addition, this equals
$6$. The remainder of division of $6$ by $5$ is $1$. Thus $2+4=1$ in the
world of ${\Bbb F}_5$.

After this point, we shall stop saying "in the world of ${\Bbb F}_q$". We
shall simply perform our addition annd multiplication according to the
rules we just described and illustrated. \endproclaim

We now introduce a "grid" ${\Bbb F}_q^d$, defined as a set of $d$-tuples
$ (a_1, a_2, \dots, a_d), $ such that $a_j$ is an element of ${\Bbb
F}_q$.

\proclaim{Example 1.2} The set ${\Bbb F}_3^2$ consists of $9$ pairs:
$(0,0)$,
$(0,1)$, $(0,2)$, $(1,0)$, $(1,1)$, $(1,2)$, $(2,0)$, $(2,1)$, and
$(2,2)$.

The set ${\Bbb F}_3^3$ consists of $27$ tiples. Write them out!

\endproclaim

\proclaim{Exercise 1.1} Show that ${\Bbb F}_q^d$, $d$ a positive integer,
consists of $q^d$ elements.

\endproclaim

We now intoduce the notion of a line in ${\Bbb F}_q^d$. Let $x \in {\Bbb
F}_q^d$ and let $v \in {\Bbb F}_q^d$ with the additional restriction that
$v \not=(0,0, \dots, 0)$. Let
$$ L(x,v)=\{x+tv: t=0,1, \dots, q-1\}. \tag1.1$$

\proclaim{Example 1.3} Let $x=(0,1)$ and $v=(1,1)$ and let $q=3$. Then
$L(x,v)$ is the "line" consisting of points $(0,1)$, $(1,2)$, and
$(2,0)$. Why is that? Well, by definition, $L((0,1),
(1,1))=\{(0,1)+t(1,1):t=0,1,2\}$. If
$t=0$, we get $(0,1)$ easily enough. If $t=1$, we get $(1,2)$, no
problem. If $t=2$, we get $(0+2 \cdot 1, 1+2 \cdot 1)=(2,0)$ since $1+2
\cdot1=3$ which is $0$ in the world of ${\Cal F}_3$. \endproclaim

\proclaim{Exercise 1.2} Let $q=3$. Show that $L((0,1),(1,1))$ is the same
line as $L((0,1),(2,2)$ and $L((1,2),(1,1)$. What is going on
here? \endproclaim

\proclaim{Exercise 1.3} We have seen in the previous exercise that given
$x \in {\Bbb F}_q^d$, there may exist $v \not=v' \not=(0,\dots,0$ such
that $L(x,v)$ and $L(x,v')$ are the same line. You probably observed that
this problem takes place if and only if $v=\lambda v'$ for some $\lambda
\in {\Bbb F}_q$. (If you did not observe this, please verify it right
away). Let $V$ be a subset of ${\Bbb F}_q^d$ with the following
properties: \roster \item If $v \in V$, $v \not=(0,\dots,0)$. \item If $v
\in V$ and $v' \in V$, there does not exist $\lambda \in {\Bbb F}_q$ such
that $v=\lambda v'$. \endroster

Suppose that $V$ is {\it maximal} in the sense that it is impossible to
add even more element to $V$ without violating one of the properties
stated above. (Note that without this restriction we may simply take $V$
to consist of a single point, say, $(1,0, \dots, 0)$). 

Compute $\# V$, the number of elements of $V$.
\endproclaim

In the standard (Euclidean) space, two different lines either do not
intersect at all, or intersect at a single point. We shall now verify
that the same is true of lines in ${\Bbb F}_q^d$.

\proclaim{Exercise 1.4} Two different lines $L(x,v)$ and $L(x',v')$ in
${\Bbb F}_q^d$ either do not intersect at all or intersect at a single
point. If $d=2$ prove that two distinct lines $L(x,v)$ and $L(x',v')$ do
not intersect if and only if there exists $\lambda \in {\Bbb
F}_q$ such that $v=\lambda v'$. \endproclaim

We are now ready to state the main problem to be studied in these notes.

\proclaim{Besicovitch/Kakeya conjecture} Let $K \subset {\Bbb F}_q^d$, $d
\ge 2$, such that for every $v \in {\Bbb F}_q^d$ with $v\not=(0, \dots,
0)$, there exists $x \in {\Bbb F}_q^d$ so that $L(x,v) \subset K$. Then
there exists $C>0$, independent of $q$, such that
$$ \# K \ge Cq^d. \tag1.2$$ \endproclaim

To put it simply, the Besicovitch/Kakeya conjecture says that if $K$
contains a line with every possible slope, then this set occupies a
positive proportion of the points in ${\Bbb F}_q^d$. In short,
$$ \text{MANY SLOPES} \ \rightarrow \ \text{MANY POINTS} \ \tag1.3$$

The Besicovitch/Kakeya conjecture is solved in two dimensions. However,
in higher dimensions, it far from being resolved. For example, the best
known result in three dimensions (see \cite{KLT00} is that
$$ \# K \ge Cq^{\frac{5}{2}+{10}^{-10}}. \tag1.4$$

One of the motivations behind this set of notes is to convince you to
dive head first into this mysterious problem which is not terribly likely
to be completely solved any time soon.

Before we start proving results pertaining to the Besicovitch/Kakeya
conjecture, we shall develop some preliminary concepts that will serve to
build up the necessary technique and intuition for the results that come
later.

\vskip.125in

\head Cauchy-Schwartz inequality and some
simple consequences\footnote{Some people call this inequality the
Cauchy-Schwartz-Buniakowski inequality. I have a suspicion that this
inequality was known and relatively widely used long before any of the
three individuals in question was born. I decided to stick with the
"Cauchy-Schwartz" usage primarily out of habit.}
\endhead

\vskip.125in

In this section we shall follow a procedure often considered nasty, but
the one I hope to convince you to appreciate. We shall work backwards,
discovering concepts as we go along, instead of stating them ahead of
time. Let $a$ and $b$ denote two real numbers. Then
$$ {(a-b)}^2 \ge 0. \tag2.1$$

This statement is so vacuous, you are probably wondering why I am telling
you this. Nevertheless, expland the left hand side of $(2.1)$. We get
$$ a^2-2ab+b^2 \ge 0, \tag2.2$$ which implies that
$$ ab \leq \frac{a^2+b^2}{2}. \tag2.3$$

Now consider
$$ A_N=\sum_{k=1}^N a_k=a_1+\dots+a_N, \ B_N=\sum_{k=1}^N
b_k=b_1+\dots+b_N, \tag2.4$$ where
$a_1, \dots, a_N$, and $b_1, \dots, b_N$ are real numbers. Let
$$ X_N={\left(\sum_{k=1}^N a_k^2 \right)}^{\frac{1}{2}}, \
Y_N={\left(\sum_{k=1}^N b_k^2 \right)}^{\frac{1}{2}}. \tag2.5$$

Our goal is to take advantage of $(2.3)$. Let's take a look at
$$ \sum_{k=1}^N a_kb_k=X_N Y_N \sum_{k=1}^N \frac{a_k}{X_N} \cdot
\frac{b_k}{Y_N}$$
$$ \leq X_NY_N \sum_{k=1}^N
\left[\frac{1}{2}{\left(\frac{a_k}{X_N}\right)}^2+
\frac{1}{2}{\left(\frac{b_k}{Y_N}\right)}^2\right]. \tag2.6$$

\proclaim{Exercise 2.1} Explain using complete English sentences how
$(2.6)$ follows from $(2.3)$. \endproclaim

\proclaim{Exercise 2.2} Explain why if $C$ is a constant, then
$\sum_{k=1}^N Ca_k=C\sum_{k=1}^N a_k$. \endproclaim

\proclaim{Exercise 2.3} Explain why $\sum_{k=1}^N
(a_k+b_k)=\sum_{k=1}^N a_k+\sum_{k=1}^N b_k$. \endproclaim

We now use Exercise 2.2 and 2.3 to rewrite $(2.6)$ in the form
$$ X_NY_N \frac{1}{2} \frac{1}{X^2_N} \sum_{k=1} a_k^2+X_NY_N
\frac{1}{2} \frac{1}{Y^2_N}\sum_{k=1}^N b_k^2$$
$$=X_NY_N \frac{1}{2} \frac{1}{X^2_N} X_N^2+X_NY_N
\frac{1}{2} \frac{1}{Y^2_N}Y_N^2$$
$$=\frac{1}{2}X_NY_N+\frac{1}{2}X_NY_N=X_NY_N. \tag2.7$$

Putting everything together, we have shown that
$$ \sum_{k=1}^N a_k b_k \leq {\left(\sum_{k=1}^N a_k^2
\right)}^{\frac{1}{2}} {\left(\sum_{k=1}^N b_k^2
\right)}^{\frac{1}{2}}. \tag2.8$$

This is known as the Cauchy-Schwartz inequality.

\proclaim{Exercise 2.4} (quite difficult if you do not know calculus) Let
$1<p<\infty$ and define the exponent $p'$ by
the equation $\frac{1}{p}+\frac{1}{p'}=1$. Then
$$ \sum_{k=1}^N a_k b_k \leq {\left(\sum_{k=1}^N {|a_k|}^p
\right)}^{\frac{1}{p}} {\left(\sum_{k=1}^N {|b_k|}^{p'}
\right)}^{\frac{1}{p'}}. \tag2.9$$

Observe that $(2.9)$ reduces to $(2.8)$ if $p=2$. Hint: prove that
$ab \leq \frac{a^p}{p}+\frac{b^{p'}}{p'}$ and proceed as in the
case $p=2$. One way to prove this inequality is to set $a^p=e^x$
and $b^{p'}=e^y$ (why are we allowed to do that?). Let
$\frac{1}{p}=t$ and observe that $0 \leq t \leq 1$. We are then
reduced to showing that for any real valued $x,y$ and $t \in
[0,1]$, $e^{tx+(1-t)y} \leq te^x+(1-t)e^y$. Let
$f(t)=e^{tx+(1-t)y}$ and $g(t)=te^x+(1-t)e^y$. Observe that
$f(0)=g(0)=e^y$ and $f(1)=g(1)=e^x$. Can you complete the
argument?

\endproclaim

\subhead Box inequality \endsubhead Let's now try to see what
Cauchy-Schwartz (C-S) inequaity is good for. Let $S_N$ be a finite set of
$N$ points in ${\Bbb R}^3=\{(x_1,x_2,x_3): x_j \ \text{is a real
number}\}$, the three-dimensional Euclidean space. Let $x=(x_1,x_2,x_3)
\in {\Bbb R}^3$ and define
$$ \pi_1(x)=(x_2,x_3), \ \pi_2(x)=(x_1,x_3), \ \text{and} \
\pi_3(x)=(x_1,x_2). \tag2.10$$

The question we ask is the following. We are assuming that $\# S_N=N$.
What can we say about the size of $\pi_1(S_N), \pi_2(S_N)$, and
$\pi_3(S_N)$? Before we do anything remotely complicated, let's make up
some silly looking examples and see what we can learn from them.

Let $S_N=\{(0,0,k): k \ \text{integer} \ k=0,1, \dots, N-1\}$. This set
clearly has $N$ elements. What is $\pi_3(S_N)$ in this case. It is
precisely the set $\{(0,0)\}$, a set consisting of one element. However,
$\pi_2(S_N)$ and $\pi_1(S_N)$ are both $\{(0,k): k=0,1,\dots,N-1\}$,
sets consisting of $N$ elements. In summary, one of the projections is
really small and the others are as large as they can be.

Let's be a bit more even handed. Let $S_N=\{(k,l,0): k,l \
\text{integers} \ 1 \leq k \leq \sqrt{N}, 1 \leq l \leq \sqrt{N}\}$, where
$\sqrt{N}$ is an integer. Again $\# S_N=N$. What do projections look
like? Well, $S_N$ is already in the $(x_1,x_2)$-plane, so
$\pi_3(S_N)=\{(k,l): k,l \
\text{integers} \ 1 \leq k \leq \sqrt{N}, 1 \leq l \leq \sqrt{N}\}$. It
follows that $\# \pi_3(S_N)=N$. On the other hand, $\pi_2(S_N)=\{(k,0):
k \ \text{integer} \ 1 \leq k \leq \sqrt{N} \}$, and
$\pi_1(S_N)=\{(l,0): l \ \text{integer} \ 1 \leq l \leq \sqrt{N} \}$,
both containing $\sqrt{N}$ elements. Again we see that it is difficult for
all the projections to be small.

Let's think about our examples so far from a geometric point of view. The
first example is "one-dimensional" since the points all lie on a line.
The second example is "two-dimensional" since the points lie on a plane.
Let's now build a truly "three-dimensional" example with as much symmetry
as possible. Let $S_N=\{(k,l,m): k,l,m \ \text{integers} \ 1 \leq k,l,m
\leq N^{\frac{1}{3}}\}$, where $N^{\frac{1}{3}}$ is an integer. Again,
$\# S_N=N$, as required. The projections this time all look the same. We
have $\pi_1(S_N)=\{(l,m): l,m \ \text{integers} \ 1 \leq l,m \leq
N^{\frac{1}{3}}\}$, a set of size $N^{\frac{2}{3}}$, and the same is true
of $\# \pi_2(S_N)$ and $\# \pi_3(S_N)$.

Let's summarize what happened. In the case when all the projections have
the same size, each projection has $N^{\frac{2}{3}}$ elements. We will
see in a moment that for any $S_N$, one of the projections must of size
at least $N^{\frac{2}{3}}$. We will see here and later in these notes
that C-S inequality is very usefull in showing that the "symmetric" case
is "optimal", whatever that means in a given instance.

Before starting a more detailed investigation, consider the two
dimensional case. Take a set of $N$ points in ${\Bbb R}^2$ and consider
projections onto $x_1$-axis and $x_2$-axis, respectively. Can we prove
by a simple geometric argument that one of these projections must contain
at least $C\sqrt{N}$ points? Well, let $S_N$ be the set in
question. Observe that $\chi_{S_N}(x_1,x_2) \leq \chi_{\pi_1(S_N)}(x_2)
\cdot \chi_{\pi_2(S_N)}(x_1)$ (see Exercise 2.5 below). It follows that
$\sum_{x_1,x_2} \chi_{S_N}(x_1,x_2)=\sum_{x_1,x_2} \chi_{\pi_1(S_N)}(x_2)
\cdot \chi_{\pi_2(S_N)}(x_1)=\sum_{x_1} \chi_{\pi_2(S_N)}(x_1) \cdot
\sum_{x_2} \chi_{\pi_1(S_N)}(x_2)$ (why?). It follows that $N=\# S_N \leq
\# \pi_1(S_N) \cdot \# \pi_2(S_N) \leq {\left(\max_{j=1,2} \# \pi_j(S_N)
\right)}^2$. We conclude that indeed $\max_{j=1,2} \pi_j(S_N) \ge
\sqrt{N}$ as promised. 

The point of considering the two-dimensional case is that while it does
not entail any of the interesting complexities of the higher dimensional
situation, it is based on the same intuition. Indeed, let us think about
the two-dimensional case from a slightly different point of view. Suppose
for a moment that $\# \pi_1(S_N)$ is smaller than $c\sqrt{N}$, where
$c$ is a constant. Then $S_N$ must consist of at most $c \sqrt{N}$
columns, by definition of $\pi_1$. On the other hand, the total number of
points in all of those columns is $N$, by assumption. It follows that one
of these columns has more than $\frac{N}{c\sqrt{N}}$ points. We conclude,
by setting $c=1$, that either $\# \pi_1(S_N) \ge \sqrt{N}$, or $\#
\pi_2(S_N) \ge \sqrt{N}$, since the latter is precisely what it means for
a column to have more than $\sqrt{N}$ points. This gives an "alternate"
argument for the two-dimensional case. Observe that the argument given in
the previous paragraph is at leat superficially mechanical, while the
argument we just went over is visual and conceptual. Are the arguments
really different, however? As an informal exercise, cut through the
mechanical non-sense of the first argument and explain why it is the same
as the second one. 

The three dimensional case is not going to fall quite so easily. To see
this, let us try to run the argument of the previous paragraph. Suppose
that $\# \pi_1(S_N) <N^{\frac{2}{3}}$. This means that $S_N$ consists of
fewer than $N^{\frac{2}{3}}$ columns of points over the $(x_2,x_3)$-axis.
Since the total number of points is $N$, this tells us that one of the
columns has more than $N^{\frac{1}{3}}$ points. This is not enough,
however, and more careful analysis is needed. The proof can be completed
this way with some work and I urge you to try! We will take a slightly
different road below in order to illustrate what a beautiful bookkeeping
tool the C-S inequality often is. 

To start our analysis of the three dimensional case we need the following
basic definition. Let $S$ be any set. Define $\chi_S(x)=1$ if $x \in S$
and $0$ otherwise.

\proclaim{Exercise 2.5} Let $S_N$ be as above. Then
$$\chi_{S_N}(x) \leq
\chi_{\pi_1(S_N)}(x_2,x_3)\chi_{\pi_2(S_N)}(x_1,x_3)
\chi_{\pi_3(S_N)}(x_1,x_2). \tag2.11$$
\endproclaim

With Exercise 2.5 in tow, we write
$$ N=\# S_N=\sum_{x} \chi_{S_N}(x) \leq
\sum_{x} \chi_{\pi_1(S_N)}(x_2,x_3)\chi_{\pi_2(S_N)}(x_1,x_3)
\chi_{\pi_3(S_N)}(x_1,x_2)$$
$$=\sum_{x_1,x_2} \chi_{\pi_3(S_N)}(x_1,x_2)
\sum_{x_3} \chi_{\pi_1(S_N)}(x_2,x_3)\chi_{\pi_2(S_N)}(x_1,x_3)$$
$$ \leq {\left(\sum_{x_1,x_2}
\chi^2_{\pi_3(S_N)}(x_1,x_2)\right)}^{\frac{1}{2}}
{\left(\sum_{x_1,x_2} {\left(\sum_{x_3}
\chi_{\pi_1(S_N)}(x_2,x_3)\chi_{\pi_2(S_N)}(x_1,x_3)\right)}^2
\right)}^{\frac{1}{2}}$$
$$=I \times II. \tag2.12$$

Now,
$$ I={\left(\sum_{x_1,x_2}
\chi^2_{\pi_3(S_N)}(x_1,x_2)\right)}^{\frac{1}{2}}={\left(\sum_{x_1,x_2}
\chi_{\pi_3(S_N)}(x_1,x_2)\right)}^{\frac{1}{2}}={(\#
\pi_3(S_N))}^{\frac{1}{2}}. \tag2.13$$

On the other hand,
$$ {II}^2=\sum_{x_1,x_2} {\left(\sum_{x_3}
\chi_{\pi_1(S_N)}(x_2,x_3)\chi_{\pi_2(S_N)}(x_1,x_3)\right)}^2$$
$$=\sum_{x_1,x_2} \sum_{x_3} \sum_{x'_3}
\chi_{\pi_1(S_N)}(x_2,x_3)\chi_{\pi_2(S_N)}(x_1,x_3)
\chi_{\pi_1(S_N)}(x_2,x'_3)\chi_{\pi_2(S_N)}(x_1,x'_3)$$
$$ \leq \sum_{x_1,x_2} \sum_{x_3} \sum_{x'_3}
\chi_{\pi_1(S_N)}(x_2,x_3)\chi_{\pi_2(S_N)}(x_1,x'_3)$$
$$=\sum_{x_2,x_3}\chi_{\pi_1(S_N)}(x_2,x_3)
\sum_{x_1,x'_3}\chi_{\pi_2(S_N)}(x_1,x'_3)=\# \pi_1(S_N) \cdot \#
\pi_2(S_N). \tag2.14$$

Putting everything together, we have proved that
$$ \# S_N \leq \sqrt{\# \pi_1(S_N)}\sqrt{\# \pi_2(S_N)}\sqrt{\#
\pi_3(S_N)}. \tag2.15$$

\proclaim{Exercise 2.6} Verify each step above. Where was C-S inequality
used? Why does $\chi^2_{\pi_j(S_N)}(x)=\chi_{\pi_j(S_N)}(x)$? \endproclaim

The product of three positive numbers certainly does not exceed the
largest of these numbers raised to the power of three. It follows from
this and $(2.15)$ that
$$ N=\# S_N \leq \max_{j=1,2,3} {(\# \pi_1(S_N))}^{\frac{3}{2}}.
\tag2.16$$

We conclude by raising both sides to the power of $\frac{2}{3}$ that
$$ \# \max_{j=1,2,3} \pi_j(S_N) \ge N^{\frac{2}{3}} \tag2.17$$ as
claimed.

\proclaim{Exercise 2.7} Let $\Omega$ be a convex set in ${\Bbb R}^3$.
This means that for any pair of points $x,y \in \Omega$, the line segment
connecting $x$ and $y$ is entirely contained in $\Omega$. Prove that
$vol(\Omega) \leq \sqrt{area(\pi_1(\Omega))} \cdot
\sqrt{area(\pi_2(\Omega))} \cdot \sqrt{area(\pi_3(\Omega))}$.

If you can't prove this exactly, can you at least prove using $(2.15)$
and its proof that $\max_{j=1,2,3} area(\pi_j(\Omega)) \ge
{(vol(\Omega))}^{\frac{2}{3}}$? This would say that a convex object of
large volume has at least one large coordinate shadow. Using politically
incorrect language this can be restated as saying that if a hyppopatamus
is overweight, there must be a way to place a mirror to make this
obvious... \endproclaim

\proclaim{Exercise 2.8} (Project question) Generalize $(2.15)$. What do I
mean, you ask... Replace three dimensions by $d$ dimensions. Replace
projections onto two-dimensional coordinate planes by projections onto
$k$-dimensional coordinate planes, with $1 \leq k \leq d-1$. Finally,
replace the right hand side of $(2.15)$ by what it should be...
\endproclaim

\subhead Incidences and matrices \endsubhead Consider a set of $n$ lines
and $n$ points in the plane. Define an incidence to be a pair $(p,l)$,
where $p$ is one of the points in our point set, $l$ is one of the lines
in our set of lines, and $p$ lies on $l$. Let $I(n)$ denote the total
number of incidences determined by a given set of $n$ points and a given
set of $n$ lines. In order to avoid needless headaches we assume that
every point in our point set lies on at least one line in our set of
lines, and every line in our line set contains at least one point in our
point set.

How large can $I(n)$ be? Well, it is clear that $I(n) \leq n^2$. This
observation is not terribly valuable, however, since $I(n)$ cannot
possibly be this large! I mean, how can every line contain every point,
and every point lie on every line?! You might retort that maybe, just
maybe, it is possible for about $n/10$ lines to contain about $n/100$
points each, and for each of those points to be contain in about $n/1000$
of those lines. We shall see that nothing like that can happen.

Our main tools in this endeavor are matrices and the C-S inequaity. Recall
that a $N$ by $N$ matrix $A$ is an array with $n$ rows and $n$ columns.
The elements of $A$ are designated by $a_{ij}$, where $i$ determines the
row and $j$ determines the column. Let's define $A$ as follows.
Enumerate the $n$ points in our point set from $1$ to $n$, and do the
same for lines in our set of lines. Let $a_{ij}=1$ if the $i$'th point
lies on the $j$'th line, and $0$ otherwise. Observe that if $j$ and $j'$
are fixed, with $j \not=j'$,
$$ a_{ij} \cdot a_{ij'}=1 \tag2.18$$ for at most one value of $i$. This
is because $a_{ij} \cdot a_{ij'}=1$ if and only if $a_{ij}=1$ and
$a_{ij'}=1$. This means that the $i$'th point is on the $j$'th lines and
also on the $j'$th line. Intersection of two distinct lines is either
empty or consists of exactly one point. It follows that indeed the
equality in $(2.18)$ can hold for at most one $i$.

We are now ready for action. What is $I(n)$? It is nothing more than the
total number of $1$s in $A$! Since $A$ consists of only $1$s and $0$s,
$$ I(n)=\sum_{i=1}^n \sum_{j=1}^n a_{ij}=\sum_{i=1}^n \left(\sum_{j=1}^n
a_{ij}\right) \cdot 1$$
$$ \leq {\left(\sum_{i=1}^n 1 \right)}^{\frac{1}{2}} {\left( \sum_{i=1}^n
{\left( \sum_{j=1}^n a_{ij} \right)}^2 \right)}^{\frac{1}{2}}=\sqrt{n}
\cdot {\left( \sum_{i=1}^n
{\left( \sum_{j=1}^n a_{ij} \right)}^2 \right)}^{\frac{1}{2}}. \tag2.19$$

Now,
$$ \sum_{i=1}^n
{\left( \sum_{j=1}^n a_{ij} \right)}^2=\sum_{i=1}^n \sum_{j=1}^n
\sum_{j'=1}^n a_{ij}a_{ij'}$$
$$=\sum_{i=1}^n \sum_{1 \leq j,j' \leq n; j \not=j'}
a_{ij}a_{ij'}+\sum_{i=1}^n \sum_{j=1}^n a_{ij}^2=apple+orange. \tag2.20$$

To estimate apple we use $(2.18)$. Indeed, since $a_{ij} \cdot a_{ij'}=1$
for at most one $i$,
$$ apple \leq \# \{(j,j'): 1 \leq j,j' \leq n; j \not=j'\}=n^2-n.
\tag2.21$$

\proclaim{Exercise 2.9} Write out the details of the equality on the
right hand side of $(2.21)$. \endproclaim


On the other hand,
$$ orange \leq \# \{(i,j): 1 \leq i,j \leq n\}=n^2. \tag2.22$$

Putting everything together and using tthe fact that $n^2-n \leq n^2$, we
see that
$$ I(n)=\sum_{i=1}^n \sum_{j=1}^n a_{ij} \leq \sqrt{2} \cdot
n^{\frac{3}{2}}. \tag2.23$$

We conclude that the number of incidences between $n$ points and $n$
lines in the plane is at most $\sqrt{2}n^{\frac{3}{2}}$. Can this
estimate be improved? Sure it can... The sharp answer is $I(n) \leq
Cn^{\frac{4}{3}}$, where $C$ is a fixed positive constant. This is the
celebrated Szemeredi-Trotter incidence theorem (\cite{ST83}) and it is
sharp in the sense that one can construct a set of $n$ lines and $n$
points such that the number of incidences is approximately
$n^{\frac{4}{3}}$, up to a constant. The proof of this result will appear
in the second part of these notes.

\proclaim{Exercise 2.10} Show that the estimate $I(n) \leq
Cn^{\frac{3}{2}}$ we just obtained for points and lines in the plane is
best possible for points and lines in ${\Bbb F}_q^2$. Hint: Take as your
point set all the points in ${\Bbb F}_q^2$ and take as your line set all
the lines in ${\Bbb F}_q^2$. \endproclaim

\proclaim{Exercise 2.11} Let $S_N$ be a subset of the plane with $N$
elements. Define $\Delta(S_N)=\{\sqrt{{(x_1-y_1)}^2+{(x_2-y_2)}^2}:
x=(x_1,x_2) \in S_N, y=(y_1,y_2) \in S_N\}$. Use $(2.23)$ to show that
$\# \Delta(S_N) \ge C \sqrt{N}$ for some constant $C$ independent of $N$.

Can you do better? The conjectured asnwer is that $\# \Delta(S_N) \ge
C\frac{N}{\sqrt{\log(N)}}$. The best known result to date, due to Katz
and Tardos (\cite{KT04}), based on the previous result due to Solymosi and
Toth (\cite{SoT01}) is $\# \Delta(S_N) \ge CN^{\beta}$, where $\beta
\approx .86$. (See \cite{KT04}).

What about higher dimensions? If $S_N \subset {\Bbb R}^d$ of size $N$,
prove that $\# \Delta(S_N) \ge CN^{\frac{1}{d}}$. Can you do better? The
conjectured answer here is $\# \Delta(S_N) \ge CN^{\frac{2}{d}}$ in
dimensions three and higher. Do you see where the exponent $\frac{2}{d}$
is coming from? Hint: Let $S_N=\{n=(n_1, \dots, n_d): n_j \in {\Bbb Z}; 1
\leq n_j \leq N^{\frac{1}{d}} \}$.

\endproclaim

\proclaim{Exercise 2.12} Show that the number of incidences between $n$
points and $n$ two-dimensional planes in ${\Bbb R}^3$ can be $n^2$.
Suppose that we further insist that the intersection of any three planes
in our collection contains at most one point. Prove that the number of
incidences is $\leq Cn^{\frac{5}{3}}$.

More generally, prove that if we have $n$ points and $n$
$d-1$-dimensional planes in ${\Bbb R}^d$, then the number of incidences
can be $n^2$. Show that the number of incidences is $\leq
Cn^{2-\frac{1}{d}}$ if we further insist that the intersection of any $d$
planes from our collection intersect at at most one point.
\endproclaim

\proclaim{Exercise 2.13} Prove that $n$ points and $n$ spheres of the
same radius in ${\Bbb R}^d$, $d \ge 4$, can have $n^2$ incidences. Use
the techniques of the chapter that when $d=2$ the number of incidences is
$\leq Cn^{\frac{3}{2}}$. What can you say about the case
$d=3$? \endproclaim

\vskip.125in

\head Besicovitch/Kakeya conjecture in two dimensions \endhead

\vskip.125in

In this section we verify $(1.2)$ in the case $d=2$. What you should be
asking yourselves at every step, is where are we using the peculiarities
of the two-dimensional space, and why this approach should be harder in
higher dimensions.

We have a set $K \subset {\Bbb F}_q^2$ which contains a line in every
direction. This means that there exist lines $L_1, L_2, \dots, L_{q+1}$
entirely contained in $K$ with the additional property that any pair of
these lines intersects at exactly one point. How do we know this? An
abnoxious answer is that you verified exactly this in Exercise 1.3 and
1.4. Let's discuss it again, however. Consider $L(x,v)$ in two
dimensions. How many choices are there for $v$? Well, $v=(v_1,v_2)$, so
there are $q^2-1$ choices, since $v=(0,0)$ is forbidden. On the other
hand, multiplying $v$ by $\lambda \in {\Bbb F}_q$ leads to the same line.
How many $\lambda$s are there? Since it makes no sense to use
$\lambda=0$, there are $q-1$ relevant $\lambda$s. It follows that $K$
indeed contains $\frac{q^2-1}{q-1}=q+1$ lines with different "slopes". By
Exercise 1.4 (not very difficult) each pair of such lines intersects at
exactly one point.

Before we get on with the precise calculations, let's try to understand
why the Besicovitch/Kakeya conjecture should be true in two-dimensions.
As we have just seen, $K$ contains $q+1$ lines of different "slopes".
Choose one of these lines and call it the stem. The other $q$ lines
intersect this stem forming a sort of a hairbrush. Since two of these
lines intersect at exactly one point, it is pretty clear that the total
number of points in $K$ is at least $(q+1) \cdot
\frac{q}{2}=\frac{q(q+1)}{2}$. Of course, we need to make this argument
precise, which is what we are about to do.

All the tools are now in place. Let $K'=\cup_{i=1}^{q+1} L_i$. Since
$K' \subset K$, it suffices to prove that $\# K' \ge Cq^2$. Let
$\chi_{L_i}(x)=1$ if $x \in L_i$ and $0$ otherwise. We must somehow take
advantage of the fact that we have $q+1$ lines with each pair
intersecting at exactly one point. How do we "encode" intersections?
Well,
$$ \sum_{x \in K'} \chi_{L_i}(x)\chi_{L_j}(x)=\# \{x \in K': x \in L_i
\ \text{and} \ x \in L_j\}=\# (L_i \cap L_j), \tag3.1$$ since $L_i
\subset K'$ and $L_j \subset K'$ by assumption.

With this observation in tow, consider
$$ \sum_{x \in K'} {[\chi_{L_1}(x)+\dots+\chi_{L_{q+1}}(x)]}^2=\sum_{x
\in K'} \sum_{i=1}^{q+1} \sum_{j=1}^{q+1} \chi_{L_i}(x)\chi_{L_j}(x)$$
$$=\sum_{i=1}^{q+1} \sum_{j=1}^{q+1} \# (L_i \cap L_j)=2q(q+1), \tag3.2$$
where to obtain the first line we used $(3.1)$ and to compute the second
line we used the fact that $\# L_i \cap L_j=1$ if $i \not=j$.

All this is very nice, but we need to somwhow get a hold on $\# K'$. This
is where Section 2 comes in handy. The left hand side of the first line
of $(3.2)$ is a sum of something squared. This immediately :) reminds us
of the Cauchy-Schwartz inequaity! Indeed, C-S tells us that
$$ {\left(\sum_{x \in K'} [\chi_{L_1}(x)+\dots+\chi_{L_{q+1}}(x)] \cdot 1
\right)}^2 \leq \# K' \cdot \sum_{x \in K'}
{[\chi_{L_1}(x)+\dots+\chi_{L_{q+1}}(x)]}^2. \tag3.3$$

Plugging in $(3.2)$ we see that
$$ {\left(\sum_{x \in K'} [\chi_{L_1}(x)+\dots+\chi_{L_{q+1}}(x)]
\right)}^2 \leq \# K' \cdot 2q(q+1), \tag3.4$$ or, equivalently,
$$ \# K' \ge \frac{{\left(\sum_{x \in K'}
[\chi_{L_1}(x)+\dots+\chi_{L_{q+1}}(x)] \right)}^2}{2q(q+1)}. \tag3.5$$

We seem to be getting somewhere provided we can evaluate the numerator of
$(3.5)$. We have
$$ \sum_{x \in K'} [\chi_{L_1}(x)+\dots+\chi_{L_{q+1}}(x)]=\sum_{x \in
K'} \sum_{j=1}^{q+1} \chi_{L_j}(x)=\sum_{j=1}^{q+1} \sum_{x \in K'}
\chi_{L_j}(x). \tag3.6$$

Since $L_j \subset K'$,
$$ \sum_{x \in K'} \chi_{L_j}(x)=\# L_j=q. \tag3.7$$

We conclude that
$$ {\left(\sum_{x \in K'} [\chi_{L_1}(x)+\dots+\chi_{L_{q+1}}(x)]
\right)}^2=q^2{(q+1)}^2. \tag3.8$$

Plugging this into $(3.5)$ yields
$$ \# K' \ge \frac{q(q+1)}{2}, \tag3.9$$ as promised.

The result we just presented was first proved in ${\Bbb R}^2$ (whatever
that means :)) by R. Davies in 1971 (\cite{D71}), though the proof above
is much closer to the one given for a related problem by A. Cordoba
(\cite{C77}). 

This seems to be the end of the story in two dimensions. Unfortunately,
(or rather fortunately) mathematicians always find a way to complicate
things. The following exercises give a taste of things to come in Part II
of these notes where the level of fun (and pain) will get wretched up
another notch.

\proclaim{Exercise 3.1} Find the smallest possible Besicovitch/Kakeya
subset of ${\Bbb F}_q^2$. We know that it contains at least
$\frac{q(q+1)}{2}$ elements. Get as close to this number as you can.
Hint: consider $S=\{(x_1,x_2) \in {\Bbb F}_q^2: x_1+x_2^2 \ \text{is a
square in} \ {\Bbb F}_q\}$. (A number $s$ is a square in ${\Bbb F}_q$ if
there exists $u \in {\Bbb F}_q$ such that $s=u^2$ in the world of ${\Bbb
F}_q$). \endproclaim

\proclaim{Exercise 3.2} Let $0<\alpha<1$. Suppose that we only assume
that $K$ is a subset of ${\Bbb F}_q^2$ with the property that for every $v
\not=(0,0)$, $v \in {\Bbb F}_q^2$, there exists an $x \in {\Bbb F}_q^2$
such that more than $q^{\alpha}$ points of $L(x,v)$ are contained in
$K'$. What can you say about $\# K$? Once you obtain an aswer, try to
determine whether your estimate is "reasonable". More precisely, for
various values of $\alpha<1$, experiment with constructions of subset of
${\Bbb F}_q^2$ satisfying the required properties. This type
of a formulation of the Kakeya problem is due to Hillel
Furstenberg. See \cite{KT01} and the references contained
therein. \endproclaim

\vskip.125in

\head A gentle entry into higher dimensions: bushes and hairbrushes
\endhead

\vskip.125in

Higher dimensional space is very annoying. It is no longer true that two
lines are either "parallel" or intersect at a single point. It is quite
easy for two lines to simply be in "parallel" planes which makes the
structure of Besicovitch/Kakeya sets much harder to understand.

\subhead Bourgain's bush argument (late 80s) \endsubhead In this section
we abandon our policy of systematically referencing the results we
present. Instead, we refer the reader to Tom Wolff's beautiful survey
article (\cite{W99}) where all the relevant references are present. 

Let's start with the following simple observation. Let $K$ be a
Besicovitch/Kakeya set in ${\Bbb F}_q^d$. How many lines must this set
contain? Well, if have completed Exercise 1.3, we know that the answer is
$\approx q^{d-1}$ (I am being obnoxious again...). Let's why that is.
Consider a line $L(x,v)$. We have $q^d-1$ choices for $v$ since $v=(0,
\dots, 0)$ is not allowed. As before, $v$ and $v'=\lambda v$, $\lambda
\in {\Bbb F}_q$, $\lambda \not=0$, lead to the same line. It follows that
the number of distinct lines in $K$ is at least $\frac{q^d-1}{q-1} \ge
\frac{q^{d-1}}{2}$.

Suppose that $\# K \leq \frac{q^{\frac{d+1}{2}}}{4}$. Then at least one
point of $K$ must lie on at least
$$L=\frac{q^{\frac{d-1}{2}}}{2} \tag4.1$$ lines
entirely contained in $K$. To see this first observe the numbers of pairs
$(p,l)$, where $p \in K$, $l$ is a line contained in $K$, and $p$ lies on
$l$, is at least $q \cdot \frac{q^{d-1}}{2}=\frac{q^d}{2}$ by the
argument in the previous paragraph. By assumption, the number of points
in $K$ is $\leq \frac{q^{\frac{d+1}{2}}}{4}$. Then $(4.1)$ follows since
$$ \frac{q^d}{2} \leq \# \cup_p \cup_l \{(p,l): p \in l\} \leq
\frac{q^{\frac{d+1}{2}}}{4} \max_p \# \{(p,l): p \in l\}, \tag4.2$$ where
$p$ is a point in $K$ and $l$ is a line in $K$. It follows that
$$  \max_p \# \{(p,l): p \in l\} \ge \frac{q^{\frac{d-1}{2}}}{4},
\tag4.3$$ as advertised. What we just proved is that there exists a point
$p_0 \in K$ which belongs to at least $\frac{q^{\frac{d-1}{2}}}{4}$ lines
in $K$. Since each of these lines contains $q-1$ points aside from
$p_0$,
$$ \# K \ge 1+L(q-1) \ge \frac{q^{\frac{d+1}{2}}}{4}. \tag4.4$$

Thus we have shown that a Besicovitch/Kakeya sets in ${\Bbb F}_q^d$ are
$\frac{d+1}{2}$ "dimensional". This is horribly unsatisfactory since our
goal is $d$, not $\frac{d+1}{2}$, and we can already do better than
$\frac{d+1}{2}$ when $d=2$. We did take an important step in the right
direction, though, as the techniques we just developed will come in handy
in a moment.

\subhead Wolff's hairbrush argument (mid 90s) \endsubhead What was the
essence of the bush argument? If lines do not intersect much, we win
because there are points all over the place. If lines do intersect, we
look for places where lots of lines intersect in the same place. We call
such a happy meeting place a bush. What we did above is first argued that
there must exist a fairly large bush. We then estimated the number of
points in this bush and obtained the estimate $(4.41)$. As cute as this
argument is, it is hopelessly naive if we are to get anywhere close to
the full Besicovitch/Kakeya conjecture.

Our next step in the direction of fame and glory (don't get too excited)
is the hairbrush construction. Let $K$ be a Besicovitch/Kakeya set and
suppose that $\# K
\leq q^{\frac{d+2}{2}}$. We repeat the argument we used in the first line
of $(3.2)$. We know by above that $K$ contains at least
$\frac{q^{d-1}}{2}$ lines with distinct "slopes". Let $K'$ be the union of
these lines. Reusing the proof of the two-dimensional Besicovitch/Kakeya
conjecture, we have
$$ q^2 k^2={\left( \sum_{x \in K'}
[\chi_{L_1}(x)+\dots+\chi_{L_k}(x)] \right)}^2 \leq \# K' \cdot \sum_{x
\in K'} {[\chi_{L_1}(x)+\dots+\chi_{L_{q+1}}(x)]}^2$$=$$ =\# K'
\cdot \sum_{x \in K'} \sum_{i=1}^{q+1} \sum_{j=1}^{q+1}
\chi_{L_i}(x)\chi_{L_j}(x)=\# K' \cdot \sum_{i=1}^k \sum_{j=1}^k \# (L_i
\cap L_j), \tag4.5$$ where $k$ is the number of lines (which by above is
$\ge \frac{q^{d-1}}{2}$).

\proclaim{Exercise 4.1} Why is the first line in $(4.5)$ true? Did we use
an inequality with a name in the second line? Which one? \endproclaim

Since we have assumed that $\# K \leq q^{\frac{d+2}{2}}$, we also have
$\# K' \leq q^{\frac{d+2}{2}}$. Plugging this into $(4.5)$ we get
$$ \sum_{i=1}^k \sum_{j=1}^k \# (L_i \cap L_j) \ge
\frac{q^{\frac{3d-2}{2}}}{2}. \tag4.6$$

It follows that there exists $i=i_0$ such that
$$ \sum_{1 \leq j \leq k; j \not=i_0} \# (L_j \cap L_{i_0}) \ge
\frac{\frac{q^{\frac{3d-2}{2}}}{2}}{\frac{q^{d-1}}{2}}-q \ge
\frac{q^{\frac{d}{2}}}{2}. \tag4.7$$

\proclaim{Exercise 4.2} How did we go from $(4.6)$ to $(4.7)$?
\endproclaim

We just proved that there exists a line $L_{i_0}$, called the base of the
hairbrush, such that at least $m=\frac{q^{\frac{d}{2}}}{2}$ other lines
contained in $K'$ intersect it. We call this collection of these $m$ lines
the hairbrush, denoted by $H$.

Let $\Pi_j$ denote the two-plane determined by $L_{i_0}$ and $L_j$.
Suppose that $\Pi_j$ contains $n_j \ge 1$ lines from the hairbrush. We
want to estimate $\# (\Pi_j \cap H)$ from below. Since $\Pi_j$ is a
two-dimensional plane, we should be able to use Section 3. Unfortunately,
in that section we only learned to deal with sets containing
approximately $q$ lines with different slopes. In this case we have $n_j$
lines, which may be smaller than $q$. This predicament forces us to
rewrite the argument in Section 3 for the purpose at hand. Let $L_1,
\dots, L_{n_j}$ be the lines in the hairbrush (after possibly doing some
relabeling) that are contained in $\Pi_j \cap H$. We have
$$ q^2 {(n_j+1)}^2={\left( \sum_{x \in \Pi_j \cap H}
[\chi_{L_1}(x)+\dots+\chi_{L_{n_j}}(x)] \right)}^2$$ $$ \leq \# (\Pi_j
\cap H) \sum_{x \in \Pi_j \cap H}
{[\chi_{L_1}(x)+\dots+\chi_{L_{n_j}}(x)]}^2$$ $$=\# (\Pi_j \cap H)
\sum_{i=1}^{n_j+1} \sum_{j=1}^{n_j+1} \# (L_i \cap L_j)=\# (P_j \cap H)
\cdot ((n_j+1)q+(n_j+1)(n_j))$$ $$\leq 2 \cdot \# (\Pi_j \cap H) \cdot
(n_j+1)q.\tag4.8$$

It follows that
$$ \# (\Pi_j \cap H) \ge \frac{1}{2} (n_j+1)q \ge \frac{qn_j}{4}.
\tag4.9$$

\proclaim{Exercise 4.3} Why does $n_j+1$ appear all over the place in
$(4.8)$ instead of $n_j$? Hint: Don't forget the base of the hairbrush...
\endproclaim

We are almost done since
$$ \# K \ge \# H \ge \frac{q}{4} \sum_{j=1}^t n_j=\frac{qm}{4} \ge
\frac{q^{\frac{d+2}{2}}}{8}. \tag4.10$$

We just proved that if $K$ is a Besicovitch/Kakeya set in ${\Bbb F}_q^d$,
then $\# K \ge \frac{q^{\frac{d+2}{2}}}{8}$. This is not quite the
Besicovitch/Kakeya conjecture, but we are getting closer!

We conclude these notes with an exercise which may well be a gateway to
further progress on the Besicovitch/Kakeya conjecture.

\proclaim{Exercise 4.4} Does a hairbrush in the argument above need to
contain a line of every "slope"? Given an explicit examples proving that
it does not. Now suppose that you have a Besicovitch/Kakeya set containing
a hairbrush containing a line with every possible "slope". Prove that $\#
K \ge Cq^d$. Prove that same conclusion follows if instead of asuming that
the hairbrush contains a line with every possible "slope", it only
contains $\ge cq^{d-1}$ lines with different "slopes".

Can you construct and example of a Besicovitch/Kakeya set $K$ such that
no hairbrush contains $\ge cq^{d-1}$ lines with different "slopes"?
\endproclaim

\newpage 

\head References \endhead 

\vskip.125in 

\ref \key C77 \by A. Cordoba \paper The Kakeya maximal function and
spherical summation multipliers \yr 1977 \jour Amer. J. Math. \vol 99
\pages 1-22 \endref 

\ref \key D71 \by R. Davies \paper Some remarks on the Kakeya problem
\jour Proc. Camb. Phil. Soc. \vol 69 \yr 1971 \pages 417-421 \endref

\ref \key I04 \by A. Iosevich \paper Geometric measure theory and Fourier
analysis \jour Birkhauser; proceedings of the series of lectures
delivered at Padova (Minicorsi) in 2002 \yr 2004 \endref

\ref \key KT01 \by N. Katz and T. Tao \paper Some connections between the
Falconer and Furstenburg conjectures \jour New York J. Math. \vol 7
\yr 2001 \pages 148-187 \endref

\ref \key KT04 \by N. Katz and G. Tardos \paper A new entropy inequality
for the Erdos distance problem \jour Towards a Theory of Geometric
Graphs. (ed.J Pach) Contemporary Mathematics \vol 342 \yr 2004 \endref

\ref \key KLT00 \by N. Katz, I. Laba, and T. Tao \paper An improved bound
on the Minkowski dimension of Besicovitch sets in ${\Bbb R}^3$ \jour
Annals of Math. \vol 152 \yr 2000 \pages 383-446 \endref 

\ref \key ST83 \by E. Szemeredi and W. Trotter \paper Extremal problems
in discrete geometry \yr 1983 \jour Combinatorica \vol 3 \pages 381-392
\endref

\ref \key SoT01 \by J. Solymosi and C. Toth \paper Distinct
distances in the plane \jour Discr. Comp. Jour. (Misha Sharir
birthday issue) \vol 25 \yr 2001 \pages 629-634 \endref

\ref \key W99 \by T. Wolff \paper Recent work connected with the Kakeya
problem \jour Prospects in Mathematics (Princeton, NJ, 1996) Amer. Math.
Soc., Providence, RI \yr 1999 \pages 129-162 \endref 




















\enddocument
