\documentstyle{amsppt}
\magnification=\magstep1
\pagewidth{5.5in}
\vsize=7in
\NoBlackBoxes
%
\def\R{{\Bbb R}}
%
%
\def\n#1{\left\vert\,{#1}\,\right\vert}
 %
\topmatter    
\title A restriction theorem for flat manifolds of codimension two 
\endtitle
\author Laura De Carli and Alex Iosevich 
\endauthor
\endtopmatter
\document
\noindent{\bf Introduction}: Let $M$ denote a submanifold of $\R^{n+2}$ of 
codimension $2$. Let $\Cal R$ denote a restriction operator
$$
{\Cal R}f(\eta) =\int e^{-i\langle x,\,\eta\rangle} f(x)dx,  
\quad  \eta\in M, \quad f\in{\Cal S}(\R^{n+2}).
\tag (1.1)r
$$
 We wish
to find an optimal range of exponents $p$ such that
$$ 
\Vert{{\Cal R}f}_{L^2(M,\,d\sigma)}\Vert \leq
C_p\Vert f\Vert_{L^p(\R^{n+2})}, \tag (1.2)
$$
where $d\sigma$ is a  compactly supported measure on $M$.
\par
Let ${\Cal F}[d\sigma]$ denote the Fourier transform of $d\sigma$. By a theorem
of Greenleaf (see [G]), the inequality ( (1.2)) holds for 
$\dsize p=\frac{2(2+\gamma)}{4+\gamma}$ if 
$$
\vert{{\Cal F}[d\sigma](R\zeta)}\Vert\leq C(1+ R)^{-\gamma},
\qquad \zeta \in S^{n+1} .
\tag (1.3)
$$
The purpose of this paper is to use Greenleaf's result to establish 
a restriction theorem for a  class of degenerate submanifolds of $\R^{n+2}$ of 
codimension $2$. We shall assume that our manifold is given as a joint graph 
of two homogeneous functions, where the first graphing 
function is homogeneous of degree 1 and 
the second graphing 
function is homogeneous of degree m. Under the appropriate curvature assumption
we will show that ( (1.3)) holds with $\gamma=\frac{n}{m}$.
\par
An application of Greenleaf's result yields a restriction theorem with
$\dsize p=\frac{2(2m+n)}{4m+n}$.
\par
We shall need the following definitions. 
\par
\noindent
{\bf Nonvanishing Gaussian curvature:} Let $\Sigma$ be a submanifold of $\R^{N+1}$ 
of codimension $1$ equiped with a smooth compactly supported measure $d\mu$.
Let $J:\Sigma \to S^{N}$ be the usual Gauss map taking each point on 
$\Sigma$ to the outward  unit normal at that point. We say that $\Sigma$
has everywhere nonvanishing Gaussian curvature if the differential of the 
Gauss map $dJ$ is always nonsingular.
\par
\noindent
{\bf Strong curvature condition:} Let $S$ be a submanifold of $\R^{N+2}$ 
of codimension $2$ equiped with a smooth compactly supported measure $d\mu$.
Suppose that $S$ is a joint graph of smooth functions $g_1$ and $g_2$, 
where $g_j:\R^N\to\R$. Let ${\Cal N}_{x_0}(S) $ denote the two dimensional
space of normals to $S$ at a point $x_0$. We say that $S$ satisfies the 
strong curvature condition (SCC) if for all $x_0 \in S$ in some neighborhood of 
support$(d\mu)$,
$$\det D^2(\nu_1g_1(x)+\nu_2g_2(x))\ne 0,\qquad 
\forall \nu\in {\Cal N}_{x_0}$$, 
where $D^2$ denotes the Hessian matrix.

One can check that the above definitions are independent of the  parametrization.
Our main result is the following:

\proclaim{Main Theorem} Let $M=\{(x,\,x_{n+1},\,x_{n+2})\in \R^{n+2}\ : \ 
x_{n+1} = \phi_1(x),\ x_{n+2} = \phi_2(x)\}$, $n \ge 2$, 
where $\phi_i\in {\Cal C}^{\infty}_0(\R^n\backslash\{0\})$, $\phi_1$ is homogeneous 
of degree $1$, and
$\phi_2$ is homogeneous of degree $m\ge 2$. 
Let $\Sigma_j = \{x \ : \ \phi_j(x) =1\}$.
Assume also that $\phi_2$ only vanishes at the origin  and that $\Sigma_2$ has 
everywhere nonvanishing Gaussian curvature. Let
$$ F(\xi,\,\lambda_1,\,\lambda_2) =
\int_{\R^{n}}
e^{i(\langle \xi,\, x\rangle +\lambda_1\phi_1(x) + \lambda_2\phi_2(x))}
\chi(x)\,dx,$$
where $\chi \in {\Cal C}^{\infty}_0(\R^n) $.
\item{ }a)  Suppose that the restriction of $\phi_1$ to the set where
$\phi_2 = 1$,  $\dsize {\phi_1}_{\vert_{\Sigma_2}}$, is constant. Then
$$\vert{F(\xi,\,\lambda_1,\,\lambda_2)}\vert\leq C(\n\xi+\n\lambda_1+\n\lambda_2)^
{-\frac{n}{m}}
\tag (1.4)$$
when $m\ge 2n$.
\item{ }b)  Let  $\dsize M_{\vert_{\{x_{n+2}=1\}}}$  denote the restriction 
of $M$ to the hyperplane $\{x_{n+2}=1\}$.  If  
$\dsize M_{\vert_{\{x_{n+2}=1\}}}$ (wiewed as a submanifold of codimension $2$
of $\{x_{n+2}=1\}$) satisfies the 
strong curvature condition, then ( (1.4)) holds for $m\ge 2$.
\par
\endproclaim
\bigskip
The conclusions of part (a) do not in general hold if $m<2n$.  
Let $\phi_1(x) = \n x$, $\phi_2(x) = \n x^m$. 
Let $\xi = (0,\, 0,\, \cdots,\, 0)$.
Then, in polar coordinates, 
$$
F(0,\,\lambda_1,\,\lambda_2) =
C\int_{0}^{\infty}
e^{i( \lambda_1 r + \lambda_2 r^m)}r^{n-1}
\chi(r)\,dr. 
$$
It is not hard to see that the best isotropic decay for this integral cannot
exceed 
\par\noindent
$\dsize O\left(\left(\sqrt{\lambda_1^2+\lambda_2^2}\right)^{-\frac{1}{2}}\right)$.
Hence the restriction $m\ge 2n$ is necessary.
\vskip.125in 
\noindent
{\bf Remarks:} (1) It is known that isotropic decay estimates for the Fourier transform
of the surface-carried measure cannot be expected to yield an optimal restriction
theorem (see e.g. [C]).
We shall apply a homogeneity argument due to Knapp  to the class of
manifolds considered in the theorem above.
\par
Let ${\Cal R}$ denote the restriction operator defined above. Let 
$\hat f_\delta = h$, where $h$ is the characteristic function of a rectangle in 
$\R^{n+2}$ with sides of lengths $ (1,\,1,\,\cdots 1,\,C,\,C)$, $C$ large.
\par
Then $$ 
\Vert{f_\delta}\Vert_p\approx \delta^{(1-1/p)(n+m+1)}\qquad
\text{and} \qquad \Vert{{\Cal R}f_\delta}\Vert_p \approx\delta^{n/2}.
$$
Hence ( (1.2)) can only hold if $\dsize p \leq \frac{2(n+m+1)}{n+2(m+1)}$.
\par
If we apply Greenleaf's result  ( (1.3)) to the Main Theorem, we see that 
( (1.2)) holds for $\dsize p\leq\frac{2(2m+n)}{4m+n}$ .
\par
The gap between this exponent and the exponent given by Knapp's homogeneity
argument suggests that the restriction theorem (1.2) may hold for a 
wider range of exponents. The result obtained using the Main Theorem is not 
sharp. In order to obtain a sharp result one would probably have to obtain
precise non-isotropic estimates for the Fourier transform of the
surface carried measure using the techniques of M. Christ (see [C]).
\par
\quad (2) The curvature conditions of the Main Theorem are not entirely
satisfying because there is no natural transition between parts (a) and (b).
\par
We hope to address these difficulties  in a subsequent paper.
\bigskip
\centerline{\bf Proof of the main result:}
\bigskip
\noindent
{\bf Notation:} \roster
\item
Given  $a$, $b > 0$ we say  that $a \approx b$ ($a$ comparable to $b$)
if there exist   $c_1$, $c_2 > 0$ such that 
$ 
c_1 a \leq b \leq c _ 2 a .
$
We say that  $a>\!>b$  ($a$  much larger than  $b$)  if  the inequality
$ a \leq  Cb$ is not satisfied for any $C >0$. The notion $a<\!<b$ is 
defined similarly.
\item
We   denote by $C$ a generic constant which may change from line to line.
\endroster
\bigskip
\demo {Proof of part (a) of the Main Theorem} 
Let $\Psi(x) =\langle \xi,\, x\rangle +\lambda_1\phi_1(x)+\lambda_2\phi_2(x) 
.$ Then $\bigtriangledown\Psi(x) = \xi+\lambda_1\bigtriangledown\phi_1(x)  +
\lambda_2\bigtriangledown\phi_2(x)$. Since ${\phi_1}_{\vert_{\Sigma_2}}$  
is constant by assumption, then $\phi_1 \ne 0$ away from the origin. Hence, 
$\bigtriangledown\phi_1(x)\ne 0$ away from the origin by the Euler homogeneity 
relation, and since every
component of $\bigtriangledown\phi_1(x)$ is homogeneous of degree zero,
we have $\vert{\bigtriangledown\phi_1(x)}\vert \ge C$ for all $x \in$ support$(\phi_1)$.
\par
Suppose that  $ \vert\xi\vert<\!< \vert{\lambda_2}\vert<\!< \vert{\lambda_1}\vert$  or 
 $ \vert{\lambda_2}\vert<\!< \n\xi <\!< \vert{\lambda_1}\vert$ . Then  
$\vert{ \bigtriangledown\Psi(x)}\vert \ge C\vert{\lambda_1}\vert$ and so an integration by parts 
argument (see theorem (1) in the appendix) shows that
$\vert{F(\xi,\,\lambda_1,\,\lambda_2) }\vert\leq C(1+\vert{\lambda_1}\vert)^{-N} \quad \forall 
N>0 $.
Similarly, if $ \vert{\lambda_1}\vert<\!< \vert{\lambda_2}\vert<\!< \vert{\xi}\vert$, or
$ \vert{\lambda_1}\vert\approx \vert{\lambda_2}\vert<\!< \vert{\xi}\vert$, then 
$\vert{F(\xi,\,\lambda_1,\,\lambda_2) }\vert\leq C(1+\vert{\xi}\vert)^{-N} \quad \forall 
N>0 $.
\par
If we rewrite 
$F$ using polar coordinates with respect to $\Sigma_2$ and assume  that 
$\chi$ is radial with respect to $\Sigma_2igmsa$, we get 
$$
F(\xi,\,\lambda_1,\,\lambda_2) =
 \int_{0}^{+\infty}r^{n-1}\chi(r)
\int_{\Sigma_2}e^{i(r\langle \xi,\, \omega\rangle +r\lambda_1+r^m\lambda_2)}
\,d\sigma(\omega)\,dr,
$$
where $ d\sigma$ is the Lebesgue measure carried by $\Sigma_2$.
Let $I(\xi)$ denote the Fourier transform of the surface-carried measure on
$\Sigma_2$, 
$$
I(\xi) = \int_{\Sigma_2}
e^{i\langle \xi,\, \omega\rangle }
\,d\sigma(\omega) .
$$
Since the Gaussian curvature on $\Sigma_2$ never vanishes, we can use 
the method of stationary phase (see theorem (3) in the appendix ) 
to write 
$I(\xi)= b(\xi)e^{iq(\xi)}$,  where $\xi $ belongs to a cone $\Gamma$ containing 
the 
normal directions to $\Sigma_2$ on the support of $d\sigma$, and where 
$b(\xi) $ is a symbol of order 
$-\frac{n-1}{2}$, $q(\xi)$ is homogeneous of degree $1$, and 
$q(\xi)\approx\n\xi$.
Away from $\Gamma$, $I(\xi)$ decays rapidly in  $\n\xi$.
\par
Suppose that we are in one of the cases where $\n\xi$ dominates: 
\roster
\item $ \vert{\lambda_2}\vert<\!< \vert{\lambda_1}\vert\approx \vert\xi\vert$,
\item $ \vert{\lambda_1}\vert<\!< \vert{\lambda_2}\vert \approx\vert\xi\vert $,
\item $ \vert{\lambda_1}\vert<\!<\vert{\lambda_2}\vert <\!< \vert\xi\vert$,
\item $ \vert{\lambda_2}\vert<\!<\vert{\lambda_1}\vert <\!< \vert\xi\vert$,
\item $ \vert{\lambda_1}\vert\approx \vert{\lambda_2}\vert \approx\vert\xi\vert $.
\endroster
Using our observation about $I(\xi)$, we write 
$$
F(\xi,\,\lambda_1,\,\lambda_2) = 
 \int_{0}^{+\infty}r^{n-1}
e^{i( rq(\xi)+r\lambda_1+r^m\lambda_2)}b(r\xi)\chi(r)dr \ .
$$
Then
$$ 
\vert{F(\xi,\,\lambda_1,\,\lambda_2)}\vert\leq C\int_{0}^{2}
r^{n-1}\n{b(r\xi)}dr.
$$
Let $s=r\n\xi $, and define $ \tilde\xi =\xi\vert\xi\vert^{-1}$. 
The integral above is bounded by 
$$
C\vert\xi\vert^{-n}\int_{0}^{2\n\xi}
s^{n-1}\vert{b(s\tilde\xi)}\vert ds = 
$$
$$
= C\vert\xi\vert^{-n}\int_{0}^{N}
s^{n-1}\vert{b(s \tilde\xi)}\vert ds + C\vert\xi\vert^{-n}\int_{N}^{2\vert\xi\vert}
s^{n-1}\vert{b(s \tilde\xi)}\vert ds,
$$
where $N$ is large. The first integral is $O(\vert\xi\vert^{-n})$ and the second 
integral is bounded by 
$$ 
C\n\xi^{-n}\int_{N}^{2\n\xi}
s^{\frac{n-1}{2}}ds  \leq C(1+\n\xi)^{-\frac{n-1}{2}} .
$$ 
Note that $\dsize \frac{n-1}{2}\ge \frac{n}{m} $ \quad when 
$\dsize m\ge \frac{2n}{n-1}$.
\par
We are left to consider the  cases where $\lambda_2$ dominates: 
\roster
\item $ \n\xi\approx\vert{\lambda_1}\vert<\!< \vert{\lambda_2}\vert$,
\item $ \n\xi<\!<\vert{\lambda_1}\vert\approx\vert{\lambda_2}\vert$,
\item $ \n\xi<\!<\vert{\lambda_1}\vert<\!<\vert{\lambda_2}\vert$,
\item $ \vert{\lambda_1}\vert<\!<\n\xi<\!<\vert\lambda_2\vert$.
\endroster
\par
As before, let
$$
F(\xi,\,\lambda_1,\,\lambda_2) =
\int_{0}^{+\infty}r^{n-1}
e^{i( rq(\xi)+r\lambda_1+r^m\lambda_2)}b(r\xi)\chi(r)dr.
$$ 
Let $s\lambda_2^{-1/m}=r$. Then  
$$
F(\xi,\,\lambda_1,\,\lambda_2) = \lambda_2^{-\frac{n}{m}}
 \int_{0}^{+\infty}s^{n-1}
e^{i (q(s\lambda_2^{-1/m}\xi)+s\lambda_2^{-1/m}\lambda_1+s^m)}
b(s\lambda_2^{-1/m}\xi)\chi(s\lambda_2^{-1/m})ds .
$$
Let 
$$
G(\xi,\,\lambda_1,\,\lambda_2) = 
 \int_{0}^{+\infty}s^{n-1}
e^{i (\lambda_2^{-1/m}sq(\xi)+s\lambda_2^{-1/m}\lambda_1+s^m)}
b(s\lambda_2^{-1/m}\xi)\chi(s\lambda_2^{-1/m})ds .
$$
It suffices to show that $\vert{G(\xi,\,\lambda_1,\,\lambda_2)}\vert$ is uniformly
bounded. When 
 $\dsize \vert\frac{\lambda_1+\n\xi}{{\lambda_2}^{\frac{1}{m}}}\vert $ is sufficiently small, 
then $\n G$ is  bounded by $\dsize C\vert{\int_{0}^{+\infty}e^{i t^m}t^{n-1} dt}\vert$.
An integration by parts argument shows that this integral converges. 
In particular  the above integral
equals 
$\dsize e^{\frac{2 \pi i }{m} }\frac{1}{m}\Gamma\left(\frac{n}{m}\right)$. 
Thus we may assume that $\dsize \vert\frac{\lambda_1+\n\xi}{{\lambda_2}^{\frac{1}{m}}}\vert \ge C$.
\par
 Let $\dsize \Phi(s)= s\frac{\lambda_1+q(\xi)}{{\lambda_2}^{\frac{1}{m}} }+s^m$. Then $\dsize \Phi'(s) =0 $ if 
$\dsize s = C\left(\frac{\lambda_1+q(\xi)}{{\lambda_2}^{\frac{1}{m}} }\right)^{\frac{1}{m-1}}$, 
and  $\dsize \Phi''(s) = m(m-1)s^{m-2}$.
\par
If we apply the  van der  Corput Lemma 
(see theorem (2) in the appendix) in the case $k=2$, and  recall that 
in particular 
$\n b$ is uniformly bounded, we see that $\n G$ 
is bounded by 
$$
C\vert\frac{\lambda_1+\n\xi}{{\lambda_2}^{\frac{1}{m}}}^{-\frac{(m-2)}{2(m-1)}+
\frac{n-1}{m-1}}\vert.
$$
The power of \ $\dsize \vert\frac{\lambda_1+\n\xi}{{\lambda_2}^{\frac{1}{m}}}\vert$ \ in the expression above is non-positive if
$m\ge 2n$, and so 
$G(\xi,\,\lambda_1,\,\lambda_2)$ is uniformly
bounded. 
This completes the proof of part (a) of the Main Theorem.
\enddemo
%
\bigskip
\demo {Proof of part (b) of the Main Theorem} As before, we rewrite $F$ 
using polar coordinateds associated to $\Sigma_2$. We get 
$$
F(\xi,\,\lambda_1,\,\lambda_2) =\int_0^{+\infty}\int_{\Sigma_2}e^{i(r\langle \omega,\,\xi
\rangle +r\lambda_1\phi_1(\omega)+\lambda_2r^m)}r^{n-1}\chi(r)d\omega dr,
$$
where, as before, $\chi$ is a smooth cutoff function which is radial with 
respect to the polar coordinates associated to $ \Sigma_2$.
Let 
$$
I(\xi,\,\lambda_1) = \int_{\Sigma_2} e^{i(\langle \omega,\,\xi
\rangle +\lambda_1\phi_1(\omega))}d\omega.
$$
Using the implicit function theorem   we can parametrize $\Sigma_2$  near a point $s_0$
by a smooth function 
$\psi: \R^{n-1}\to \R$. 
Without loss of generality,  we can assume that  $\bigtriangledown
\phi_1(s_0) = 0$ and that $\bigtriangledown
\phi_2(s_0) = (0,\, 0,\, \cdots,\,0,\,1)$. Thus, we can  locally
write $\Sigma_2 =\{(\omega',\,\omega_n) \ :\  \omega_n = \psi(\omega')\}.$ 
The restriction 
of $M$ to the hyperplane $\{x_{n+2}=1\}$ can thus be locally parametrized
by the functions $\psi(\omega')$ and $ \phi_1(\omega',\,\psi(\omega'))$.
If we let $\xi = (\xi',\,\xi_n)$, we can  write 
$I(\xi,\,\lambda_1)$ as a finite sum of terms of the form
$$
 \int_{\R^{n-1}} e^{i(\langle \omega',\,\xi'
\rangle +\xi_n\psi(\omega')+ \lambda_1\phi_1(\omega',\,\psi(\omega')))}
\chi_1(\omega')d\omega',
\tag  (1.5) $$
where $\chi_1$ is a smooth cutoff function supported in a neighborhood of
$s_0$.
It was observed by M. Christ (see [C]) that  the strong curvature 
condition  (see the introduction) implies the following result.
\proclaim{Lemma} 
Let $\Omega$ be a submanifold of $\R^{N+2}$ of codimension $2$ locally
parametrized by smooth functions $g_1$ and $g_2$, where $g_j : \R^N\to \R$.
Let $d\sigma$ denote a smooth measure on $\Omega$. Suppose that $\Omega$
satisfies the strong curvature condition. Then 
$$
\n{{\Cal F}[d\sigma](R\eta) } \leq C(1+R)^{-\frac{N}{2}}.
$$
\endproclaim
The proof of the lemma shows that the integral in ( (1.5)) can be written 
as 
$\dsize  b(\xi,\,\lambda_1)e^{iq(\xi,\,\lambda_1)}$, 
where  $(\xi,\lambda_1) $ belongs to a cone  containing 
the normal directions to $M_{\vert_{\{x_{n+2}=1\}}}$ on the support of 
$d\sigma$, $b(\xi,\,\lambda_1) $
is a symbol of order 
$-\frac{n-1}{2}$, $q(\xi,\,\lambda_1)$ is homogeneous of degree $1$, and 
$\n{q(\xi,\,\lambda_1)}\approx(\n\xi+\n{\lambda_1})$.
\par
We must analyze the following integral:
$$ 
 \int_{0}^{+\infty}r^{n-1}
e^{i( rq(\xi,\,\lambda_1)+r^m\lambda_2)}b(r\xi, \,r\lambda_1)\chi(r)dr .
\tag (1. 6)
$$
We may assume that $\n{q(\xi,\,\lambda_1) } \leq C\n{\lambda_2}$, since if
$\n{q(\xi,\,\lambda_1) } \ge c\n{\lambda_2}$ for a sufficiently large 
$c >0$, then the integral in ( (1. 6)) decays rapidly in 
$\n\xi +\n{\lambda_1} $. (See theorem (1) in the appendix.)
\par
Let $s=r\lambda_2^{\frac{1}{m}} $. Then,  the integral in
 (1. 6) can be written as
$$
\lambda_2^{-\frac{n}{m}}\int_{0}^{+\infty}s^{n-1}
e^{i( s\lambda_2^{\frac{1}{m}}q(\xi,\,\lambda_1)+s^m)}
b(s\lambda_2^{\frac{1}{m}}\xi, \,s\lambda_2^{\frac{1}{m}}\lambda_1)
\chi(s\lambda_2^{\frac{1}{m}})dr .
$$
Let 
$$
G(\xi,\,\lambda_1,\,\lambda_2) = 
\int_{0}^{+\infty}s^{n-1}
e^{i( s\lambda_2^{\frac{1}{m}}q(\xi,\,\lambda_1)+s^m)}
b(s\lambda_2^{\frac{1}{m}}\xi, \,s\lambda_2^{\frac{1}{m}}\lambda_1)
\chi(s\lambda_2^{\frac{1}{m}})dr .
$$
As before, it suffices to show that $\n{G(\xi,\,\lambda_1,\,\lambda_2)}$ is 
uniformly bounded . When
 $\dsize \vert\frac{\n{\lambda_1}+\n{\xi}}{{\lambda_2}^{\frac{1}{m}}}\vert $ is sufficiently small, 
then $\n G$ is  bounded by $\dsize C\n{\int_{0}^{+\infty}e^{i t^m}t^{n-1} dt}$.
Hence we can assume  $\dsize \vert
\frac{\n{\lambda_1}+\n{\xi}}{{\lambda_2}^{\frac{1}{m}}}\vert \ge C$. We can write 
$\dsize G(\xi,\,\lambda_1,\,\lambda_2) = 
\int_{0}^{N}+\int_{N}^{C\n{\lambda_2}^{\frac{1}{m}}}$, $ N$ large. The first integral 
is uniformly bounded. In order to handle the second integral  let 
$\dsize \Phi(s) = s\lambda_2^{\frac{1}{m}}q(\xi,\,\lambda_1)+s^m$. 
Then $\dsize \Phi'(s) = 0$ if 
$\dsize s = 
c_m\left(\lambda_2^{-\frac{1}{m}}q(\xi,\,\lambda_1)\right)^{\frac{1}{m-1}}$, 
and 
$\dsize \Phi''(s) = m(m-1)s^{m-2}$. If the critical point is smaller than 
$N$ the integral has rapid decay, so we may assume that 
$\n{\lambda_2^{-\frac{1}{m}}q(\xi,\,\lambda_1)}$ is large. 
If we recall that $\dsize 
\n{q(\xi,\,\lambda_1)} \approx \n\xi+\n{\lambda_1}$, then by the van der 
Corput lemma (see theorem (2) in the appendix)  we get 
$$
\int_{N}^{C\n{\lambda_2}^{\frac{1}{m}} }\leq\n{ \frac{\n{\lambda_1}+\n{\xi}}{{\lambda_2}^{\frac{1}{m}}}}^{-\frac{(m-2)}{2(m-1)} +
\frac{n-1}{m-1}-\frac{n-1}{2(m-1)} -\frac{n-1}{2}}.
\tag (1. 7)
$$
Note that the third and the fourth terms in the power of $\dsize \n{\frac{\n{\lambda_1}+\n{\xi}}{{\lambda_2}^{\frac{1}{m}}}}$
arise from the fact that $b$ is a symbol of order $-\frac{n-1}{2}$, and 
$\dsize \vert\frac{\n{\lambda_1}+\n{\xi}}{{\lambda_2}^{\frac{1}{m}}}\vert$ is large.
\par
The power of $\dsize \vert\frac{\n{\lambda_1}+\n{\xi}}{{\lambda_2}^{\frac{1}{m}}}\vert$ in ( (1. 7)) is nonnegative provided that 
$m\ge 2$. Hence, $\vert{ G(\xi,\,\lambda_1,\,\lambda_2)}\vert$ is bounded and the 
proof is complete.
\enddemo
\bigskip

\centerline{\bf Appendix}
\bigskip
In this section we recall a few classical results that we used to prove the Main 
Theorem.
The first two theorems, which deal with  oscillatory integrals,  can be found e.g. 
in [St].
\proclaim
{\bf Theorem 1}\quad Suppose  $\phi \in {\Cal C}^{\infty}_0{\R^n}$ and suppose that
$\psi$ is a real-valued and smooth function which has no critical points on the 
support of $\phi$. Then 
$$
\n{\int_{\R^n} e^{i\lambda\psi(x)}\phi(x)dx} = O(\lambda^{-N}) 
$$
as $\lambda\to \infty$, for every $N \ge 0$.
\endproclaim
\medskip
\proclaim
{\bf Theorem 2}\quad Suppose that $\psi$ is  real-valued and smooth 
and that $\phi$ is complex-valued and smooth in $[a,\,b]$. If
$\n{\psi^{(k)}(x)}\ge 1$, then
$$
\n{\int^b_a e^{i\lambda\psi(x)}\phi(x)dx}\leq C_k\lambda^{-\frac{1}{k}}
\left[\n{\phi(b)}+\int^b_a\n{\phi'(t)}\,dt\right]
$$
holds when
\roster
\item $ k\ge 2$
\item or $k=1$, if in addition it is assumed that $\psi'(x)$ is monotonic.
\endroster
\endproclaim
\medskip
\proclaim
{\bf Theorem 3}\quad 
 Let $S$ be a smooth hypersurface in $\R^{n}$ 
with nonvanishing Gaussian curvature, and let 
$d\sigma $
be a ${\Cal C}^\infty$ measure  on $S$.
Then
$$
\n{\widehat{ d\mu}(\xi)} \leq C(1+\n\xi)^{-\frac {n-1}{2}} .
$$
Moreover suppose that $\Gamma\subset \R^n\backslash\{0\} $ is the cone consisting
of all $\xi$ which are normal to some point $x\in S$ belonging to
some compact neighborhood ${\Cal N}$ of support$(d\mu)$. Then,
$$
\frac{\partial^{\alpha}}{\partial{\xi}}\widehat{ d\mu}(\xi) = O\left((1+\n\xi)^{-N}\right), 
\qquad \forall N, \text{ if }
\xi\not\in\Gamma,
$$
$$
\widehat{ d\mu}(\xi) =\sum
= a_j(\xi)e^{i\langle x_j,\,\xi\rangle},\qquad \text{ if }\xi\in\Gamma,
$$
where the finite sum is taken over all points $x_j\in {\Cal N}$ having
$\xi$ as a normal and 
$$
\vert
\frac{\partial^{(\alpha)}}{\partial\xi}\widehat {d\mu}(\xi)\vert \leq C_\alpha(1+\n\xi)^
{-\frac {n-1}{2}-\n\alpha}.
$$
\endproclaim
\demo{Proof} See [So] pag. 50-51.
\enddemo
\bigskip
\centerline{\bf References}
\bigskip
\item{{\bf [C]}} M.  Christ,\quad {\it Restriction of the Fourier transform to 
submanifolds of low codimension}, Thesis, University of Chicago (1982). 
\par
\item{{\bf [DC]}} L. De Carli,\quad {\it Unique continuation for higher order
elliptic operators}, Thesis, UCLA (1993). 
\par
\item{{\bf [G]}} A. Greenleaf,\quad {\it Principal curvatures in harmonic analysis},
Ind. Univ. Math. J.  {\bf 30}  (1981), 519-537.
\par
\item{{\bf [H]}} L. Hormander,\quad {\it 
The analysis of linear partial differential operators I}, 
 New York, 1990.
\par
\item{{\bf [I1]}} A. Iosevich,\quad {\it Maximal Operators associated to 
families of flat curves and hypersurfaces }, Thesis, UCLA (1993). 
\par
\item{{\bf [I2]}} A. Iosevich,\quad {\it Maximal Operators associated to 
families of flat curves in the plane.} To appear in Duke Math.J.  
\par
\item{\bf [L]} W. Littman,\quad {\it Fourier transforms of surface-carried measures 
and differentiability of surface averages}, Bull. Am. Math. Soc.  {\bf 69} (1963), 766-770.
\par
\item{{\bf [So]}} C.D.Sogge,\quad 
{\it Fourier integrals in classical analysis}, 
Cambridge  University Press,  Cambridge, 1993.
\par
\item{{\bf [St]}} E. M. Stein,\quad 
{\it Harmonic Analysis},  
Princeton University Press, Princeton, 1993.

\end




