%Revised shortened version, September 16, 1999
\documentstyle{amsppt}
\pagewidth{6in}\vsize8.5in\parindent=6mm\parskip=3pt\baselineskip=14pt
\tolerance=10000\hbadness=500
%\magnification=1100
\NoRunningHeads
%\NoLogo
\loadbold
\topmatter
\title
On averaging operators associated  with 
convex hypersurfaces of finite type
\endtitle
\author
Alexander Iosevich \ \  Eric Sawyer \ \ Andreas Seeger
\endauthor
\thanks Research supported in part by  NSF grants.  \endthanks
\address
%Wright State University, Mathematics and Statistics Department,
%Dayton OH 45435
Georgetown University, Department of Mathematics, Washington, DC 20057
\endaddress
\email iosevich\@gumath1.math.georgetown.edu\endemail
\address 
Indiana University - Purdue University Indianapolis,
Department of Mathematical Sciences, Indianapolis, IN 46202
\endaddress
\email esawyer\@math.iupui.edu \endemail
\address 
University of Wisconsin-Madison, Department of
 Mathematics, Madison, WI 53706
\endaddress 
\email seeger\@math.wisc.edu \endemail
%\subjclass\endsubjclass
\endtopmatter
%\input defne

\def\leaderfill{\leaders\hbox to 1em{\hss.\hss}\hfill}

\def\a{{\alpha}}
\def\R{{\Bbb R}}
\def\SS{{\Cal S}}
\def\MM{{\Cal M}}
\def\NN{{\Cal N}}
\def\HH{{\Cal H}}
\def\EE{{\Cal E}}
\def\CC{{\Cal C}}
\def\ZZ{{\Cal Z}}
\def\Z{{\Bbb Z}}
\def \tSi{{\widetilde \Sigma}}
\def\thcr{{\theta_{\text{cr}}}}

\define\eg{{\it e.g. }}
\define\cf{{\it cf}}
\define\Rn{{\Bbb R^n}}
\define\Rd{{\Bbb R^d}}
\define\sgn{{\text{rm sign }}}
\define\rank{{\text{\rm rank }}}
\define\corank{{\text{\rm corank }}}
\define\coker{{\text{\rm Coker }}}
\redefine\ker{{\text{\rm Ker }}}
\define\loc{{\text{\rm loc}}}
\define\spec{{\text{\rm spec}}}
\define\comp{{\text{\rm comp}}}
\define\Coi{{C^\infty_0}}
\define\dist{{\text{\rm dist}}}
\define\diag{{\text{\rm diag}}}
\define\supp{{\text{\rm supp }}}
\define\rad{{\text{\rm rad}}}
\define\Lip{{\text{\rm Lip}}}
\define\inn#1#2{\langle#1,#2\rangle}        
\define\rta{\rightarrow}
\define\lta{\leftarrow}
\define\noi{\noindent}
\define\lcontr{\rfloor}
\define\lco#1#2{{#1}\lcontr{#2}}
\define\lcoi#1#2{\imath({#1}){#2}}
\define\rco#1#2{{#1}\rcontr{#2}}
\redefine\exp{{\text{\rm exp}}}
\define\bin#1#2{{\pmatrix {#1}\\{#2}\endpmatrix}}
\define\meas{{\text{\rm meas}}}

\define\card{\text{\rm card}}
\define\lc{\lesssim}
\define\gc{\gtrsim}
%Greek letters
\define\ga{\gamma}             \define\Ga{\Gamma}
\define\eps{\varepsilon}                
\define\ep{\epsilon}
\define\la{\lambda}             \define\La{\Lambda}
\define\sig{\sigma}             \define\Sig{\Sigma}
\define\si{\sigma}              \define\Si{\Sigma}
\define\vphi{\varphi}
\define\ome{\omega}             \define\Ome{\Omega}
\define\om{\omega}              \define\Om{\Omega}
\define\ka{\kappa}
\define\fa{{\frak a}}
\define\bbE{{\Bbb E}}
\define\bbR{{\Bbb R}}
\define\bbZ{{\Bbb Z}}
\define\cA{{\Cal A}}
\define\cB{{\Cal B}}
\define\cE{{\Cal E}}
\define\cF{{\Cal F}}
\define\cJ{{\Cal J}}
\define\cL{{\Cal L}}
\define\cM{{\Cal M}}
\define\cR{{\Cal R}}
\define\cS{{\Cal S}}
\define\cT{{\Cal T}}
\define\tchi{{\widetilde \chi}}
\define\tka{{\widetilde \kappa}}
\define\txi{{\widetilde \xi}}
\define\teta{{\widetilde \eta}}
\define\tomega{{\widetilde \omega}}
\define\tzeta{{\widetilde \zeta}}
\define\tpsi{{\widetilde \psi}}
\define\tphi{{\widetilde \phi}}

\document


%\subheading{1.  Introduction}
\head
{\bf 1.  Introduction}
\endhead

Let $\Sigma$ be a smooth  convex hypersurface 
in $\R^d$, $d\ge 3$, and 
denote surface measure  on $\Sigma$ by $d\sigma$.
Let $\chi$ be a compactly supported $C^\infty$ function
and let
$$\tSi= \supp \chi\cap \Sigma.$$
For $t>0$
define the convolution operator $\Cal A_t$ by
$$
\cA_t f(x)=\int f(x-ty')\chi(y')d\sigma(y')\tag 1.1
$$
and an associated maximal function
$$
\cM f(x) =\sup_{t>0}|\cA_t f(x)|. \tag 1.2
$$


The main issues in this paper are 
 the $L^p$ boundedness of the maximal operator
$\Cal M$ and the regularity properties of
the averaging operator $\cA\equiv \cA_1$.




Stein \cite{22}  showed that if $\Sigma$ is a $(d-1)$-dimensional sphere in 
$\Bbb R^{d}$, $d\ge 3$,
 then $\cM$ is bounded on $L^p(\Bbb R^d)$ for $p>d/(d-1)$ and unbounded 
for $p\le d/(d-1)$.
Greenleaf \cite{11}  proved similar results
under the  conditions on the decay of the Fourier transform 
$\widehat {d\sigma}$. In particular 
if  $\Sigma$ is a hypersurface  and
the Gaussian curvature of $\Sigma$ does not vanish, one obtains the 
same result as for the sphere.
 The two dimensional version of Stein's result was proved by Bourgain
\cite{1}.

If  the Gaussian curvature is allowed to vanish one would like to determine  
 the best possible value of $p_0$ such that $L^p$ boundedness holds 
for $p > p_0$. Cowling and Mauceri \cite{7} showed that there are surfaces 
where $p_0\in(2,\infty)$ and  Sogge and Stein \cite{21} showed that such 
$p_0<\infty$ exists if the Gaussian 
curvature is assumed to vanish of only finite order.
 The extension of Bourgain's result to plane curves of  finite type
was obtained 
in \cite{12} using  scaling; this method does not readily apply in 
higher dimensions.

In this paper we consider a {\it convex } surface $\Sigma$
of {\it finite line type} in $\Bbb R^d$,  $d\ge 3$,
{\it i.e.} it  is assumed that each tangent line has finite order of contact. 
Bruna, Nagel and Wainger \cite{2} expressed the decay of the Fourier transform
$\widehat{d\sigma}$ using the caps
$$
B(x,\delta)=\{y\in \Sigma: \dist(y,H_x(\Sigma))<\delta\};
$$
here  $H_x(\Sigma)$ denotes the tangent plane at $x\in \Sigma$ 
(considered as an affine subspace of $\Bbb R^d$ passing through $x$). The 
estimate is $$|\widehat {d\sigma}(\xi)|\le C \big[
|B(x_+,|\xi|^{-1})|+|B(x_-,|\xi|^{-1})| \big]
$$
where $x_\pm$ are the points on $\Sigma$ for which $\xi$ is a
normal vector and  $|B|$ denotes  the surface measure of $B$.
The behavior of the maximal operator $\cM$ is not just determined 
by the size of the balls of given height $\delta$, but also by the number 
of balls of height $\delta$  and fixed   diameter $\gg \delta^{(d-1)/2}$. 
Taking  this into account
 Nagel, Wainger and the third author \cite{18}  proved  maximal theorems 
on $\Bbb R^d$, $d\ge 3$,
using the quantity 
$$
\Gamma_r(\delta)=(\int_\tSi|B(x,\delta)|^{r-1}d\sigma(x))^{1/r} \tag 1.3
$$
for $r>1$.
Note that if $\sup_x |B(x,\delta)|=O( \delta^a)$ then 
$\Gamma_r(\delta)=O(\delta^{a(1-1/r)})$; however if $a<(d-1)/2$ then 
$\Gamma_r(\delta)$ tends to be significantly smaller.
The first theorem in \cite{18} addresses the case $p>2$. Suppose that
$$
\int_0^1
\delta^{-1/p}\Gamma_{\frac{p}{p-2}}(\delta)\frac {d\delta}{\delta}
\le A<\infty \quad \text{ and } \quad p>2
\tag 1.4
$$
then
$\cM$ is bounded on $L^p(\Bbb R^d)$.


Another  theorem was proved by the first two authors 
in \cite{14} and \cite{13}, completely settling the case  $p>2$.
Namely let 
$d(y,H_x(\Sigma))$ be the distance of $y\in \Sigma$ to the tangent plane 
$H_x(\Sigma)$ through $x$; then  
the maximal operator 
 $\cM$ is bounded on $L^p(\Bbb R^d)$, for $p>2$,
if $$d(\cdot, H_x(\Sigma))^{-1/p}\in L^1(\widetilde \Sigma)\tag 1.5$$
for every
$x\in \widetilde\Sigma$;
conversely,  the condition (1.5) at points with $\chi(x)\neq 0$
is  necessary  for $L^p$ boundedness.
In \S4 we shall use a variant of the argument in \cite{14}
to show that the sufficiency of (1.5) actually follows from
the sufficiency of (1.4). It follows a posteriori
that  for $p>2$ the $L^p$ boundedness of $\cM$, the finiteness of the integral
 (1.4) and
the condition  (1.5) are  equivalent
if $\Sigma$ is closed and $\chi\equiv 1$.


We remark that the hypothesis $(1.4)$ implies $L^p$ boundedness for a 
class of convex hypersurfaces, with the $L^p$ bounds depending only
 on $A$ and certain 
admissible constants (for the definition of admissibility see \S2).
 On the other hand, 
for a single convex body the assumption (1.5) 
is often easier to verify.





The analogue of (1.4) for $p<2$ is the condition
$$
\int_0^1
[\log(1+\delta^{-1})]^{\frac 1p-\frac 12}\delta^{-\frac 1 p}
\Gamma_{\frac{p}{2-p}}(\delta)\frac {d\delta}{\delta}
<\infty;
\tag 1.6
$$
if $p<2$ and (1.6) is satisfied then
$\cM$ is bounded on $L^p(\Bbb R^d)$. This statement is 
(implicitly) contained in \cite{18} ({\it cf.} Theorem 2.5 below).
Note that if the curvature does not vanish then
  $|B(x,\delta)|\approx \delta^{(d-1)/2}$ and 
$\Gamma_{\frac{p}{2-p}}(\delta)\approx\delta^{(d-1)(1-1/p)}$ so the integral
(1.6) converges if and only if $p>d/(d-1)$, which is Stein's maximal theorem.
The nonvanishing of the curvature is not necessary; as one can see by
checking (1.6) for surfaces of the form
$$x_d=-c+\sum_{i=1}^{d-1}|x_i|^{a_i},\qquad   2\le a_1\le \dots\le a_{d-1},
\tag 1.7
$$ where the $a_i$ are even integers. In this case $L^p$ boundedness holds for
$p>d/(d-1)$ if $a_i\le d$ for $i=1,\dots,d-1$. 
In \S2 a related 
result will be deduced from (1.6) in \S2; namely $L^p$ boundedness holds
if the Gaussian curvature belongs to $L^\gamma(\tSi)$ for all 
$\gamma<1/(d-2)$.



It is not presently known whether for $p<2$ the condition 
(1.6) always gives the
correct range of $L^p$ boundedness up to endpoints.
 Moreover it is not known
precisely how  (1.7) relates to the notions of type and multitype.
One purpose  of this paper is to prove some partial results in this direction
and obtain a  fairly complete picture in three dimensions.

\medskip 

In order to formulate our results we now
review the definitions of type and
 multitype. For convex hypersurfaces  in $\Bbb R^d$ 
a natural notion of multitype has been implicitly 
 introduced by Schulz \cite{20}. Various related and  more general 
notions of multitype 
had been previously formulated in complex analysis, see in particular 
Catlin's paper \cite{3}; 
later Yu \cite{26} has given a simple formulation of Catlin's 
multitype condition for convex domains in $\Bbb C^n$, building on the 
results in \cite{20}.

We  first consider a smooth real valued
 function $\Phi$ defined in a neighborhood of the origin in
an $n$-dimensional vector space
$\Bbb E_n$  so that $\Phi(0)=\nabla \Phi(0)=0$.
We say that a vector
 $v$ in $\Bbb E^n$  has contact of order $m+1$ if
$$
\Phi(sv)=O(s^{m+1})\quad \text {if }\quad  s\to 0.
$$
Let $$S^m=\{v\in \Bbb E^n: \text{ $v$ has contact of order $m+1$.}\}
\tag 1.8$$
It is shown in \cite{20} that $S_m$ is a linear subspace of 
$\Bbb E^n$ and that   there are   even integers $m_1,\dots, m_k$ so that
$m_1<\dots<m_k$, $1\le k\le n$ and $m_0:= m_1-1\ge 1$ and
$$
0=S^{m_k}\subsetneq\dots \subsetneq S^{m_0}:=\Bbb E^{n};
$$
and the sequence is maximal, {\it i.e.}
$$S^m=S^{m_{k}}\text {  if } m_{k-1}< m\le  m_k.$$


The largest number $m_k$ is  the {\it type } of $\Phi$ at $0$.
Let $\dim S^{m_i}=n_i$, so that $n_0=n$ and $n_{k}= 0$.
For $i=1,\dots, n$ let
$$a_i=m_j\quad\text{ if }\quad
n-n_{j-1}<i\le n-n_j, \quad j=1,\dots,k;
$$
the  $n$-tuple $\fa=(a_1,\dots, a_n)$ is then called the {\it multitype}
of $\Phi$ at $0$. 
Clearly this definition is independent of  the {\it linear}
 coordinate system on $\Bbb E_n$.



Now let $\Sigma$ be a convex hypersurface in $\Bbb R^d$ and let $P\in \Sigma$.
Then near $P$ the surface is a graph over its tangent plane at $P$.
For a suitable choice of the 
 unit  normal vector $n_P$ at $P$ the surface can be parametrized
by
$$
\aligned
&T_P\Sigma\to \Bbb R
\\
&v\mapsto P+v+ \Phi(v) n_P
\endaligned
\tag 1.9
$$
where $\Phi$ is a convex function  vanishing of second order at the origin.
We say that $\Sigma$ is of multitype $\fa=(a_1,\dots, a_{d-1})$ at $P$
if $\Phi$
has multitype $\fa$ at the origin. This notion is invariant under
affine transformations in $\Bbb R^d$. Moreover, if $\Sigma$ is given as 
a graph $w_n=\Psi(w')$ then it is easy to see that
the multitype at $P=(w',\Psi(w'))$  is equal to the multitype of the function
$$y'\mapsto \Psi(w'+y')-\Psi(w')-\inn{y'}{\nabla_{w'}\Psi(w')}.$$



Calculations in \cite{18} on examples of the form (1.7)
suggest the following

\noi{\bf Conjecture:} Let $\Sigma$ be a convex surface in $\Bbb R^{d-1}$,
 let
$P\in \Sigma$ and let $\fa=(a_1,\dots, a_{d-1})$ be the multitype at $P$.
 Define $\nu_k$ by
$$
\nu_k=\sum_{j=k}^{d-1}\frac 1{a_j},\quad k=1,\dots,d-1; \qquad \nu_d=0.
%\aligned&\nu_k=\sum_{j=k}^{d-1}\frac 1{a_j},\quad k=1,\dots,d-1,\\
%&\nu_{d}=0\endaligned
\tag 1.10
$$ 
We conjecture that
$\cM$ is bounded on $L^p(\Bbb R^d)$ if the support of $\chi$ 
is contained in a sufficiently small neighborhood of $P$ and 
if
$$p>\max_{k=1,\dots,d} \frac{k}{k-1+\nu_k}.
\tag 1.11$$


Note that among the numbers (1.11) only the one corresponding to
$k=1$ can be  $\ge 2$ so that the condition for 
$L^p$ boundedness for $p>2$ reduces to  
$$p>\sum_{j=1}^{d-1}\frac 1{a_i}.\tag 1.12$$
As observed in \cite{14}
the condition (1.12) is equivalent to the integrability condition
(1.5) for $x=P$ so that the equivalence of (1.5) and (1.4)
mentioned above amounts to the equivalence of (1.12) and (1.4).
More generally,   one may also  conjecture that $L^p$ boundedness holds
if for every $l\in\{0,1,\dots, d-1\}$ and for every $l$-plane $E$ through
$P$ the function $x\mapsto [\dist (x,E)]^{-1}$ belongs to $L^{(d-l)/p}$
(here the $0$-plane through $P$ is just $\{P\}$). 

In the present paper we shall concentrate on the simplest
case, $d=3$.


\proclaim{Theorem 1.1}
Let $\Sigma$ be a smooth  convex hypersurface of finite line
type in $\R^3$. Let $P\in \Sigma$, let $\fa=(a_1, a_2)$
be the multitype at $P$ and let $K(x)$ be the Gaussian curvature at $x$.

Let  $\cM$ be the  maximal operator 
as defined in (1.2). There is a neighborhood $U$ of $P$ in $\Sigma$ 
so that the 
following statements hold if $\chi$ is supported in $U$.


(i) Suppose that $a_1>2$.
 Then $\cM$ is bounded 
if and only if $p>(\frac 1{a_1}+\frac 1{a_2})^{-1}$.

(ii) Suppose that $a_1=2$, $0<\gamma<1$ and $K^{-\gamma}\in L^1(U)$.
Then $\cM$ is bounded  if 
$p> \frac{2a_2(1-\gamma)+2+4\gamma}{a_2(1-\gamma)+2+2\gamma}$.

(iii) If $a_1=2$ 
then  $\cM$ is bounded for $p>\max\{\frac 3 2,
\frac {2a_2}{a_2+1}\}$.
\endproclaim


We note that (i) is already contained in \cite{14}, but we shall 
give a different proof in \S4 by deducing it from (1.4).
Also note that (i) and (iii) together verify the 
above conjecture in three dimensions; 
however there are cases where (ii) gives a better result (see \S4).
Statement (iii) follows from statement (ii) by using

\proclaim{Theorem 1.2}
Let $\Sigma$ be a smooth  convex hypersurface of finite line
type $\le m$ in $\Bbb R^3$, and let $K$ be the Gaussian curvature function on $\Sigma$.
If $\gamma<(m-2)^{-1}$ then  $K^{-\gamma}$ is locally integrable on $\Sigma$.
\endproclaim

We now discuss
the  regularity properties of the averaging 
operator $\cA=\cA_1$. A positive 
and apparently quite precise result for Besov spaces 
\footnote{\rm Recall that
$\|f\|_{B^p_{\beta,r}}\approx 
(\sum_{k=0}^\infty[2^{k\beta}\|\cL^k f\|_p]^r)^{1/r}$ with
 suitable Littlewood-Paley cutoffs $\cL^k$ localizing frequencies to annuli
$|\xi|\approx 2^k$ if $k>0$.} 
$B^p_{\alpha,q}$ and Sobolev spaces $L^p_\alpha$
 can be 
formulated in terms of the balls $B(x,\delta)$, using a condition similar
to (1.4), (1.6). 


\proclaim{Theorem 1.3}
 Suppose  $\Sigma\subset \Bbb R^d$ 
is convex, smooth and  of finite line 
type.
Suppose that $1\le p\le 2$ and suppose that 
$$\sup_{\delta>0} \delta^{\frac 1q-\frac 1p-\alpha}
\Big(\int_\tSi|B(x,\delta)|^{\frac{2q(p-1)}{p+q-pq}}d\sigma(x)
\Big)^{\frac 1p+\frac 1q-1}<\infty
\tag 1.13$$
holds for some $(p,q)$ with $p\le q$.
Then $\cA$ maps the Besov space $B^p_{\beta,r}$ boundedly to 
$B^q_{\beta+\alpha,r}$.

Moreover, if $1<p\le 2$, $p\le q<\infty$,  then $\cA$ is bounded 
from $L^p(\Bbb R^d)$ to $L^q_\alpha(\Bbb R^d)$ if $q\ge 2$ and 
bounded 
from $L^p(\Bbb R^d)$ to $L^{q-\eps}_\alpha(\Bbb R^d)$ if $p\le q\le 2$.
\endproclaim

Clearly the second  assertion about  Sobolev estimates is a consequence of 
the first assertion for Besov spaces, by standard embedding theorems 
({\it i.e.} Littlewood-Paley inequalities).


Again one can try to  relate the condition (1.13) to the multitype.
Consider the model example (1.7)
%$x_d=\sum_{i=1}^{d-1}|x_i|^{a_i}$ 
where $a_1\le  \dots\le a_{d-1}$ are even integers, $\nu_k$ as in (1.10).
% $\nu_d=0$ and 
%$\nu_k=\sum_{j=k}^{d-1}\frac 1{a_j}, k=1,\dots,d-1$.
We note that for this example  a complete
description of the  $L^p\to L^q$ estimates for $\cA$ has been given by
Ferreyra, Godoy and Urciuolo \cite{10} (without the restriction that the 
$a_i$ are even integers), see also the paper 
by Sang Hyuk Lee \cite{16}. Both proofs relied on a method introduced by
 Christ \cite{5}.

A calculation for the model example  shows that   (1.13) is satisfied when
$$
\alpha\le \min_{1\le k\le d}\big[ \nu_k+k-1-\frac{k+\nu_k}{p}+
\frac{\nu_k+1}q],
\tag 1.14
$$
see \S3.
For $\alpha=0$ this becomes $\frac 1q\ge \frac{\nu_k+k}{\nu_k+1}
\frac 1p-\frac{\nu_k+k-1}{\nu_k+1};$
this is the condition given in \cite{10}.
Concerning the case $p=q$ one obtains (for the model example) that
 $\cA$ is bounded from
$B^p_{\beta,r}$ to $ B^p_{\beta+\alpha,r}$ 
and from $B^{p'}_{\beta,r}$ to $ B^{p'}_{\beta+\alpha,r}$ provided
that 
$\alpha\le \nu_{k+1}+k/p$, if $a_k\le p\le a_{k+1}$. 



To formulate a conjecture for $L^p_\beta\to L^{q}_{\beta+\alpha}$
regularity (or related Besov-type estimates) in the general case
 one simply replaces 
$(a_1,\dots,a_{d-1})$ in the model example by the multitype at $P$ and
 assumes that $\chi$ has small support near $P$. Then (1.14)
 should imply the 
$L^p\to L^{q}_\alpha$ for the averaging operator, if $p<q$, and $1<p\le 2$.
Clearly by duality the boundedness region is symmetric with respect to 
the diagonal
$1/p+1/q=1$, so it suffices to consider the case $p\le 2$.
One expects that at least
 for the case $p=q$ boundedness may fail at the vertices
of the boundedness region, see \cite{6} 
for counterexamples 
in two dimensions.
We note that complete $L^p\to L^q$ results
in two dimensions are  in
 \cite{19}, \cite{5}. 



In three dimensions we prove the conjecture up to certain endpoint results.

\proclaim{Theorem 1.4}
Let $\Sigma$ be a smooth compact convex hypersurface of finite line
type in $\R^3$, let $P\in \Sigma$ and let
$\fa=(a_1, a_2)$ be the multitype of $\Sigma$ at $P$.
Let  $\nu_1=a_1^{-1}+a_2^{-1}$, $\nu_2= a_2^{-1}$ and let
$\cT(P)$ be
the set of all $(\frac 1p, \frac 1q,\alpha)$ with $p\le q$ satisfying the conditions
$$
\align
\alpha&\le \nu_1-\frac {1+\nu_1}p+\frac{1+\nu_1}q
\tag 1.15.1
\\
\alpha &<\nu_2+1-\frac {2+\nu_2}p+\frac{1+\nu_2}q
\tag 1.15.2
\\
\alpha &\le 2-\frac 3p+\frac 1q
\tag 1.15.3
\endalign
$$
and
$$
\align
\alpha&\le  \nu_1+\frac{1+\nu_1}q-\frac {1+\nu_1}p
\tag 1.16.1
\\
\alpha &<\nu_2+\frac{2+\nu_2}q-\frac {1+\nu_2}p
\tag 1.16.2
\\
\alpha &\le\frac 3q-\frac 1p
\tag 1.16.3
\endalign
$$
Then there is a neighborhood $U$ of $P$ such that 
$\cA$ is bounded from $B^p_{\beta,r}(\Bbb R^3)$ to 
$B^q_{\beta+\alpha,r}(\Bbb R^3)$ if
$\supp \chi\in U$ and 
$(1/p,1/q,\alpha)$ belongs to $\cT(P)$.

Moreover 
$\cA$ is bounded from $L^p_{\beta}(\Bbb R^3)$ to 
$L^q_{\beta+\alpha}(\Bbb R^3)$ if 
   $(1/p,1/q,\alpha)$ 
belongs to the interior of $\cT(P)$.
\endproclaim


\remark{\bf Remark 1.5}
If $p\le 2\le q$ and
the  $B^p_{0,r}\to B^{q}_{\alpha,r}$  estimate holds for a given $p$, $q$ with
$p\le 2\le q$ then the $L^p_\beta\to L^q_{\alpha+\beta}$ estimate follows;
this yields partial endpoint results for the Sobolev estimates.
\endremark

\remark{\bf Remark 1.6}
(i) A natural conjecture for $L^p\to L^q$ estimates
is given in terms 
of distances to tangent lines and planes.
Let $\Delta_j(p,q)=1/p-j/q$ and 
$\rho_l(d,p,q)=d-l-1+ \Delta_{d-l}/(1-\Delta_1)$.
Suppose that for  $l=0,1,\dots, d-1$ and for all $l$-planes $E$ through
$P$ the functions $x\mapsto [\dist (x,E)]^{-1}$ belong to 
$L^{\rho}(\Sigma)$ for $\rho=\rho_l(d,p,q)$. One may conjecture that
$\cA$ maps $L^p(\Bbb R^d)$ to $L^q(\Bbb R^d)$ (provided, of course, that
$\chi$ is supported in a sufficiently small neighborhood of $P$). 

If $d=3$ then the description of multitype together with estimates in $\S3$
can be used to show that the above conditions are equivalent with the 
conditions given in Theorem 1.4.


(ii) It is easy
to see that the condition for $l=d-1$
in (i) is necessary, by testing $\cA$ 
on characteristic functions of cylinders 
with base $B(P,\delta)$ and height $\delta$. 

(iii) Analogously, one can formulate a conjecture for 
 the $L^p$ boundedness of the maximal operator in terms of  
distances to tangent planes and lines. The conjecture is that ${\Cal M}$ 
maps $L^p({\Bbb R}^d)$ to $L^p({\Bbb R}^d)$ if for
$l=0, \dots, d-1$ and for all $l$-planes $E$ through $P$ 
the functions $x\mapsto [\dist(x,E)]^{-1}$ belong to
$L^{\frac{d-l}{p}}(\Sigma)$. 

\endremark



The paper is organized as follows. In \S2 we shall derive estimates for operators associated 
to certain classes of convex functions, 
emphasizing  uniformity of
 these estimates. In \S3 we shall discuss various properties of the multitype 
and the associated scaling; in particular we prove  versions of Theorem 1.2.
The proofs of Theorems 1.1 and 1.4 are contained in \S4, and some examples 
are considered in \S5.


%\subheading{2. Operators associated to convex functions of finite line type}
\head{\bf 2. Operators associated to convex functions of finite line type}
\endhead

In this section we collect facts which are either immediate consequences
of estimates
for classes of  convex functions 
of finite type in  \cite{2}, \cite{9} or \cite{18}, or can be 
obtained by modifications of arguments in those papers.

 
Let $B_{T}\subset \Bbb R^{n}$  denote the open ball  
of radius $T$ centered at $0$. In what follows it is always assumed that 
$T\le 1$.
For $0<b\le M$,  $N\in\Bbb Z^+$, $2\le m< N$,  let
 $\Cal S_T^n(b,M, m,N)$ be the class of all $C_N(\overline{B_T})$ functions $g$
 with the property
that for all $x\in B_T$
$$
\aligned
&g(0)=\nabla g(0)=0
\\
&\frac{d^2}{(dt)^2} g(x+t\theta)\big|_{t=0}\ge 0 \text{ for all }
\theta\in S^{n} 
\\
%&\max_{x\in B_T}
&\max_{2\le j\le m} \Big|\big(\frac {d}{dt}\big)^j g(x+t\theta)
\big|_{t=0}\Big|\ge 
b \text{ for all }
\theta\in S^{n} 
\\
%&\max_{x\in B_T}
& \max_{|\alpha|\le N} \Big|\big(\frac{\partial}{\partial x}
\big)^\alpha g(x)\Big|\le M
\endaligned
\tag 2.1
$$


Next let $\fa=(a_1, a_2,\dots, a_n)$ an $n$-tuple with even integers
so that
$2\le a_1\le \dots\le a_n$.
We define
 $\cS_T^n(b,M, \fa,N)$ to
 be the class of all functions in
 $\Cal S_T^n(b,M, a_n,N)$ with the property that
$$
\max_{2\le j\le a_i} \Big|\big(\frac {\partial}{\partial x_i}\big)^j g(x)
\Big|\ge b .
\tag 2.2
$$
We also set
$$\nu_k=\sum_{j=k}^{n}\frac 1{a_j}, \quad k=1,\dots, n,
\quad\text{ and }\qquad\nu_{n+1}=0.
\tag 2.3
$$
We note that if $\Sigma$ is convex and of finite line type
and if $P\in \Sigma$ is of multitype $\fa$ then 
there is a neighborhood of $P$ in $\Sigma$ where 
$\Sigma$ can be parametrized by (1.9) and so that $\Phi\circ L
\in\cS^{d-1}_T(b,M,\fa,N)$  for a  rotation $L$ and suitable constants
$T,b,M$.

Constants in estimates which will depend only on the parameters $n$, $b$, $M$,
$m$ or $\fa$, $N$  are called {\it admissible}.
 All constants in this 
section will be admissible, but statements involving the multitype
 in \S3 and \S4 below  will contain ``nonadmissible''  constants.

Notice that if $\Phi\in \cS^n_{2T}(b, M,m,N)$  the functions
$$
w\mapsto\Phi(y+w)-\Phi(y)-\inn{w}{\nabla\Phi(y)}
$$
belong to the class $\cS^n_{T}(b, 3M,m,N)$ for all $|y|\le T$. A 
similar remark applies to the class
$\cS^n_{2T}(b, M,\fa,N)$.

We now recall an important inequality from \cite{2} (see also  variants 
in \cite{9}, \cite{18}).
Let $|w|\le T$ and let
$$\align
P_{w,y}(s)&=\sum_{j=2}^m \frac 1{j!}{\inn{w}{\nabla}}^j\Phi(y) 
\frac{s^j}{j!}+M\frac {s^{m+1}}{(m+1)!}
\\
\widetilde P_{w,y}(s)&=\sum_{j=2}^m \frac 1{j!}|{\inn{w}{\nabla}}^j\Phi(y)| 
\frac{s^j}{j!}+M\frac {s^{m+1}}{(m+1)!}
\endalign
$$
Then there exists an admissible constants $C_1$, so that for $|y|\le T$, 
$|w|\le T$,
$0\le s\le 1$,
$$
C_1^{-1}\widetilde P_{w,y}(s)\le
\Phi(y+sw)-\Phi(y)-\inn{w}{\nabla\Phi(y)}\le C_1 P_{w,y}(s).
\tag 2.4
$$

Notice that  by (2.4) there exists an 
admissible constant $c_0>0$ so that for all
$$
\delta\le c_0 T^{m}=:\delta_0
\tag 2.5
$$
the sets
$$\cB(x,\delta)=
\{y: \ |y|\le T; \ |\Phi(y)-\Phi(x)-\inn{\nabla\Phi(x)}{y-x}|
\le \delta\}
\tag 2.6
$$
are contained in $\{|x|\le 2T\}$.
If $\Sigma=\text{graph}(\Phi)$ then 
these sets are comparable
 to projections of the  balls  $B(y,\delta)$
defined in the introduction.
% The constant $c$ will  be chosen smaller 
%than necessary for the previous statement in order to be able to prove the
%following Proposition.



\proclaim{Proposition 2.1}
 Let $\Phi\in \cS^{d-1}_{2^n T}(b, M,\fa,N)$, $m=a_n$, $N\ge m+1$.
There are admissible  constants $C_1,\dots, C_5$, 
$\mu_1\ge 1$, $C_0> c_0^{-1}$, 
so that the following statements hold.

(i) Let $1\le l\le n$ and let  $E$ be an $l$-plane through the origin. Let 
$\delta\le C_0^{-1} T^m$, $\mu_1\le \mu\le C_1^{-1}\delta^{-1/m}$.
Then  for all $|w|\le T$ the set 
$\cB(w, \mu\delta)$ is contained in
$\{|w|\le 2T\}$. Moreover
if  $V_E(x,w_0,\mu\delta)$
 is the $l$-dimensional volume of the cross sections
$(x+E)\cap \cB(w_0, \mu\delta)$, then
for $w_1,w_2\in \cB(w_0,\delta)$ one has
$$C_2^{-1}\Big(\frac{\mu}{\mu_1}\Big)^{l/m} V_E(w_1,w_0,\mu_1\delta)\le
 V_E(w_1,w_0,\mu\delta)
\le C_3
V_E(w_2,w_0,\mu\delta)\le C_2C_3 
\Big(\frac{\mu}{\mu_1}\Big)^{l/2} V_E(w_2,w_0\mu_1\delta).
\tag 2.7
$$

(ii) 
Let $\delta\le C_0^{-1} T^m$,
and let
$\cB(x,\delta)$ be as in (2.6),  $\nu_k$ as in (2.3).
Then  for $|x|\le T$,  
 $$
|\cB(x,\delta)|\le C_4 \delta^{\nu_1}.
\tag 2.8
$$

(iii)  
For $k=1,\dots, n$ let 
$$K_k(x)=\det\pmatrix 
&\Phi_{x_1x_1}&\dots&\Phi_{x_1x_k}
\\&\vdots &{} &\vdots
\\
&\Phi_{x_kx_1}&\dots&\Phi_{x_kx_k}
\endpmatrix.
%\Phi_{x_ix_j}(x')
%\endpmatrix_{\Sb i=1,\dots, k\\ j=1,\dots, k\endSb}.
\tag 2.9
$$
Then for $\delta\le C_0^{-1}T^m$
$$
|\cB(x,\delta)|\le 
C_5 \delta^{\frac k2+\nu_{k+1}}
\big[\sup_{y\in \cB(x,\delta)}
|K_k(y)|\big]^{-1/2}. 
\tag 2.10
$$
\endproclaim


\demo{Proof}
%The estimates follow from inequalities  for families of normalized
%convex functions of finite type 
The chain of inequalities (2.7) is an easy consequence 
 of \cite{18, Corollary  2.6} which  in  turn was based on (2.4).
Inequality (2.8) is proved by induction over the dimension.
It is true for $n=1$ by (2.4).
Let $n> 1$.
Then again by (2.4) one sees that the set
$$\cJ(\delta)=\{x_n: \text{ there is } 
x'\in \bbR^{n-1} \text{ so that } (x',x_n)\in 
\cB(x,\delta)\}
\tag 2.11
$$
is contained in an interval of length $\le C\delta^{1/a_n}$.
The functions $y'\mapsto \Phi(y,y_n)$ belong to
$\cS^{n-1}_{2^{n-1}T}(b, M,\fa',N)$, with $\fa'=(a_1,\dots, a_{n-1})$.
 By  the induction hypothesis 
the $n-1$-dimensional slices through $\cB(x,\delta)$  at height $y_n\in I$ 
have volume $\le C\delta^{1/a_1+\dots1/a_{n-1}}$. 
The assertion follows by integrating over $\cJ(\delta)$.

We now turn to the estimate (2.10), and consider first the case $k=n$.
 In \cite{9} it is shown  for arbitrary polynomials 
of degree
$\le q+1$ that 
$$\max_{|u|\le n}|\det P''(u)|\le C_{n,q}\max_{|u|\le 1}|P(u)|^n
\tag 2.12
$$
where $C_{n,q}$ is an absolute constant.
Now by estimates for functions in $\cS^n_{2T}(b, M,m,N)$ 
(\cite{2, \S3})
there are constants $c_0$, $C_0$ 
 and a polynomial $P_{\delta,x}$ of degree $\le m$, vanishing of second order
at $x$, so that
$$
\{y: P_{\delta,x}(y)\le c_0\delta\}
\subset
\cB(x,\delta)
\subset
\{y: P_{\delta,x}(y)\le C_0\delta\};
\tag 2.13
$$
here 
the constants $c_0$, $C_0$ do not depend on $x$ and $\delta$.
Following \cite{9} we apply a result of  John to wit
 there is a translation $\tau_{-x}$
and a  symmetric positive definite  linear
transformation $T$ so that $ B(1)\subset  T(\tau_{-x} \cB(x,\delta)\subset
B(n)$ where $B(1)$ and $B(n)$ denote the balls of radii $1$ and 
$n$, centered at the origin. 
By (2.13)
$$
\det T^{-1} \max_{|u|\le n}|\det 
P_{\delta,x}''(x_0+T^{-1}u)|^{1/2}
\le C_{n,m}\max_{|u|\le 1} |P_{\delta,x}(x_0+T^{-1}u)|^{n/2}
$$
and since  $\det T^{-1}$ is comparable with the 
measure of $\cB(x,\delta)$ the assertion follows for $k=n$.

To show (2.10) we argue by induction on $n$, the case $k=n$ is already 
taken care of. Let 
$n>k$. Pick $z\in \cB(x,\delta)$ so that
$K_k(z) \le 2 \min_{y\in B(x,\delta)} K_k(y)$.
Let $V_x(y_n,\delta)$ be the $n-1$ dimensional
 slice of $\cB(x,\delta)$ at height $y_n$.
Then
by the induction hypothesis 
$$
\align
|V_x(z_n,\mu_1\delta)|&\le 
C\delta^{k/2+\sum_{i=k+1}^{n-1}a_i^{-1}}
\big[\max_{z':
(z',z_n)\in \cB(x,\mu_1\delta)} K_k((z',z_n)\big]^{-1/2}
\\&\le
C\delta^{k/2+\sum_{i=k+1}^{n-1}a_i^{-1}}
\big[\max_{z:
z\in \cB(x,\delta)} K_k(z)\big]^{-1/2};
\endalign
$$
in this formula the sum in the exponent is not present when $k=n-1$. By (2.7) 
$$V_x(y_n,\delta)\le V_x(y_n,\mu_1\delta)\le C V_x(z_n,\mu_1\delta)$$
and integrating over
$y_n \in J(\delta)$ yields another factor of $\delta^{1/a_n}$, as in  the 
proof of (2.8). \qed
\enddemo

        



We  now let $n=d-1$ and
 consider the regularity properties
of the following integral operator acting on functions in $\Bbb R^d$,
$$
\cA_t f(x)=\int f(x'-y', x_{d}-t(\Phi(y')+c_d)) \chi(y')dy'.
\tag 2.14
$$
Here
$\Phi\in \Cal S_{r}^{d-1}(b,M, m,N)$,  and the smooth cutoff function
$\chi$ is supported in $\{x':|x'|\le T\}$, $T<2^{-d+1} r$. 
We shall not try to minimize smoothness and  therefore always assume 
that $N$ is large; by ``large'' we mean  $N\ge 10 d m$, which 
is  assumed in the remainder of this section.


Our first result is an estimate for $\cA_1$ after a
 localization in frequency space. Let $\delta>0$ be small, and let 
$\beta\in C^\infty_0({\Bbb R^d})$ be supported in $\{\xi: 1/2<|\xi|\le 2\}$.
Define $\cL_\delta$ by
$$\widehat {\cL_\delta f}(\xi)=\beta(\delta\xi) \widehat f(\xi).
\tag 2.15
$$

\proclaim{Proposition 2.2}


Suppose that $1\le p\le 2$ and
$1/r= 1/p+1/q-1$. Then
$$\|\cL_\delta \cA_1 f\|_q\le C
\delta^{\frac 1q-\frac 1p}
\Big(\int_{|w|\le T}|\cB(w,\delta)|^{r(1-\frac 1p+\frac 1q)-1}
dw\Big)^{1/r}
\|f\|_p 
\tag 2.16
$$
for all $f\in L^p(\Bbb R^d)$.
\endproclaim

\demo{Proof}
The proof follows  a pattern 
of \cite{18}  and we shall  be brief. Observe $\cA_1 f=d\mu*f$
where $d\mu$ is a smooth density on
$\Sigma$. We split $d\mu=\sum_j d\mu_j$ 
 where each $d\mu_j$ is supported in a cap $\cB_j$
 of height $\approx \delta$ and the caps (or ``balls'') have finite overlap.
This splitting is done by using a partition of unity subordinated 
to  the  $\cB_j$, see \cite{2} for the metric properties of the caps and
\cite{18} for the necessary quantitative bounds for the partition of unity.

For  sequences $\gamma =\{\gamma_j\}$ consider the bilinear operator
$$
T_\delta[\gamma,f]=\sum_j \gamma_j L_\delta[d\mu_j*f].
$$
The inequality (2.16) follows by choosing $\gamma=(1,1,1,\dots)$ from
from the following more general  estimate, valid for $p\le 2$:
$$
\|T_\delta[\gamma,f]\|_q\le C\delta^{\frac 1q-\frac 1p}
\Big(\sum_j [|\gamma_j|\,
|B_j|^{1-\frac 1p+\frac 1q}]^{r}
\Big)^{1/r}
\|f\|_p, \qquad\frac 1r=\frac 1p+\frac 1q-1.
\tag 2.17
$$
Indeed (2.17) is clear  for $p=1=q$, and also for $p=1$, $q=\infty$ 
(where $r=\infty$).
The nontrivial part is the case  $p=2=q$ (again then $r=\infty$); but this 
estimate  is a consequence of Theorem 2.2 in \cite{18}.
The general case follows by interpolation.\qed
\enddemo

The next result is an immediate consequence, and  also proves Theorem 1.3.
\proclaim{Corollary 2.3}
Suppose that $1\le p\le 2$ and suppose that 
$$\sup_{\delta>0} \delta^{\frac 1q-\frac 1p-\alpha}
\Big(\int_{\{|w|\le T\}}|\cB(w,\delta)|^{\frac{2q(p-1)}{p+q-pq}}dw
\Big)^{\frac 1p+\frac 1q-1}\le A<\infty
\tag 2.18$$
holds for some $(p,q)$ with $p\le q$.
Then $\cA$ maps the Besov space $B^p_{\beta,r}(\Bbb R^d)$ boundedly to 
$B^q_{\beta+\alpha,r}(\Bbb R^d)$
\endproclaim

\remark{\bf Remark 2.4} 
For the model example (1.7), {\it i.e.}
$x_d=c_d-\sum_{j=1}^{d-1}|x_j|^{a_j}$ one has
$$
\int|\cB(w,\delta)|^\eta dw
\le C \max_{1\le k\le d} \delta^{(1+\eta)\nu_k+\frac{(k-1)\eta}2};
$$
see  \cite{18, formula (5.2)}. From this  the sharp estimates 
for the maximal 
operator have been deduced in \cite{18}; moreover
Corollary 2.3 implies that
the averaging operator maps 
 $B^p_{\beta,r}$ to 
$B^q_{\beta+\alpha,r}$ if  $p\le 2$ and (1.14) is satisfied.
Concerning $L^p\to L^q$ estimates this  gives an extension to some 
endpoints of estimates in \cite{}, namely in the cases where
 $p\le 2\le q$.
\endremark

In order to prove the maximal Theorem 1.1 we shall rely on the following
 result implicitly in \cite{18}.

\proclaim{Theorem 2.5}
Let $\cA_t f$ be as in (2.14) and define the associated maximal function
 by $\cM f(x) =\sup_{t>0}|\cA_t f(x)|$.
Suppose that
$$\Gamma_{\frac{p}{2-p}}(\delta)=
(\int_\tSi|\cB(w,\delta)|^{\frac{2p-2}{2-p}}dw)^{\frac{2-p}{p}},
$$
and the inequality
$$
\int_0^1
[\log(1+\delta^{-1})]^{\frac 1p-\frac 12}\delta^{-1/p}
\Gamma_{\frac{p}{2-p}}(\delta)\frac {d\delta}{\delta}
\le A<\infty
$$
holds.
Then
$\cM$ is bounded on $L^p$;  the operator norm is dominated 
by $CA\|\chi\|$ where $C$ is admissible and $\|\chi\|$ is a 
suitable Sobolev norm of $\chi$.
\endproclaim

\demo{Proof} Let $H_{\delta,t}(x)= t^n \cL_\delta [d\mu](tx)$ where
$\cL_\delta$ is as in the proof of Proposition 2.2. By \cite{18, (4.4)}
$$\big\|\sup_{t>0}|H_{\delta,t}*f\big \|_p\le C 
[\log \delta^{-1}]^{\frac 1p-\frac 12}\delta^{-1/p} 
\Gamma_{\frac{p}{2-p}}(\delta)
$$
for small $\delta$ 
and the statement of the  theorem follows  by introducing a dyadic 
decomposition for large frequencies 
 and summing  the estimates for the operators corresponding
to the pieces.\qed
\enddemo



A consequence of Theorem 2.5 and Proposition 2.1 is

 

\proclaim{Proposition 2.6}
 Let $\Phi\in \cS^{d-1}_{2T}(b, M,\fa,N)$, $N>m+1$ and let 
$\cB(w,\delta)$ be defined as in (2.6). 
Let  $k\in \{1,\dots, d-1\}$,  $\beta>0$ and $\eta>1/2$. 
Suppose that
$$
|\cB(w,\delta)|\le C\delta^\eta
\tag 2.20
$$
(in particular we can choose $\eta=\nu_1$ if $\nu_1>1/2$)
 and 
$$
\int K_k(x)^{-\beta} d\sigma(x)\le A.
\tag 2.21
$$

Then the following statements hold.

(i) $\cM$ is $L^p$ bounded
 for $p>\frac{1+2\eta}{2\eta}$.

(ii) If  $\beta\ge\frac 1{k-1+2\nu_{k+1}}$ then
 $\cM$ is $L^p$ bounded for
$p>\frac{k+1+2\nu_{k+1}}{k+2\nu_{k+1}}$.

(iii) If 
$\beta<\frac 1{k-1+2\nu_{k+1}}$ then
 $\cM$ is $L^p$ bounded for
$p>p_0(\beta,\eta, k)=\frac{1+2\eta-2\beta(k+2\nu_{k+1}-2\eta)}
{2\eta-\beta(k+2\nu_{k+1}-2\eta)}.$
\endproclaim


\demo{Proof}
First note that the restriction $\eta>1/2$
implies that 
$\frac{k+1+2\nu_{k+1}}{k+2\nu_{k+1}}<2$ and $p_0(\beta,\eta,k)<2$ if
$\beta<\frac 1{k-1+2\nu_{k+1}}$. Therefore
it suffices to check the condition (1.6). (i) follows immediately from Theorem 2.5; however this special case follows already from Greenleaf's 
paper \cite{11}.

We may therefore assume that  $(k+1+2\nu_{k+1})/(k+2\nu_{k+1})
<p\le (2\eta+1)/2\eta$.
Let $$
\align
&\underline\theta=1+\beta-\frac \beta{p-1}\\
&\overline \theta=\frac{(p-1)(k+2\nu_{k+1})-1}{(p-1)(k+2\nu_{k+1}-2\eta)}.
\endalign
$$
Note that $\overline\theta>0$ since 
$(k+1+2\nu_{k+1})/(k+2\nu_{k+1})<p$ and $\overline\theta\le 1$
 since
$p\le (2\eta+1)/2\eta$.
Moreover a computation shows  that the inequality 
$\underline \theta< \overline \theta$ is equivalent with
$$1+2\eta-2\beta(k+2\nu_{k+1}-2\eta)<p
\big( 2\eta-\beta(k+2\nu_{k+1}-2\eta)\big).
\tag 2.22
$$
If  
$\beta\ge\frac 1{k-1+2\nu_{k+1}}$
then (2.22) holds for all
$p\in (\frac{k+1+2\nu_{k+1}}{k+2\nu_{k+1}},2)$ and if $\beta<
\frac 1{k-1+2\nu_{k+1}}$ then (2.22) is satisfied precisely
 for $p>p_0(\beta,\eta,k)$. In either case it is therefore possible to choose
$0<\theta<1$ such that $\underline \theta\le\theta<\overline\theta$.
We now estimate  using Proposition 2.1
$$
\align
\Big(\int_{|w|\le T} |\cB(w,\delta)|^{\frac{p}{2-p}-1}&dw
\Big)^{\frac{2-p}{p}}
=
\Big(\int_{|w|\le T} |\cB(w,\delta)|^{\frac{2p-2}{2-p}(1-\theta)
+\frac{2p-2}{2-p}\theta
}dw\Big)^{\frac
{2-p}{p}}
\\
&\le 
\Big( (A_1\delta^\eta)^{\frac{2p-2}{2-p}\theta}
(A_2\delta^{\frac{k}2+\nu_{k+1}})^{\frac{2p-2}{2-p}(1-\theta)}
\int_{|w|\le T} [ K_k(w)]^{-\frac 12\frac{2p-2}{2-p}(1-\theta)} dw
\Big)^{\frac{2-p}p}
\\
&\le C \delta^{(2\eta\theta+(k+2\nu_{k+1})(1-\theta))\frac{p-1}p}
\Big(\int_{|w|\le T}
 [ K_k(w)]^{-\frac{p-1}{2-p}(1-\theta)} dw\Big)^{\frac{2-p}p}.
\endalign
$$
The integral is finite if
$\frac{p-1}{2-p} (1-\theta)\le \beta$; a short computation shows that this is
equivalent to the condition $\theta\ge \underline{\theta}$ hence 
satisfied in view of our choice of $\theta$.
Now according to Theorem 2.5 the $L^p$ boundedness holds if 
$(2\eta\theta+(k+2\nu_{k+1})(1-\theta))\frac{p-1}p>\frac 1p$ 
and another computation
shows that this is precisely the restriction $\theta<\overline\theta$.
\qed
\enddemo


As an easy consequence we obtain


\proclaim{Theorem 2.7}
Let $\Sigma\subset \Bbb R^d$, $d\ge 3$,
be a  convex hypersurface of finite line type and 
let $K(x)$ the Gaussian curvature. 
Suppose that 
% $\beta>(2d-4)^{-1}$ 
$$\int_\tSi [K(x)]^{-\beta} d\sigma(x)
<\infty \quad \text{ for all } \beta<\frac 1{d-2}.
$$
Then the maximal operator in (1.2) is bounded on $L^p(\Bbb R^d)$, for
$p>d/(d-1)$.
%Denote  by $\cM$ the maximal operator in (1.2) associated to $\Sigma$.
%(i) If $\beta\ge (d-2)^{-1}$ then $\cM$ is bounded on 
%$L^p(\Bbb R^d)$ for $p>d/(d-1)$.
%
%(ii) If $(2d-4)^{-1}< \beta<(d-2)^{-1}$ then $\cM$ 
% is bounded on 
%$L^p(\Bbb R^d)$ for $p>\frac{1+2\beta}{\beta(d-1)}$.
\endproclaim


\demo{Proof} After localization we may assume that the 
averaging operator is of the form (2.14).
Note that $|B(x,\delta)|\approx |B(y,\delta)|$ if
$y\in B(x,\delta)$.
Therefore by Proposition 2.1
$$
|B(x,\delta)|^{1+2\beta}\lc 
\int |B(y,\delta)|^{2\beta} d\sigma(y)
\lc \delta^{(d-1)\beta}
\int |K(y)|^{-\beta} d\sigma(y)
$$
Therefore 
$|B(x,\delta)|\lc\delta^{\eta_\beta}$ with $\eta_\beta=
\frac{(d-1)\beta}{1+2\beta}$
and $\eta_\beta>1/2$ if $\beta>(2d-4)^{-1}$.
The assertion follows from an application of  Proposition 2.6 with $k=d-1$,
$\eta=\eta_\beta$, the critical exponent in case (ii) is then 
$p=\eta_\beta^{-1}$ and for $\beta=1-\eps$ we see that
$\eta_\beta^{-1}=d/(d-1) +O(\eps)$.\qed
\enddemo






%\subheading{3. Auxiliary Results}
\head{\bf 3. Auxiliary Results}\endhead
According to a result of Schulz \cite{20} one can decompose  a convex function
at a given point into a main term, which after an affine change of variable
exhibits some  homogeneity, and a remainder term. We first need the
following


\definition{Definition} Define the dilations $A_s$ by 
$$A_sx=(s^{\frac{1}{a_1}} x_1 , \ldots , s^{\frac{1}{a_{n}}} x_{n} ).
\tag 3.1$$
  We say that a smooth function $Q : \R^{n} 
\rightarrow \R$ is mixed homogeneous of degree $(a_1, a_2, \ldots
, a_{n})$, $a_j>0$, if
$$Q(A_sx)=
sQ(x), s > 0 .
\tag 3.2$$
\enddefinition

The following Proposition summarizes and  extends
a  result of  \cite{20}; the fact (3.5) below  was already  applied 
in the proof of Theorem 10 in \cite{14}.

\proclaim {Proposition 3.1} 
  Let $\Phi\in \Cal S_T^n(b,M,m,3N+2)$, where $N>m$.
% and set $\widetilde N=N-m$.
Suppose that $a_1\le \dots\le a_n\le m$ and 
$\fa=(a_1,a_2,\dots, a_n)$ is the multitype of $\Phi$ at $0$. 
Then the following statements hold.


There is a rotation $L$ on $\Bbb R^{n}$  so that
$$
\Phi(L x)=Q(x)+R(x), \ |x|\le T
\tag 3.3
$$
where $Q$ is a convex mixed homogeneous polynomial of degree 
$(a_1,\dots, a_n)$,   the  $a_i$ are even positive integers with
$a_1\le \dots\le a_{n}$, the graph of $Q$ is of finite line type 
$\le a_n\le  m$ 
and $(a_1,\dots, a_{n})$ is the multitype at $0$ of
 the graph of $\Phi$ (considered  as a subset of $\Bbb R^{n+1}$.)
If $a_j<a_{j+1}$ then the linear subspace $S^{a_j}$ 
 consisting of all
$v$ such that $(\inn{v}{\nabla})^j[\Phi\circ L](0)=0$ for $j< a_{j+1}$
 is  the image of $\text{span}\{e_{j+1},\dots, e_n\}$ under $L^{-1}$.
Moreover
$$Q(x)>0\quad \text{ if } x\neq 0\tag 3.4
$$
and
$$
|Q(x)|\le C_1|x||\nabla Q(x)|\le C_2|x|^2 \sum_{i,j}
\Big|\frac{\partial^2 Q}{\partial x_i \partial x_j  }(x)\Big|.
\tag 3.5
$$
The  remainder term $R$ satisfies
$$
%\lim_{s\rightarrow 0} 
\Big|s^{-1}\frac{\partial^{|\alpha |}}
{ \partial x^\alpha}\big(R(A_s x)\big) \Big| \le C_{M, N} s^{1/m}
\tag 3.6
$$
 for $|x|\le T$ and  all multiindices $\alpha=(\alpha_1,\dots, \alpha_{d-1})$ 
with $|\alpha|\le  N$; $A_s$ is as in  (3.1). 



If $a_1 = \cdots = a_k = 2$ for some $k$, then the rotation $L$ can be 
chosen so that
$$
Q(x) = c_1 x^2_1 + \cdots + c_k x^2_k + \widetilde Q
  (x_{k + 1}, \ldots , x_{n}) 
\tag  3.7
$$
where $\widetilde Q$ is mixed homogeneous of degree
 $(a_{k+1} , \ldots , a_{n})$; {\it i.e.}
$\widetilde Q(s^{\frac 1{a_{k+1}}}x_{k+1}, \dots, s^{\frac 1{a_n}}x_{n})=
s\widetilde Q(x_{k+1},\dots x_{n})$ for all $x\in \Bbb R^n$.
\endproclaim


\remark{Remark }
We note that   if $\Phi$ belongs to 
$\cS_T^n(b,  M, \fa, 3N+1)$ then
 $Q$ belongs  to a  family
$\cS_T^n(\widetilde b, C M, m, 3N+1)$, with $\widetilde b>0$,
 but unfortunately
 there is no good  lower bound for 
$\widetilde b$ in 
terms of $b$.
\endremark


\demo{Proof of Proposition 3.1} 
The decomposition (3.3) was obtained by Schulz \cite{20} and the
 construction involved the subspaces 
 $S^{m_i}$ mentioned 
in the introduction. 
The polynomial $Q$ was obtained as a Taylor-polynomial
$\sum c_\gamma x^\gamma$ of $\Phi\circ L$
 where each multiindex $\gamma$  satisfies
$\sum_{i=1}^{n}\gamma_i/a_i=1 $; the convexity and (3.4) is verified 
in \cite{20}.
As observed in \cite{14}, (3.5) is a consequence of Euler's homogeneity 
relation $Q(x)=\sum x_ia_i^{-1}Q_{x_i}(x)$.
To see (3.6), fix $\alpha$,
% choose
%$N>|\alpha |+\max\{ a_1,\ldots ,a_{n-1}\}$,
 and use Taylor's formula to
write 
$$R(x)=P_{2N}(x)+R_{2N}(x)
$$
where $P_{2N}(x)$ is a linear combination of monomials 
$G_\beta(x):=x^\beta$ with
$|\beta |\le 2N$ and $ \sum^{n}_{k=1}{\beta_k\over\alpha_k}>1$.
If $\alpha_i\le \beta_i$, $i=1,\dots,n$ it follows
immediately that
$$\frac{\partial^{|\alpha|}}{\partial x^\alpha}\Big[
s^{-1}G_\beta(A_sx)\Big]
=
c_{\alpha,\beta} x^{\beta-\alpha}
  s^{-1+\sum^{n}_{k=1}\frac{\beta_k}{\alpha_k}}
$$
which is $\le C s^{1/m}$ since $\beta_k$ assume only integer values and
$m^{-1}\le  a_n^{-1}$.
Thus
$$
\Big|
\frac{\partial^{|\alpha|}}{\partial x^\alpha}
\Big[
s^{-1}P_{2N}(A_s x)\Big]\Big|\le 
C_{M,N} s^{1/m}.
$$
  Finally, the remainder  $R_{2N}(x)$ satisfies $|\partial_\alpha R_{2N}(x)|\le
C_N|x|^{2N+1-|\alpha|}$, for $|\alpha|\le N$. Therefore

$$
\Big|\frac{\partial^{|\alpha|}}{\partial x^\alpha}\Big[
s^{-1}R(A_s x)\Big]\Big|
\le C|x|^{N+1} s^{-1}\max s^{(N+1)/a_i}\le C' |s|^{1/m}
$$
by the definition of $N$. This finishes the proof of (3.6).

We now turn to proving (3.7)  and discuss first the case 
$k=1$. Split $x=(x_1,x')$ with $x'=(x_2,\dots,x_{d-1})$.
Then $Q$ can be decomposed as
$$
Q(x)=c_1x^2_1+x_1A(x')+B(x'),
$$
where $B$ is mixed homogeneous of degree $(a_2,\dots, a_{d-1})$, 
and $A$ is mixed
homogeneous of degree $(a_2/ 2, \dots, a_{n}/2)$.
%For $2\le i,j\le d-1$ set
% $A_{ij}=\frac{\partial^2A}{\partial x_i\partial x_j}$ and
% $B_{ij}=\frac{\partial^2 B}{\partial x_i\partial x_j}$. 
In order to prove that
$A=0$ it suffices to show that the partial derivatives $A_{x_ix_j}$ 
vanish for all $i,j\ge 2$. To see this we  use homogeneity.
Define $$\delta_s x'=(s^{1/a_2}x_2,\dots,s^{1/a_{n}}x_{n})$$ 
and observe that
$$\aligned
B_{x_ix_j}(\delta_sx')&=s^{1-1/a_i-1/a_j}B_{x_ix_j}(x')\\ 
A_{x_ix_j}(\delta_sx')&=s^{1/2-1/a_i-1/a_j}A_{x_jx_j}(x')
\endaligned
\tag 3.8
$$
for $s>0$.

By the convexity of $Q$ we have 
$$\inn\eta{\nabla^2Q(x) \eta}\ge 0
\tag 3.9$$ 
for all  $x$ near $0$ and all $\eta$.
With $\eta=e_j$, $j=2,\dots,n$ this yields
$$
0\le B_{x_jx_j}(x') +x_1 A_{x_jx_j}(x').
\tag 3.10
$$

Suppose now 
that $A_{x_jx_j}(\tilde x')\neq 0$;
 then $G_j=B_{x_jx_j}/A_{x_jx_j}$ satisfies
$G_j( \delta_s x')=s^{1/2}G_j(x')$ for $x'$ near $\tilde x'$.
Using this homogeneity property
we see from (3.10) that if
$A_{x_jx_j}$ is not identically  zero, then $e_j^t\nabla^2Q(x)e_j$ changes sign
arbitrarily close to the origin, a contradiction.
Therefore $A_{x_jx_j}$ vanishes identically, for $j=2,\dots,n$.

Next we show that $A_{x_ix_j}=0$ for $i\neq j$. 
We apply (3.9) with $\eta=\xi_ie_i+\xi_je_j$. Since $A_{x_jx_j}=0$,
(3.10) becomes
$$
0\le B_{x_ix_i}(x')\xi^2_2+2B_{x_ix_j}(x')\xi_i\xi_j+
B_{x_jx_j}(x')\xi^2_j+ 2x_1 A_{x_ix_j}(x')
\xi_i\xi_j.
\tag 3.11
$$
Assume that $A_{x_ix_j}(\tilde x')\neq 0$; by homogeneity we have then 
$A_{x_ix_j}(\delta_s x')\neq 0$ for $x'$ near $\tilde x'$.
By (3.8) and (3.11)  it follows that
$$
\align
0&\le x_1+
\frac{\inn{\eta}{\nabla^2B( \delta_s x')\eta}}
{\inn{\eta}{\nabla^2A(\delta_sx')\eta}}\\
&=x_1+
\frac{
B_{x_ix_i}(x')\xi^2_2s^{1/2-1/a_i+1/a_j}+2B_{x_ix_j}(x')\xi_i\xi_j
s^{1/2}+B_{x_jx_j}(x')\xi_j^2
s^{1/2+1/ a_i-1/ a_j}}
{2A_{x_ix_j}(x')\xi_i\xi_j}
\endalign
$$
and this expression tends to $x_1$ as 
 $s\to 0$ since $|a_i^{-1}-a_j^{-1}|\le 1/2$.  Thus for each $s$
sufficiently small, we can find a value of $x_1$, 
such that the right side of (3.11) vanishes.  We see 
that the expression changes
sign arbitrarily close to the origin, a
contradiction.  Hence $A_{x_ix_j}$ also vanishes.

We now turn to the case $k>1$. Split $x=(x',x'')$ with $x'=(x_1,\dots, x_k)$;
then
$$
Q(x)=Q_0(x')+\sum_{i=1}^k x_i A_i(x'')+B(x'').
$$
where $Q_0(x')$  is a positive definite quadratic form on $\Bbb R^{k}$, 
the functions $A_i$ are mixed homogeneous of degree
$(a_{k+1}/2,\dots, a_{n}/2)$ and $B$ is mixed homogeneous of degree
$(a_{k+1},\dots, a_{n})$. By performing a rotation in the $x'$ variables we 
can assume that $Q_0(x')=\sum_{i=1}^k c_ix_i^2$. Then we can apply  
the case $k=1$  already proved
to the functions
$(x_i,x'')\mapsto Q(x_ie_i,x'')$ and deduce that $A_i=0$.
\qed
\enddemo






\proclaim{Lemma 3.2} Suppose that 
$\Phi\in \cS^n_{2T}(b,M,m,N)$, $N>4m$, $a_2>2$,  and suppose that
$$
\frac{\partial^2\Phi}{\partial x_1^2}(0)\neq 0,
\qquad
\frac{\partial^{a_2}\Phi}{\partial x_2^{a_2}}(0)\neq 0,
\qquad
\frac{\partial^j\Phi}{\partial x_2^j}(0)= 0 \text{ if } j<a_2.
$$
Let $K_2[\Phi]=\Phi_{x_1x_1}\Phi_{x_2x_2}-(\Phi_{x_1x_2})^2$.
Then
$$
\frac{\partial^{a_2-2} K_2[\Phi]}{\partial x_2^{a_2-2}}(0)\neq 0.
\tag 3.12$$
Moreover there is $\epsilon>0$, $\delta>0$ and $C_\gamma$ (all depending on 
$\Phi$) so that 
$$
\sup\Sb
x_1,x_3,\dots, x_n\\ \in [-\delta,\delta]
\endSb
\int_{-\delta}^\delta\big(K_2[\Psi](x)\big)^{-\gamma} dx_2<C_\gamma, 
\qquad\text{if }  \gamma<(m-2)^{-1},
\tag 3.13
$$ 
for all
$\Psi\in\cS^n_{r}(b/2, 2M,m,N)$ with 
$\|\Phi-\Psi\|_{C^N(|x|\le r)}\le \epsilon$ .

\endproclaim
\demo{Proof}
We define $\phi(y_1,y_2)=\Phi(y_1,y_2,0)$. Then $(1,0,...)$
 is an eigenvector
 of the Hessian of $\phi$ and we can apply Proposition 2.1
 to $\phi$, without performing a rotation. Thus
$$\phi(y)=\frac{c_1}2 y_1^2 +c_2 y_2^{a_2}+R(y)
$$
where $c_1>0$, $c_2>0$ and  $R$ satisfies (3.6). 
Now
$$
K(y)=
c_1c_2 a_2(a_2-1) y_2^{a_2-2}+E(y)$$ where the error $E(y)$ is given by
$$E=(c_1+R_{y_1y_1})R_{y_2y_2}+
c_2a_2(a_2-1)R_{y_1y_1}y_2^{a_2-1}-
R_{y_1y_2}^2
%\endalign
$$

Expanding $R$ we see that
$$
R(y)=\sum_\beta c_\beta y^\beta+R_{a_2+1}(y);
\tag 3.14
$$
here we sum over multiindices $\beta$  so that $|\beta|\le m$ and
$ \beta_1/2+\beta_2/a_2>1$.
All derivatives of order
 $\le a_2$  of $R_{a_2+1}$
vanish for $y=0$.

In order to show (3.12) we shall show that $\partial_{y_2}^{a_2-2} E(0)=0$.
To see this let $G_\beta(y)=y^\beta$. We have to verify that
$$\align
&\frac{\partial^{a_2} G_\beta}{\partial y_2^{a_2}} =O(y)
\\
&
\frac{\partial^{2+\ell} G_\beta}{\partial y_1^2 \partial y_2^\ell}
\frac{\partial^{a_2-\ell}G_{\beta'}}{\partial y_2^{a_2-\ell}}
=O(y),\qquad \ell\le a_2-2
\\
&\frac{\partial^2 G_\beta}{\partial y_1^2} =O(y)
\\
&\frac{\partial^{\ell+1}G_\beta}{\partial y_1\partial y_2^\ell}
=O(y),
\qquad 1\le \ell\le \frac {a_2}2
\endalign
$$
whenever $\beta$ or $\beta'$ occur in the sum (3.14).
Considering the term 
$\frac{\partial^{2+\ell} G_\beta}{\partial y_1^2 \partial y_2^\ell}
\frac{\partial^{a_2-\ell}G_{\beta'}}{\partial y_2^{a_2-\ell}}$
%$\frac{\partial^\ell}{\partial y_2^\ell}
%\frac{\partial^2 G_\beta}{\partial y_1^2}
%\frac{\partial^{a_2-\ell}G_{\beta'}}{\partial y_2^{a_2-\ell}}$
 it is clearly $O(y)$ unless  
$\beta_1=2$, $\beta_2=\ell$, $\beta_1'=0$, $\beta_2'=a_2-\ell$ and $\beta_j=
\beta_j'=0$ for $j\ge 3$. But this implies that $a_2-\ell=a_2$, hence 
$G_\beta(y)=y_1^2$, but $y_1^2$ is not an admissible  term in (3.14).
We argue similarly for each of the other terms
and (3.12) is proved.

To see the second assertion we use a result 
related to van der Corput's lemma which is due to M. Christ \cite{4}
(alternatively one may use the Malgrange preparation theorem).
It states that for any $k\in\bbZ_+$ 
there is a constant $A_k$ such that for any interval $I\subset \bbR$,
any $f\in C^k(I)$ and any $\gamma>0$
$$\Big|\{t\in I:|f(t)|\le \gamma\}\Big|\le A_k \gamma^{1/k} \inf_{s\in I}
|D^k f(s)|^{-1/k}.
\tag 3.15
$$
By continuity we know that 
$\frac{\partial^{a_2}K_2}{\partial x_2^{a_2}}(x)\neq 0$ for small $x$
and we can apply (3.15) with $k=a_2$ to obtain (3.13)
\qed
\enddemo



\proclaim{Proposition 3.3}
Let $n=2$, $\Phi\in \cS^2_T(b,M,m,N)$ for large $N$ and 
suppose that $(a_1,a_2)$ is the multitype at $0$;
moreover assume
$$\aligned
&\frac{\partial^j \Phi}{\partial x_i^j}(0)=0 \quad \text{ for $j<a_i$},
\\&\frac{\partial^{a_i} \Phi}{\partial x_i^{a_i}}(0)\neq 0,
\endaligned
\tag 3.16$$ 
for $i=1,2$.
Let $\rho(x)=x_1^{a_1}+x_2^{a_2}$ and let 
$$\Omega_\ell=\{x:2^{-\ell-1}\le \rho(x)\le 2^\ell\}.$$
Let $\nu=1/a_1+1/a_2$.
There is $\ell_0>0$ so that for $\ell>\ell_0$
$$\int_{\Omega_\ell} |\det \Phi''(x)|^{-\gamma} d\sigma(x)\le
C_\gamma 2^{\ell(2\gamma-2\gamma\nu-\nu)}, \quad { for }\quad
\gamma<\frac 1{a_2-2}. 
\tag 3.17
$$
Moreover $[\det \Phi'']^{-\gamma}$ is integrable over a neighborhood of the origin.
\endproclaim



\demo{Proof}
In view of  assumption (3.16)  we may decompose 
$\Phi=Q+R$  where $Q$ is mixed homogeneous of degree $(a_1,a_2)$, 
in fact
$Q(x)\le c_1\rho(x)\le c_2 Q(x)$ for small $x$,
 by the homogeneity and positivity of $Q$ and $\rho$.
The function  $Q$ is of type $\le a_2$ near $0$ 
and by homogeneity considerations it is easy to see that $Q$ is of type 
$\le a_2$ everywhere. Moreover,   by 
(3.5) the rank of $Q$ in
$\Omega_1$  is at least $1$.

Let $\Gamma=\{x\in \Omega: \det \Phi''(x)=0\}$ and fix $x^0\in \Gamma$. 
Then there is a rotation $L_{x^0}$ so that
$\psi(y)=Q(x^0+L_{x^0}y)$ satisfies the assumption of Lemma 3.2
and therefore we can integrate $[\det Q'']^{-\gamma}$ over a small 
neighborhood  of $x^0$; moreover the bound persists for small $C^N$
 perturbations of $Q$.
Using compactness arguments we see that there is $\epsilon>0$ so that
$$\int_{\Omega_1 }[\det \Psi'']^{-\gamma} dx\le C_\gamma
\tag 3.18$$
if $\|\Psi-Q\|_{C^N(\Omega_{1})}\le \eps $ and $\gamma<1/(a_2-2)$.

Let, for large $\ell$
$$
\Phi_\ell(y)=2^\ell\Phi(2^{-\ell/a_1}y_1, 2^{-\ell/a_2}y_2).
$$
Then
$$\Phi_\ell=Q+ R_\ell \tag 3.19$$
 and all derivatives of  $R_\ell$ tend to $0$  uniformly in
$\{y:\rho(y)\le 1\}$.
Therefore there is $\ell_0$, $2^{-\ell_0}\ll 1$ 
 so that  (3.18) applies for $\Psi=\Phi_\ell$,
 $\ell>\ell_0$, with a bound independent of $\ell$.
Since
$$\det \Phi_\ell''(y)= 2^{2\ell(1-\nu)} \det \Phi''( 2^{-\ell/a_1}y_1,
2^{-\ell/a_2}y_2)
\tag 3.20$$
we obtain for $\ell>\ell_0$
$$\align
\int_{\Omega_\ell}|\det\Phi''(y)|^{-\gamma} dy&=
\int_{\Omega_1}
2^{-\ell\nu}|
\det\Phi''(2^{-\ell a_1}y_1,2^{-\ell a_2}y_2)|^{-\gamma} dy
\\&= 
2^{-\ell\nu}
2^{2\gamma\ell(1-\nu)}\int_{\Omega_1}| \det \Phi_\ell''(y)|^{-\gamma} dy.
\tag 3.21
\endalign
$$
If $\gamma<(a_2-2)^{-1}$ we can dominate 
the integrals by a constant independent of $\ell$ and the estimate (3.17)
 is proved. 

%The integrability of $[\det \Phi'']^{-\gamma}$ in a neighborhood of 
%the origin is trivial  if $a_1=a_2=2$.
Since $a_1\le a_2$ we see that
$-\nu+2\gamma(1-\nu)\le \frac 2{a_2}((a_2-2)\gamma-1)<0$ and therefore
we can sum the estimates (3.21) to obtain the integrability of 
$[\det\Phi'']^{-\gamma}$ near the origin.
\qed
\enddemo


\demo{\bf Proof of Theorem 1.2} Immediate from Proposition 3.3\qed\enddemo

%\remark{Remark} There exist examples for which the critical 
%exponent $(a_2-1)^{-1}$ in Proposition 3.3 is not sharp, {\it cf.} 
% the examples in \S4.
%\endremark


We now examine the size of the balls in (2.6) near a point of given multitype.


\proclaim{Proposition 3.4}
  Let $\Phi\in \Cal S_T^n(b,M,m,N)$, where $N$ is large, and let
$\fa=(a_1,a_2,\dots, a_n)$ be the multitype of $\Phi$ at $0$.
We assume that (3.16) holds for $i=1,\dots,n$.
%$\frac{\partial^j \Phi}{\partial x_i^j}(0)=0$ for $j<a_i$,
%$\frac{\partial^{a_i} \Phi}{\partial x_i^{a_i}}(0)\neq 0$.

Let $\nu=\sum_{j=1}^n \frac 1{a_j}$,
 let $\rho(y)=\sum_{i=1}^n y_i^{a_i}$
and
$\Omega_\ell=\{x:2^{-\ell-1}\le \rho(x)\le 2^\ell\}.$
Then there are constants $C_1$, $C_2$ so that
for $C_1\delta\le 2^{-\ell}\le C_2$, $y\in \Omega_\ell$
$$
|\cB(y,\delta)|\le C_\alpha \delta^{\alpha} 2^{\ell(\alpha-\nu)}
\quad \text{ if }  \quad \alpha\le \frac 12+\frac 1{a_n}.
\tag 3.22
$$

\endproclaim

\demo{Proof} Decompose $\Phi=Q+R$ as in (3.3). By our assumptions this 
holds with the rotation $L$ being the identity.
By the metric properties of the balls $\cB(y,\delta)$ (in particular the 
triangle inequality for the pseudo-distance 
associated to these balls \cite{2}) 
it follows that there are constants $C_1>1$, $C_2>1$
so that
$$
\cB(y,\delta)\subset \{ x: C_1^{-1} Q(x)\le Q(y)\le  C_1 Q(x)\}\quad
\text{ if }\quad Q(y)\ge C_2\delta .
 $$

Now let $Q(y)\ge C_2\delta$ and
set $\Phi_\ell(w)=2^\ell\Phi(A_{2^{-\ell}} w)$; note that
$\Phi_\ell=Q+R_\ell$ where $R_\ell$ tends to zero in the $C^\infty$ topology.
Let $\ell $ be large so that $2^{-\ell-1}\le Q(y)\le 2^{-\ell}$.
Then one computes  that with 
$W=\{y':C_1^{-1}/2\le Q(y')\le  C_1\}$ and  
$Y_\ell=A_{2^\ell}y\in W$ 
$$
\{A_{2^{\ell}}z: z\in \cB(y,\delta)\}
=
\{w:\Phi_\ell(w)-\Phi_\ell (Y_\ell)-\inn{w-Y_\ell}{\nabla\Phi_\ell(Y_\ell)}
\le 2^\ell\delta\}
:=W_{\ell,y,\delta}
$$
and  $W_{\ell,y,\delta}$ is contained in 
$W$. 
By Proposition 3.1  there is $C_2>0$ 
and $\ell_0>0$
such that  for any $y\in W$ there is a unit vector $\theta$ with
${\inn {\theta}{\nabla}}^2\Phi_\ell(y)\ge C$, for all $\ell>\ell_0$.
Moreover 
$\Phi_\ell$ is of line type $\le a_n$, with uniform bounds for $\ell>\ell_0$,
 since this is the case for $Q$.
This implies that
$$|W_{\ell,y,\delta}|\le C (2^\ell\delta)^{\alpha}$$
for $0\le \alpha\le 1/2+1/a_n$.
Since the Jacobian of the change of variable $z\to A_{2^\ell z}$ is
$2^{\ell\nu}$
we obtain that 
$$
|\cB(y, \delta)|\le C\delta^{\alpha} 2^{-\nu\ell+\alpha\ell}
$$
 and since 
$Q(y)\approx \rho(y)$ the assertion follows.\qed
\enddemo

\remark{Remark}
Let $\alpha\le 1/2+1/a_n$. The estimate $|\cB(y,\delta)|
\le C \delta^{\alpha} [\Phi(y)]^{\nu-\alpha}$, for small $y$,
 is an easy consequence of Proposition 3.4.
\endremark


%\subheading{4. Estimates  involving the multitype}
\head
{\bf 4. Estimates  involving the multitype}
\endhead

We shall first give a different proof of the 
following Theorem proved by the first two authors in  \cite{14}.

\proclaim{Theorem 4.1} Let $\cM$ be as in (1.2). Suppose that
 $(a_1,a_2,\dots, a_d)$ is the multitype  at $P$ and
 that $\nu=\sum_{j=1}^{d-1}\frac 1{a_{j}}\le\frac 12$ .
Then there is a neighborhood $U$ of $x_0$ so that 
$\cM$ is bounded on $L^p(\Bbb R^d)$ if $p>\nu^{-1}$, 
provided that $\supp \chi\subset U$.
\endproclaim


\demo{Proof}
We may assume that our averages are of the form (2.14) and 
 $P=(0,c_d)$.
Since $\nu^{-1}\ge 2$ we just have to verify (1.4).
We now use Proposition 3.4, with $\alpha=\nu$ in the first term 
below and $\alpha<\nu$ in the second, and obtain
$$\align\Gamma_{\frac{p}{p-2}}(\delta)
&\le
 \Big(\int_{\rho(w)\le C_1\delta}
|\cB(w,\delta)|^\frac 2{p-2}dw\Big)^{\frac{p-2}p}
+\sum_{ C_1\delta\le 2^{-\ell}\ll 1}
\Big(\int_{\rho(w)\approx 2^{-\ell}}
|\cB(w,\delta)|^\frac 2{p-2}dw\Big)^{\frac{p-2}p}
\\
& \le C\big(\delta^\nu+\sum_{ C_1\delta\le 2^{-\ell}\ll 1}
(\delta^\alpha 2^{\ell(\alpha-\nu)})^{\frac 2p}
\delta^{\nu\frac{p-2}{p}}\big)\le C'\delta^\nu.
\endalign
$$
This implies (1.4) since $\nu> 1/p$.\qed
\enddemo


\demo{\bf Proof of Theorem 1.1}
If $a_1>2$ then $\nu\le 1/4+1/a_2\le 1/2$ and the assertion (i) follows from
Theorem 4.1 (the necessity of the condition had also been shown in \cite{13}).
Now let $a_1=2$. Assertion (ii) follows from Proposition 2.6
(with $k=2$, $\nu_3=0$, $\eta=1/2+1/a_2$), and by Proposition 3.3
the hypothesis of (ii) is satisfied with $\beta< (a_2-2)^{-1}$; 
this shows assertion (iii).\qed
\enddemo

\demo{\bf Proof of Theorem 1.4}
It is sufficient to assume that $\cA$ is of the form (2.14)
 so that the multitype 
at $0$ is $\fa=(a_1,a_2)$ and $\chi$ is supported near the origin; moreover
we may assume that (3.16) holds for $i=1,2.$
%$\frac{\partial^j \Phi}{\partial x_i^j}(0)=0$ for $j<a_i$,
%$\frac{\partial^{a_i} \Phi}{\partial x_i^{a_i}}(0)\neq 0$, i=1,2


We have boundedness for the cases $p=1=q$ trivially.
Since  $|\cB(y,\delta)|\le C \delta^\nu$
for small $y$ and $\delta$
 it follows from Theorem 1.3 that
$\cA$ maps $B^{p}_{\beta,r}$ to $B^{p'}_{\beta+\alpha}$ if 
$1\le p\le 2$, $ \alpha\ge -\nu$ and $\frac 1p-\frac 12\le 
\frac{\alpha+\nu}{2(\nu+1)}$. This is the asserted estimate for $1/p+1/q=1$.
 We remark that this result is well known and 
follows  just  from the assumption
that $\widehat {d\sigma}(\xi)=O(|\xi|^{-\nu})$, see {\it e.g} \cite{23, p. 371}
and also  the original references \cite{25}, \cite{17}.




We shall now consider the case $1/p+1/q< 1$ and prove boundedness 
under the conditions
(1.15.1-3); 
boundedness for $1/p+1/q> 1$ under the conditions (1.16.1-3) follows then by
duality.
We shall verify the condition (1.13) by estimating the volume of the balls
$\cB(w,\delta)$ using Proposition 2.1 and then apply either 
Proposition 3.3 or Proposition 3.4 or both.

In what follows define $r$ and $\sigma$ by
$$\aligned
&\frac 1r=\frac 1p+\frac 1q-1\\
&\sigma=\frac{2q(p-1)}{p+q-pq}
\endaligned
$$ so that
$\sigma/r=2/p'$.
First observe that by Proposition 2.1
$$
\delta^{-\alpha-\frac 1p+\frac 1q}
\Big(\int_{\rho(w)\le C_2\delta}|\cB(w,\delta)|^\sigma dw\Big)^{\frac 1r}
\le C \delta^{\nu(\frac{\sigma+1}r)-\alpha-\frac 1p+\frac1q}=
C\delta^{-\alpha+\nu-\frac{\nu+1}p+\frac {\nu+1}q}
\tag 4.1
$$
which is bounded uniformly in $\delta$, by 1.15.1.
Here we assume that $C_2$ is as in the statement of
 Proposition 3.4.


We use Proposition 2.1 to estimate $\cB(w,\delta)$ 
and our conclusion follows if we can verify the estimate
$$\Big(\int_{C_2\delta\le \rho(w)\le c}
\Big(\frac{\delta}{\sqrt{\det\Phi''(w)}}\Big)^{\sigma(1-\theta)}
|\cB(w,\delta)|^{\sigma\theta}dw\Big)^{1/r}
\le C\delta^{\alpha+\frac 1p-\frac 1q}
\tag 4.2
$$
for suitable $\theta\in [0,1]$ and small $c$.

In the present relevant case  $1/p+1/q> 1$ we distinguish three subcases
$$
\align
&(a_2-1)(1-\frac 1p)-\frac 1q\ge  0 \quad\text{ and }\quad
\frac{a_1-1}{p}+\frac 1{q}<a_1-1,
\tag 4.3.1
\\
&(a_2-1)(1-\frac 1p)-\frac 1q\ge 0 \quad\text{ and }\quad
\frac{a_1-1}{p}+\frac 1{q}\ge a_1-1,
\tag 4.3.2
\\
&(a_2-1)(1-\frac 1p)-\frac 1q< 0.
\tag 4.3.3
\endalign
$$

We begin by assuming that the third estimate (4.3.3) holds. 
Here we check (4.2) with $\theta=0$; 
by Proposition 3.3 the desired estimate  holds if
$$\align
&\sigma>\frac 1{a_2-2} \tag 4.4
\\&\frac \sigma r\le \alpha+\frac 1p-\frac 1q.
\tag 4.5
\endalign
$$
It is easily checked that (4.4) is equivalent  to (4.3.3) which is 
presently assumed and (4.5) is equivalent to the assumption (1.15.3).


Next we assume that the inequalities (4.3.2) hold. 
In order to carry out the integration in
(4.2) we have to assume
that
$\sigma(1-\theta)<(a_2-1)^{-1}$ which is equivalent to saying that
$\theta$ is larger than the critical value
$$\thcr=
\frac 1{a_2-2}\Big(a_2-1-\frac p{(p-1)q}\Big).
\tag 4.6
$$
Under the conditions 
$(a_2-1)(1-\frac 1p)-\frac 1q\ge 0$ (i.e. in (4.3.1) and (4.3.2))
 we have that $\thcr\ge 0$;
moreover one can check that the assumption $1/p+1/q>1$ is equivalent with
$\thcr<1$. We may therefore choose $\theta=\thcr+\eps<1$ where $\epsilon$
is small.


Let $\Omega_\ell=\{w:2^{-\ell-1}\le\rho(y)\le 2^{-\ell}\}$.
By Propositions 3.3 and 3.4 we estimate
$$
\align
&\delta^{-\alpha-\frac 1p+\frac 1q}
\Big(\int_{\Omega_\ell}
\Big(\frac{\delta}{\sqrt{\det\Phi''(w)}}\Big)^{\sigma(1-\theta)}
|\cB(w,\delta)|^{\sigma\theta}dw\Big)^{1/r}
\\&\qquad\le C
\delta^{-\alpha-\frac 1p+\frac 1q+\frac{\sigma(1-\theta)}{r}+
(\frac 12+\frac 1{a_2})\theta\frac{\sigma}{r}}
2^{\frac \ell r(-\nu+\sigma(1-\theta)(1-\nu)+
(\frac 12+\frac 1{a_2}-\nu)\sigma\theta)}
\tag 4.7
\endalign
$$
Now one computes
$$
\frac 1r\Big(
-\nu+\sigma(1-\theta)(1-\nu)+
(\frac 12+\frac 1{a_2}-\nu)\sigma\theta\Big)
=\frac{a_1-1}{a_1}-\frac 1p(\frac {a_1-1}{a_1})-\frac 1{a_1 q}
-\frac{\eps}{p'}\frac{a_2-2}{a_2}
$$
so that (4.3.2) implies the sum  $\sum_{\ell>0} 2^{\ell(\dots)} $ in  (4.7)
 converges.
Moreover
$$
-\alpha-\frac 1p+\frac 1q+\frac{\sigma(1-\theta)}r+
(\frac 12+\frac 1{a_2})\theta\frac{\sigma}{r}=
\widetilde\alpha-\eps(\frac{a_2-2}{a_2} p')-\alpha
$$
where
$$\align 
\widetilde\alpha&=2-\frac 3p+\frac 1q-\thcr\frac{a_2-2}{a_2p'}
\\
&=\frac{a_2+1}{a_2}-(2+\frac 1{a_2})\frac 1p+(1+\frac 1{a_2})\frac 1q.
\endalign
$$
Therefore if (4.3.2) is satisfied we can choose $\eps=\theta-\thcr$ 
so small that
the exponent of $\delta$  in (4.7) becomes nonnegative. This settles the estimate in case (4.3.2).

Finally assume that (4.3.1) holds, and again choose $\theta=\thcr+\eps$.
The assumption
$\frac{a_1-1}{p}+\frac 1{ q}<a_1-1$ implies that the terms
$2^{\ell(\dots)}$ in (4.7) form   an increasing geometric
progression if $\eps>0$ is 
chosen small enough. 
We compute
$$
\align& 
\delta^{-\alpha-\frac 1p+\frac 1q}\sum_{2^{-\ell}\ge C_2\delta}
\Big(\int_{\Omega_\ell}
\Big(\frac{\delta}{\sqrt{\det\Phi''(w)}}\Big)^{\sigma(1-\theta)}
|\cB(w,\delta)|^{\sigma\theta}dw\Big)^{1/r}
\\&\qquad\le C
\delta^{-\alpha-\frac 1p+\frac 1q
+\frac{\sigma(1-\theta)}{r}
+(\frac 12+\frac 1{a_2})\theta\frac {\sigma}{r}}
%\sum_{2^{-\ell}\ge C_2\delta}
\delta^{- (-\frac{\nu}r+
\frac{\sigma}{r}(1-\theta)(1-\nu)+(\frac 12+\frac 1{a_2}-\nu)\frac {\sigma}{r}
\theta)}
\\&\qquad= C
\delta^{-\alpha-\frac 1p+\frac 1q+\nu\frac{\sigma+1}{r}}
=C\delta^{-\alpha+\nu-\frac {\nu+1}p
+\frac{\nu+1}q}.
\endalign
$$
We have  proved the 
asserted estimate in the remaining case (4.3.1).\qed
\enddemo



%\subheading{Some examples}
\head{\bf 5. Some Examples}\endhead
As pointed out before Theorems 1.1 and 1.4 are sharp for the surfaces
given as a graph
$x_3=x_1^{a_1}+x_2^{a_2}$. We now discuss a class examples for which the
multitype does not suffice to get the best possible results.
In order to prove improved  $L^p\to L^q_\alpha$ results we shall
directly apply Theorem 2.5. 


\subheading{Maximal operators}
Let $\Sigma\subset \Bbb R^3$ 
be the graph of
$$
\Phi(x)=x_1^2+x_2^4+x_1^2x_2^2-c_2
\tag 5.1
$$
over the set $|x_1|+|x_2|\le 1/4$
and consider the averages  (2.14), 
with $\chi$ supported where $|x_1|+|x_2|\le 1/8$.
The Hessian
$$\det \Phi''=4x_1^2+24 x^2_2(1+x_2^2)-16x_1^2x_2^2$$
is nonnegative in the support of $\chi$ 
and since $\text{trace}(\Phi'') \ge 1$ we see that
$\Phi''$  has two positive eigenvalues away from $0$. 
Therefore  $\Phi$ is convex, of multi-type $(2,4)$ at  $0$ and of type
$2$ at $(x_1,x_2)\neq 0$ near  $0$.
The sufficient condition for $L^p$ boundedness which only 
depends on the multitype yields boundedness  for $p>8/5$,
by Theorem 1.1 (iii).
However $|\det \Phi''|^{-1+\eps}$  is integrable near $0$, for
all $\eps>0$, and therefore we obtain $L^p$ boundedness for $p>3/2$, which 
the best possible result.


More generally we  consider
$$\Phi(x)= x_1^2+x_2^M+x_1^ax_2^b-c_2
\tag 5.2$$
where $a$ and $b$ are positive even integers
with $a/2+b/M>1$. The graph of $\Phi$ is convex near the origin and the
 multitype  at $(0,0)$ is $(2,M)$. 
Therefore, if the cutoff function $\chi$ has small support one
ceratinly  obtains boundedness for $p>2(M+1)/(M+2)$.
One computes
$$\det \Phi''(x) =c x_2^{M-2}+d x_1^ax_2^{b-2}+
o( x_2^{M-2}+ x_1^ax_2^{b-2})
$$
with $c,d>0$. Then for small $\eps$
$$\int_{|x|\le \eps} [\det\Phi'']^{-\gamma} dx<\infty
$$
if
$$\gamma<\gamma_{\text{cr}}=
\min\big\{\frac 1{b-2}, \frac 1{M-2}+\frac{M-b}{a(M-2)}\big\}
$$
Note that $\gamma_{\text{cr}}>(M-2)^{-1}$ if $b<M$. In this case
part (ii) of 
Theorem 1.1
gives us $L^p$ boundedness for
 $p>p_0$ where the critical exponent $p_0$ is less than
$2(M+1)/(M+2)$.                                                                 
\subheading{{\bf $\boldkey{L}^{\boldkey p}\to \boldkey{L}^{\boldkey q}$-estimates}}
Consider again the example (5.1).
Let $Q_0=(6/5, 1/2)$, $Q_0^*=(1/2, 1/6)$,
 and $R=(5/7, 2/7)$.
Then the result of Theorem 1.4 implies $L^p\to L^q$ boundedness in the interior
of the 
convex hull of the points $(0,0)$, $(1,1)$, $Q_0$, $Q_0^*$ and $R$.


Let $\ell$ be the line 
$2-3/p+1/q=0$ and let
$\sigma$
 be the lower edge of the boundary of the boundedness region
which contains the point $(1,1)$. All points on $\ell$ with
abscissae
$1/p\in [5/6,1]$ belong to $\sigma$. We shall show that 
this segment is in fact longer and 
thereby obtain a larger boundedness region.
We use the estimate (2.10) with $k=2$ and $\nu_{k+1}=0$.
$L^p$ to $L^q$ boundedness ($p\le q$, $p\le 2$) holds by Theorem 1.3 if
$$ \delta^{\frac 1q-\frac 1p+\frac{2q(p-1)}{p+q-pq}
(\frac 1p+\frac 1q-1)}
\Big(\int_\tSi|\det \Phi''|^{-\frac{q(p-1)}{p+q-pq}}dx
\Big)^{\frac 1p+\frac 1q-1}<\infty
$$
and the exponent of $\delta$ is positive. The last requirement is 
equivalent to the restriction 
$2-3/p+1/q>0$. Since $|\det \Phi''|^{-\gamma}$ is integrable for $\gamma<1$
we obtain boundedness if the restriction
$\frac{q(p-1)}{p+q-pq}<1$ is satisfied. A computation shows that
all points on $\ell$ with
abscissae $1/p \in [4/5,1]$ belong to $\sigma$.
Therefore 
if $(1/p, 1/q)$ 
belongs to the interior of the pentagon with
vertices
 $(1,1)$, $(4/5, 2/5)$, $(5/7,2/7)$, $(3/5, 1/5)$ and $(0,0)$
then
the averaging operator maps $L^p$ to $L^q$.
Similar considerations yield improved $L^p\to L^q_\alpha$ estimates.


We remark that the preceding $L^p\to L^q$ estimates for the example
in (5.1)  could also 
be obtained by a scaling argument in the spirit of
\cite{15}; one uses isotropic dilations since the curvature vanishes at 
an isolated point. The rescaled operators  can be embedded in
analytic  families and the estimations are  variants of those in \cite{17}.






\Refs
%\nofrills{REFERENCES}


\ref\no  1
\by J. Bourgain
\paper Averages in the plane over convex curves and maximal operators
\jour J. Analyse Math.
\vol 47
\yr1986
\pages 69--85
\endref

\ref\no  2 \by J. Bruna, A. Nagel and S. Wainger
\paper Convex hypersurfaces and Fourier transform
\jour Ann. Math.
\vol 127
\yr 1988
\pages 333--365
\endref

\ref\no 3 \by D. Catlin\paper Boundary invariants of pseudoconvex domains
\jour Ann. Math. \vol 120
\yr 1984\pages 529--586\endref


\ref\no 4\by M. Christ\paper
Hilbert transforms along curves, I. Nilpotent groups
\jour Ann. Math.\vol 122\yr 1985\pages 575--596\endref

\ref \no 5\bysame
\paper Endpoint bounds for singular fractional integral operators
\jour preprint 1988
\endref

\ref\no 6\bysame\paper
 Failure of an endpoint estimate for integrals along curves
\inbook Fourier analysis and partial differential equations
\bookinfo  ed. by J. Garcia-Cuerva, E. Hernandez, F. Soria and J. L. Torrea
\publ CRC Press \yr 1995
\endref

\ref \no 7  \by M. Cowling and G. Mauceri
\paper Inequalities for some maximal functions II
\jour Trans. Amer. Math. Soc.
\vol 296 \yr 1986 \pages 341--365
\endref


\ref \no 8  \bysame
\paper Oscillatory integrals and Fourier
 transforms of surface carried measures
\jour Trans. Amer. Math. Soc.
\vol 304
\yr 1987
\pages 53--68
\endref


\ref\no  9 \by M. Cowling, S. Disney, G. Mauceri and D. M\"uller
\paper Damping oscillatory integrals
\jour Invent. Math
\vol 101 
\yr 1990
\pages 237--260
\endref




\ref\no 10 \by E. Ferreyra, T. Godoy and M. Urciuolo
\paper Endpoint bounds
for convolution operators with singular measures\jour 
Coll. Math. \vol 76\yr 1998\pages 35--47\endref



\ref\no  11 \by A. Greenleaf
\paper Principal curvature in harmonic analysis
\jour Indiana Math. J.
\vol 30
\year 1981
\pages 519--537
\endref


\ref\no 12  \by A.\ Iosevich
\paper Maximal operators associated to families of flat curves in the
plane
\jour Duke Math. J.
\vol 76
\year 1994
\pages  633-644
\endref

\ref\no  13 \by A. Iosevich and E. Sawyer
\paper Oscillatory integrals and maximal averages over homogeneous
surfaces
\jour Duke Math. J.
\vol 82
\year 1996
\pages 103--141
\endref

\ref\no  14 \bysame
\paper Maximal averages over surfaces
\jour Adv. Math.\vol 132\yr 1997\pages 46--119
\endref

\ref\no   15\bysame
\paper Sharp $L^p\to L^q$ estimates for a class of averaging operators
\jour Ann. Inst. Fourier \vol 46\yr 1993\pages 903--927
\endref

\ref\no 16\by S. H. Lee\paper Convolution operators with singular measures
\jour preprint\endref




\ref\no 17\by W. Littman \paper $L^p-L^q$-estimates for singular integral 
operators \jour Proc. Symp. Pure and Appl. Math. Amer. Math. Soc.
\vol 23 \yr 1973 \pages 479--481
\endref

\ref\no 18 \by A. Nagel, A. Seeger and S. Wainger
\paper Averages over convex hypersurfaces
\jour Amer. J. Math.
\vol 115
\yr 1993
\pages 903--927
\endref


\ref\no 19\by F. Ricci and E. M. Stein
\paper Harmonic analysis on nilpotent groups and singular
integrals I. Oscillatory integrals
\jour J. Funct. Anal. \vol 73\yr 1987\pages 179--194
\endref

\ref\no 20 \by H. Schulz
\paper Convex hypersurfaces of finite type and the asymptotics of their
Fourier transforms
\jour Indiana U. Math. J.
\vol 40 
\yr 1991
\pages 1267--1275
\endref

\ref\no 21  \by C. D. Sogge and E. M. Stein
\paper Averages of functions over hypersurfaces in $\R^n$
\jour Invent. Math.
\vol 82
\yr 1985
\pages 543--556
\endref

\ref\no 22 \by E. M. Stein
\paper Maximal functions: Spherical Means
\jour Proc. Nat. Acad. Sci. USA
\vol 73
\yr 1976
\pages 2174--2175
\endref

\ref\no 23 \bysame
\book Harmonic Analysis
\publ Princeton University Press
\yr 1993
\endref


\ref \no 24 \by E. M. Stein and S. Wainger
\paper Problems in harmonic analysis related to curvature
\jour Bull. Amer. Math. Soc. \vol 84
\yr 1978 \pages 1239--1295
\endref

\ref\no 25\by R. Strichartz\paper 
Convolutions with kernels having singularities
on a sphere
\jour Trans. Amer. Math. Soc. \vol 148\yr 1970\pages 461--471\endref

\ref\no 26\by J. Yu\paper Multitypes of convex domains
\jour Indiana Math. J. \vol 41\yr 1992\pages 837--849\endref
\endRefs
 
\enddocument


