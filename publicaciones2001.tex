%% September 2000
\documentstyle{amsppt}
\pagewidth{6in}\vsize8.5in\parindent=6mm\parskip=3pt\baselineskip=14pt
\tolerance=10000\hbadness=500
%\magnification=1100
\NoRunningHeads
%\NoLogo
\loadbold
\topmatter
\title
Two problems associated with convex finite type domains 
\endtitle
\author
Alexander Iosevich \ \  Eric Sawyer \ \ Andreas Seeger
\endauthor
%\date \enddate
\thanks Research supported in part by  NSF grants.  \endthanks
\address
%Wright State University, Mathematics and Statistics Department,
%Dayton OH 45435
A. Iosevich, 
% Department of Mathematics, Georgetown University,
%Washington, DC 20057, USA
%\endaddress
%\curraddr  
Mathematics Department, University of Missouri, Columbia, MO 65211, USA
%\endcurraddr
\endaddress
\email iosevich\@math.missouri.edu \endemail
\address 
E. Sawyer,
%Indiana University - Purdue University Indianapolis,
%Department of Mathematical Sciences, Indianapolis, IN 46202  ???
Department of Mathematics and Statistics, 1280 Main Street West,
 Hamilton, Ontario  L8S 4K1, Canada
\endaddress
\email 
%esawyer\@math.iupui.edu
sawyer\@mcmaster.ca \endemail
\address 
A. Seeger, Department of
 Mathematics, 
University of Wisconsin-Madison, 
Madison, WI 53706, USA
\endaddress 
\email seeger\@math.wisc.edu \endemail
%\subjclass\endsubjclass
\abstract
We use scaling properties of convex surfaces of finite line  type to 
derive new estimates for two problems arising in harmonic analysis.
For Riesz means associated to such surfaces we obtain sharp $L^p$ 
estimates for $p>4$, generalizing the Carleson-Sj\"olin theorem. 
Moreover we obtain estimates for the remainder term in the lattice point problem associated to  convex bodies; these estimates are sharp
in some instances involving sufficiently flat
boundaries.
\endabstract
\TagsOnRight
\endtopmatter
%\input defnew




\def\leaderfill{\leaders\hbox to 1em{\hss.\hss}\hfill}

\def\a{{\alpha}}
\def\R{{\Bbb R}}
\def\SS{{\Cal S}}
\def\MM{{\Cal M}}
\def\NN{{\Cal N}}
\def\HH{{\Cal H}}
\def\EE{{\Cal E}}
\def\CC{{\Cal C}}
\def\ZZ{{\Cal Z}}
\def\Z{{\Bbb Z}}
\def \tSi{{\widetilde \Sigma}}
\def\thcr{{\theta_{\text{cr}}}}

\define\eg{{\it e.g. }}
\define\cf{{\it cf}}
\define\Rn{{\Bbb R^n}}
\define\Rd{{\Bbb R^d}}
\define\sgn{{\text{\rm sign}}}
\define\rank{{\text{\rm rank }}}
\define\corank{{\text{\rm corank }}}
\define\coker{{\text{\rm Coker }}}
\redefine\ker{{\text{\rm Ker }}}
\define\loc{{\text{\rm loc}}}
\define\spec{{\text{\rm spec}}}
\define\comp{{\text{\rm comp}}}
\define\Coi{{C^\infty_0}}
\define\dist{{\text{\rm dist}}}
\define\diag{{\text{\rm diag}}}
\define\supp{{\text{\rm supp }}}
\define\rad{{\text{\rm rad}}}
\define\Lip{{\text{\rm Lip}}}
\define\inn#1#2{\langle#1,#2\rangle}        
\define\rta{\rightarrow}
\define\lta{\leftarrow}
\define\noi{\noindent}
\define\lcontr{\rfloor}
\define\lco#1#2{{#1}\lcontr{#2}}
\define\lcoi#1#2{\imath({#1}){#2}}
\define\rco#1#2{{#1}\rcontr{#2}}
\redefine\exp{{\text{\rm exp}}}
\define\bin#1#2{{\pmatrix {#1}\\{#2}\endpmatrix}}
\define\meas{{\text{\rm meas}}}

\define\card{\text{\rm card}}
\define\lc{\lesssim}
\define\gc{\gtrsim}
%Greek letters
\define\ga{\gamma}             \define\Ga{\Gamma}
\define\eps{\varepsilon}                
\define\ep{\epsilon}
\define\la{\lambda}             \define\La{\Lambda}
\define\sig{\sigma}             \define\Sig{\Sigma}
\define\si{\sigma}              \define\Si{\Sigma}
\define\vphi{\varphi}
\define\ome{\omega}             \define\Ome{\Omega}
\define\om{\omega}              \define\Om{\Omega}
\define\ka{\kappa}
\define\fa{{\frak a}}
\define\bbE{{\Bbb E}}
\define\bbR{{\Bbb R}}
\define\bbZ{{\Bbb Z}}
\define\cA{{\Cal A}}
\define\cB{{\Cal B}}
\define\cC{{\Cal C}}
\define\cE{{\Cal E}}
\define\cF{{\Cal F}}
\define\cJ{{\Cal J}}
\define\cL{{\Cal L}}
\define\cM{{\Cal M}}
\define\cN{{\Cal N}}
\define\cR{{\Cal R}}
\define\cS{{\Cal S}}
\define\cT{{\Cal T}}
\define\cU{{\Cal U}}
\define\cV{{\Cal V}}
\define\ic{{\imath}}
\define\tchi{{\widetilde \chi}}
\define\tka{{\widetilde \kappa}}
\define\txi{{\widetilde \xi}}
\define\teta{{\widetilde \eta}}
\define\tomega{{\widetilde \omega}}
\define\tzeta{{\widetilde \zeta}}
\define\tpsi{{\widetilde \psi}}
\define\tphi{{\widetilde \phi}}
\define\bom{{\partial \Om}}
\define\fV{{\frak V}}
\define\fS{{\frak S}}
\define\fA{{\frak A}}
\define\fB{{\frak B}}
\define\fC{{\frak C}}
\define\fD{{\frak D}}
\define\bbV{\Bbb V}

%We now turn to the question of $L^p$ convergence of 
%Riesz means associated to convex domains.
%\subheading{ 1. Introduction}
\head {\bf 1. Introduction}\endhead
 Let $\Omega$ be a convex domain in $\bbR^d$ with smooth boundary.
We assume that
$\partial \Omega$ is of {\it finite line type}, that is, at each point each tangent line has finite order of contact.

We discuss two problems in this paper.
Both problems  have in common that progress can be made using some approximate scaling  properties of $\bom$.
We derive an extension 
of the Carleson-Sj\"olin theorem concerning 
$L^p$ convergence results for Riesz means defined by a distance function 
associated  to $\Omega$; we assume  that $1\le p\le 4/3$.
We also  give asymptotics for the number of integer lattice points
inside large dilates of $\Omega$; the
bounds for the error terms are sharp in some cases where
all lines tangent to the boundary
have  high order of contact with $\partial\Om$.






\subheading {1.1. Riesz means}
We assume that the origin belongs to the interior of $\Omega$. Let $\rho:\Bbb R^d\to [0,\infty)$ be homogeneous of degree $1$
be the Minkowski functional associated to $\Omega$; {\it i.e.}  $\rho$ is 
homogeneous of degree one, so that $\rho(\xi)=1$ if $\xi\in\partial \Omega$.
The boundary 
$\Sigma_\rho:=\partial \Omega$ is then the  unit sphere for the generalized
distance function $\rho$.
The Bochner-Riesz operator associated to  $\rho$ is defined by
$$
\widehat{S_{\la,\rho}f}(\xi)=
(1-\rho(\xi))^\la_+
\widehat f(\xi);\tag 1.1
$$
here our definition of the Fourier transform is
$\widehat f(\xi)=\int f(y)e^{-\ic\inn{y}{\xi}} dy$.
It is well known that if $1\le p<\infty$ the  $L^p$ boundedness of the Bochner-Riesz operator 
implies $L^p$ convergence of the Riesz means
$\Cal F^{-1}[(1-\rho/t)^\la_+\widehat f]$
to the limit $f$ if $f\in L^p$ and $t\to\infty$.

A necessary condition for $L^p$ boundedness is
$$\la>\la(p)=d|1/p-1/2|-1/2
\tag 1.2$$
Indeed in view of the compact support of the multiplier 
it is necessary for $L^p$ boundedness that the inverse Fourier transform of 
$(1-\rho)^\la_+$ belongs to $L^p$. Using standard asymptotic expansions
one can show (working near points on $\Sigma_\rho$ where the 
curvature  does not vanish) that (1.2) is necessary for 
$\Cal F^{-1}[(1-\rho)^\la_+]\in L^p$.

It is known \cite{9}, \cite{29} that the validity of an $L^2$ restriction  theorem 
 for the 
Fourier transform
implies the $L^p$ boundedness of the Bochner-Riesz operator.
 Since $\Sigma_\rho$ is of finite type, 
say $\le n$, it follows from \cite{3} that the Fourier transform of
$\widehat{d\sigma}(\xi) $ of a smooth density carried by  $\Sigma_\rho$ is
$O(|\xi|^{-\mu})$ for some $\mu$ with  $\mu\ge (d-1)/n$.
Using  the appropriate versions of the
 Stein-Tomas restriction theorem (\cite{10}) one can show that
 $L^p$ boundedness holds for
 $1\le p\le 2(\mu+1)/(\mu+2)$ and $\la>\la(p)$ ({\it cf.} \cite{29}). Note that
$2(\mu+1)/(\mu+2)=(2n+2d-2)/(2n+d-1)$ for the 
example $x_d=\sum_{i=1}^{d-1} x_i^n$ with even $n$, so that the range obtained in this 
way is small for large $n$.

\proclaim{Theorem 1.1} Suppose that
$d\ge 2$, $1\le p\le 4/3$, $\la>d(1/p-1/2)-1/2$  and that
 $\Sigma_\rho$ is of finite line type. Then $S_{\la,\rho}$ is 
bounded on $L^p(\Bbb R^d)$.
\endproclaim


It is conjectured that $L^p$ boundedness holds for the same range of exponents
as for the sphere.
The conjecture for the sphere is that $L^p$ boundedness should hold
for $\la>\la(p)$ for 
$p<2d/(d+1)$. This is currently known only in two dimensions,
see Carleson and Sj\"olin
\cite{4}.
Sj\"olin \cite{28} extended this result  to arbitrary planar domains with
 smooth boundary, for some 
 variants concerning convex  domains in the plane 
 with nonsmooth boundary see also the more  recent paper by Ziesler and the 
third author \cite{27}. 
For partial results 
in higher dimensions, in the case that the 
 Gau\ss \ curvature of $\Sigma_\rho$ 
does not vanish,
we refer to Bourgain \cite{1} and for background to
\cite{29}. Our proof of  Theorem 1.1  uses a variant of C\'ordoba's 
 geometrical proof \cite{6} of the Carleson-Sj\"olin theorem and rescaling.


%%%%\proclaim{Theorem 2} Suppose that $d=3$ and $\partial \Omega=\Sigma_\rho$ 
%%%%is of finite line type. Then there is $p_0>4/3$ so that
%%%%$S_{\la,\rho}$ is 
%%%%bounded on $L^p(\Bbb R^3)$ for $1\le p<p_0$, provided 
%%%%that  $\la>3(1/p-1/2)-1/2$.
%%%%\endproclaim

\subheading{1.2 Multitype and an estimate for the Fourier transform of 
surface carried measure}

A precise estimate  of the  Fourier transforms of surface carried measure is 
due to Bruna, Nagel and Wainger \cite{3}. Let $\Sigma=\bom$ and 
$H_P(\Sigma)$  the affine tangent plane at $P\in \Sigma$, and let
$$
B(P,\delta)=\{y\in \Sigma: \dist(y,H_P(\Sigma))<\delta\}.
\tag 1.3
$$
Then $$|\widehat {d\sigma}(\xi)|\le C \big[
|B(P_+,|\xi|^{-1})|+|B(P_-,|\xi|^{-1})| \big]
\tag 1.4
$$
where $P_\pm$ are the points on $\Sigma$ for which $\xi$ is a
normal vector and  $|B|$ denotes  the surface measure of $B$.
For many problems it is important to know not just the size of the 
balls but also  the distribution function  of $x\mapsto |B(x,\delta)|$
and how it relates to the notions of multitype and type.
We review the definition of multitype which is implicit in 
Schulz \cite{26}, see also \cite{17}.

Consider a smooth real valued
 function $\Phi$ defined in a neighborhood of the origin in
a  $d-1$-dimensional Euclidean  vector space
$\Bbb E^{d-1}$  so that $\Phi(0)=\nabla \Phi(0)=0$.
We say that a vector
 $v$ in $\Bbb E^{d-1}$  has contact of order $n+1$ if
$$
\Phi(sv)=O(s^{n+1})\quad \text {if }\quad  s\to 0.
$$
The sets  $$S^n=\{v\in \Bbb E^n: \text{ $v$ has contact of order $n+1$}\}
\tag 1.5 $$
are  linear subspaces of 
$\Bbb E^{d-1}$ and   there are   even integers $m_1,\dots, m_k$ so that
$m_1<\dots<m_k$, $1\le k\le d-1$ and $m_0:= m_1-1\ge 1$ and
$$
0=S^{m_k}\subsetneq\dots \subsetneq S^{m_0}:=\Bbb E^{n};
\tag 1.6
$$
moreover the sequence is maximal, in the sense that
$S^n=S^{m_{k}}$  if $ m_{k-1}< n\le  m_k.$
Define
%Let $\dim S^{m_i}=n_i$, so that $n_0=n$ and $n_{k}= 0$ and define
$$
a_i=m_j \quad\text{  if } d-1-\dim S^{m_{j-1}}<i\le d-1-\dim S^{m_j},\,\,j=1,\dots,k.
\tag 1.7
$$
 The  $d-1$-tuple 
$\fa=(a_1,\dots, a_{d-1})$ is then called the  multitype
of $\Phi$ at $0$. 
%Clearly this definition is independent of  the {\it linear}
% coordinate system on $\Bbb E_n$.
%%%Associated with it comes a {\it flag} of subspaces


We now fix  $P\in \Sigma$, choose a unit normal
$n_P$ and parametrize $\Sigma$ near $P$ as  a graph over 
its tangent plane at $P$. Thus the parametrization is given by
$$v=\Gamma(v)\mapsto P+v+ \Phi(v) n_P
\tag 1.8
$$ for $v\in T_P\Sigma$, and 
 $\Phi$ is a convex function  vanishing of second order at the origin.
We perform the above construction for $\Phi(v)$  defined on 
$\bbE^{d-1}=T_P\Sigma$ and obtain a flag of subspaces
$$0=S^{m_k}_P\subsetneq\dots \subsetneq S^{m_0}_P= T_P\Sigma. 
\tag 1.9
$$
%With it comes a dual flag
%$$T_P^*\Sigma=(S^{m_k}_P)^*\supsetneq\dots \supsetneq (S^{m_0}_P)^*= 0
%$$
%
Let  $W_j$ be the orthogonal complement of
$S^{m_{j}}_P$ in $S^{m_{j-1}}_P$, $j=1,\dots,k$, then
$$
T_P\Sigma= W_1\oplus\cdots\oplus W_{k}
\tag 1.10
$$
We denote by $\Pi_j^P$ the orthonormal projection on $T_P\Sigma$  to
$W_j$.
We also have a similar decomposition and projections $\Pi_j^P$ to $W_j^*$
on $T^*_P\Sigma$, here we let $W_j^*$ the space of linear functionals 
on $W_j$ extended by $0$ on the orthogonal complement of $W_j$.
We can extend these projections to linear maps  on 
$T_P^*\Bbb R^d\simeq (\bbR^d)^*$ by
defining $\Pi_j^P n_P=0$.


On $T_P^*\Sigma$ we define a nonisotropic  distance function $\rho_*$ by
$$
\rho_*(\eta)=
\sum_{j=1}^k 
|\Pi_j^P\eta|^{\frac{m_j}{m_j-1}};
\tag 1.11
$$
here $|\cdot|$ denotes the Euclidean distance in $W_j$.
If $\xi\in T_P^*\Bbb R^d$ is taken from a suitable 
 conic neighborhood of  $n_P$ and $\Pi^P$ denotes the  projection
to $T_P^*\Si$ 
 we define
 $$\Theta_P(\xi)=\rho_*\big(\frac{\Pi^P \xi}{\inn{\xi}{n_P}}\big).
\tag 1.12
$$
Finally   we set  for $l\le d-2$
$$\nu_l(P)= \sum_{i=l}^{d-1} a_i^{-1}=\sum_{j=1}^k 
\frac {\dim S_P^{m_{j-1}}-\dim S_P^{m_j}}{m_j}
\tag 1.13
$$
and write $\nu(P)\equiv \nu_1(P)$. An alternative description of $\nu(P)$ 
 (see \cite{16}) is
$$\nu(P)=\sup\{q:\dist(\cdot,H_P\Sigma)\in L^q(\Sigma)\};
\tag 1.14
$$
in fact for $q=\nu(P)$ the function $\dist(\cdot,H_P\Sigma)$ belongs to
the space $ L^{q,\infty}(\Sigma)$.

Our result for the Fourier transform of surface carried measure is 

\proclaim{Proposition 1.2}
Let $P\in \bom$.
% and  suppose that all principal curvatures of $\bom$ 
%vanish  at $P$. 
Then there is a neighborhood $U$ of $P$ and a conic 
neighborhood $V$ of  $\{\pm n_P\}$ in $\bbR^d$ so that for all 
$\chi\in C^\infty_0(U)$ and all $\xi\in V$ with $|\xi|\ge 1$ we have
$$|\widehat{\chi d\sigma}(\xi)|\lc\|\chi\|_{C^{N}}\,
\min\{|\xi|^{-\nu},   |\xi|^{-\frac 12-\nu_2} [\Theta_P(\xi)]^{\nu-\nu_2-\frac 12}\};
$$
here $\|\chi\|_{C^N}=\max_{\alpha\le N}\|\chi^{(\alpha)}\|_{L^\infty(U)}$ and $N$ is sufficiently large.
\endproclaim

In this statement  $N>d+m_k$ will suffice.


\subheading{1.3. A lattice point estimate}

Let $$\cN_\Om(t)=\card(t\Omega\cap\bbZ^d).
\tag 1.15
$$
It is well known  (and elementary) that $\cN_\Om(t)$ is asymptotic
to $t^d\text{vol}(\Om)$
as $t\to \infty$ and that the error term
$$ 
E_\Om(t)=\cN(t)-t^d\text{vol}(\Om)
\tag 1.16
$$
as $O(t^{d-1})$.
Moreover if $\bom$ has suitable curvature properties then the error term 
improves; in particular if the
Fourier transform of the surface measure on the boundary satisfies
$\widehat {d\sigma}(\xi)=O(|\xi|^{-\alpha})$ then  the classical method
(see  {\it e.g.} \cite{11},
 \cite{13, Theorem 7.7.16} and \cite{24}) yields 
$E_\Om(t)=O(t^{d-1-\frac{\alpha}{d-\alpha}})$.
This estimate however is not sharp, and 
several authors beginning with van der Corput have obtained  
improvements for the case of nonvanishing
 Gau\ss \ curvature; see the monographs by Kr\"atzel \cite{18}
and Huxley \cite{14}, and in  particular 
the papers by  Kr\"atzel and Nowak \cite{20} and recent improvements by
W.  M\"uller \cite{22}
for results on general convex bodies with nonvanishing curvature
 in higher dimensions.
In \cite{24, I}, \cite{25} Randol  obtained better estimates for the case 
of convex domains in the plane with  finite type boundary;
these are sharp for 
$\Omega=\{ x:x_1^k+x_2^k\le 1\}$ where $k\le 4$ is even. See also 
Nowak \cite{23} for more refined results.
Generalizations to domains of the form 
$\Omega=\{ x:x_1^k+...+x_d^k\le 1\}$ are in \cite{24, II}, \cite{19}.


Here we give a version for general convex bodies with finite type boundary
in higher dimensions. Let $\nu(P)=\nu_1(P)$ and  $\nu_2(P)$ as in (1.13) above.



\proclaim{Theorem 1.3}
%Let  $\Gamma$ be the set of points in $\partial\Omega$ at which all 
%principal curvatures vanish. Let
Let $$\nu=\min_{P\in\bom} \nu(P), \qquad 
\mu= \frac 1 2+ \min_{P\in  \bom} \nu_2(P).$$
Then there is a constant $C$ depending on $\Omega$ so that
$$
|E_\Omega(t) |\le C_\Om (1+ t^{d-1-\nu}+ t^{d-1-\frac{\mu}{d-\mu}}).
\tag 1.17
$$

Specifically, if $\Gamma$ is the set of all points $P\in \bom$ 
at which all principal curvatures vanish then
$$
E_\Omega(t)= \sum_{P\in \Gamma} t^{d-1-\nu(P)} G_P(t) + 
O(t^{d-1-\frac{\mu}{d-\mu}})
\tag 1.18
$$
where $G_P(t)$ is bounded as $t\to\infty$.
If the normal line determined by $n_P$ coincides with
$\Bbb Re_i$ for some $i\in\{1,\dots, d\} $ then  
$\limsup_{t\to \infty} |G_P(t)|>0$.
%
%Finally for almost all rotations $A\in SO(d)$ the estimate
%(1.17)  for the 
%error term improves, in the sense that
%$$
%|E_{A\Omega}(t)|\le C(A) 
%t^{d-1-\frac{\mu}{d-\mu}} 
%\tag 1.19
%$$ and $C(A)<\infty$ for almost every rotation.
\endproclaim

We note that 
the number $\mu/(d-\mu)$ is greater then
$(2d-1)^{-1}$ since $\mu>1/2$.
In  particular if the Gau\ss \ curvature only vanishes 
at one point at the surface and if 
$\nu<\mu/(d-\mu)$ then 
there is  $A\in SO(d)$ 
so that
$\limsup_{t\to \infty} t^{\nu-d+1} |E_{A \Om}(t) |$
is  positive (for other model cases  compare \cite{19}, \cite{23}). 
Note that the sum in (1.18) over $P\in \Gamma$, since $\Gamma$ is a discrete subset of 
$\bom$ (as noted in \cite{16}, {\it cf.} the proof of Lemma 2.2. below).
We remark that it is well known that for almost all rotations $A\in SO(d)$ the error terms 
$E_{A\Om}(t)$ improve, see
  \cite{5}, \cite{31},
\cite{32},
\cite{23}, \cite{15} and \cite{2}.



{\it Notation:} 
Given two quantities $A$, $B$ we write $A\lesssim B$ if there is a absolute positive constant
$C$ so that $A\le C B$.
We write $A\approx B$ if $A\lc B$ and $B\lc A$.




\head{\bf 2. An estimate for Fourier transforms of surface carried measures}
\endhead

We begin by reviewing  some facts about classes of convex functions in 
\cite{3}, \cite{26}, \cite{16}, \cite{17}.

Let    $B_{T}\subset \Bbb R^{n}$  denote the open ball  
of radius $T$ centered at $0$;
it is always assumed that 
$T\le 1$.

Fix a flag $\fV$ 
 of subspaces $0=\bbV_k\subsetneq \dots\subsetneq \bbV_0$
 of $\Bbb E^{d-1}$, with
$\bbV_0=\Bbb E^{d-1}$,
and let 
$m=(m_1,\dots,m_k)$ be a $k$-tuple of even positive integers with
$m_1<\dots<m_k$.
For $0<b\le M$,  $N\in\Bbb Z^+$, $N>m_k$,  let
 $\fS_T^{d-1}(b,M,\fV, m,N)$ 
be the class of all $C_N(\overline{B_T})$ functions $g$
 with the property
that 
$$
\aligned
&g(0)=\nabla g(0)=0
\\
&\frac{d^2}{(dt)^2} g(x+t\theta)\big|_{t=0}\ge 0 \text{ for all }
\theta\in S^{d-2}, x\in B_T
\\
%&\max_{x\in B_T}
&\max_{2\le j\le m_l} \Big|\big(\frac {d}{dt}\big)^j g(x+t\theta)
\big|_{t=0}\Big|\ge 
b \text{ for all }
\theta\in S^{d-2}\cap \bbV_{l-1}, x\in B_T
\\
%&\max_{x\in B_T}
& \max_{|\alpha|\le N} \Big|\big(\frac{\partial}{\partial x}
\big)^\alpha g(x)\Big|\le M
\text{ for all } x\in B_T.
\endaligned
\tag 2.1
$$
Here $S^{d-2}$ denotes the unit sphere in $\bbE^{d}$.
We also define
$\fa(\fV,m)= (a_1(\fV,m),\dots, a_l(\fV,m))$ by
$$
a_i(\fV,m)=m_j(\fV,m) \quad \text{ if } d-1-\dim\bbV_{j-1}<i\le d-1-\dim\bbV_j,
\tag 2.2
$$
in analogy to (1.7).

Now if
 $P\in \Sigma$ (with $\Sigma=\bom$ as in the introduction) and
$\bbE^{d-1}=T_P\Sigma$ 
then let $\bbV_j=S^{m_j}_P\subset T_P\Sigma$ as in  (1.5).
 Let $\Phi$ be as in (1.8). Then there is $T>0$ and a neigborhood $\cU$ of 
$0$ so that for all $w\in \cU$
 the functions $y\mapsto \Phi(w+y)-\Psi(w)- \inn{y}{\nabla_{w}\Phi(w)}$ 
 belong to $\fS_T^n(b,M,\fV, m,N)$; moreover there are positive constants 
$c_0,C_0, C_1$ so that 
$$\cB(w,\delta)= \{y: |\Phi(y)-\Phi(w)-\inn{\nabla_w \Phi(w)}{y-w}|\le \delta
\}
\tag 2.3$$
belongs to $B_T$ if $\delta\le c_0 T^{m_k}$ and satisfies
$$\meas(\cB(w,\delta))\le C\delta^{\nu};\tag 2.4$$
see Proposition 2.1  in \cite{17}.



\proclaim{Lemma 2.1} Suppose that $\Phi
\in\fS_T^{d-1}(b,M,\fV, m,N)$ and suppose that $\fa=(a_1,\dots, a_{d-1})$
is the multitype at the origin. 
Let $\Psi^w(y)=\Phi(y)-\Phi(w)-\inn{\nabla_w \Phi(w)}{y-w}$ and 
let $\fa(w)= (a_1(w),\dots, a_{d-1}(w))$ be the multitype of 
$\Psi^w$ at the origin. Then there is a neighborhood $\cU$ of the origin so 
that  $a_i(w)\le a_i$ for $i=1,\dots, d-1$ and  all 
$w\in\cU$.
%%, moreover
%%$a_1^w=2$ if $w\in \cU\setminus\{0\}$.
\endproclaim

\demo{Proof} Let $S^{m_i} $ be as in (1.5) and let  $\ell>\dim S^{m_i}$. Recall that $S^{n}=S^{m_{j-1}}$ for $m_j<n\le m_{j-1}$.
Using continuity and compactness arguments together with the definition of 
the spaces $S^{m_i}$   we see that there  is a   
  neighborhood $\cU\subset \widetilde \cU$ of the origin  so that for every 
$w\in \widetilde \cU$, every $y\in \cU$  and every 
$\ell$-tuple of orthonormal vectors 
$\{u_1,\dots, u_\ell\}$
$$
\sum_{i=1}^\ell \sum_{s\le m_j}\big|(\inn{u_i}{\nabla_y})^s\Psi^w(y)\big|\ge b_0>0.
\tag 2.5
$$
The result of the Lemma follows quickly from the definition of 
the multitype.\qed
\enddemo

We now let $\Sigma$ denote the graph of $\Phi$. On $T_0\Sigma=\bbR^{d-1}$ we define a nonisotropic  distance function $\rho$ by
$$
\rho(y)=
\sum_{j=1}^k 
|\Pi_jy|^{m_j};
\tag 2.6
$$
note that that the unit ball for $\rho^*$ in (1.11) is the polar set for 
the unit ball  for $\rho$.

 The following Lemma gives an improvement of estimates in 
\cite{16} and  \cite{17}.  A rescaling argument is used
as in those papers; the present improvement  is obtained using
a more careful argument for the rescaled pieces.

%%, the argument in Proposition 3.4 of the latter paper 
%%can be used to reach the  same conclusion 
%%under the more restrictive assumption $\alpha\le 1/2 +(d-1)/m_k$.


\proclaim{Lemma 2.2} Let $\Phi$ be a convex smooth function 
defined in a neighborhood of the origin
in $\Bbb R^{d-1}$, so that $\Phi(0)=\nabla\Phi(0)=0$. Let
$\fV$ be the flag of subspaces $\{S^{m_j}\}$ defined as in (1.5). 
Let $\fa$ be the multitype of $\Phi$ near $0$, $\cB(w,\delta)$ as in (2.3) and
$\rho$ as in (2.6). Let $\nu=\sum_{i=1}^{d-1} a_i^{-1}$ ,
$\nu_2=\sum_{i=2}^{d-1} a_i^{-1}$. 

Then there is a neighborhood $\cU$ of the origin and 
 $\delta_0> 0$ so that for all $0<\delta\le\delta_0$ and all  $w\in \cU$
$$\meas(\cB(w,\delta))\le C\delta^\alpha [\rho(w)]^{\nu-\alpha},
\quad\nu\le \alpha\le \frac 12 +\nu_2.
$$

\endproclaim

\demo{Proof}
We may assume that $a_1>2$ since otherwise the theorem follows already from the
estimate (2.4).
Let $\{u_1,\dots, u_{d-1}\}$ an orthonormal basis of $\Bbb R^{d-1}$ so that
$$S^{m_j}=\text{span}\{u_i, \, d-1-\dim S^{m_j}<i\le d-1\}\tag 2.7
$$ for $j=0,\dots, k-1$.
By performing a rotation we may assume that the $u_i$ are the standard coordinate vectors. 

Define dilations $A_t$ by
$$A_tx= (t^{\frac{1}{a_1}}x,\dots, t^{\frac{1}{a_{d-1}}}x).\tag 2.8
$$
According to \cite{26}, \cite{16} we may split 
$$\Phi(x) =Q(x)+R(x)$$
where $Q$ is a convex polynomial satisfying 
$$Q(A_t x)=tQ(x) \tag 2.9
$$
and
$$
0<|Q(x)|\le C_1|x||\nabla Q(x)|\le C_2|x|^2 \sum_{i,j}
\Big|\frac{\partial^2 Q}{\partial x_i \partial x_j  }(x)\Big|.
\tag 2.10
$$
and the remainder term $R$ satisfies 
$$
%\lim_{s\rightarrow 0} 
\Big|s^{-1}\frac{\partial^{|\alpha |}}
{ \partial x^\alpha}\big(R(A_s x)\big) \Big| \lc s^{1/m}
\tag 2.11
$$
 for $|x|\le T$ and  all multiindices $\alpha=(\alpha_1,\dots, \alpha_{d-1})$ 
with $|\alpha|\le  N$. 
Since $Q$ is positive away from the origin and homogeneous with respect to dilations $(A_t)$  we have that
$$
Q(y)\approx \rho(y)
$$
where $\rho$ is as  in (2.6); in fact 
$\rho(y)\approx \sum_{i=1}^{d-1}|\inn{y}{u_i}|^{a_i}.$

Set $\Phi_\ell(y)=2^\ell\Phi(A_{2^{-\ell}} y)$ and note that
$\Phi_\ell(y)=Q(y)+R_\ell(y)$ where $R_\ell$ and its derivatives
 tend to zero uniformly on compact sets, as $\ell\to \infty$.


Denote by $\fa(w)=(a_1(w),\dots, a_{d-1}(w))$ the multitype of $Q$ at $w$. 
Then $\fa(0)=\fa$ and
 by Lemma 2.1   there 
is $M>0$ so that $a_i(w)\le a_i$ for $0\le \rho(w)\le 2^{-M+2}$ and, by (2.10/11),  $a_1(w)=2$
for $0< \rho(w)\le 2^{-M+2}$; note that nothing is said about the position of the 
spaces $S^m(w)$.
Now  for any point $w$ there is an open ball $U(w)$ of radius $T(w)/4$
and a  flag $\fV(w)$  consisting of $l(w)$ nested subspaces  and
an $l(w)$-tuple  $m(w)$ so that for $x\in U(w)$ the functions
$$h\mapsto Q^x(h)= Q(x+h)-Q(x)-\inn{\nabla Q(x)}{h}$$
belong to   a class 
$\fS_{T(w)}^{d-1}(b(w),M(w),\fV(w), m(w),N)$ 
so that $a_i(\fV(w), m(w))\ge a_i$ and  $a_1(\fV(w), m(w))=2$.



By the metric property of the nonisotropic  balls $\cB(w,\delta)$ there are 
constants $C_2\gg C_1\gg 1$ and $\delta_1\ll 1$ so that
$$
\cB(y,\delta)\subset \{ x: C_1^{-1}\rho(y)\le\rho(x)\le  C_1\rho(y)\}
\quad \text{ if } \rho(y)\ge C_2\delta; 
\tag 2.12
$$
we may assume that $C_1\ge 2^{2M+4}$.

We shall now show that there are constants $c_0>0$, $C_0>1$ so that
 for $2^{-\ell}\le c_0$
$$
%\aligned
|\cB(y,\delta)|\le \delta^\alpha 2^{\ell(\alpha-\nu)} 
%\\
\text { if } 2^{-l-M}\le \rho(y)\le 2^{-l-M+1},  \,
\delta\le C_0^{-1} 2^{-M-\ell}, \,
0\le \alpha\le \nu_2+\frac 1 2. 
%\endaligned
\tag 2.13
$$

Let $$W=\{y: C_1^{-2} 2^{-M-2}\le \rho(y)\le C_1^{-1}  2^{-M+2}\}
\tag 2.14$$
which because of
$C_1\ge 2^{2M+4}$
 is contained in the open 
ball of radius $2^{-M}$ centered at the origin.
 We may cover the compact 
annulus $W$ by finitely many open balls $U_i$ with  center $w_i\in W$ and radius $T(w_i)/4$
 so that
$Q^x\in \fS_{T(w_i)}^{d-1}(b(w_i),M(w_i),\fV(w_i), m(w_i),N)$  provided that 
$|x-w_i|\le T(w_i)/2$.

Since $\Phi_\ell$ converges to $Q$ 
in the $C^N$-topology uniformly on compact sets. There  
is a positive constant $c_0$ so that for $2^{-\ell}\le c_0$
the functions 
$$h\mapsto \Phi_\ell(x+h)-\Phi_\ell(x)-\inn{\nabla\Phi_\ell(x)}{h}
\tag 2.15
$$
belong to
$\fS_{T(w_i)}^{d-1}(\tfrac{b(w_i)}2,2M(w_i),\fV(w_i), m(w_i),N)$  if $|x-w_i|\le T(w_i)/2$.
By the finite type property there is a $\delta_0>0$ so that for $\gamma\le \delta_0$  and 
$x\in U_i$
the caps 
$$
W_\ell(x,\gamma) \subset 
\{z:|\Phi_\ell(z)-\Phi_\ell(x)-\inn{\nabla\Phi_\ell(x)}{z-x}|\le \gamma\}
$$ are contained in the double  of $U_i$; moreover we have
$$
|W_\ell(x,\gamma)|\le C \gamma^{\frac 12+\nu_2}, \quad \gamma\le \delta_0,
\tag 2.16
$$
by the analogue of (2.4) with exponent $1/2+\nu_2$; here $C$ is independent of $\ell$.

Now in order to show that
(2.13) holds
we assume that
$ C_1^{-1}2^{-l-M}\le \rho(y)\le C_1^{-1}2^{-l-M+1} $ 
and observe that
the image of $\cB(y,\delta)$ under the linear transformation $A_{2^\ell}$ is 
$W_\ell(A_{2^\ell}y,2^\ell\delta)$ which is contained in $W$, in fact in a $U_i$ if 
$2^\ell\delta\le \delta_0$.
Since  $\det A_{2^\ell}=2^{\ell\nu}$
we have $|\cB(y,\delta)|\lc 2^{-\ell\nu}|W_\ell(A_{2^\ell}y,2^\ell\delta)|$ and (2.13) follows.

Finally if
$\delta\le C_0^{-1} 2^{-M-\ell}$ we use 
$|\cB(y,\delta)|=O(\delta^\nu)$ instead and observe  that in this range
$\delta^\alpha 2^{\ell(\alpha-\nu)}\lc \delta^\nu$, provided that $\alpha\ge \nu$. This together with (2.13) proves the asserted statement.\qed
\enddemo

\proclaim{Lemma 2.3} Let $\Phi$, $\fV$, $\fa$,
 $\cU$  be as in Lemma 2.2, $N>d+a_{d-1}$.
For $\xi\in \bbR^d$ define
$$F(\xi)=\int \chi(y)e^{-\ic(\inn {\xi'}{y}+\xi_d\Phi(y))}dy.$$
There is a neighborhood $\widetilde \cU\subset \cU$ of the origin and a conic neighborhood $\cV$
 of 
$e_d$ so that for $\xi\in\cV$
$$
|F(\xi)|\le C\|\chi\|_{C^{N}}
|\xi|^{-\alpha}\Big(\sum_j \Big[\frac{\Pi_j \xi}{|\xi_d|}
\Big]^{\frac{m_j}{m_j-1}}
\Big)^{\nu-\alpha},
\quad\nu\le \alpha\le \frac 12 +\nu_2, \tag 2.17
$$

\endproclaim
\demo{Proof}
We may assume that (2.7) holds and that the $u_i$'s form the standard basis in $\bbR^{d-1}$.
 Observe that then
$$\sum_j |\Pi_j \eta|^{\frac{m_j}{m_j-1}}\approx \sum_{i=1}^{d-1}|\xi_i|^{a_i'}$$
with $a_i'=a_i/(a_i-1)$.
%; in the proof we may assume $a_1>2$.


Assume that $s/2\le \rho(x)\le 2s$ and $s$ is small.
Then $|A_{1/s} x|\approx 1$ and $|Q_{x_i}(A_{1/s} x)|\le C$. But 
$Q_{x_i}(A_{1/s} x)=s^{-1+1/a_i}Q_{x_i}(x)$ so that $|Q_{x_i}(x)|\lc s^{1-\frac 1{a_i}}.$
Similarly by (2.11) the remainder term $R_{x_i}$ satisfies the same estimate so that
$$|\Phi_{x_i}(x)|\lc \Big(\sum_{k=1}^{d-1}|x_k|^{a_k}\Big)^{1-\frac 1{a_i}}
$$
for small $x$ 
and therefore 
$$\sum_{i=1}^{d-1} |\Phi_{x_i}(x)|^{a_i'}\lc 
\sum_{k=1}^{d-1}|x_k|^{a_k}.$$
Now let $x(\xi)$ be the unique point at which $\xi$ is normal to the graph of $\Phi$.
By the Bruna-Nagel-Wainger estimate for the Fourier transform  (1.4) and Lemma 2.2  we have that
$$|F(\xi)|\lc |\xi|^{-\alpha} \rho(x(\xi))^{\nu-\alpha}$$
and since  $x(\xi)$ is determined by $\xi_i/\xi_d=\pm \Phi_{x_i}(x(\xi))$ for $i=1,\dots, d-1$, 
 the estimate (2.17) follows.
\qed\enddemo




\head{\bf 3. Lattice point estimates}\endhead

In this section we prove Theorem 1.3.
We use a variant of the  classical proof (see   Randol \cite{24} for the
 two-dimensional case).
Choose $\zeta\in C^\infty_0(\Bbb R^d)$ so that $\zeta$ is nonnegative,
$\zeta(x)=0 $ if $|x|\ge 1$ and
$\int\zeta(x) dx=1$. Define $\zeta_\eps(x)=\eps^{-d}\zeta(\eps^{-1}x)$.
We work with the $\eps$-regularization $\chi_\Omega* \zeta_\eps$ of the characteristic function
of $\Omega$ and define
$$
\cN_\eps(t)= \sum_{k\in \Bbb Z^d}\chi_{t\Om}*\zeta_{\eps t}(k).
$$
By the Poisson summation formula
$$
\align
\cN_\eps(t)&= \sum_{k\in \Bbb Z^d}t^d 
\widehat{\chi_{\Om}}(2\pi t k)\widehat{\zeta}(2\pi\eps tk)
\\
&=t^d\text{vol}(\Omega)+ \cR_\eps(t)
\tag 3.1
\endalign
$$
where
$$\cR_\eps(t)=
 \sum_{k\neq 0}t^d \widehat{\chi_{\Om}}(2\pi t k)\widehat{\zeta}(2\pi\eps tk).$$
By the divergence theorem
$$
\widehat {\chi_\Omega}(\xi)=\int_\Omega e^{-\ic \inn{x}{\xi}} dx =\ic \sum_{i=1}^d 
\frac{\xi_i}{|\xi|^{2} } 
F_i(\xi)
\tag 3.2
$$
where
$$
F_i(\xi)=\int_\Si n_i(y)e^{-\ic\inn{y}{\xi}}d\sigma(y)
\tag 3.3
$$
and $n_i$ denotes the $i^{\text {th}}$ component of the outer normal vector $n_P$. 

Let $\Gamma$ be the set of points $P\in\Si$ at which all principal curvatures
 vanish. As noticed in \cite{16} it follows from (2.10/11) that the set 
$\Gamma$ is discrete, thus finite by compactness.
For every $P \in \Gamma$ we choose a 
narrow  conic symmetric  neighborhood $V_P$  of the normals $\{\pm n_P\}$, a small
neighborhood  $U_P$ of $P$  in $\Sigma$ and a $C^\infty_0$ 
function $\chi_P$ whose restriction to $\Si$ vanishes off $\cU$ 
and so that $\chi_P$ equals one in a neighborhood of $P$. 
We may arrange these neighborhoods so that the sets $\overline V_P\cap\{\xi:|\xi|\ge 1\}$,
$P\in \Gamma$ are pairwise disjoint and that  the normals to all points 
in a neighborhood of $\overline U_P$ are contained 
in $V_P$ (thus the $\overline U_P$'s  are disjoint too).

Define 
$$F_{i,P}(\xi)=\int_\Si \chi_P(y)n_i(y)e^{-\ic\inn{y}{\xi}}d\sigma(y)
$$
If
the cones  $V_P$ are chosen sufficiently narrow, we have 
$$F_{i,P}(\xi)\lc \cases  \min\{|\xi|^{-\nu(P)},\,  \xi^{-(\frac 12+\nu_2(P))} \}
\Theta_{P}(\tfrac{\Pi^P \xi}{\inn{n_P}{\xi}}) \quad &\text{ if }\xi\in V_P
\\
C_N|\xi|^{-N}&\text{ if } \xi \notin V_P.
\endcases
\tag 3.4$$
The estimate for $\xi\in V_P$ follows from Proposition 1.2, and the estimate for $\xi\notin V_P$ follows by a simple integration by parts;
namely if $t\mapsto \gamma(t)$ parametrizes $\Si$ near $P$ then $|\inn{\gamma'(t)}{\xi}|\approx
|\xi|$ for $\gamma(t)\in U_P$ and $\xi\notin V_P$.

Moreover by the Bruna-Nagel-Wainger estimate we have 
$$
|F_i(\xi)-\sum_{P\in\Gamma} F_{i,P}(\xi)|\lc|\xi|^{-\mu}, \qquad
\mu=\frac 12 
+\inf_{P\in\Si} \nu_2(P)
\tag 3.5
$$
here we used the definition of $\Gamma$ and the fact that  $\chi_P$ equals one near $P$.

We now estimate the remainder term $R_\eps(t)$ where $\eps\ll 1/t$ will be suitably chosen.
Let $\dist_\infty$ denote the distance  taken with respect to the $\ell^\infty$ metric 
in $\bbR^d$, or $\bbZ^d$.
For $P\in \Gamma$ let
$$
\align
\cA_P&=\{ k\in V_P\cap\Bbb Z^d: k\neq 0, \dist_\infty (k, \bbR n_P)\le 3/4\}
\\
\cB_P&=\{ k\in V_P\cap\Bbb Z^d: k\neq 0, \dist_\infty (k, \bbR n_P)> 3/4\}
\\
\cC&=\{ k\in \bbZ^d: k\neq 0, k\notin \cup_{P\in\Gamma}V_P\}.
\endalign
$$
Let 
$$
\align
\fA^i_P(t)&=
 \sum_{k\in \cA_P}t^d  \widehat{\zeta}(2\pi\eps tk)
\frac{2\pi k_i}{|2\pi k|^2} F_{i,P}(2\pi  tk)
\\
\fB^i_P(t)&=
 \sum_{k\in \cB_P}t^d \widehat{\zeta}(2\pi\eps tk)
\frac{2\pi k_i}{|2\pi k|^2} F_{i,P}(2\pi t k)
\\
\fC^i_P(t)&=
 \sum_{k\in \cC}t^d \widehat{\zeta}(2\pi\eps tk)
\frac{2\pi k_i}{|2\pi k|^2} F_{i,P}(2\pi t k)
\\
\fD^i(t)&=
 \sum_{k\neq 0}t^d \widehat{\zeta}(2\pi\eps tk)
\frac{2\pi k_i}{|2\pi k|^2}
 (F_i(2\pi t k)-\sum_{P\in \Gamma}F_{i,P}(2\pi t k))
\endalign
$$
then

$$
\cR_\eps(t)=\sum_{i=1}^d (\fD^i(t)+\sum_{P\in \Gamma}(\fA^i_P(t)+\fB^i_P(t) +\fC^i_P(t))).
\tag 3.6
$$



When evaluating $\fA^i_P$ we essentially sum over  integers in a tubular neighborhood of a line and by the estimate (2.4)  we certainly get
$$
|\fA^i_P(t)|\lc 
\sum_{k\in \cA_P}  t^d |tk|^{-1-\nu}\lc t^{d-1-\nu}.
\tag 3.7
$$
Next for the estimation of $\fD^i_P$ we use the rapid decay estimate in (3.4) to obtain
$$
|\fD^i_P(t)|\lc 
\sum_{k\neq 0}t^d   |tk|^{-N}\lc t^{d-N}
\tag 3.8
$$
and for  $\fC^i_P$ we use (3.5) which yields
$$
|\fC^i_P(t)|\lc C_N
 \sum_{k\neq 0}t^d (1+|\eps tk|)^{-N} (1+|tk|)^{-\mu-1}
\lc \eps^{\mu+1-d}
\tag 3.9
$$
Finally
$$
|\fB^i_P(t)|\lc 
 \sum\Sb k\neq 0\\ k\in V_P \endSb t^d |tk|^{-\frac 32-\nu_2(P)}\Theta_{P}
\big(\tfrac {\Pi^P k}{\inn k{n_P}}\big)
(1+|\eps tk|)^{-N}
$$ and we claim that
for $\la\ge 1$
$$
 \sum\Sb |k|\approx \la\\ k\in V_P\endSb t^d |tk|^{-\frac 32-\nu_2(P)}\Theta_{P}
\big(\tfrac {\Pi^P k}{\inn k{n_P}}\big)
(1+|\eps tk|)^{-N} \lc \la^{d-\frac 32 -\nu_2(P)}\min\{ 1, (\la\eps t)^{-N}\}
\tag 3.10
$$
which implies 
$$
|\fB^i_P(t)|\lc \eps^{\frac 32 +\nu_2(P)-d}\lc \eps^{-(d-1-\mu)}.
\tag 3.11
$$
We verify (3.10). Let $\fa=\fa(P)$ be the multitype at $P$.
In view of $\dist(k,\bbR^{n_P})\ge 3/4$ it is  straightforward to check that
$$
\Theta_{P}\big(\tfrac {\Pi^P k}{\inn k{n_P}}\big)\approx 
\Theta_{P}\big(\tfrac {\Pi^P \xi}{\inn \xi{n_P}}\big)\quad
\text{  if $|\xi-k|_\infty\le 1/2$, $k\in \cB_P$}.
$$
Thus we may replace the sum in (3.10) by an integral. 
After performing a suitable rotation in this integral  we have to show that
$$
\int_{|\xi_d|\approx \la}\int_{|\xi'|\le \la}|\xi|^{-3/2-\nu_2(P)}\Big(\sum_{i=1}^{d-1}
\frac{|\xi_i|^{a_i'}}{|\xi|^{a_i'}}\Big)^{\nu-\nu_2(P)-\frac 12}
d\xi' d\xi_d\lc \la^{d-\frac 32-\nu_2(P)}.
\tag 3.12
$$
Now 
$(\sum_{i=1}^{d-1}
(|\xi_i|/|\xi|)^{a_i'})^{\nu-\nu_2-\frac 12}\lc 
(|\xi_1|/|\lambda|)^{a_1'(1/a_1-1/2)}
$ with $a_1'(1/a_1-1/2)>-1$,
 and therefore the integral in (3.12) is bounded by
$$
\la^{d-1-3/2-\nu_2(P)} \int_{|\xi_1|\le \la}
(|\xi_1|/|\lambda|)^{a_1'(1/a_1-1/2)} d\xi_1\lc 
\la^{d-3/2-\nu_2(P)}.
$$
This shows (3.10).

To finish the proof we note that 
$$N_\eps(t(1-C\eps))\le \cN_\Om(t)\le 
N_\eps(t(1+C\eps))
$$
where $C$ is a constant depending only on the geometry of $\Omega$.
Thus, by taking into account the leading term  in (3.1) we see that
$$
E_\Om(t)\lc (t^{d-1-\nu}+t^d\eps +\eps^{-(d-1-\mu)})
$$
and the desired estimate follows if we choose 
$\eps= t^{-d/(d-\mu)}$.
This completes the proof of (1.17).

\demo{Lower bounds}
To show (1.18) we work with our choice 
$\eps=\eps(t)= t^{-d/(d-\mu)}$.
For  (1.18) we simply set
$$G_{P}(t) =\sum_{i=1}^d t^{\nu(P)-d-1}\fA^i_P(t)$$
which we already showed to be bounded above. However we have to
verify  the claim that
 $\limsup_{t\in \infty} |G_P(t)|>0$
 in the case where  $n_P=\pm e_i$.


We now assume that $n_P =e_i$ (the case $n_P=-e_i$ is handled in the same 
  way). Then define

$$G_{P}(t) = 
t^{\nu(P)+1-d}
\sum_{\ka\in \Bbb Z\setminus\{0\}} (2\pi)^{-1} t^d\widehat 
\zeta(2\pi t^{-\mu/(d-\mu)}\kappa e_i) \sgn(\kappa)|\kappa|^{-1} 
F_{i,P}(2\pi t\kappa e_i).
$$


We split this sum into  parts $G_P(t)=I(t)+II(t)$ where
$$
\aligned
I_P(t) &=  (2\pi)^{-1} t^{\nu(P)+1}\sum_{\ka\in \Bbb Z\setminus\{0\}} 
 \sgn(\kappa)|\kappa|^{-1} 
F_{i,P}(2\pi t\kappa e_i)
\\
II_P(t) &=  (2\pi)^{-1} t^{\nu(P)+1}\sum_{\ka\in \Bbb Z\setminus\{0\}} 
(1-\widehat 
\zeta(2\pi t^{-\mu/(d-\mu)}\kappa e_i)) \sgn(\kappa)|\kappa|^{-1} 
F_{i,P}(2\pi t\kappa e_i).
\endaligned
$$

For the estimation of $II$ we note that
$|(1-\widehat \zeta(2\pi t^{-\mu/(d-\mu)}\kappa e_i))|\lc
\min\{1, t^{-\mu/(d-\mu)\kappa}\}
$
%because $\int\zeta(x) x^{\alpha} dx=0$ for all multiindices $\alpha$ with 
%$0<|\alpha|\le M$, 
and since
$F_{i,P}(2\pi t\kappa e_i)=O((t\kappa)^{-\nu}$ we get the estimate
$$
|II(t)|\lc t^{-\frac{\mu}{d-\mu}}.
$$

To examine  $I(t)$ we parametrize 
 by our assumption on $n_P=e_i$  
$$
F_{i,P}(2\pi t\kappa e_i)= e^{-\ic\kappa \inn{P}{e_i}}
\int_{y'\in \Bbb R^{d-1}} \chi_0(y' ) (1+|\nabla \Phi(y')|^2)^{1/2}
e^{\ic\ka\Phi(y')} dy'
$$
where $\Phi\equiv \Phi^P$ is convex, vanishes of second order at the origin of $\bbR^{d-1}$
and has multitype $\fa(P)$ there; $\chi_0$ is smooth, compactly supported 
and equal to one in a neighborhood of the origin. By  the convexity
 $\inn{P}{n_P}=\inn{P}{e_i}\neq 0$.
To examine the integral we 
 may use an  asymptotic expansion  derived in Schulz \cite{26}
(stated there for $\kappa\to \infty$, but the statement for $\kappa\to -\infty$
follows similarly). We obtain
$$
F_{i,P}(2\pi t\kappa e_i)= e^{-2\pi \ic t\kappa \inn{P}{e_i}} \kappa^{-\nu} 
c_0 (P) e^{\frac{\pi i}{2\nu}\sgn(\kappa)} +O(\kappa^{-\nu-\eta})
$$
where $c_0(P)>0$ and $\eta$ is the reciprocal of the 
least common multiple of $a_1,\dots, a_n$. 
Thus
$$I(t) = c_0(P) \pi^{-1}
\sum_{\ka>0}|\ka|^{-\nu-1}\sin\big(2\pi \ka t\inn{P}{e_i}-\pi/(2\nu)\big)
+O(\kappa^{-\nu-1-\eta}).
$$
The sum defines a periodic function which is not identically zero, 
by the uniqueness theorem for Fourier series. 
Combining this with the estimation for the error term $II(t)$ we see that
$\limsup_{t\to \infty} |G_P(t)|>0$.
\enddemo



\remark{\bf Remark} For almost all rotations the estimates for the error term improves. There is $r>2$ so that
$$|E_{A\Om}(t)| \le \cC(A) t^{d-1-\frac{d-1}{d+1}} \log^{1/r}(2+t)
$$(indeed $\cC$ is in $L^q(SO(d))$ for $q<r$.
As in \cite{2} this is proved using a result  on the maximal function
$$
M(\theta)=\sup_{r>0} r^{(d+1)/2}|\widehat{\chi_\Omega}(r\theta)|
$$
which was shown by Svensson \cite{30} 
 to be in $L^{q_0}(S^{d-1})$ for some $q_0>2$ (under our assumption of 
finite  line type, see also Randol \cite{25} for a 
similar result with additional real analyticity assumption).
Indeed, let $\cR_{\eps,A}(t)=\sum_{k\neq 0}\chi_\Om(2\pi tAk)\widehat \zeta(2\pi \eps tk)$ and
$$\cM_j(A)=\sup_{2^j\le t\le 2^{j+1}}|\cR_{\eps_j,A}(t)|,\quad \text{ with } \eps_j= 2^{-2jd/(d+1)}
$$
then for $q\le q_0$
$$\align\|\cM_j\|_{L^q(SO(d))}&\le 2^{jd}\sum_{k\neq 0} (1+|\eps_j 2^j|k|)^{-N} 
(2^j|k|)^{-(d+1)/2}
(\int|M(A\tfrac k{|k|})|^qdA)^{1/q}
\\&\lc 2^{j(d-1-\frac{d-1}{d+1})} \|M\|_{L^q(S^{d-1})}
\endalign
$$
by the  (standard) choice of $\eps_j$.
But  $$
|E_{A\Om}(t)|  t^{-(d-1-\frac{d-1}{d+1})}\log^{-1/r}(2+t)
\lc 1+ \Big(\sum_{j>0}|\cM_j(A)
 2^{-j(d-1-\frac{d-1}{d+1})} (1+j)^{-1/r}|^q\Big)^{1/q}
$$
which is in $L^q(SO(d))$ for $r<q_0$.

We remark that the methods in W. M\"uller's paper \cite{22} could
 be used to improve
the above bound to $
|E_{A\Om}(t)|\le \cC(A)  t^{d-1-\frac{d-1}{d+1}-\beta}
$ 
where $\beta=\beta(\Om)>0$ and $\cC$ is finite almost everywhere.
\endremark


\head{\bf 4.  Bochner-Riesz multipliers -
 the case of one nonvanishing principal curvature}\endhead


In this  section we shall prove 
 a general theorem concerning multipliers of Bochner-Riesz type associated to
surfaces with at least 
 one nonvanishing principal curvature. Then, in the subsequent section, we
 shall   deduce Theorem 1.1
by rescaling arguments.

In what follows $M_p$ will be the space of Fourier multipliers on 
$L^p(\Bbb R^d)$; $\|m\|_{M_p}$ is the operator norm of the
operator  $T_m$ defined by $\widehat {T_m f}(\xi)=m(\xi) \widehat f(\xi)$.


We split variables   
in $\Bbb R^d$ as $\xi=(\txi,\xi_d)$ and in the statement of the 
 Proposition we further split $\txi=(\xi_1,\xi')\in \bbR\times\bbR^{d-2}$.
The  proof of the following result 
uses the ideas from the two-dimensional case, see
\cite{9}, \cite{6}. 


\proclaim{Proposition 4.1}
Let $\eps>0$, $N\ge d+1+2/\eps$ and let $g \in C^{N}(\Bbb R^{d-1})$.
Suppose  that there is a cube $U$ centered at the origin 
and $a>0$ 
so that 
$$\frac{\partial^2 g}{\partial \xi_1^2}(\xi_1,\xi')\ge a$$ 
in $U$. Let $\chi$ be supported in $U$ and 
let $\phi$ be a smooth function supported in $(1/2,2)$.
Let $0<\delta\ll 1$
and
$$m_\delta(\xi)=\chi(\xi) \phi(\delta^{-1}(\xi_d-g(\xi_1,\xi'))).$$
Then 
%for $\eps>0$
 $$\|m_\delta\|_{M_4}\le C_\eps \delta^{-\frac {d-2}4-\eps},$$
where $C_\eps$ depends only on $a$, $\eps$, $U$, the 
$C^{N}(U)$ norms of the functions $g$, $\chi$  and the $C^{d+1}$ norm of 
$\phi$.
\endproclaim






\demo{\bf Proof} We may assume that $U$ is the unit cube, and that the support of $\chi$ 
has small diameter.
We decompose
$m_{\delta}=\sum_k m_{\delta,k}$ where $k=(k_2,\dots, k_{d-1})$ ranges over 
$(d-2)$-tuples of integers $k_i\le C\delta^{-1/2}$ and 
$$m_{\delta,k}(\xi)=
m_\delta(\xi)\prod_{i=2}^{d-1}\psi(\delta^{-1/2}\xi_i-k_i ) 
$$
for suitable $\psi\in C^\infty_0$ satisfying 
$\sum_{n=-\infty}^\infty\psi(s-n)=1$, so that $\supp \psi\subset [-1,1]$.
Let $\tpsi\in C^\infty_0([-2,2])$  so that $\tpsi$ is 
equal to $1$ on the support
of $\psi$.

Denote by $T_k$ the convolution operator with 
Fourier multiplier  $m_{\delta,k}$ and by 
$R_k$ the convolution operator 
with 
 Fourier multiplier  $\tpsi(\delta^{-1/2}\xi'-k)$.
Note that
$\|R_k\|_{L^p\to L^p}\le C$, $1\le p\le \infty$.
Then for $2\le p\le \infty$
$$\Big\|\sum_k R_k g_k\Big\|_p
\lc \Big(\sum_k\big\|g_k\big\|_p^{p'}\Big)^{1/p'}$$
which follows for  $p=\infty$ from Minkowski's inequality and for $p=2$
by orthogonality; for $2<p<\infty$ one uses interpolation.
Since $T_k=R_kT_kR_k$ it follows that
$$
\Big\|\sum_k T_k\Big\|_{L^4\to L^4}
\le C\delta^{-(d-2)/4} \sup_k\|T_k\|_{L^4\to L^4}
$$
and therefore it suffices to show that 
$$\|T_k\|_{L^4\to L^4}\lc \delta^{-\eps}.
\tag 4.1
$$

The estimate (4.1)  is proved using  arguments in C\'ordoba \cite{6}
 which we  will sketch.
For $\nu\in \Bbb Z$ we define
operators $T_{k,\nu}$ and $S_\nu$ by 
$\widehat {S_\nu f}(\xi)=\tpsi(\delta^{-1/2}\xi_1-\nu)$ and
$\widehat {T_{k,\nu} f}(\xi)=\psi(\delta^{-1/2}\xi_1-\nu)\widehat{T_k f}(\xi)$.
Then
$T_k=\sum_\nu T_{k,\nu}S_\nu f$ where the sum is extended over integers $\nu$ 
with $|\nu|\ll \delta^{-1/2}$ since we assume that
 the support of $\chi$  is small.


Now
$$
\align
\Big\|\sum_\nu T_{k,\nu} S_\nu f\Big\|_4^2 &=
\Big\|\sum_{\nu,\nu'} (T_{k,\nu} S_\nu f)
(T_{k,\nu'} S_{\nu'} f)
\Big\|_2
\\
&\le \sum_{\ell:2^\ell\delta^{1/2}\ll 1}
\Big\|\sum\Sb (\nu,\nu'):\\
|\nu-\nu'|\approx 2^\ell\endSb
 (T_{k,\nu} S_\nu f)
(T_{k,\nu'} S_{\nu'} f)
\Big\|_2
\tag 4.2
\endalign
$$
It can be checked that the family of  functions
$ (T_{k,\nu} S_\nu f)
(T_{k,\nu'} S_{\nu'} f)$ has an orthogonality property which implies that
$$
\Big\|\sum\Sb (\nu,\nu')\\
|\nu-\nu'|\approx 2^\ell\endSb
 (T_{k,\nu} S_\nu f)
(T_{k,\nu'} S_{\nu'} f)
\Big\|_2\lc
\Big\|\Big(\sum_\nu |T_{k,\nu} S_\nu f|^2\Big)^{1/2}\Big\|_4^2.
\tag 4.3
$$
The proof of (4.3) is based on an idea of C. Fefferman \cite{9};
in higher dimensions one uses the following
\proclaim{Lemma 4.2} Suppose that $a'\in \Bbb R^{d-2}$, $|a'|\ll 1$,
and the vectors $\txi$, $\teta$, $\tzeta$, $\tomega$ satisfy 

(i) $\xi+\eta-\tzeta-\tomega=0$,

(ii) $\xi_1>\zeta_1> 0$, $\eta_1<\omega_1<0$,

(iii) $|\txi|, |\teta|, |\tzeta|, |\tomega|\in [2^{\ell-1}\delta^{1/2},
2^{\ell+1}\delta^{1/2}]$,

(iv) $\xi'$, $\eta'$, $\zeta'$ and $\omega'$ belong to the cube of sidelength
$4\delta^{1/2}$ centered at $a'$.

Then
$$
g(\txi)+g(\teta)-g(\tzeta)-g(\tomega)\ge c 2^{\ell}\delta^{1/2}
\big(|\xi_1-\zeta_1|+
|\eta_1-\omega_1|\big)
\tag 4.4
$$

In (4.4), $c$ depends only on the lower bound of $g_{\xi_1\xi_1}$ and the 
$C^4$ norm of $g$ in $\supp \chi$.
\endproclaim

\demo{Sketch of Proof}
A Taylor expansion about the origin yields
$$
g(\txi)+g(\teta)-g(\tzeta)-g(\tomega)=I+II+III+IV
$$
where
$$\align
I&=\frac 12 g_{\xi_1\xi_1}(0)(\xi_1^2+\eta_1^2-\zeta_1^2-\omega_1^2)
\\
II
&=\frac 12\big(
\xi_1\inn {g_{\xi_1\xi'}(0)}{\xi'}+
\eta_1\inn {g_{\xi_1\xi'}(0)}{\eta'}-
\zeta_1\inn {g_{\xi_1\xi'}(0)}{\zeta'}-
\omega_1\inn {g_{\xi_1\xi'}(0)}{\omega'}
\big)
\\
III
&=\frac 12
\big(
\inn{\xi'}{g_{\xi'\xi'}(0)\xi'}+
\inn{\eta'}{g_{\xi'\xi'}(0)\eta'}-
\inn{\zeta'}{g_{\xi'\xi'}(0)\zeta'}-
\inn{\omega'}{g_{\xi'\xi'}(0)\omega'}
\big)
\\
IV&= r(\txi)+r(\teta)-r(\tzeta)-r(\tomega)
\endalign
$$
where $r$ vanishes of third order at the origin.
(4.4) is proved by verifying 
$$
\align
I&\approx \, 2^{\ell}\delta^{1/2}
(|\xi_1-\zeta_1|+
|\eta_1-\omega_1|)
\\
II&\le C 2^\ell\delta
\\
III&\le C\delta
\\
IV&\le C 2^{2\ell}\delta
(|\xi_1-\zeta_1|+
|\eta_1-\omega_1|).
\endalign
$$
The straightforward calculation is omitted; we note that formula (6.30) 
in \cite{21} turns out to be
useful in order to  carry it out.\qed
\enddemo
\demo{Proof of Proposition 2.1, cont}
By (4.3) it remains to show that
$$\Big\|\Big(\sum_\nu |T_{k,\nu} S_\nu f|^2\Big)^{1/2}\Big\|_4\lc 
\delta^{-\eps}\|f\|_4.
\tag 4.5
$$

Let $\Gamma_k(t)=(-\nabla_{\txi}g(t,\delta^{1/2} k), 1)$
which gives a one parameter family of
vectors normal  
to $\Sigma_\rho$.

For $\sigma\ge 2$ let $\cR_{k,\sigma}$ be the set of all cylinders 
 whose base is 
a $d-2$ dimensional ball of radius $s$ and whose height is 
$\sigma s$ (any $s>0$),
so that  the axis is parallel to
$\Gamma_{k}(t)$ for some $|t|\le 1$. 

Define the maximal function 
$$M_{k,\sigma} f(x)=\sup\Sb x\in R\\R\in \cR_{k,\sigma}\endSb \frac 1{|R|}
\int_R |f(y)| dy.
$$
Then arguing as in \cite{6} and 
using standard estimates for the kernel of $T_{k,\nu}$ we see that
$$
\int\sum_\nu|T_{k,\nu} S_\nu f(x)|^2 w(x) dx\lc
\int\sum_\nu|S_\nu f(x)|^2 
M_{k,\delta^{-1/2}} w(x) dx.
$$


The $L^p$ norm of $(\sum_\nu|S_\nu f|^2)^{1/2}$ 
is bounded by the $L^p$ norm of 
$f$, for $p\ge 2$ (see \cite{6})  and therefore   we can finish our proof
by using duality and  showing that
$$
\|M_{k,\sigma} f\|_2\le C_\eps \sigma^{\eps} \|f\|_2
\tag 4.6
$$
uniformly in $k$.




If we knew that for every $\xi$ the function 
$t\mapsto \inn{\xi}{\Gamma_k(t)}$ changed sign at most 
$M$ times then it would follow 
from a result by C\'ordoba \cite{7} that (4.6) holds with 
$\sigma^\eps$ replaced by
$C_1 M [\log \sigma]^{C_2}$. 
This hypothesis  may not be satisfied, but we can get around this point 
by  a simple  approximation. Namely devide $[-1,1]$ into $\sigma^{\eps/2}$ 
intervals $[a_j,b_j]$ of lengths $\sigma^{-\eps/2}$. 
Let $P_{k,j}(t)$ be the
vector valued  
Taylor polynomial of 
degree $[2/\eps]$ of $\nabla_\txi g(\cdot, \delta^{1/2}k)$ 
 expanded about $a_j$, and let $\Gamma_{k,j}(t)=(-P_{k,j}(1),1)$.
Then $|\Gamma_k(t)-\Gamma_{k,j}(t)|\le C\sigma^{-1}$ for $t\in [a_j,b_j]$.



Let $\cR_{k,\sigma,j}$ be the set of all cylinders  whose base is 
a $d-2$-dimensional ball of radius $s$  whose height is 
$\sigma s$,
so that  the axis is parallel to
$\Gamma_{k,j}(t)$ for some $|t|\le 1$.  If $M_{k,\sigma,j}$ denotes the 
associated maximal operator then it is immediate that
$M_{k,\sigma}f\le \sum_j M_{k,\sigma,j}f$  where the sum contains only
$O(\sigma^{\eps/2})$ terms. C\'ordoba's result yields the
$L^2$ bound $C_\eps [\log\sigma]^{C_2}$ for each $M_{k,\sigma,j}$.
This finishes the proof of (4.6).\qed
\enddemo
\enddemo



\head{\bf 5. Proof of Theorem 1.1}\endhead
The $L^1$ version of the theorem is well known, and therefore by an 
interpolation argument one has to show the boundedness on $L^{4/3}(\Bbb R^d)$,
or, equivalently, on $L^4(\Bbb R^d)$.


We split $(1-\rho(\xi))_+^\la= h_0(\rho(\xi))+h_1(\rho(\xi))$ where
$h_0$ is supported in $\{t:t\le 1-\epsilon_0\}$ for suitable
small $\epsilon_0$ and 
$h_1$ is supported in $\{t:t> 1-2\epsilon_0\}$. Then
 $h_0(\rho(\xi))$ is a Fourier
 multiplier in $M_1$; the mild singularity at the origin can be handled  e.g.
by an averaging
argument in \cite{8, p. 248}, replacing $\rho$ by $\rho^N$ for large $N$.

Let $\xi^0\in \Sigma_\rho$. It suffices to show that there exists a 
neighborhood $V$ of $\xi^0$ (in $\Bbb R^d$) so that
$h_1(\rho(\xi))\tchi$ is a multiplier 
on $\Bbb R^d$ for $\la>(d-2)/4$ if $\tchi\in C^\infty$ and supported in
$V$. 
The multiplier norm is invariant under rotations and we may assume that
$\Sigma_\rho$ can be parametrized as a graph
$\xi_d=G(\txi)$, $\txi\in \Bbb R^{d-1}$ 
near $\xi^0$, so that $\rho(\xi)<1$ if $\xi_d>G(\txi)$.
We write
$$
\chi(\xi)h_1(\rho(\xi))=\chi(\xi) H(\xi) 
(\xi_d-G(\txi))_+^\la
\quad\text{ where } \quad
H(\xi)= \Big( \frac{1-\rho(\xi)}{\xi_d-G(\txi)}\Big)^\la.
$$
A Taylor expansion of $\rho$ about $\xi_d=G(\txi)$ shows that
 $H$ is smooth on $\supp \chi$; therefore by the algebra property 
of $M_p$ it suffices to show that
$\tchi(\xi) 
(\xi_d-G(\xi_1,\xi'))_+^\la$ belongs to $M_4$ if 
$\supp \tchi$ is sufficiently close to $\xi^0$.

Let $\fa=(a_1,\dots, a_{d-1})$ be the multitype of $\Sigma_\rho$ at $\xi^0$, in the sense 
of \S 1.2.
By an affine transformation 
we may assume that $\xi^0=0$, $G(0)=\nabla G(0)=0$ and that 
$G=Q+R$ where $Q$ and $R$ are as in the proof of Lemma 2.2: The function  $Q$ is mixed homogeneous of degree
$(a_1,\dots, a_{d-1})$,
i.e. if
$A_s(\txi)=
(s^{\frac{1}{a_1}} \xi_1 , \dots , s^{\frac{1}{a_{d-1}}} \xi_{d-1} )$
then $Q$ satisfies $Q(A_s(\txi))=s Q(\txi)$. The remainder term $R$ satisfies
$
\Big|s^{-1}\frac{\partial^{|\alpha |}}
{ \partial \xi^\alpha}\big(R(A_s \txi)\big) \Big| \le C_{M, N} s^{1/m}
$
 for small $x$ and $s$ and  all multiindices $\alpha=(\alpha_1,\dots, \alpha_{d-1})$ 
with $|\alpha|\le  N$.  In particular 
$|R(\txi)|\le
Q(\txi)/10$ if $Q(\txi)\le 2^{-r_0+2}$ for suitably large $r_0$.

Next we set $R_r(\txi) =2^r R(A_{2^{-r}}\txi)$, so that
$G_r=Q+R_r$ tends to $G$ in the $C^\infty$ topology,
 as $r\to\infty$.
Since the  Hessian of $Q$ has rank $1$ where $1/4<Q(\txi)\le 4$ 
(see (2.10)) the same is
true for $G_r=Q+R_r$ if $r$ is large; we may assume that the matrix norm of
$(Q+R_r)''$ is bounded below uniformly in $r$ if $r\ge r_0$.
 

Let $\phi_1$ be supported in $(1/2,2) $ such that $\sum_{k\ge 0} 
\phi_1(2^ks)=1$ for $0<s\le 1$.  Then we have to show
a bound for the $M_4$ norm of
$$\kappa_j(\xi)=\tchi(\xi) \phi_1(2^j(\xi_d-G(\xi_1,\xi')))
(\xi_d-G(\xi_1,\xi'))_+^\la.$$
Here we may assume that $\tchi(\xi)=0$ when $Q(\txi)\ge 2^{-r_0}$.


We now perform a further decomposition in terms of $G(\txi)$.
Let $\eta\in C_0^\infty(\Bbb R)$ so that $\eta(s)=1$ if $|s|\le 1/2$ and
$\eta(s)=0$ if $|s|\ge 1$; also let $\eta_0=\eta$ and for integer $r>0$ let
$\eta_r(s)=\eta(2^{-r}s)-\eta(2^{-r+1}s)$.
Let 
$$\kappa_{j,n}(\xi)= \kappa_j(\xi)\eta_n(2^jG(\txi))$$
so that
$\kappa_{j,n}$ is supported where
$|\xi_d-G(\txi)|\approx 2^{-j}$ and
$G(\txi)\approx 2^{n-j}$ if $n\ge 0$ and $G(\txi)\lc 2^{-j}$ if
$n=0$. Using the  assumption on the 
support of the  cutoff function $\tchi$ we see that
$\ka_{j,n}=0$ for $j\le n+r_0$.

For the pieces $\kappa_{j,n}$ we employ a scaling argument
(for a similar argument in two dimensions see \cite{12}).
For the scaling we use
the dilations $\xi\mapsto (A_{2^{n-j}}(\txi), 2^{n-j}\xi_d)$.
Define  for $n>0$
$$
\widetilde \kappa_{j,n}(\txi,\xi_d)=
 \phi_1(2^n(\xi_d-G_{j-n}(\xi_1,\xi')))
(\xi_d-G_{j-n}(\xi_1,\xi'))_+^\la
\eta_1(G_{j-n}(\txi));$$
for $n=0$ we use  the same formula but 
 with $\eta_1$ replaced by $\eta=\eta_0$. Then
$$
\kappa_{j,n}(A_{2^{n-j}}\txi, 2^{n-j}\xi_d)
= 2^{(n-j)\la}\tchi(A_{2^{n-j}}\txi, 2^{n-j}\xi_d)
\widetilde \kappa_{j,n}(\txi,\xi_d)
$$
 so that
$$\|\ka_{j,n}\|_{M_p}\lc 2^{(n-j)\la}\|\widetilde \ka_{j,n}\|_{M_p}
$$

It is now easy to see that the $C^4$ norm of $\widetilde\kappa_{j,0}$ is 
$\lc 2^{-j\la}$ and 
$\widetilde\kappa_{j,0}$ is supported in a fixed ball with diameter independent of $j$.

Therefore
$$\|\widetilde \kappa_{j,0}\|_{M_p}\lc
2^{-j\la}, \qquad 1\le p\le \infty.
$$


Note that for  $j-n\ge r_0$ the multipliers $\tka_{j,n}$ are supported where
$1/4<Q(\txi)<4$, and by construction
the matrix norm of $G_{j-n}''$ is in this region
bounded above and below, for $j-n\ge r_0$.
We may apply Proposition 4.1 (with $\delta=2^{-n}$), 
to see that for $0<n\le j-r_0$
$$
\|\widetilde \kappa_{j,n}\|_{M_4}\lc
2^{(n-j)\la} 2^{-n(\la-\frac{d-2}4)}
$$
and the assertion of Theorem 1.1 follows by summing over $0<n\le j-r_0$, $j>0$.
\qed


\Refs
%\nofrills{REFERENCES}


\ref\no 1
\by J. Bourgain
\paper Besicovich-type maximal operators and applications to Fourier analysis
 \jour Geom. and Funct. Anal. \vol 1
\yr 1991\pages 147--184
\endref

\ref\no 2\by L. Brandolini, L. Colzani, A. Iosevich, A. Podkorytov and 
G. Travaglini\paper Geometry of the Gauss map and Lattice points in convex domains\jour Preliminary report, June 2000\endref

\ref\no  3 \by J. Bruna, A. Nagel and S. Wainger
\paper Convex hypersurfaces and Fourier transform
\jour Ann. Math.
\vol 127
\yr 1988
\pages 333--365
\endref




\ref \no 4 \by L.Carleson and P. Sj\"olin 
\paper Oscillatory integrals and a
multiplier problem for the disc \jour Studia Math. \vol44 \yr 1972
\pages 287--299 
\endref 


\ref \no 5 \paper Nombre de points entiers dans une famille
homoth\'etique de domaines de ${\Bbb R}^2$ \by Y. Colin de Verdi\`ere \yr
1977
\jour Ann. Scient. Ec. Norm. Sup. \vol 10 \endref


\ref\no 6 \by A. C\'ordoba
\paper A note on Bochner-Riesz operators \jour Duke Math. J.
\vol 46 \yr 1979 \pages 505-511.
\endref

\ref\no 7 \bysame
\paper  Geometric Fourier analysis 
\jour Ann. Inst. Fourier
\vol  32 \yr 1982
\pages 215--226
\endref

\ref \no 8 \by H. Dappa and W. Trebels
\paper On maximal functions generated by Fourier multipliers
\jour Ark. Mat. \vol 23 \yr 1985 \pages 241--259
\endref

\ref\no 9\by C. Fefferman\paper A note on spherical summation multipliers\jour israel Math. J.
\vol 15\yr 1973\pages 44-52\endref


\ref\no  10 \by A. Greenleaf
\paper Principal curvature in harmonic analysis
\jour Indiana Math. J.
\vol 30
\year 1981
\pages 519--537
\endref

\ref\no 11\by E. Hlawka\paper \"Uber Integrale auf konvexen K\"orpern I
\jour Monatshefte Math\vol 54 \yr 1950\pages 1-36
\moreref
\paper II
\jour Monatshefte Math\vol 54 \yr 1950\pages 81-99
\endref


\ref \no 12 \by L. H\"ormander 
\paper Oscillatory integrals and multipliers on $FL^p$
\jour Ark. Mat.\vol 11\yr 1973\pages 1--11
\endref


\ref\no 13 \bysame  \book The analysis of linear partial 
differential operators Vol. I \publ Springer-Verlag \publaddr New York,
Berlin \yr 1983 \endref



\ref \no 14
\by M. N. Huxley
 \book Area, Lattice Points, and Exponentials Sums \yr 1996 \bookinfo 
London Mathematical Society Monographs
New Series 13\publ Oxford Univ. PressI \endref

\ref \no 15 
 \by A. Iosevich \paper Lattice points and generalized Diophantine 
conditions
 \jour preprint
\endref



\ref\no  16 \by A. Iosevich and E. Sawyer
\paper Maximal averages over surfaces
\jour Adv. Math.\vol 132\yr 1997\pages 46--119
\endref

\ref\no  17 \by A. Iosevich, E. Sawyer and A. Seeger
\paper On averaging operators associated with convex 
hypersurfaces of finite type
\jour Journal d'Analyse Math\'ematiques \vol 79\yr 1999\pages 159--187
\endref


\ref \no 18 \by E. Kr\"atzel  \book  Lattice points\publ Mathematics and its applications,  
Kluwer\yr 1988
 \endref



\ref \no 19 \by E. Kr\"atzel  \paper Lattice points in three-dimensional
convex bodies\jour Math. Nachr. \vol 212\yr 2000\pages 77-90\endref

\ref \no 20\by E. Kr\"atzel and W. G. Nowak \paper Lattice
points in large convex bodies \jour Monatshefte Math. \vol 112\yr 1991\pages 61--72
\moreref\paper  II \jour Acta Arithmetica \vol 62 \pages 285-295 \yr 1992 \endref


%\ref \no\by E. Landau \book Zahlentheorie \yr 1927 \publaddr 
% Leipzig (reprinted by Chelsea, 1969)
%\endref



\ref \no 21\by G. Mockenhaupt, A. Seeger and C.D. Sogge
\paper Local smoothing of Fourier integral operators and Carleson-Sj\"olin
estimates\jour J. Amer. Math. Soc.\vol 6\yr 1993\pages 65--130
\endref

\ref\no 22\by W. M\"uller\paper Lattice points in large convex bodies\jour
Monatshefte Math.\vol 128\pages 315--330\yr 1999\endref

\ref\no 23 \by W. G. Nowak\paper Zur Gitterpunktlehre in der Euklidischen Ebene
\jour Indag. Math\vol 46\yr 1984\pages 209-223
\moreref \paper II\jour 
\"Osterreich. Akad. Wiss. Math. Natur. Kl. Sitzungsber. II\vol 194\yr 1985\pages 31--37\endref

\ref \no 24\paper A lattice point problem \by B. Randol \yr 1966
\jour Trans. Amer. Math. Soc. \vol 121\pages 257--268\moreref paper II
\jour Trans. Amer. Math. Soc.\vol 125\pages 101--113 \endref

\ref\no 25\bysame 
\paper On the Fourier transform of the indicator function of a planar set
\jour Trans. Amer. Math. Soc.\vol139\yr 1969\pages 271--278
\moreref
\paper On the asymptotic behavior of the  Fourier transform of the indicator function of a 
convex  set
\jour Trans. Amer. Math. Soc.\vol 139\yr 1969\pages 278--285
\endref


\ref\no 26 \by H. Schulz
\paper Convex hypersurfaces of finite type and the asymptotics of their
Fourier transforms
\jour Indiana U. Math. J.
\vol 40 
\yr 1991
\pages 1267--1275
\endref

\ref\no 27 \by A. Seeger and S. Ziesler
\paper Riesz means associated with convex
domains in the plane\jour Math. Z.\toappear
\endref


\ref \no 28 \by P. Sj\"olin \paper Fourier multipliers and estimates of Fourier
transforms of smooth measures carried by curves in $\bold R^2$
\jour Studia Math. \vol 51 \yr 1974 \endref

\ref\no 29 \by E. M. Stein
\book Harmonic Analysis
\publ Princeton University Press
\yr 1993
\endref
\ref\no 30
\by I. Svensson\paper Estimates for the Fourier transform of the characteristic function of a convex set\jour 
Arkiv Mat.\vol 9\yr 1971\pages 11-22\endref





\ref \no 31 \paper On the number of lattice points in planar
domains
\by M. Tarnopolska-Weiss \jour Proc. Amer. Math. Soc. \vol 69 \yr 1978
\pages 308--311\endref

\ref\no 32\by A. Varchenko\paper Number of lattice points in families of homothetic
domains in $\bbR^n$\jour Funct. Anal. Appl. \vol 17\yr 1983\pages 79--83\endref



\endRefs
\enddocument






