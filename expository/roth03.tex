\documentstyle{amsppt}
\tolerance 3000 \pagewidth{5.5in} \vsize7.0in
\magnification=\magstep1 \NoRunningHeads \widestnumber
\key{AAAAAAAAA} \topmatter
\title Roth's theorem on arithmetic progressions
\endtitle
\author Alex Iosevich
\endauthor
\date September 17, 2003
\enddate
\endtopmatter
\document

The purpose of this paper is to provide a simple and
self-contained exposition of the celebrated Roth's theorem on
arithmetic progressions of length three. The original result is
proved in \cite{Roth53}, while the proof given below is very
similar to the exposition of Roth's original argument given in
\cite{GRS1990}.

\definition{Definition} We say that a subset of positive integers
$A$ has positive upper density if
$$ \lim_N \sup \frac{|A \cap [1,N]|}{N}>0. \tag1$$
\enddefinition

\proclaim{Roth's Theorem} If $A$ is a subset of positive integers
of positive upper density, then $A$ contains a three term
arithmetic progression. \endproclaim

\subhead Basic setup \endsubhead Let $S(n)$ denote the largest
number of integers in $[1,n]$ that can be chosen so that no three
term arithmetic progression is formed. Let
$$ c=\lim_{n \to \infty} \frac{S(n)}{n}. \tag2$$

The existence of this limit exists follows from the fact that
$S:{\Bbb Z}^{+} \to {\Bbb R}$ is a sub-additive function. The
(easy) details are left to the reader.

Let
$$ \epsilon=\frac{c^2}{10^6}, \tag3$$ and let $m$ be large enough
so that
$$ c \leq \frac{S(n)}{n}<c+\epsilon \ \text{for} \ 2m+1 \leq n.
\tag4$$

Let $2N$ be sufficiently large in the sense that will become clear
below. Let $A \subset \{1,2 \dots, 2N\}$ with $|A| \ge 2Nc$ which
contains no arithmetic progressions of length three. Let
$$ A=\{u_1, u_2, \dots, u_r\}. \tag5$$

Let $A_{even}$ denote the set of even elements of $A$,
$$ A_{even}=\{2v_1, 2v_2, \dots, 2v_s\}; \ \frac{1}{2}A_{even}
=\{v_1, v_2, \dots, v_s\}. \tag6$$

By assumption,
$$ 2Nc \leq r \leq 2N(c+\epsilon), \tag7$$ and
$$ N(c-\epsilon) \leq s \leq N(c+\epsilon). \tag8$$

\subhead Fourier transform \endsubhead Let
$$ \widehat{A}(\alpha)=\sum_{i=1}^r e(\alpha u_i), \tag9$$ and
$$ \widehat{A_{even}}(\alpha)=\sum_{j=1}^s e(\alpha v_j), \tag10$$
where
$$ e(t)=e^{2 \pi i t}. \tag11$$

As we shall see below, the key to the proof of Roth's theorem is
to show that away from $\alpha=0$, $|\widehat{A}(\alpha)|$ is much
smaller than $N$, the number of terms in the sum. The idea behind
this is that the enemy of an exponential sum is presence of
arithmetic progression. These progressions play the same role in
the discrete Fourier transforms as that of lack of curvature in
the decay properties of the Fourier transform of a surface carried
measure.

\subhead The basic idea \endsubhead Let $\int$ denote the sum over
$\alpha=\frac{i}{2N}$, $i=0,1, \dots, 2N-1$. Observe that
$$ \int e(\alpha u)=2N \ \text{if} \ u=0, \ \text{and} \ 0 \
\text{otherwise}. \tag12$$

The key observation is that
$$ \int \widehat{A}(\alpha) {(\widehat{A_{even}})}^2(-\alpha)=2N \# \{(i,j,k):
u_i-v_j-v_k=0; u_i \in A, v_j, v_k \in \frac{1}{2}A_{even} \}.
\tag13$$

Why is this important? If $u_i-v_j-v_k=0$, then the set
$$ \{2v_j, u_i, 2v_k\} \tag14$$ is an arithmetic progression of
length three, except for the trivial cases when
$$ 2v_j=u_i=2v_k. \tag15$$

By assumption our $A$ does not contain arithmetic progressions of
length three, therefore the right hand side of $(13)$ equals
$$2Ns<3cN^2, \tag16$$ where the inequality follows by $(8)$.

We shall obtain a contradiction in the following manner. First,
$$ \widehat{A}(0) {(\widehat{A_{even}})}^2(0)=rs^2>c^3N^3.
\tag17$$

On the other hand,
$$ \widehat{A}(0) {(\widehat{A_{even}})}^2(0)<\left|\int
\widehat{A}(\alpha)
{(\widehat{A_{even}})}^2(-\alpha)\right|+\left|\int_{\alpha
\not=0} \widehat{A}(\alpha)
{(\widehat{A_{even}})}^2(-\alpha)\right|=I+II. \tag18$$

We already know that
$$ I \leq 3cN^2. \tag19$$

We will see that
$$ II \leq 18 \epsilon c N^3. \tag20$$

This will be accomplished as follows. We will show that when
$\alpha \not=0$,
$$ |\widehat{A}(\alpha)| \leq 6 \epsilon N. \tag21$$

This is the main estimate of the paper which we shall carry out
shortly. With $(21)$ in tow, we complete the proof in the
following way. We observe that
$$ II \leq \left( \max_{\alpha \not=0} \left|
\widehat{A}(\alpha)\right| \right) \int
|{(\widehat{A_{even}}(-\alpha))}^2| \leq 6 \epsilon N \int
|{(\widehat{A_{even}}(-\alpha))}^2|, \tag22$$ so $(20)$ would
follow if we could show that
$$ \int |{(\widehat{A_{even}}(-\alpha))}^2| \leq 3cN^2. \tag23$$

To establish $(23)$ observe that the left hand side of $(23)$
equals
$$ \sum_{j=1}^s \sum_{k=1}^s \int e(\alpha(v_j-v_k))=2Ns \leq
3cN^2 \tag24$$ by $(8)$ and $(12)$.

Using $(17)$, $(18)$, $(19)$, and $(20)$ we see that
$$ c^3N^3 \leq 3cN^2+18 \epsilon c N^3, \tag25$$ which is not
possible if $N$ is sufficiently large. Thus the proof of Roth's
theorem is reduced to establishing $(21)$.

\subhead Estimate $(21)$ \endsubhead We need the following basic
fact about diophantine approximation which can be found in any
book on elementary number theory and/or deduced easily from the
pigeon-hole principle. For $\alpha$ arbitrary and $M>0$, there
exist integers $p,q$ with
$$ \alpha=\frac{p}{q}+\beta, \ 1 \leq q \leq M \ \text{and} \
q|\beta| \leq \frac{1}{M}. \tag26$$

We also need the following elementary estimate which we do not
prove...
$$ \left|\frac{1}{2}(e(x)+e(-x))-1\right|=|\cos(x)-1| \leq \frac{x^2}{2}.
\tag27$$

The first step in establishing the estimate $(21)$ is "smearing".
More precisely, we show that
$$ \left|\widehat{A}(\alpha)-\frac{1}{2m+1} \sum_A \sum_{|i| \leq
m} e(\alpha(u+iq)) \right| \leq \frac{|A|m^2}{2M^2} \leq
\frac{Nm^2}{2M^2}. \tag28$$

We shall then estimate the "smeared" part by showing that
$$ \left|\frac{1}{2m+1} \sum_A \sum_{|i| \leq
m} e(\alpha(u+iq)) \right| \leq 5N \epsilon. \tag29$$

Combining $(28)$ and $(29)$ yields $(21)$.

\vskip.125in

We first establish $(28)$, which is a bit easier. From $(27)$ we
deduce that
$$ \left| \frac{1}{2m+1} \sum_{|i| \leq m}
[e(\alpha+i\gamma)-e(\alpha)] \right| \leq
\frac{{(m\gamma)}^2}{2}. \tag30$$

Let $M$ denote the largest integer smaller than $\sqrt{N}$. Let
$\alpha \not=0$ and let $p,q, \beta$ be as in $(26)$ above. Then
$$ e(\alpha(u+iq))=e(\alpha u+i (\beta q)). \tag31$$

It follows that
$$ \left|\frac{1}{2m+1}\sum_{|i| \leq m} [e(\alpha(u+iq))-e(\alpha
u)]\right|=\left|e(\alpha u)-\frac{1}{2m+1}\sum_{|i| \leq m}
e(\alpha(u+iq)) \right|$$ $$ \leq \frac{{(m \beta q)}^2}{2} \leq
\frac{m^2}{2M^2}, \tag32$$ and $(28)$ follows instantly.

We now turn our attention to $(29)$. Let
$$ W_s=\{s+iq: |i| \leq m\} \ \text{calculated modulo 2N}.
\tag33$$

It follows that
$$ \frac{1}{2m+1}\sum_A \sum_{|i| \leq m}
e(\alpha(u+iq))=\sum_{s=0}^{2N-1} e(\alpha s) \frac{|W_s \cap
A|}{2m+1}. \tag34$$

We want to show that
$$ \left|\sum_{s=0}^{2N-1} e(\alpha s) \frac{|W_s \cap
A|}{2m+1}\right| \leq 5N \epsilon \tag35$$ which will establish
$(29)$.

From $(35)$, $(34)$ and $(28)$ it will then follow that
$$ |\widehat{A}(\alpha)| \leq \frac{m^2N}{2M^2}+5N\epsilon \leq 6N \epsilon
\tag36$$ for a sufficiently large $N$ which will establish $(21)$.

We now prove $(35)$. Let
$$ E_s=\frac{|W_s \cap A|}{2m+1}-c. \tag37$$

Observe that for $mq \leq s \leq 2N-mq$, $W_s$ forms an arithmetic
progression of length $2m+1$ in $\{1, 2, \dots, 2N\}$. It follows
that
$$ |W_s \cap A| \leq (2m+1)(c+\epsilon) \tag38$$ by $(7)$.
It follows that for these values of $s$, $E_s \leq \epsilon$. For
the other $2mq$ values of $s$ we use the trivial bound $E_s \leq
1$.

Our next observation is that the average value of $E_s$ is
positive. Indeed, since each $a \in A$ appears in exactly $2m+1$
sets $W_s$, we have
$$ \frac{1}{2m+1} \sum_{s=0}^{2N-1} |W_s \cap A|=|A|
\frac{2m+1}{2m+1}=|A|, \tag39$$ so
$$ \frac{\sum_{s=0}^{2N-1} E_s}{2N}=\frac{|A|}{2N}-c, \tag40$$
which is a positive number.

Now, using $(40)$ and the discussion following $(38)$,
$$ \sum_{s=0}^{2N-1} |E_s| \leq 2 \sum_{0 \leq s \leq 2N-1: E_s>0}
E_s \leq 2(2N \epsilon+2mq) \leq 4 \epsilon N+4mM \leq 5N\epsilon
\tag41$$ if $N$ is sufficiently large.

For $\alpha \not=0$, $\sum_{s=0}^{2N-1} e(\alpha s)=0$, so
$$ \left| \sum_{s=0}^{2N-1} e(\alpha s) \frac{|W_s \cap A|}{2m+1}
\right|=\left|\sum_{s=0}^{2N-1} e(\alpha s) E_s \right|$$ $$ \leq
\sum_{s=0}^{2N-1} |E_s| \leq 5N \epsilon. \tag42$$

This establishes $(35)$ and proof of Roth's theorem is complete.






\newpage

\head References \endhead

\ref \key GRS1990 \by R. Graham, B. Rothschild, and J. Spencer
\paper Ramsey theory \jour John Wiley and Sons, Inc. \yr 1990
\endref

\ref \key Roth53 \by K. Roth \paper On certain sets of integers
\jour J. London Math Soc. \yr 1953 \vol 28 \pages 104-109 \endref










\enddocument
