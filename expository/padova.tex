\documentstyle{amsppt}
\tolerance 3000 \pagewidth{5.5in} \vsize7.0in
\magnification=\magstep1 \NoBlackBoxes \NoRunningHeads
\widestnumber \key{AAAAAAAAAAAAAA} \topmatter
\author Alex Iosevich
\endauthor
\title Szemeredi-Trotter incidence theorem, related results and some
amusing consequences
\endtitle
\date February 11, 2002
\enddate
\thanks Research supported in part by the NSF grant DMS00-87339
\endthanks
\endtopmatter
\document

This is an expository paper based on the lectures the author gave
in Padova at the Minicorsi di Analisi Matematica last summer. The
author wishes to thank the organizers, the participants, and the
fellow lecturers for many interesting and useful remarks.

The main theme of this paper is an old and beautiful subject of
geometric combinatorics. We will not even attempt to cover
anything resembling a significant slice of this broad and
influential discipline. See, for example, \cite{AgPa95} for a
thorough description of this subject. The purpose of this article
is to give an elementary proof of Szemeredi-Trotter incidence
theorem (\cite{ST83}), a result that found a tremendous number of
applications in combinatorics, analysis, and analytic number
theory. We shall describe some of the consequences of this seminal
result.

\definition{Definition} An incidence of a point and a line is a pair
$(p,l)$, where $p$ is a point, $l$ is a line, and $p$ lies on $l$.
\enddefinition

\proclaim{Theorem 0.1}(Szemeredi-Trotter) Let $I$ denote the
number of incidences of a set of $n$ points and $m$ lines (or $m$
strictly convex closed curves). Then
$$ I \lesssim n+m+{(nm)}^{\frac{2}{3}}, \tag0.1$$ where here and
throughout the paper, $A \lesssim B$ means that there exists a
positive constant $C$ such that $A \leq CB$.
\endproclaim

\proclaim{Corollary 0.2} Let $S$ be a subset of ${\Bbb R}^2$ of
cardinality $n$. Let $\Delta(S)=\{|x-y|: x,y \in S\}$, where
$|\cdot|$ denotes the Euclidean norm. Then
$$ \# \Delta(S) \gtrsim n^{\frac{2}{3}}. \tag0.2$$
\endproclaim

This estimate is not sharp. It is conjectured to hold with the
exponent $1$ in place of $\frac{2}{3}$. For the best known
exponents to date (around $.86$), see \cite{SolToth2001} and
\cite{STT2001}. However, Corollary 0.2 is still quite useful as we
shall see in the final section of this paper.

\proclaim{Corollary 0.3} Let $A$ be a subset of ${\Bbb R}$ of
cardinality $n$. Then either $A+A=\{a+a': a,a' \in A\}$ or $A
\cdot A=\{aa': a,a' \in A\}$ has cardinality $\gtrsim
n^{\frac{5}{4}}$. \endproclaim

This paper is organized as follows. In the first section we prove
Theorem 0.1. In the second section we deduce the corollaries. In
the last section we give some applications of Theorem 0.1 to
problems in analysis and convex geometry, including the
non-existence of orthogonal Fourier bases of the disc. In the
final section of the paper, we describe a combinatorial principle
due to Moser which allows us to prove non-existence of orthogonal
Fourier bases for a ball in any dimension greater than $1$.

\vskip.125in

\head Section I: Proof of Theorem 0.1 \endhead

\vskip.125in

We shall deduce Theorem 0.1 from the following graph theoretic
result due to Ajtai et al (\cite{Ajtai86}), and, independently, to
Leighton.

\definition{Definition} The crossing number of a graph, $cr(G)$, is
the minimal number of crossings over all the planar drawings of
this graph. A crossing is an intersection of two edges not at a
vertex.
\enddefinition

\proclaim{Theorem 1.1} Let $G$ be a graph with $n$ vertices and
$e$ edges. Suppose that $e \ge 4n$. Then
$$ cr(G) \gtrsim \frac{e^3}{n^2}. \tag1.1$$

\endproclaim

Before proving Theorem 1.1, we show how it implies Theorem 0.1.
Take the points in the statement of the Theorem as vertices of a
graph. Connect two vertices with an edge if the two corresponding
points are consecutive on some line. It follows that
$$ e=I-m. \tag1.2$$

If $e>4n$ we get $I<4n+m$, which is fine with us. If $e \ge 4n$,
we invoke Theorem 1.1 to see that
$$ cr(G) \gtrsim \frac{e^3}{n^2}=\frac{{(I-m)}^3}{n^2}. \tag1.3$$

Combining $(1.3)$ with the obvious estimate $cr(G) \leq m^2$, we
complete the proof of Theorem 0.1.

We now turn our attention to the proof of Theorem 1.1. Let $G$ be
a planar graph with $n$ vertices, $e$ edges and $f$ faces. Euler's
formula (easy to prove...) says that
$$ n-e+f=2. \tag1.4$$

Combined with the observation that $3f \leq 2e$, we see that in
such a planar graph
$$ e \leq 3n-6. \tag1.5$$

It follows that if $G$ is any graph, then
$$ cr(G) \ge e-3n. \tag1.6$$

We now convert this linear estimate into the estimate we want by
randomization. More precisely, let $G$ be as in the statement of
Theorem 1.1 and let $H$ be a random subgraph of $G$ formed by
choosing each vertex with probability $p$ to be chosen later.
Naturally, we keep an edge if and only if both vertices survive
the random selection. Let ${\Bbb E}()$ denote the usual expected
value. An easy computation yields
$$ {\Bbb E}(vertices)=np, \tag1.7$$
$$ {\Bbb E}(edges)=ep^2, \tag1.8$$
$$ {\Bbb E}(crossing \ number \ of \ H) \leq p^4cr(G). \tag1.9$$

It follows that
$$ cr(G) \ge \frac{e}{p^2}-\frac{3n}{p^3}. \tag1.10$$

Choosing $p=\frac{4n}{e}$ we complete the proof of Theorem 1.1.

\vskip.125in

\head Proof of Corollary 0.2 and Corollary 0.3 \endhead

\vskip.125in

To prove Corollary 0.3, draw a circle of fixed radius around each
point in $S$. By Theorem 0.1, the number of incidences is
$\lesssim n^{\frac{4}{3}}$. This means that a single distance
cannot repeat more than $\approx n^{\frac{4}{3}}$ times. It
follows that there must be at least $\approx n^{\frac{2}{3}}$
distinct distances since the total number of distances is $\approx
n^2$. In other words, we just proved that $\Delta(S) \gtrsim
n^{\frac{2}{3}}$ as promised.

The choice of lines and points is less obvious in the proof of
Corollary 0.3. Let $P=(A+A) \times (A \cdot A)$. Let $L$ be the
set of lines of the form $\{(ax, a'+x): a,a' \in A\}$. We have
$$ \# P=\# (A+A) \times \# (A \cdot A), \tag2.1$$
$$ \# L=n^2, \tag2.2$$ while the number of incidences is clearly $n
\times n^2=n^3$. It follows that
$$ n^3 \lesssim {(\# P)}^{\frac{2}{3}}n^{\frac{4}{3}}, \tag2.3$$ which
means that
$$ \# P \gtrsim n^{\frac{5}{3}}. \tag2.4$$

It follows that either $\# (A+A)$ or $\# (A \cdot A)$ exceeds a
constant multiple of $n^{\frac{5}{4}}$. This completes the proof
of Corollary 0.3.

\vskip.125in

\head Application to Fourier analysis \endhead

\vskip.125in

\definition{Definition} We say that a domain $\Omega \subset {\Bbb
R}^d$ is spectral if $L^2(\Omega)$ has an orthogonal basis of the
form ${\{e^{2 \pi i x \cdot a}\}}_{a \in A}$. \enddefinition

The following result is due to Fuglede (\cite{Fuglede74}). It was
also proved in higher dimensions by Iosevich, Katz and Pedersen
(\cite{IKT99}).

\proclaim{Theorem 2.1} A disc, $D=\{x \in {\Bbb R}^2: |x| \leq r
\}$, is not spectral. \endproclaim

\demo{Proof of Theorem 2.1} Let $A$ denote a putative spectrum. We
need the following basic lemmas:

\proclaim{Lemma 2.2} $A$ is separated in the sense that there
exists $c>0$ such that $|a-a'| \ge c$ for all $a,a' \in A$.
\endproclaim

\proclaim{Lemma 2.3} There exists $s>0$ such that any square of
side-length $s$ contains at least one element of $A$. \endproclaim

For a sharper version of Lemma 2.3 see \cite{IP2000}.

The proof of Lemma 2.2 is straight-forward. Orthogonality implies
that
$$ \int_D e^{2 \pi i x \cdot (a-a')} dx=0, \tag2.1$$ whenever $a
\not=a' \in A$. Since $\int_D dx=2 \pi r$ and the function $\int_D
e^{2 \pi i x \cdot \xi}dx$ is continuous, the left hand side of
$(2.1)$ would have to be strictly positive if $|a-a'|$ were small
enough. This implies that $|a-a'|$ can never be smaller than a
positive constant depending on $r$.

The proof of Lemma 2.3 is a bit more interesting. By Bessel's
inequality we have
$$ \sum_A {|\widehat{\chi}_D(\xi+a)|}^2 \equiv {|D|}^2, \tag2.2$$ for
almost every $\xi \in {\Bbb R}^d$, since the left hand side is a
sum of squares of Fourier coefficients of the exponential with the
frequency $\xi$ with respect to the putative orthogonal basis
${\{e^{2 \pi i x \cdot a} \}}_{a \in A}$. We have
$$ \sum_{A_{\xi}} {|\widehat{\chi}_D(a)|}^2=\sum_{A_{\xi} \cap
Q_s} +\sum_{A_{\xi} \cap Q_s^c}=I+II, \tag2.3$$ where
$A_{\xi}=A-\xi$ and $Q_s$ is a square of side-length $s$ centered
at the origin.

We invoke the following basic fact. See, for example,
\cite{Stein93}. We have
$$ |\widehat{\chi}_D(\xi)| \lesssim {|\xi|}^{-\frac{3}{2}}. \tag2.4$$

It follows that
$$ II \lesssim  \sum_{A_{\xi} \cap Q_s^c} {|a|}^{-3} \lesssim s^{-1}.
\tag2.5$$

Choosing $s$ big enough so that $s^{-1}<<{|D|}^2$, we see that $I
\not=0$, and, consequently, that $A_{\xi} \cap Q_s$ is not empty.
This completes the proof of Lemma 2.3.

We are now ready to complete the proof of Theorem 2.1. Intersect
$A$ with a large disc of radius $R$. By Lemma 2.2 and Lemma 2.3,
this disc contains $\approx R^2$ points of $A$. We need another
basic fact about $\widehat{\chi}_D(\xi)$, that it is radial, and
in fact equals, up to a constant, to ${|\xi|}^{-1} J_1(2 \pi
|\xi|)$, where $J_1$ is the Bessel function of order $1$. We also
need to know that zeros of Bessel functions are separated in the
sense of Lemma 2.2. This fact is contained in any text on special
functions. See also \cite{SteinWeiss71}.

With this information in tow, recall that orthogonality implies
that $|a-a'|$ is a zero of $J_1$. Since the largest distance in
the disc of radius $R$ is $2R$ and zeros of $J_1$ are separated,
we see that the total number of distinct distances between the
elements of $A$ in the disc or radius $R$ is at most $\approx R$.
This is a contradiction since Corollary 0.2 says that $R^2$ points
determine at least $R^{\frac{4}{3}}$ distinct distances. This
completes the proof of Theorem 2.1.



\enddemo










\vskip.125in

\head Application to convex geometry \endhead

\vskip.125in

The following result is due to Andrews (\cite{Andrews61}).

\proclaim{Theorem 3.1} Let $Q$ be a convex polygon with $n$
integer vertices. Then $n \leq {|Q|}^{\frac{1}{3}}$. \endproclaim


\subhead Proof of Theorem 2.1 \endsubhead Let ${\Cal C}$ denote a
strictly convex curve running through the vertices of $Q$. Let
$\Omega$ denote the convex domain bounded by ${\Cal C}$. Let $L$
denote the set of strictly convex curves obtained by translating
${\Cal C}$ by every lattice point inside $\Omega$. Let $P$ denote
the set of lattice points contained in the union of all those
translates. By Theorem 0.1 the number incidences between the
elements of $P$ and elements of $L$ is $\lesssim
{|\Omega|}^{\frac{4}{3}}$ since $\# L \approx \# P \approx
|\Omega|$. Since each translate of ${\Cal C}$ contains exactly the
same number of lattice points,
$$ \# {\Cal C} \cap {\Bbb Z}^2 \lesssim
\frac{{|\Omega|}^{\frac{4}{3}}}{|\Omega|}={|\Omega|}^{\frac{1}{3}}.
\tag3.1$$

This completes the proof of Theorem 3.1. See \cite{Iosevich2002}
for a more general version of this result.

What sort of an incidence theorem would be required to prove a
more general version of this result? Well, suppose we had a
theorem which said that the number of incidences between $n$
points and $n$ strictly convex hyper-surfaces in general position
in ${\Bbb R}^d$ is $\lesssim n^{\alpha}$. By general position we
simply mean that intersection of any $d$ of these hyper-surfaces
contains at most $2$ points. Repeating the argument above, we
would arrive at the conclusion that if $P$ is a convex polyhedron
with $N$ lattice vertices, then
$$ |P| \gtrsim N^{\frac{1}{\alpha-1}}. \tag3.2$$

However, a higher dimensional version of the aforementioned
theorem of Andrews says that $|P| \gtrsim N^{\frac{d+1}{d-1}}$.
This leads us to conjecture the following. \proclaim{Conjecture
3.2} The number of incidences between $n$ points and $n$ strictly
convex hyper-surfaces in general position is $\lesssim
n^{2-\frac{2}{d+1}}$. \endproclaim

This result would be sharp in view of $(3.2)$ and the following
result due to Barany and Larman (\cite{BL98}). \proclaim{Theorem
3.3} The number of vertices of $P_R$, the convex hull of the
lattice points contained in the ball of radius $R>>1$ centered at
the origin is $\approx R^{d\frac{d-1}{d+1}}$. \endproclaim



\vskip.125in

\head Higher dimensions \endhead

\vskip.125in

\definition{Definition} We say that $A \subset {\Bbb R}^d$ is
well-distributed if the conclusions of Lemma 2.2 and Lemma 2.3
hold for $A$. \enddefinition

\proclaim{Theorem 4.1} If $R>0$ is sufficiently large, then
$$ \# (\Delta(A \cap {[-R,R]}^d) \gtrsim R^{2-\frac{1}{d}}. \tag4.1$$
\endproclaim

\proclaim{Corollary 4.2} The ball $B_d=\{x: |x| \leq 1\}$ is not
spectral in any dimension greater than $1$. \endproclaim

Corollary 4.2 follows from Theorem 4.1 in the same way as Theorem
2.1 follows from Corollary 0.2. Lemma 2.2 and Lemma 2.3 go through
without change except that in ${\Bbb R}^d$,
$$ |\widehat{\chi}_{B_d}(\xi)| \lesssim
{|\xi|}^{-\frac{d+1}{2}}, \tag4.2$$ $\widehat{\chi}_{B_d}(\xi)$ is
a constant multiple of
$$ {|\xi|}^{-\frac{d}{2}} J_{\frac{d}{2}}(2 \pi |\xi|), \tag4.3$$ and
the zeroes of $J_{\frac{d}{2}}$ are still separated.

See \cite{Stein93}, \cite{SteinWeiss71} and/or any text on special
functions for the details. Better yet, prove it yourself- it is
not very difficult.

We are left to prove Theorem 4.1. Since $A$ is well-distributed,
there is $s>0$ such that every cube of side-length $s$ contains at
least one point of $A$. Fix a reference cube of side-length $s$
and consider a row of consecutive cubes in each of the coordinate
directions with respect to the reference cube. Chose a point of
$A$ in the $10$th cube in each coordinate direction. Name those
points $P_1, P_2, \dots, P_d$. Let $O$ denote the center of the
reference cube. Construct a system of annuli centered at $O$ of
width $Md$, with the first annulus of radius $\approx R$.
Construct $\approx R$ such annuli.

It follows from the assumption that $A$ is well distributed that
each constructed annulus ${\Cal A}$ has $\approx R^{d-1}$ points
of $A$. Let
$$\cup_{i=1}^d \{|x-P_i|: x \in {\Cal A}\}=\{d_1, \dots, d_k\}.
\tag4.4$$

Let
$$ A_j^l=\{x \in {\Cal A} \cap A: |x-P_l|=d_j\}. \tag4.5$$

It is not hard to see that
$$ A_j^l=\cup_{1 \leq j_m \leq k} \cup_{m=1}^{d-1} A_j^l \cap_{l'
\not=l} A_{j_m}^{l'}. \tag4.6$$

Taking unions of both sides in $j$ and counting, we see that
$$ R^{d-1} \lesssim k^d, \tag4.7$$ where we have used the fact that
the intersection of $d$ spheres in question consists of at most
two points. Taking $d$'th roots and using the fact that we have
$\approx R$ annuli with $\approx R^{d-1}$ point of $A$, we
conclude that
$$ \# \Delta(A \cap {[-R,R]}^d) \gtrsim
R^{1+\frac{d-1}{d}}=R^{2-\frac{1}{d}}, \tag4.8$$ as desired.

\head Some comments on finite fields \endhead

\vskip.125in

In this section we consider incidence theorems in the context of
finite fields. More precisely, let $F_q$ denote the finite field
of $q$ elements. Let $F_q^d$ denote the $d$-dimensional vector
space over $F_q$. A line in $F_q^d$ is a set of points $\{x+tv: t
\in F_q\}$ where $x \in F_q^d$ and $v \in F_q^d \ \backslash \ (0,
\dots,0)$. A hyperplane in $F_q^d$ is a set of points $(x_1,
\dots, x_d)$ satisfying the equation $A_1x_1+\dots+A_dx_d=D$,
where $A_1, \dots, A_d, D \in F_q$ and not all $A_j$'s are $0$.

It is clear that without further assumptions, the number of
incidences between $n$ hyper-planes and $n$ points is $\approx
n^2$ and no better, since we can take all $n$ planes to be rotates
of the same plane about a line where all the points are located.
We shall remove this "difficulty" by operating under the following
non-degeneracy assumption.
\definition{Definition} We say that a family of hyperplanes in $F_q^d$
is non-degenerate if the intersection of any $d$ (or fewer) of the
hyper-planes in the family contains at most one point.
\enddefinition

The main result of this section is the following:
\proclaim{Theorem 5.1} Suppose that a family ${\Cal F}$ of $n$
hyper-planes in $F_q^d$ is non-degenerate. Let ${\Cal P}$ denote a
family of $n$ points in $F_q^d$. The the number of incidences
between the elements of ${\Cal F}$ and ${\Cal P}$ is $\lesssim
n^{2-\frac{1}{d}}$. Moreover, this estimate is sharp. \endproclaim

We prove sharpness first. Let ${\Cal F}$ denote the set of all the
hyper-planes in $F_q^d$ and ${\Cal P}$ denote the set of all the
points in $F_q^d$. It is clear that $\# {\Cal F} \approx {\Cal P}
\approx q^d$. On the other hand, the number of incidences is
simply the number of hyper-planes times the number of points on
each hyper-planes, which is $\approx q^{2d-1}$. Since
$q^{2d-1}={(q^d)}^{2-\frac{1}{d}}$, the sharpness of the Theorem
5.1 is proved.

We now prove the positive result. Consider an $n$ by $n$ matrix
whose $(i,j)$ entry if $1$ if $i$'th point lies on $j$'s line, and
$0$ otherwise. The non-degeneracy condition implies that this
matrix does not contain a $d$ by $2$ sub-matrix consisting of
$1$'s. Using Holder's inequality we see that the number of
incidences,
$$ I=\sum_{i,j} I_{ij} \leq {\left( \sum_i {\left(\sum_j I_{ij}
\right)}^d \right)}^{\frac{1}{d}} \times n^{\frac{d-1}{d}}
\tag5.1$$
$$={\left( \sum_i \sum_{j_1, \dots, j_d} I_{ij_1} \dots I_{ij_d}
\right)}^{\frac{1}{d}} \times n^{\frac{d-1}{d}} \lesssim n \times
n^{\frac{d-1}{d}}=n^{2-\frac{1}{d}}, \tag5.2$$ because when
$j_k$'s are distinct, $I_{ij_1} \dots I_{ij_d}$ can be non-zero
for at most one value of $i$ due to the non-degeneracy assumption.
If $j_k$'s are not distinct, we win for the same reason. This
completes the proof of Theorem 5.1.

Why should the finite field case be different from the Euclidean
case? The proof of Szemeredi-Trotter theorem given above suggests
that main difference may be the notion of order. In the proof of
Szemeredi-Trotter we used the fact that points on a line may be
ordered. However, no such notion exists in a finite field.
Nevertheless, Tom Wolff conjectured that if $q$ is a prime, then
there exists $\epsilon>0$ such that the number of incidences
between $n$ points and $n$ lines in $F_q^2$ should not exceed
$n^{\frac{3}{2}-\epsilon}$ for $n \approx q$. This fact has
recently been proved by Bourgain, Katz and Tao (\cite{BKT2003}).
The notion of order in $F_q^2$ when $q$ is a prime is partially
addressed in \cite{IosevichII2002}.









\newpage

\head References \endhead

\vskip.125in

\ref \key AgPa95 \book Combinatorial geometry \publ A
Wiley-Interscience publication \yr 1995 \endref

\ref \key Ajtai86 \by M. Ajtai, V. Chvatal, M. Newborn, and E.
Szemeredi \paper Crossing-free subgraphs \jour North-Holland Math
Studies \vol 60 \yr 1986 \endref

\ref \key Andrews63 \by G. Andrews \paper A lower bound for the
volume of strictly convex bodies with many boundary lattice points
\jour Trans. Amer. Math. Soc. \vol 106 \yr 1963 \pages 270-279
\endref

\ref \key BKT2003 \by J. Bourgain, N. Katz, and T. Tao \paper A
sum-product estimate in finite fields, and applications \jour
(preprint) \yr 2003 \endref

\ref \key BL98 \by I. Barany and D. Larman \paper The convex hull
of the integer points in a large ball \yr 1999 \jour Math. Ann.
\vol 312 \pages 167-181 \endref

\ref \key Iosevich2002 \by A. Iosevich \paper A generalization of
Andrew's theorem \jour (in preparation) \yr 2002
\endref

\ref \key IosevichII2002 \by A. Iosevich \paper The notion of
order in a two-dimensional vector space over a finite field with
prime number of elements. \jour (in preparation). \yr 2002 \endref

\ref \key IKT99 \by A. Iosevich, N. Katz, and S. Pedersen \paper
Fourier bases and a distance problem of Erdos \jour Math. Res.
Lett. \vol 6 \yr 1999 \pages 251-255 \endref

\ref \key IP2000 \by A. Iosevich and S. Pedersen \paper How large
are the spectral gaps? \jour Pacific J. Math. \vol 192 \yr 2000
\endref

\ref \key Fuglede74 \by B. Fuglede \paper Commuting self-adjoint
partial differential operators and a group theoretic problem \jour
J. Func. Anal. \vol 16 \yr 1974 \pages 101-121 \endref

\ref \key SolToth2001 \by J. Solymosi and Cs. D. Toth \paper
Distinct distances in the plane \jour Discrete Comp. Geom. (Misha
Sharir birthday issue) \vol 25 \yr 2001 \pages 629-634 \endref

\ref \key STT2001 \by J. Solymosi, G. Tardos, and C. D. Toth
\paper Distinct distances in the plane \vol 25 \yr 2001 \pages
629-634 \endref

\ref \key Stein93 \by E. M. Stein \book Harmonic Analysis \yr 1993
\publ Princeton University Press \endref

\ref \key ST83 \by E. Szemeredi and W. Trotter \paper Extremal
problems in discrete geometry \jour Combinatorica \vol 3 \yr 1983
\pages 381-392
\endref

\ref \key SteinWeiss71 \by E. M. Stein and G. Weiss \book
Introduction to Fourier analysis on Euclidean spaces \yr 1971
\publ Princeton University Press \endref












\enddocument
