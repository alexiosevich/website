\documentstyle{amsppt}
\tolerance 3000
\pagewidth{5.5in}
\vsize7.0in
\magnification=\magstep1
\widestnumber \key{AAAAAAAAAAA}
\topmatter
\author Alex Iosevich    
\endauthor 
\thanks Research at MSRI supported in part by NSF grant DMS97-06825
\endthanks 
\address Department of Mathematics Wright State University Dayton Ohio; 
email: iosevich $\@$zara.math.wright.edu \endaddress 
\curraddr Mathematical Sciences Research Institute Berkeley California; 
email: iosevich $\@$msri.org \endcurraddr
\abstract We shall give simple sufficient conditions for the Orlicz type 
bounds for the averaging operators and restriction operators associated 
with infinitely flat curves in the plane. Our results, obtained by scaling,  
can be used to recover, up to the endpoints, the results previously obtained in 
\cite{BMO91}, \cite{Bak94}, and \cite{Bak94-2}. We also prove some three  
dimensional analogs of those results. 
\endabstract 
\title Scaling properties of infinitely flat curves and surfaces  
\endtitle
\endtopmatter 

\head Introduction \endhead 
\vskip.125in 

Let $\gamma \in C^{\infty}([0, \infty))$, $\gamma(0)=\gamma'(0)=0$,  
$\gamma''(s)\ge 0$, and $\gamma''(s)=0$ iff $s=0$. Let  
$$ T_{\gamma}f(x)=\int_{0}^{2} f(x_1-s, x_2-\gamma(s))ds=f*\mu(x). \tag1$$  

If $\gamma''(s)>0$ on $[0,2]$, it is well known that  
$T_{\gamma}: L^p({\Bbb R}^2) \rightarrow L^q({\Bbb R}^2)$ if and only if 
$(\frac{1}{p}, \frac{1}{q})$ is contained in the triangle with 
endpoints $(0,0)$, $(1,1)$ and $(\frac{2}{3}, \frac{1}{3})$. (See 
\cite{Str70}, \cite{Litt73}).  If $\gamma''$ vanishes of order $m-2$,  
$m \ge 2$, then $T_{\gamma}:L^p({\Bbb R}^2) \rightarrow L^q({\Bbb R}^2)$ if
and only if $(\frac{1}{p}, \frac{1}{q})$ is contained in the trapezoid with
the endpoints $(0,0)$, $(1,1)$, $(\frac{2}{m+1}, \frac{1}{m+1})$, and 
$(\frac{m}{m+1}, \frac{m-1}{m+1})$. (See \cite{RiSt88}). 

If 
$\gamma''$ vanishes of infinite order, the estimate $T_{\gamma}:L^p({\Bbb R}^2) 
\rightarrow L^q({\Bbb R}^2)$ may not hold for any $q>p$. However, the 
Orlicz space estimates may be possible. For example, in \cite{BMO91}, the
authors showed that for some flat curves in ${\Bbb R}^2$ there exists a 
Young's function $\Phi$, with $\lim_{t \rightarrow \infty} 
\frac{\Phi(t)}{t^2}=0$ such that the estimate 
$$ {||T_{\gamma}f||}_{L^2({\Bbb R}^2)} \leq 
C{||f||}_{L^{\Phi}({\Bbb R}^2)} \tag2$$
holds, where $L^{\Phi}({\Bbb R}^2)$ denotes the standard Orlicz space,  
associated to an increasing Young function $\Phi$, equipped with the norm 
$$ {||f||}_{\Phi}\equiv \inf \left\{s>0: \int \Phi \left(\frac{|f(x)|}{s}
\right)dx \leq 1 \right\}. \tag3$$ 

More precisely, the result proved in \cite{BMO91} is the following. 
\proclaim{Theorem I} (Bak, McMichael, and Oberlin \cite{BMO91}) Let 
$T_{\gamma}$ be as above. If there exist constants $d \in (0,2]$, 
$\epsilon>0$, and $\beta>0$ such that 
$$ \int_{0}^{\epsilon} D(\beta s) {(\gamma(s))}^{\frac{2}{d}} 
s^{3(2/d-1)} ds<\infty, \tag4$$ then there exists a constant $C$ and 
a Young function $\Phi$ with $\Phi(t) \approx {(tH_{+}^{-1}(t^{-d}))}^2$ 
such that $T_{\gamma}: L^{\Phi}({\Bbb R}^2) \rightarrow L^2({\Bbb R}^2)$,
where $H_{+}^{-1}(x)=1$ for $x>1$, $H^{-1}(x)$ for $x \leq 1$ with 
$H(x)=x^3\gamma(x)$, and $D(s)=|\{ \xi \in {\Bbb R}^2: |\widehat{\mu}(\xi)| 
>s \}|$. \endproclaim 

In \cite{Bak94} the following three dimensional result was proved. 
\proclaim{Theorem II} (Bak \cite{Bak94}) Let 
$$ T^n_{\gamma}f(x)=\int_{\{0 \leq |y| \leq 2\}} f(x'-y, x_n-\gamma(|y|))dy, 
\tag5$$ where $(x',x_n) \in {\Bbb R}^{n-1} \times {\Bbb R}$,  
$\gamma(0)=\gamma'(0)=\gamma''(0)=0$, $\gamma''(s)>0$ for $s>0$, and 
$\frac{\gamma'(s)}{s}$ is non-decreasing for $s>0$. Let $\gamma_n(s)=t^n
\gamma'(s)$. Let $n+1 \leq q<\infty$ and assume that for some $\delta>0$ and
$c=c(q)>0$ $\frac{\gamma_n(s)}{s^{q+\delta}}$ is non-decreasing for 
$0<s \leq c$. Let $\Phi=\Phi_q$ be a Young's function such that $\Phi^{-1}(s)
\approx \Psi^{-1}(s)=\frac{s^{\frac{1}{q}}}{{(\gamma_n^{-1}(1/s))}^{n-1}}$ 
if $s \ge 1$, and $\Psi^{-1}(s)=s^{\frac{1}{q}}$ if $s<1$. If $n=2$ or $3$ 
there exists a constant $C$ such that for every Borel set $E \subset 
{\Bbb R}^n$ with $|E|<\infty$ $T^n_{\gamma}: L^{\Phi}({\Bbb R}^n) 
\rightarrow L^q({\Bbb R}^n)$, $f=\chi_E$, the characteristic function of $E$.
\endproclaim 

In this paper we shall give a simple set of sufficient, and in many cases,
necessary conditions, such that $T_{\gamma}:L^{\Phi}({\Bbb R}^2) 
\rightarrow L^{\Psi}({\Bbb R}^2)$. We shall also see that the techniques
of this paper can be used to obtain $L^{\Phi}({\Bbb R}^2) \rightarrow 
L^2({\Bbb R}^2)$ bounds for the restriction operator 
${\Cal R}f=\hat{f}{|}_{\{(s, \gamma(s))\}}$. See Theorem 3. 
We will also show that these results generalize, in a 
straightforward way, to surfaces of rotation in ${\Bbb R}^3$. See 
the section "Three dimensions" below. Our result for the restriction 
operator is motivated by the following result due to Bak. 

\proclaim{Theorem III} (Bak, \cite{Bak94-2}) Suppose
that $\frac{\gamma(s)}{s^3}$ is increasing on $[0,2]$. Suppose that 
$\frac{\gamma''(s)}{s}$ is increasing on $(0,\delta)$ for some $\delta>0$. 
Then for $1 \leq q<\infty$ and $0<d<1$ there exists a constant $C=C_{d,q}$ 
such that for all $f \in {\Cal S}({\Bbb R}^2)$, the class of rapidly decreasing
functions, 
$$ {\left( \int_{0}^{\delta} {|\hat{f}(s, \gamma(s))|}^q ds \right)}^
{\frac{1}{q}} \leq C+C\int_{{\Bbb R}^2} |f(x)| \cdot {[\gamma_q^{-1}(|f(x)|)]}
^d dx, \tag6$$ where $\gamma_q(s)=s^{q-1}\gamma(s^q)$. 
\endproclaim 

The main idea behind our two-dimensional results is the following lemma 
motivated by the results in \cite{CCVWW89} where it was observed that 
even though finite type convex curves (e.g. $\{(s, s^m): -1 \leq s \leq 1\}$) 
behave very well under the usual diagonal scaling of the form 
$(x_1,x_2) \rightarrow
(2^jx_1, 2^{mj}x_2)$, infinite type curves (e.g. $\{(s, e^{-1/s^2}): 
-1 \leq s \leq 1\}$) require a more careful non-diagonal scaling.  

\proclaim{Lemma 1} Let $\tau_jf(x)=f(2^{-j}x_1, \gamma_jx_1+h_jx_2)$, 
where $\gamma_j=\gamma(2^{-j})$ and $h_j$ is defined as above. Let 
$$ T^{*}_jf(x)=2^j \tau_j T_j {\tau_j}^{-1}f(x), \tag7$$ where 
$$ T_jf(x)=\int_{2^{-j}}^{2^{-j+1}} f(x_1-s, x_2-\gamma(s))ds. \tag8$$ 

Then $T^{*}_j:L^p({\Bbb R}^2) \rightarrow L^q({\Bbb R}^2)$ for 
$\left(\frac{1}{p}, \frac{1}{q} \right)\in {\Cal T}$ with constants 
independent of $j$.
\endproclaim 

\demo{Proof} 
Since $T^{*}_j$ clearly maps $L^1 \rightarrow L^1$ and 
$L^{\infty} \rightarrow L^{\infty}$, with constants independent of $j$, 
it suffices to check that $T^{*}_j: L^{\frac{3}{2}}({\Bbb R}^2) \rightarrow
L^3({\Bbb R}^2)$ with constants independent of $j$. The classical proof 
of the $L^p$ improving properties of measures supported on surfaces with
non-vanishing Gaussian curvature (see e.g. \cite{Str70} and the subsection 
"Simplification" below) shows that it 
suffices to check that if we write $T^{*}_jf(x)=f*\nu_j(x)$, then 
$$ |\hat{\nu_j}(\xi)| \leq C{(1+|\xi|)}^{-\frac{1}{2}}, \tag9$$ with 
$C$ independent of $j$. \enddemo  

Let $T_jf(x)=f*\mu_j(x)$. Let $A_j$ denote the matrix given by the 
equation $\tau_jf(x)=f(A_jx)$. Then
$$ f*\nu_j(x)=2^j \mu_j*\tau^{-1}_jf(A_jx). \tag10$$ 

Using the elementary properties of the Fourier transform, 
$$ \hat{f}(\xi) \hat{\nu_j}(\xi)=2^j \hat{f}({(A_j^{-1})}^t A_j^t\xi) 
\hat{\mu_j}({(A_j^{-1})}^t\xi)=2^j\hat{f}(\xi) 
\hat{\mu_j}({(A_j^{-1})}^t\xi), \tag11$$ since 
${(A_j^{-1})}^t A_j^t=I$, where $A_j^t$ denotes the transpose matrix 
of $A_j$. 

By Proposition 4.2 of \cite{CCVWW89}, 
$$ 2^j|\hat{\mu_j}({(A_j^{-1})}^t\xi)| \leq C{(1+|{(A_j^{-1})}^t A_j^t\xi|)}
^{-\frac{1}{2}}=C{(1+|\xi|)}^{-\frac{1}{2}}. \tag12$$ 

Diving both sides of $(11)$ by $\hat{f}(\xi)$ and using $(12)$ completes the
proof of Lemma 1. 

The main idea behind the three dimensional results in this paper is the 
estimate of the Fourier transform of the measure carried by a family of
radial surfaces dependent on a parameter. See Lemma 6 below. It turns 
out that while it is difficult to find a scaling that gives the 
optimal decay in all directions, even the most naive scaling allows one
to get the optimal decay in the direction normal to the surface at the 
origin. This turns out to be enough to obtain the desired results for 
the averaging and restriction operators. 

Our plan is as follows. 
In the section titled "Scaling" we shall give sufficient 
conditions for the $L^{\Phi}({\Bbb R}^2) \rightarrow L^{\Psi}({\Bbb R}^2)$
boundedness of the operator $T_{\gamma}$ and the $L^{\Phi}({\Bbb R}^2) 
\rightarrow L^2({\Bbb R}^2)$ boundedness of the restriction operator 
${\Cal R}$ in terms of the Orlicz norms of the family of dilation operators
of the form $\tau_Af(x)=f(Ax)$ where $A$ is an invertible matrix. 

In the subsection titled "Simplification" we will show that the assumptions
of our main result can be simplified if we are willing to assume that 
$\sup_{0<a<b}\frac{\gamma''(a)}{\gamma''(b)} \leq C$. 

In the subsection titled "Three dimensions" we shall extend our 
results to radial surfaces in ${\Bbb R}^3$. 

In the section  titled "Orlicz norms of dilation operators" we shall compute
upper and lower bounds for the Orlicz norms of the operators $\tau_A$ 
under various assumptions on the Young functions in question. 

We shall conclude the paper with the subsection on examples.  

\head Scaling \endhead 
\vskip.125in 

The following definition is motivated by interpolation results for 
$L^p$ and Orlicz spaces (see e.g. \cite{Tor76}, \cite{Tor86}).

\definition{Definition 2} Let ${\Cal T}$ denote the triangle with the 
endpoints $(0,0)$, $(1,1)$, and $(\frac{2}{3}, \frac{1}{3})$. We say that
$(\Phi, \Psi) \subset {\Cal T}$ if every linear operator bounded from 
$L^p({\Bbb R}^2) \rightarrow L^q({\Bbb R}^2)$, 
$\left(\frac{1}{p}, \frac{1}{q}\right) \in {\Cal T}$, is bounded from 
$L^{\Phi}({\Bbb R}^2) \rightarrow L^{\Psi}({\Bbb R}^2)$. 

Let ${\Cal T}'=\{p: 1 \leq p \leq \frac{6}{5} \}$. We say that $\Phi \subset 
{\Cal T}'$ if every linear operator bounded from $L^p({\Bbb R}^2) 
\rightarrow L^2({\Bbb R}^2)$, $p \in {\Cal T}'$, is bounded from 
$L^{\Phi}({\Bbb R}^2) \rightarrow L^2({\Bbb R}^2)$. 
\enddefinition 

Our main results are the following. 
\proclaim{Theorem 3} Let $T_{\gamma}$ be as above. Suppose that there 
exists $\epsilon>0$ so that $h'(t)> \epsilon \frac{h(t)}{t}$ for every 
$t>0$, where $h(t)=t \gamma'(t)-\gamma(t)$. Let $(\Phi, \Psi) \subset 
{\Cal T}$. 

Let $\tau_jf(x)=f(2^{-j}x_1, \gamma_jx_1+h_jx_2)$ where 
$\gamma_j=\gamma(2^{-j})$ and $h_j=h(2^{-j})$. Let $N_j(\Phi)$ denote 
the $(L^{\Phi}({\Bbb R}^2), L^{\Phi}({\Bbb R}^2))$ norm of the operator
$\tau_j$, and let $N^{-1}_j(\Psi)$ denote the $(L^{\Psi}({\Bbb R}^2), 
L^{\Psi}({\Bbb R}^2))$ norm of the operator $\tau^{-1}_j$. 

Suppose that 
$$ \sum_{j=0}^{\infty} 2^{-j} N_j(\Phi) N^{-1}_j(\Psi)< \infty. \tag13$$ 

Then $T_{\gamma}:L^{\Phi}({\Bbb R}^2) \rightarrow L^{\Psi}({\Bbb R}^2)$. 
\endproclaim 

\remark{Remark} Note that by duality it suffices to check $T_{\gamma}: 
L^{\Psi^{*}}({\Bbb R}^2) \rightarrow L^{\Phi^{*}}({\Bbb R}^2)$. In 
other words, it suffices to check the condition $(13)$ with 
$N_j(\Phi)$ replaced by $N_j(\Psi^{*})$, and $N_j^{-1}(\Psi)$ 
replaced by $N_j^{-1}(\Phi^{*})$. \endremark 

\proclaim{Theorem 4} Let ${\Cal R}f$ be defined as above. Let $\gamma$ satisfy 
the assumptions of Theorem 3. Let $\sigma_jf(x)=|A_j|f(A_j^tx)$ with
$A_j$ defined by $\tau_jf(x)=f(A_jx)$, where $\tau_j$ is as above. Let 
${\Cal N}^{-1}_j(\Phi)$ denote the 
$(L^{\Phi}({\Bbb R}^2), L^{\Phi}({\Bbb R}^2))$ norm of the operator 
$\sigma_j^{-1}$. Let $\Phi \in {\Cal T}'$. 

Suppose that 
$$ \sum_{j=0}^{\infty} 2^{-\frac{j}{2}} {\Cal N}_j^{-1}(\Phi)<\infty. \tag14$$

Then ${\Cal R}: L^{\Phi}({\Bbb R}^2) \rightarrow L^2(\Gamma)$, where 
$\Gamma=\{(s, \gamma(s)): 0 \leq s \leq 2\}$. 
\endproclaim 

We shall now prove Theorem 3 and Theorem 4.  
\vskip.125in 

\head Proof of Theorem 3 \endhead 
\vskip.125in 

Let $T_j$, $T^{*}_j$ be defined as above. Then 
$$ {||T_jf||}_{\Psi}=2^{-j}{||\tau_j^{-1} T^{*}_j \tau_jf||}_{\Psi}. 
\tag15$$ 

By Lemma 1, $T^{*}_j: L^p({\Bbb R}^2) \rightarrow L^q({\Bbb R}^2)$ for 
$\left(\frac{1}{p}, \frac{1}{q} \right) \in {\Cal T}$ with constants 
independent of $j$. By definition of $N_j(\Phi)$ and $N_j^{-1}(\Psi)$ it
follows that $T: L^{\Phi}({\Bbb R}^2) \rightarrow L^{\Psi}({\Bbb R}^2)$ if
$(13)$ holds. This completes the proof of Theorem 3. 

\head Proof of Theorem 4 \endhead 
\vskip.125in 

The proof of Theorem 4 is along the same lines as the proof of Theorem 3. Let
${\Cal R}_jf=\hat{f}{|}_{\Gamma_j}$, where $\Gamma_j=\{(s, \gamma(s)): 
2^{-j} \leq s \leq 2^{-j+1}\}$. Let 
$\sigma_jf(x)=|A_j|f(A_j^tx)$. Using the elementary properties of the Fourier 
transform it is not hard to see that ${\Cal R}_j \sigma_jf=
\hat{f}(A_j^{-1}\Gamma_j(s))$. It follows that 
$$ {||{\Cal R}_j\sigma_jf||}_2=2^{-\frac{j}{2}} 
{||\hat{f}(A_j^{-1}\Gamma_0(2^{-j} \cdot)||}_2. \tag16$$ 

By Proposition 4.2 in \cite{CCVWW89}, the Fourier transform of the 
Lebesgue measure on $A_j^{-1}\Gamma_0(2^{-j} \cdot)$ is bounded by 
$C{(1+|\xi|)}^{-\frac{1}{2}}$, where $C$ is independent of $j$. The 
proof of the restriction theorem for curves with non-vanishing Gaussian 
curvature (see e.g. \cite{St93}) implies that 
$$ {||{\Cal R}_j \sigma_jf||}_2 \leq C2^{-\frac{j}{2}}{||f||}_{\Phi}, \tag17$$ 
for $\Phi \subset {\Cal T}'$. It follows that 
$$ {||{\Cal R}_j f||}_2 \leq C2^{-\frac{j}{2}} {||\sigma_j^{-1}f||}_{\Phi}. 
\tag18$$  

This completes the proof of Theorem 4. 

\subhead Simplification \endsubhead 
It is not hard to see that the statements of Theorem 3 and 
Theorem 4 can be simplified if we are willing to assume that $\gamma''$ is 
increasing on $[0,2]$, or, even, that $\sup_{0<a \leq b} \frac{\gamma''(b)}
{\gamma''(a)} \ge C$. More precisely, under this assumption, everywhere in
the statements of Theorem 3 and Theorem 4, we can replace the scaling 
transformation $\tau_jf(x)=f(2^{-j}x_1, \gamma_jx_1+h_jx_2)$ by a simpler 
scaling transformation $\tau_j'f(x)=f(2^{-j}x_1, \gamma_jx_2)$. 
This is the consequence of the following 
rephrasing of the classical $L^p$-improving result for curves with 
everywhere non-vanishing curvature and the Stein-Tomas restriction 
theorem in ${\Bbb R}^2$. 

\proclaim{Lemma 5} Let $Tf(x)=\int_I f(x_1-s, x_2-\phi(s))ds$, 
${\Cal R}f=\hat{f}{|}_{\{(s, \phi(s)): s \in I\}}$, where $I$ is a 
compact interval and $\phi$ is a smooth function. Suppose that 
$\phi''(s)\ge 1$ on $I$. Then $T: L^p({\Bbb R}^2) \rightarrow 
L^q({\Bbb R}^2)$, $\left(\frac{1}{p}, \frac{1}{q} \right) \in {\Cal T}$, 
and ${\Cal R}: L^p({\Bbb R}^2) \rightarrow L^p({\Bbb R}^2)$, 
$p \in {\Cal T}'$, with constants independent of $\phi$. 
\endproclaim 

Before proving the lemma, let's see how it implies the claim in the first
paragraph. The analog of the operator $T^{*}_j$ in 
the proof of Theorem 3 is just the averaging operator over the curve 
$(s, \phi_j(s))$, $1 \leq s \leq 2$, where $\phi_j(s)=\frac{\gamma(2^{-j}s)}
{\gamma_j}$. By assumption and the mean value theorem $\phi_j''(s) \ge 1$, 
so Lemma 5 implies that the new $T^{*}_j: L^p({\Bbb R}^2) \rightarrow 
L^q({\Bbb R}^2)$, $\left(\frac{1}{p}, \frac{1}{q} \right) \in {\Cal T}$. The 
rest of the proof goes through as before and we get the claim in the first 
paragraph. The argument in the restriction theorem is basically the same. 

We now prove Lemma 5. Again, we prove the case of the $L^p$-improving 
theorem, the restriction proof being similar. Let 
$K_z(x)=\frac{1}{\Gamma(z)} {(x_2-\phi(x_1))}^{z-1} \chi_I(x_1)$, where 
$\Gamma$ denotes the standard gamma function and $\chi_I$ is the characteristic
function of the interval $I$. Let $T_zf(x)=f*K_z(x)$. It is clear that if
$Re(z)=1$, then $T_z: L^{1}({\Bbb R}^2) \rightarrow L^{\infty}({\Bbb R}^2)$ 
with universal constants. It remains to show that when $Re(z)=-\frac{1}{2}$, 
then $T_z: L^2({\Bbb R}^2) \rightarrow L^2({\Bbb R}^2)$ with constants 
independent of $\phi$. Let's compute $m_z(\xi)=\hat{K_z}(\xi)$. Well, 
$$ m_z(\xi)=\int e^{-i\langle x, \xi \rangle} K_z(x)dx. \tag19$$ 

Let $y_1=x_1$, $y_2=x_2-\phi(x_1)$. We get 
$$ \frac{1}{\Gamma(z)} \int \int_I e^{-i(y_1\xi_1+\phi(y_1)\xi_2)} 
e^{-iy_2\xi_2} y_2^{z-1} dy_1dy_2. \tag20$$ 

When $Re(z)=-\frac{1}{2}$, the expression in $(20)$ is controlled by 
$|\widehat{d\sigma}(\xi)| {|\xi_2|}^{\frac{1}{2}}$, so, by Plancherel's 
theorem it remains to show that $|\widehat{d\sigma}(\xi)| 
\leq C{|\xi_2|}^{-\frac{1}{2}}$
with constants independent of $\phi$. This immediately follows by 
the van der Corput lemma since $\phi''(s) \ge 1$ on $I$. 
\vskip.125in 

\subhead Three dimensions \endsubhead 
In this section we shall see that if 
$S=\{x=(x',x_3) \in {\Bbb R}^3: x_3=\gamma(|x'|)\}$, where $\gamma$ satisfies  
the assumptions of the previous subsection (Simplification), 
then one can easily 
generalize Theorem 3 and Theorem 4 to three dimensions by replacing the 
scaling transformation $\tau_j'$ by its three dimensional version, 
$\tau_j'f(x)=f(2^{-j}x', \gamma_jx_3)$, using the following three dimensional
version of Lemma 5. 
\proclaim{Lemma 6} Let $Tf(x)=\int_B f(x'-y, x_n-G(y))dy$, 
${\Cal R}f(x)=\hat{f}{|}_{\{(y, G(y)): y \in B\}}$, where $B$ is the annulus 
$\{y: 1\leq |y| \leq 2\}$, and $G(y)=\phi(|y|)$. 
Suppose that $\min\{\phi', \phi''\}
\ge 1$ on $B$. Then $T:L^p({\Bbb R}^3) 
\rightarrow L^q({\Bbb R}^3)$ for $\left(\frac{1}{p}, \frac{1}{q} \right) \in
{\Cal T}_3$, the triangle with the endpoints $(0,0)$, $(1,1)$, and 
$\left(\frac{3}{4}, \frac{1}{4}\right)$, and ${\Cal R}: L^p({\Bbb R}^3)
\rightarrow L^2({\Bbb R}^3)$, $p \in {\Cal T}'_3=\{p: 1 \leq p \leq 
\frac{4}{3}\}$, with constants independent of $G$. \endproclaim 

As we noted above, we can now prove the obvious analogs of Theorem 3 and
Theorem 4. We let 
$\tau_jf(x)=f(2^{-j}x', \gamma_jx_3)$. The analog of the operator $T_j^{*}$ 
is the averaging operator over the hypersurface $\{x: 1 \leq |x'| \leq 2; \ \ 
x_3=G_j(x')\}$, where $G_j(x')=\frac{\gamma(2^{-j}|x'|)}{\gamma_j}$. The 
determinant of the hessian matrix of $G_j$ is 
$2^{-2j}\gamma''(2^{-j}r) {\left(\frac{2^{-j} \gamma'(2^{-j}r)}{r}\right)}
$, where $r=|x'|$. By assumption and the mean value theorem this 
quantity is bounded below by $1$. The rest of the argument is the same 
as the proof of Theorem 3, with Lemma 6 replacing Lemma 1. 
The argument for the restriction operator is similar. 

We shall now prove Lemma 6. By the proof of Lemma 5 it suffices to show that
$$ |F(\xi)|=\left|\int_B e^{i(\langle x', \xi \rangle+\xi_3 G(x'))} dx'\right| 
\leq C{|\xi_3|}^{-1}, \tag21$$ with $C$ independent of $G$. 

Going into polar coordinates, applying standard stationary phase, and making 
a change of variables sending $r \rightarrow r{\xi}_3^{-\frac{1}{2}}$, we 
get ${\xi_3}^{-1}$ times 
$$ I(A,t)=\int_{t^{\frac{1}{2}}[1,2]} 
e^{i(rAt^{-\frac{1}{2}}-t\phi(t^{-\frac{1}{2}}r))} r^{1}b(rAt^{-\frac{1}{2}})
dr, \tag22$$ where $A=|x'|$, $t=\xi_3$, and $b$ is a symbol of order 
$-\frac{1}{2}$. It suffices to prove that $I$ is uniformly bounded with 
constants independent of $A$, $t$, and $\phi$. 

Suppose  that either $A \approx |t|$ or $A >> |t|$. 
Now, by the van der Corput lemma, the integral $I(A,t)$ is bounded by 
$$ C|t^{\frac{1}{2}} b(A)|, \tag23$$ where $C$ is a universal constant, 
since the second derivative of the phase function of $I$, 
$\phi''(rt^{-\frac{1}{2}})$, is bounded below by $1$ on 
$[t^{\frac{1}{2}}, 2t^{\frac{1}{2}}]$ by assumption. The 
expression in $(23)$ is bounded above by another universal constant $C'$ 
since $b$ is a symbol of order $-\frac{1}{2}$, and $|t| \leq A$. 

It remains to handle the case when $A << t$. We undo the change of variables 
sending $r \rightarrow rt^{-\frac{1}{2}}$, and we let $h(r)=rA-t\phi(r)$. 
It is not hard to see that $|h'(t)| \ge |A-t| \ge |t|$. Van der Corput 
lemma gives us the decay $\frac{C}{t}$ and the proof of Lemma 6 is complete.  
\vskip.125in 

\head Orlicz norms of dilation operators \endhead 
\vskip.125in 

\proclaim{Lemma 7} Let $\Phi$ be a Young function such that  
$$ \Phi(a)\Phi(b) \leq \Phi(Cab) \tag24$$ for all $a,b>0$ with some 
$C>0$. Let $\det(A)=t$. Then  
$$ \frac{c}{\Phi^{-1}(t)}{||f||}_{\Phi} \leq {||\tau_Af||}_{\Phi} 
\leq C\Phi^{-1}\left(\frac{1}{t}\right) {||f||}_{\Phi}. \tag25$$ 
\endproclaim 

\proclaim{Lemma 8} Let $\Phi$ be a Young function such that 
$$ \Phi(a)\Phi(b) \ge \Phi(Cab) \tag26$$ for all $a,b>0$ with some 
$C>0$. Then 
$$ c \Phi^{-1}\left(\frac{1}{t} \right){||f||}_{\Phi} \leq 
{||\tau_{A}||}_{\Phi} \leq \frac{C}{\Phi^{-1}(t)} {||f||}_{\Phi}. \tag27$$ 
\endproclaim 

\proclaim{Lemma 9} Let $\Phi^{*}$ denote the conjugate function of $\Phi$
given by the equation 
$$ \Phi^{*}(s)=\inf_t (ts-\Phi(t)). \tag28$$ 

Let $N_A(\Phi)$ denote the $(L^{\Phi}, L^{\Phi})$ norm of the operator 
$\tau_A$. Then 
$$ \frac{1}{t} \leq N_A(\Phi)N_A(\Phi^{*}). \tag29$$ 
\endproclaim 

\head Proof of Lemma 7 and Lemma 8 \endhead 
\vskip.125in 

We must estimate $\inf \{s>0\}$ such that 
$$ \int \Phi \left( \frac{|f(Ax)|}{s}  
\right) dx \leq 1, \tag30$$ where we may assume that $\int \Phi(|f(x)|)dx=1$.

Making a change of variables and using $(24)$ this immediately reduces to 
$\frac{1}{\Phi(s)} \leq t$ which implies that $\Phi(s) \ge \frac{1}{t}$. It
follows that $s \ge \Phi^{-1}\left(\frac{1}{t} \right)$. Taking the $\inf$ 
proves the second inequality of $(25)$. Replacing $A$ by $A^{-1}$ we see
that the second inequality implies the first. This completes the proof of
Lemma 7. 

Making a change of variables and using $(26)$ reduces $(30)$ to 
$\Phi\left(\frac{1}{s} \right) \leq t$ which implies that 
$s \ge \frac{1}{\Phi^{-1}(t)}$. This proves the second inequality in 
$(27)$. Replacing $A$ by $A^{-1}$ we see again that the second inequality 
implies the first. This completes the proof of Lemma 8. 

\head Proof of Lemma 9 \endhead 
\vskip.125in 

By Holder's inequality 
$$ {||\tau_Af||}^2_2 \leq {||\tau_Af||}_{\Phi} 
{||\tau_Af||}_{\Phi^{*}}. \tag31$$

Since ${||\tau_Af||}^2_2=\frac{1}{t} {||f||}_2$, the conclusion of  
Lemma 9 follows. 

The following is a sample result obtained by combining Theorem 3, Lemma 7 
and Lemma 8. 

\proclaim{Theorem 10} Let $T_{\gamma}$ be as above. Suppose that $\gamma$
satisfies the conditions of Theorem 3. 

Suppose that $\Phi$ and $\Psi$ satisfy $(24)$.  
Then $T_{\gamma}: L^{\Phi}({\Bbb R}^2) 
\rightarrow L^{\Psi}({\Bbb R}^2)$ if 
$$ \sum_{j=0}^{\infty} 2^{-j} \Phi^{-1}\left(\frac{2^j}{h_j}\right) 
\Psi^{-1}(2^{-j}h_j)<\infty. \tag32$$ 

Suppose that $\Phi$ and $\Psi$ satisfy $(26)$. 
Then $T_{\gamma}: L^{\Phi}({\Bbb R}^2) \rightarrow 
L^{\Psi}({\Bbb R}^2)$ if 
$$ \sum_{j=0}^{\infty} \frac{2^{-j}}{\Phi^{-1}(2^{-j}h_j) 
\Psi^{-1}\left(\frac{2^j}{h_j}\right)}<\infty. \tag33$$ 

Suppose that $\Phi$ satisfies $(24)$ and $\Psi$ satisfies 
$(26)$. Then $T_{\gamma}: L^{\Phi}({\Bbb R}^2) \rightarrow 
L^{\Psi}({\Bbb R}^2)$ if 
$$ \sum_{j=0}^{\infty} 2^{-j} \frac{\Phi^{-1}\left(\frac{2^j}{h_j}\right)}
{\Psi^{-1} \left(\frac{2^j}{h_j} \right)}<\infty. \tag34$$ 

Suppose that $\Phi$ satisfies $(26)$ and $\Psi$ 
satisfies $(24)$. Then $T_{\gamma}: L^{\Phi}({\Bbb R}^2)
\rightarrow L^{\Psi}({\Bbb R}^2)$ if 
$$ \sum_{j=0}^{\infty} 2^{-j} \frac{\Psi^{-1}(2^{-j}h_j)}
{\Phi^{-1}(2^{-j}h_j)}<\infty. \tag35$$ 
\endproclaim 

\remark{Remark}
A three dimensional version of this result can be generated 
using Lemma 6 and the proof of Theorem 3 under the 
assumption that $\gamma''$ is increasing on $[0,2]$ by replacing 
$h_j$ by $\gamma_j$ and $2^j$ by $2^{2j}$ in Theorem 10 above. 
\endremark 

The following is a sample theorem obtained by combining Theorem 4, Lemma 7
and Lemma 8. 

\proclaim{Theorem 11} Let ${\Cal R}$ be as above. Suppose that $\gamma$ 
satisfies the conditions of Theorem 4. 

Suppose that $\Phi$ satisfies $(24)$. Then 
${\Cal R}: L^{\Phi}({\Bbb R}^2) \rightarrow L^2(\Gamma)$ if 
$$ \sum_{j=0}^{\infty} \frac{2^{\frac{j}{2}} \Phi^{-1}(2^{-j}h_j)}{h_j}
<\infty. \tag36$$ 

Suppose that $\Phi$ satisfies $(26)$. Then 
${\Cal R}: L^{\Phi}({\Bbb R}^2) \rightarrow L^2(\Gamma)$ if 
$$ \sum_{j=0}^{\infty} \frac{2^{\frac{j}{2}}}{h_j \Phi^{-1} \left( 
\frac{2^j}{h_j} \right)}<\infty. \tag37$$ 
\endproclaim 

\remark{Remark} Three dimensional analogs can be generated using Lemma 6
and the proof of Theorem 4 by replacing $h_j$ by $\gamma_j$ and $2^j$ 
by $2^{2j}$. \endremark 

\head Examples \endhead 
\vskip.125in 

\proclaim{Example 1} 
Let $\Phi(s)=\int_{0}^{s} \phi(t)dt$, where $\phi(t)=t^{q-1}\log^{-l}(t)$, 
$l>0$. Let $\Psi(s)=s^{q}$. Let $\gamma(s)=e^{-\frac{1}{s^{\alpha}}}$. 
A calculation shows that $\Phi(s) \approx s^{q} \log^{-l}(s)$ and that
$\Phi$ satisfies the condition $(24)$. Since $\Psi$ satisfies
any condition you want, Theorem 10, along with the subsection "Simplification"  
implies that 
$T_{\gamma}: L^{\Phi}({\Bbb R}^2) \rightarrow L^{\Psi}({\Bbb R}^2)$ if 
$$ \sum_{j=0}^{\infty} 2^{-j} 2^{j\frac{\alpha l}{q}}<\infty, \tag38$$ since 
$ \Phi^{-1}(s) \leq s^{\frac{1}{q}} \log^{\frac{l}{q}}(s)$ 
(see \cite{Bak95}, Example 1.3).

The sum in $(38)$ converges if $\alpha l<q$. 
\endproclaim 

\proclaim{Example 2} Let $\Phi^{-1}(s) \approx s\log^{-l}(s)$. 
Let $\gamma$ be as in Example 1. It is not 
hard to see that $\Phi$ satisfies $(26)$. Theorem 11 and the subsection 
"Simplification" imply that ${\Cal R}: L^{\Phi}({\Bbb R}^2) \rightarrow 
L^2(\Gamma)$ if 
$$ \sum_{j=0}^{\infty} 2^{-\frac{j}{2}} 2^{j\alpha l}<\infty, \tag39$$ 
which takes place if $\alpha l<\frac{1}{2}$. 
\endproclaim 

The necessity results in \cite{Bak94}, \cite{Bak94-2}, and \cite{BMO91}, 
associated with Theorems I, II, and III in the introduction, 
show that Examples 1 and 2 give 
optimal results, at least up to the endpoints. Indeed, to test the sharpness of 
Example 1, we just test $T_{\gamma}$ against a characteristic of a cube 
with side-lengths $\delta$ and $\gamma(\delta)$, (see \cite{BMO91}), 
whereas the sharpness of 
Example 2 follows by a variant of the classical Knapp homogeneity 
argument. See \cite{Bak94}.

{\bf Acknowledgements:} The author wishes to thank J. Vance   
for many helpful conversations and for pointing out the reference 
\cite{CCVWW89}.  
\vskip.125in 

\newpage 

\head References \endhead 
\vskip.25in 

\ref \key Bak94 \paper Restrictions of Fourier tranforms to flat curves in 
${\Bbb R}^2$ \by J.-G. Bak \jour Illinois J. Math. \vol 38 \yr 1994 \endref

\ref \key Bak94-2 \by J.-G. Bak \paper Sharp convolution estimates for 
measures on flat surfaces \jour (preprint) \yr 1994 \endref 

\ref \key Bak95 \by J.-G. Bak \paper Averages over surfaces with infinitely 
flat points \jour J. Func. Anal. \yr 1995 \vol 129 \endref 

\ref \key BMO91 \by J.-G. Bak, D. McMichael, and D. Oberlin \paper Convolution
estimates for some measures on flat curves \jour J. Func. Anal. \vol 101
\yr 1991 \endref 

\ref \key CCVWW89 \by A. Carbery, M. Christ, J. Vance, S. Wainger, and 
D. Watson \paper Operators associated to flat plane curves: $L^p$ estimates
via dilation methods \jour Duke Math. J. \vol 59 \yr 1989 \endref 

\ref \key CarSj72 \by L. Carleson and P. Sjolin \paper Oscillatory integrals
and the multiplier problem for the disc \jour Studia Math. \vol 44 
\yr 1972 \endref 

\ref \key Litt73 \by W. Littman \paper $(L^p, L^q)$ estimates for 
singular integral operators \jour Proc. Symp. Pure Math. \vol 23 
\yr 1973 \endref 

\ref \key RiSt88 \by F. Ricci and E. Stein \paper Harmonic analysis on 
nilpotent groups and singular integrals II \jour J. Func. Anal. 
\vol 78 \yr 1988 \endref 

\ref \key St93 \by E. Stein \paper Harmonic Analysis 
\jour Princeton Univ. Press \yr 1993 \endref 

\ref \key Str70 \by R. Strichartz \paper Convolution with kernels having
singularities on a sphere \jour Trans. Amer. Math. Soc. \vol 148 
\yr 1970 \endref 

\ref \key Tom75 \by P. Tomas \paper A restriction theorem for the Fourier 
transform \jour Bull. Amer. Math. Soc. \vol 81 \yr 1975 \endref 

\ref \key Tor76 \by A. Torchinsky \paper Interpolation of operators and 
Orlicz classes \jour Studia Math. \vol 59 \yr 1976 \endref 

\ref \key Tor86 \by A. Torchinsky \paper Real variable methods in harmonic
analysis \yr 1986 \jour Academic Press, Orlando \endref 

\enddocument 

