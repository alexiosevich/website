\documentstyle{amsppt}
\tolerance 3000 \pagewidth{5.5in} \vsize7.0in
\magnification=\magstep1 \widestnumber \key{AAAAAAAAAAA}
\NoRunningHeads \topmatter \title How to complete the square
\endtitle
\author 
\endauthor
\endtopmatter
\document

This is a quick reminder on how to complete the square. More precisely,
suppose that we want to rewrite 
$$ ax^2+bx+c \tag1$$ in the form 
$$ M{(x-z)}^2+T. \tag2$$ 

We will end up with a formula in a moment, but let's play with $(1)$ and
$(2)$ for a moment to see how such a formula comes about. We have 
$$ M{(x-z)}^2+T=Mx^2-2Mzx+Mz^2+T, \tag3$$ and we want this expression to
$(1)$ for every possible $x$. This means that all the coefficients must
be equal, so 
$$ M=a, \ -2Mz=b, \ \text{and} \ Mz^2+T=c. \tag4$$ 

Let's unravel this puzzle piece by piece. The first equality in $(4)$
gives us 
$$M=a. \tag5$$ 

We now plug $(5)$ into the the second equality in $(4)$ and obtain 
$$ z=-\frac{b}{2M}=-\frac{b}{2a}. \tag6$$ 

We must now deal with the third equality in $(4)$. We get 
$$ c=Mz^2+T=a \cdot {\left( -\frac{b}{2a} \right)}^2+T. \tag7$$ 

We conclude that 
$$ T=c-a \cdot {\left( -\frac{b}{2a} \right)}^2=c-\frac{b^2}{4a}. \tag8$$

Putting evverything together we see that 
$$ M=a, \ z=-\frac{b}{2a}, \ T=c-\frac{b^2}{4a}, \tag9$$ and we have
ourselves a formula. 

\vskip.125in 

\proclaim{Example} Let $f(x)=4x^2+24x+32$. Complete the square. By the
formula, $a=4, b=24$, and $c=32$. It follows that $M=4$, $z=-3$, and
$T=-4$. It follows that 
$$ 4x^2+24x+32=4{(x+3)}^2-4. \tag10$$

\endproclaim 

\proclaim{Ugly Example} Sometimes numbers do not divide nicely. Consider 
$f(x)=3x^2-5x+7$. Applying the formula above we see that 
$$ 3x^2-5x+7=3{\left(x-\frac{5}{6}\right)}^2+\frac{59}{12}. \tag11$$

\endproclaim









\enddocument 
