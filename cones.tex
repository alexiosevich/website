\documentstyle{amsppt}

\magnification=\magstep1

\pagewidth{5.5in}

\vsize=7in

%

\def\n#1{\left\vert\,{#1}\,\right\vert}

\def\N#1{\left\Vert\,{#1}\,\right\Vert}

\def\d#1#2{{{\partial^{#1}}\over{\partial{#2}}}}

%

\def\R{{\Bbb R}}

\def\C{{\Bbb C}}

%

\def\cal{\Cal}

%

\topmatter    

\title     $L^p$ estimates for the Cauchy kernel with respect to convex cones 

\endtitle

\author Laura De Carli and Alex Iosevich 

\endauthor

\endtopmatter

\document

%

\noindent{\bf Introduction}:

Let $\Gamma$  be an open and convex cone in $\R^{n+1} $ with vertex at $0$. 

Define the {\it dual cone of $\Gamma$} as 

$$ \Gamma ^0 =\lbrace \ y \in \R^{n+1}

\,: \langle y,\zeta \rangle > 0 \quad \forall \zeta \in

\Gamma\, \rbrace, 

\tag 0.1

$$

where $\langle\ ,\ \rangle$ denotes the usual scalar product in $\R^{n+1}$.



If we assume that $\Gamma$    

does not contain any straight line, $\Gamma^0\ne \emptyset$.

%

We define  the {\it Cauchy kernel with respect to  $\Gamma$ } as the 

holomorphic function

$ 

K_{\Gamma} : \R^{n+1}+i\Gamma \rightarrow  \C$,

$$

K_{\Gamma} (x+iy) = {\strut \int_{\Gamma^0}  e^{i\langle x+iy,\ \zeta 

\rangle}\,d\zeta.} \tag 0.2

$$

Let $f\in {\Cal S}( \R^{n+1}) $. Define the 

{\it Cauchy transform  of $f$ with respect to $\Gamma$ } 

as the  holomorphic function 

$

T_{\Gamma}(f): \R^{n+1}+i\Gamma  \rightarrow  \C,

$ 

$$

T_{\Gamma}f\,(x+iy)  =  (2\pi)^{-n-1}\,\int_{\Gamma^0} \, \hat f(\zeta) e^

{i\langle  x+iy,\,\zeta \rangle} \, d\zeta,  \tag 0.3

$$

where $\hat f$ denotes the Fourier transform of $f$. The definitions and the

basic properties of the Cauchy kernels and  the Cauchy transform with respect to

a convex cone can be found in [V].   Let $T_{\Gamma,\,y}f(x)= T_\Gamma f(x+iy)$.

Observe that

$T_{\Gamma}f(x+iy)= (K_{\Gamma,y}*f)(x)$, where

$*$ denotes the  usual convolution, and we have set $K_{\Gamma,y}(x)=

K_\Gamma(x+iy)$. Observe also that  

$$

\lim_{{y\to 0}\atop{y\in \Gamma^0}}T_{\Gamma,\,y}f =

f*\widehat{\chi_{\Gamma}},\qquad f \in {\Cal S}(\R^{n+1}),

$$

where  $\chi_\Gamma(x)$ is the characteristic function of  $\Gamma$,

and the above limit is intended in distribution sense. We wish to find optimal

ranges of exponents  $(p,\,q)$ such that  

$$

\N{T_{\Gamma,\,y}f}_q\leq C(y,\,\Gamma)\N f_p

, \qquad f \in {\Cal S}(\R^{n+1}),

\tag 0.4

$$

where $\dsize C(y,\, \Gamma)= \sup_{||f||_p=1}\N{T_{\Gamma,\,y}f}_q< 

\infty$ is a positive constant which depends on

$\Gamma$ and on

$|y|$, and we have denoted by $\N{\ }_r$ the usual norm on $L^r(\R^{n+1})$. 



An easy scaling argument  shows that if

$(0.4)$ holds, then $q\ge p$ and $C(y,\, \Gamma)\leq$ \newline$

|y|^{-(n+1)\left|\frac 1q-\frac 1p\right|}C(\Gamma,\, \frac{y}{|y|})$.   See also

[H]. Let

${\cal T}_\Gamma$ be the set of the $\left(\frac 1p,\,  \frac 1q\right)$'s such

that $(0.4)$ holds.  Let ${\Cal Q}$ be the closed  square of  vertices $(\frac 12,

\,0)$, $(1,\, 0)$, $(\frac 12,\, \frac 12)$, $(1,\,\frac 12)$, and let ${\Cal T} $

be  the  closed triangle of vertices $(0,\,0)$, $(1,\,1)$, $(1,\, 0)$.  In the next

section we will show that, for every cone $\Gamma$,  ${\Cal Q}\subset {\cal

T}_\Gamma\subset {\Cal T}$. These results can be summarized in the following 



\proclaim{Proposition 0.1} 

${\cal T}_\Gamma \supset \left\{\left(\frac 1 p,\, \frac 1 q\right) : q\ge 2\

\hbox{and}\  p\leq 2\right\}$  for every convex cone  $\Gamma\subset \R^n$. 

Moreover, if  $\left(\frac 1 p,\, \frac 1 q\right)\in {\cal T}_\Gamma$, then 

$q\ge p$ and 

$$C(y,\, \Gamma)\leq

|y|^{-(n+1)\left|\frac 1q-\frac 1p\right|}C\left(\Gamma, 

\frac{y}{|y|}\right). $$

\endproclaim

%

We will prove Proposition $0.1$ in the next section.

When  $\Gamma$ is  the intersection of a finite number of 

half spaces passing through the origin of $\R^{n+1}$, one can easily see that 

$\chi_\Gamma(x)$ is a $L^p$ Fourier multiplier for $p\in (1,\, \infty)$.

Moreover, $\ K_{\Gamma ,y}  

\in L^r\,(\R^{n+1})$ for $ r\in (1,\, \infty]$, 

$y \in \Gamma$,  and

$$ 

\N{K_{\Gamma,y}}_r  \leq   C\,\hbox{dist}\,(y,\,\partial\Gamma)^{-(n+1)(1-\frac 1

r)}\qquad

\hbox{for all}

\quad y\in \Gamma.


\tag 0.5

$$

The  proof is  in [D1]. Using   Young's inequality for

convolution and $(0.5)$ we can prove  the following.   

\proclaim{Theorem 0.2}

Let  $\Gamma$ be as above. 

Then  $T_{\Gamma,y}(f)(x) \in L^{q}(\R^{n+1})$  for every $ y \in \Gamma$,  

$ q \in (1, \, + \infty]$ and   $ f  \in  {\Cal S}(\R^n)$, and, for 

every $p \leq q$, $p \ne \infty$,  

$$

\N{T_{\Gamma ,y} f}_q  \leq   C(\Gamma)\,\hbox{dist}\,(y,\,\partial\Gamma)

^{-(n+1)\left|\frac 1q-\frac 1p\right|} 

\N{ f}_p.

\tag 0.6

$$

\endproclaim 

\medskip 

\noindent

Thus, ${\Cal T}_\Gamma= {\Cal T}$ if $\Gamma$ is as in Theorem $0.2$. $(0.6)$ does

not hold when 

$p=\infty$, (and consequently $q=\infty$).  Let for

example $\Gamma =(0,\,+\infty)$, and  let $f(x)=\chi_{(a,\,b)}$, where $0<

a<b$.  Then one can check that 

$T_\Gamma f(x+iy)= \int_a^b\frac{1}{t-x-iy}\,dt

=\log\left(\frac{b-\zeta}{a-x-iy}\right)$, which is not uniformly bounded with

respect to 

$y$.

\medskip

It is well known that nothing as strong as 

Theorem $0.2$ can hold when $\Gamma$ is not the intersection of half spaces.

However, the following result is proved in [D1].

%

\proclaim{ Proposition 0.3} Suppose that, after perhaps a rotation and dilation

of coordinates,  

$\Gamma$  is  the "light cone"  $\{ (x_0,\,x_1,\,\cdots, x_n)\in \R^{n+1} :

x_0^2\ge x_1^2+\cdots x_n^2\}$. Then $K_{\Gamma ,y}  

\in L^r\,(\R^{n+1})$ for $ r >\frac{ 2n}{n+1}$, 

$y \in \Gamma$,  and 

$$ \N{K_{\Gamma,y}}_r  \leq   C(\Gamma)|y|^{-(n+1)\left|1-\frac 1 r\right|}.

$$

The above range of exponents for $r$ is sharp.

\endproclaim

The following Proposition is an easy consequence of  Proposition $0.3$

and Young inequality for convolution.   




\proclaim{Proposition 0.4} Let $\Gamma$ be as in Proposition $0.3$.

Then, for every $(p,\,q)$ 

satisfying \newline $\frac 1p> \frac{n-1}{2n} + \frac 1q$ 

$$ 

\N{T_{\Gamma,y} f}_q  \leq  C(\Gamma)\,

|y|^{-(n+1)\left|\frac1p-\frac1q\right|}\,\N{ f}_p.    \tag 0.7

$$ 

\endproclaim

If $\Gamma$ is as in Proposition $0.4$, ${\cal T}_\Gamma$ contains 

the open triangle of vertices

$(1,\,0)$, $(\frac 12 - \frac 1{2n}, \, 0)$, $(0,\, \frac 12 + \frac 1{2n})$.

Since ${\cal T}_\Gamma$ contains ${\Cal Q}$, by the M. Riesz interpolation theorem

${\cal T}_\Gamma$ contains the open 

trapezoid  ${\Cal Q}_n$ of vertices

$(1,\,0)$, $(\frac 12 - \frac 1{2n}, \, 0)$, $(0,\, \frac 12 + \frac 1{2n})$,

$(\frac 12,\, \frac 12)$. 

\medskip

Our main aim is to generalize

Proposition $0.4$ to more general cones. We recall the following definitions. 

Let $\Sigma$ be a submanifold of 

${\R}^{N+1}$ 

of codimension $1$ equipped with a smooth compactly supported measure $d\mu$.

Let $J:\Sigma \to S^{N}$ be the usual Gauss map taking each point on 

$\Sigma$ to the outward  unit normal of $\Sigma$ at that point. We say that

$\Sigma$ has everywhere {\it nonvanishing Gaussian curvature} if the differential

of the  Gauss map $dJ$ is always nonsingular.

The {\it principal curvatures} of  $\Sigma$ are the eigenvalues of $dJ$.



In Section $1$ we will extend  Proposition $0.4$  to cones

whose boundaries have  $n-1$  everywhere  nonvanishing

principal curvature. 

\vskip.25in

\centerline{SECTION  1} 

\bigskip

The main aim of this section is to prove Proposition $0.1$ and  generalize

Proposition  $0.4$. We begin with the proof of Proposition $0.1$. 



\medskip

\noindent

{\it Proof of Proposition $0.1$.} We first prove that if $\Gamma$ is a  convex

cone, and if

$(0.4)$ holds for $(p,\, q)$, then     $C(y,\,

\Gamma)\leq 


|y|^{-(n+1)\left|\frac 1q-\frac 1p\right|}C(\Gamma,\, \frac{y}{|y|})$. For  

this purpose we use a classical scaling argument.



Let $y  \in \Gamma$, with   $|y|=\epsilon$.


Let

$f_\epsilon(x)= f(\epsilon^{-1} x)$. It is not difficult to see that

$T_\Gamma f_\epsilon(x+iy)= T_\Gamma f(\epsilon^{-1} (x+i y))$, and hence

$\N{T_{\Gamma,\,y} f_\epsilon }_q= \epsilon^{\frac{n+1}{q}}\N{T_{\Gamma,

\epsilon^{-1}y} f }_q$. Moreover, $\N{f_\epsilon }_p =

\epsilon^{\frac{n+1}{p}}\N f_p$. By $(0.4)$, 

$$

\N{T_{\Gamma,\,\epsilon^{-1}y} f }_q\leq C(\epsilon^{-1}y,\, \Gamma) 

\N{f}_p, $$

which is equivalent to  

$$

\frac{\N{T_{\Gamma, \, y}\, f_\epsilon  }_q}{\N{f_\epsilon}_p}\leq

\epsilon^{-(n+1)\left|\frac{1}{q}-\frac{1}{p}\right|}C( \epsilon^{-1}y,\,

\Gamma).

$$

By assumption,

$\dsize 

\sup_{||f||_p=1} \N{T_{\Gamma, \, y}\, f }_q=

C(y,\,\Gamma)

$

and thus $$

 C(y,\,\Gamma)\leq 

\epsilon^{-(n+1)\left|\frac{1}{q}-\frac{1}{p}\right|}\,C( 


\epsilon^{-1}y,\, \Gamma),

$$

as required.





\medskip

Let ${\Cal Q}$ be the closed  square of  vertices $(\frac 12,\,

0)$, $(1,\, 0)$, $(1,\, \frac 12)$, $(\frac 12,\, \frac 12)$, and let ${\Cal T} $

be  the  closed triangle of vertices $(0,\,0)$, $(1,\,1)$, $(1, 0)$.

We show that ${\Cal T}_\Gamma$ contains ${\Cal Q}$. If $q\ge 2$ and $y\in

\Gamma$, by  the Hausdorff-Young inequality 

$$\N{T_{\Gamma,\,y} f}_q\leq \N{

e^{-\langle y, \, .\rangle }\hat f }_{L^{q^\prime}(\Gamma^0)},

\tag 1.2

$$ where $q^\prime$

denotes the dual exponent of $q$.  Note that $q^\prime \leq 2$. Take  $p\leq

2$. Let $\frac 1r$ $=$ $\frac{1}{q^\prime} - \frac {1}{p^\prime}$ $=$ $\frac

1p - \frac 1q$. By the H\"older inequality, 

$$

\N{

e^{-\langle y, \, .\rangle }\hat f }_{L^{q^\prime}(\Gamma^0)} \leq 

\N{ e^{-\langle y, \, .\rangle }}_{L^r(\Gamma^0)}\N{\hat f }_{p^\prime}.

\tag 1.3

$$

It is not difficult to see  that $\dsize\left(\int_{\Gamma^0}e^{-r\langle y, \,

\zeta\rangle }d\zeta\right)^{\frac1r} = C(\Gamma)|y|^{-\frac{n+1}r}$. By the

Hausdorff-Young inequality,

$(1.2)$,

$(1.3)$, and the above observation, we obtain  

$$

\N{T_{\Gamma,\,y} f}_q \leq C(\Gamma)|y|^{-(n+1)\left|\frac 1q-\frac1p\right |}\N

f_p,

$$

as required.





To prove that ${\Cal T}_\Gamma\subset {\Cal T}$ we  

consider the quadrant 

$L = \{x= (x_0, \, \cdots, \, x_n)  : x_j > 0 \}$. It is not difficult to see

that $L= L^0$, and that $K_L(x+iy)= \Pi_{j=0}^{n}\frac{i}{x_j+iy_j}$. Let $e_0=

(1,\, 1,\, \cdots,\, 1)$. Let

$\hat f(\zeta)=e^{-\langle \zeta ,\, \epsilon e_0\rangle }\chi 

_{L}(\zeta)$, whith $\epsilon >0$. Then,

$

f(x)= K_L(x+i\epsilon e_0)$ and 

$T_Lf(x+i\epsilon e_0)=   K_L(x+2i\epsilon e_0)

$.

A direct computation shows that 

$\N f_p= c(p) \epsilon^{-\frac{n+1}{p}}$, and $\N{T_{L,

\,\epsilon e_0}f}_q =  c(q) \epsilon^{-\frac{n+1}{q}}$,

where $c(p)$ and $c(q)$ are positive constants that depend only on  $p$ and $q$.

Thus, 

$$

\frac{\N{T_{L, \,\epsilon e_0}f}_q}{\N f_p}= O\left( 

\epsilon^{\frac{n+1}{p}- \frac{n+1}{q}}\right).

\tag 1.4

$$

When $p>q$ the power of $\epsilon $ in $(1.4)$ is negative. Since $\epsilon $

can be taken arbitrarily small, $(0.4)$ does not  hold. 

 

We state and prove the main result of this paper.

\proclaim{Theorem 1.1} 

Let $\Gamma$ be a convex cone in  $\R^{n+1} $. 

Suppose that   $\partial \Gamma\slash \{0\}$   has  

$n-1$  everywhere  nonvanishing principal curvatures. 

Then, for every $(\frac 1p,\,\frac 1q)\in {\Cal Q}_n$,  

$$ 

\N{T_{\Gamma,y} f}_q  \leq  C(\Gamma) |y|^{-(n+1)\left|\frac 1p-\frac

1q\right|}\,\N{ f}_p.    \tag 1.5

$$ 

\endproclaim

We will prove Theorem 

$1.1$ using Young inequality for convolution and the following

$L^p$ estimates for the  norm of the Cauchy kernel with respect to $\Gamma$.   



\proclaim{  Theorem 1.2} Let $\Gamma$ be as in Proposition $1.1$ Then $

K_{\Gamma,y} \in L^r(\R^{n+1})$ for $\frac{ 2n}{n+1}< r <\infty$,

$y \in \Gamma$,  and   

$$ 

\N{K_{\Gamma,y}}_{r}  \leq  C(\Gamma) 

 |y|^{-(n+1)(1-\frac 1r)}.

$$ 

The above range of exponents for $r$ is sharp.

\endproclaim



To prove Theorem $1.2$  we shall use the classical sharp estimates for 

the Fourier 

transform of the surface-carried measure that we recall here. 

%

\proclaim{ Lemma 1.3}

Let $S$ be a smooth hypersurface in $\R^{n}$ 

with nonvanishing Gaussian curvature, and let 

$d\sigma $

be a $ C^\infty$ measure  on $S$.

Then

$$

\n{\widehat{ d\mu}(\xi)} \leq C(1+\n\xi)^{-\frac {n-1}{2}} .

\tag 1.6

$$

Moreover suppose that $\Gamma\subset \R^{n}\backslash\{0\} $ is the cone

consisting of all $\xi$ which are normal to some point $x\in S$ belonging to

some compact neighborhood ${\Cal N}$ of support$(d\mu)$. Then,

$$

\d{\alpha}{\xi}\widehat{ d\mu}(\xi) = {\Cal O}\left((1+\n\xi)^{-N}\right), 

\qquad \forall N, \text{ if }

\xi\not\in\Gamma,

$$

$$

\widehat{ d\mu}(\xi) =\sum

b_j(\xi)e^{i\langle x_j,\,\xi\rangle},\qquad \text{ if }\xi\in\Gamma,

$$

where the finite sum is taken over all points $x_j\in {\Cal N}$ having

$\xi$ as a normal, and 

$$

\n{\d{\alpha}{\xi} b_j(\xi)} \leq C_\alpha(1+\n\xi)^

{-\frac {n-1}{2}-\n\alpha}.

\tag 1.7

$$

\endproclaim

\bigskip

\noindent

{\it Proof.} See  [So] pg. 50-51.

\bigskip

\noindent

Before proving  Theorem $1.2$ we shall prove the following

technical lemma.

\proclaim{ Lemma 1.4} 

Let $\Gamma \subset \R^{n+1}$ be a convex cone. Then    

$\partial \Gamma\slash \{0\}$ is  smooth and  has 

$n-1$  everywhere  nonvanishing principal curvatures if and only if $\partial

\Gamma^0\slash

\{0\}$   is  smooth and has   $n-1$  everywhere  nonvanishing

principal curvatures.

\endproclaim

\bigskip

\noindent

{\it Proof.} Since 

$\Gamma= (\Gamma^0)^0$  for every   convex cone $\Gamma$, we need only to

prove that  when   

$\partial \Gamma\slash \{0\}$ is  smooth and has  $n-1$  everywhere 

nonvanishing principal curvatures,  then  also  

$\partial \Gamma^0 \slash \{0\}$ is smooth and has  $n-1$  everywhere 

nonvanishing principal curvatures.



Suppose for now that $\Gamma = \{ (x_0, x) :x_0 \ge \phi(x)\}$, where $\phi\in

C^{\infty}(\R^n\slash\{0\})$ is a homogeneous function of degree $1$ that

vanishes only at the origin. Without loss of generality,  

$\phi(x) >0$ when $x \ne 0$. Then, $\Gamma

\subset \{ x_0 \ge 0\}$. Since $\Gamma$ is convex and

$\partial \Gamma\slash \{0\}$  has $n-1$  everywhere 

nonvanishing principal curvatures, the level set $S= \{x\in \R^n :

\phi(x) \leq 1\}$ is  convex, (and also compact), and the  Hessian matrix of

$\phi$ has $n-1$ nonvanishing eigenvalues on $\R^n\slash \{0\}$. 

We recall that 


$$

\Gamma^0 =\{(\zeta_0, \, \zeta)\in\R\times\R^{n} : x_0 \zeta_0 + \langle

x,\, \zeta\rangle \ge 0 \ \forall (x_0, \, x) \in \Gamma\},

\tag 1.8 $$

where $\langle \ ,\ \rangle$  denotes the scalar product in $\R^n$. 

Thus, one can  see that   $\Gamma^0\subset\{\zeta_0\ge 0\}$. Moreover, $(0, \,

\zeta)\in\Gamma^0$ if and only if $\zeta =0$, and 

$(\zeta_0,\,\zeta)\in \Gamma^0 $ if and only if  

$$

\min_{x \in S}\left(1+\left\langle

\frac x{x_0},\, \frac\zeta{\zeta_0}\right\rangle\right) \ge 0

\tag 1.9

$$

for  $x_0>0$, $\zeta_0 >0$. Without loss of generality we can let

$x_0=\zeta_0 =1$. 

We shall  find the minimum of  $F_\zeta(x)=1+ \langle x,\, \zeta

\rangle$ on  $S$.  Since the gradient of $F_\zeta(x)$

vanishes in the interior of $S$ if and only if $\zeta =0$, we shall find the 

stationary points of $F_\zeta(x)$ on the boundary of $S$. By the  Lagrange

multiplier theorem, these points are the solutions of  the system

$$

\cases \zeta = \lambda \nabla \phi(x)

\cr

\phi(x)=1 

\endcases

\tag 1.10

$$

for some $\lambda \ne 0$. By the Bolzano-Weirstrass

theorem $F_\zeta$ has a minimum and a maximum on $S$, and hence the system 

$(1.10)$ has two solutions.  Let  $(x_1(\zeta),\, \lambda_1(\zeta))$ be one of

these solutions. Let

$\Sigma =

\{�x : \phi(x)=1\}$. If $\lambda_1(\zeta)>0$,

$x_1(\zeta)$ is the point of $\Sigma$ having $\frac\zeta {\n \zeta}$ as the

outward unit normal. Since $S$ is convex, there exists 

$x_2(\zeta)\in \Sigma$ having $-\frac\zeta {\n \zeta}$ as the

outward unit normal. The second solution of $(1.10)$ is thus $(x_2(\zeta), \,

\lambda_2(\zeta))$, with $\lambda_2(\zeta) < 0$.  Thus,

$$

F_\zeta(x_j(\zeta)) = 1+\lambda_j(\zeta)\langle \nabla \phi(x_j(\zeta)),\,

x_j(\zeta)\rangle, \qquad j=1,\ 2.

$$  

Since $\phi$ is homogeneous of degree $1$, by  Euler's homogeneity relation, 

$$

F_\zeta(x_j(\zeta)) = 1+\lambda_j(\zeta)\phi(x_j(\zeta))= 1+\lambda_j(\zeta). 

$$

Then, $\zeta$ satisfies $(1.9)$ if and only if 

$$

1+\lambda_j(\zeta) \ge 0, \qquad j=1,\ 2.

\tag 1.11

$$

Since $\lambda_1(\zeta)>0$, $(1.11)$ is satisfied if and only if

$1+\lambda_2(\zeta)

\ge 0$.  We will denote $(x_2(\zeta), \,

\lambda_2(\zeta))$ by $(x(\zeta), \,\lambda(\zeta))$ from now on. We show

that

$\lambda(\zeta)$ is a homogeneous function of degree

$1$ and is smooth away from the origin. If we  scalar multiply  both

sides of the first equation of $(1.10)$ by  $x(\zeta)$, by  Euler's homogeneity

relation, 

$$

\lambda = \langle x(\zeta), \, \zeta\rangle.

\tag 1.12

$$

After differentiating both sides of $(1.12)$,

$$

\nabla\lambda(\zeta) = x(\zeta) + Dx(\zeta)\zeta,

\tag 1.13

$$

where $Dx(\zeta)$ denotes the differential of $x(\zeta)$. Since one can

easily see that  

$$

0 = \nabla_\zeta(\phi(x(\zeta)) = \lambda^{-1}Dx(\zeta)\zeta,

\tag 1.14

$$

from $(1.14)$ and Euler's homogeneity relation follows that 

$x(\zeta)$ is homogeneous of degree $0$, and from $(1.12)$ and  $(1.13)$ that 

$\lambda(\zeta)$ is homogeneous of degree $1$.  Observe that  $(1.13)$ and the

fact that

$Dx(\zeta)\zeta=0$ yields  $\nabla\lambda(\zeta) = x(\zeta)$. 



Until now  we have  not used the assumption that $\Sigma$ has

nonvanishing Gaussian curvature. This assumption will turn out to be

necessary to prove that $\lambda$ is smooth away from the origin. We recall

in fact  that $\dsize {x(\zeta)_\vert}_{S^{n-1}}$ is the inverse of the Gauss map

of $\Sigma$, which is smooth and nonsingular by assumption. Then $x(\zeta)$ is

also smooth and nonsingular on $\R^{n}\slash \{0\}$, being homogeneous of

degree $0$. By $(1.12)$, $\lambda(\zeta)$ is also smooth on $\R^{n}\slash

\{0\}$. Since $\nabla\lambda(\zeta) = x(\zeta)$, the Gauss map of the set 

$\Sigma^0 =\{ \zeta : \lambda(\zeta) =1\}$ is $Dx(\zeta)$, which is

nonsingular. This concludes the proof of the Lemma.

\bigskip

\noindent{\it Remark:} Consider the  cone

$$

\Gamma = \{ x \in \R^{3} : x_0 \ge (x_1^4+x_2^4)^{\frac 14}\}.

$$

The curvature of  $\Sigma= \{x_1^4+x_2^4 =1\}$ vanishes at $(0,\, \pm 1)$ and 

$(\pm 1,\, 0)$. To compute $\Gamma^0$ we  repeat the argument that we used to

prove   Lemma $1.4$. We obtain   

$$

\Gamma^0 = \{ \zeta \in \R^{3} : \zeta_0 \ge (\zeta_1^{\frac43}

+\zeta_2^{\frac43})^{\frac 34}\},

$$

and $\Sigma^0 = \{ \zeta_1^{\frac43}+\zeta_2^{\frac43}=1\}$ is not

everywhere smooth.  \bigskip

\noindent

{\it Proof of Theorem $1.2$.} Since $\Gamma$ is convex, there is not loss of

generality to assume that the "light cone" $\{x : x_0 \ge (x_1^2 + \cdots +

x_n^2)^{\frac 12}\}$ contains $\Gamma$.  Let 

$y\in\Gamma$. Let $L$ be the Lorentz transformation that maps  

$ y$ to  $(y_0,\, 0,\, \cdots ,  \, 0)$, $y_0 > 0 $. Let 

$L(\Gamma)=\Gamma^\prime$. One can easily see that $\Gamma^\prime$ is a convex cone

and  $\partial \Gamma^\prime

\slash \{0\}$ has everywhere $n-1$ nonvanishing principal curvatures. Moreover, 

$(\Gamma^\prime)^0 = L(\Gamma^0)$. In fact, $\zeta \in (\Gamma^\prime)^0 $ if

and only if $\langle x^\prime,\, \zeta\rangle \ge 0$ for every $x^\prime \in

\Gamma^\prime$. Since $x^\prime = L(x)$ for some  $x \in \Gamma$, we have

that $\zeta

\in (\Gamma^\prime)^0 $ if and only if $L^*(\zeta) \in \Gamma^0$, where $L^*$

denotes the adjoint of $L$. Since $L^* = L^{-1}$,  $\zeta \in

L(\Gamma^0)$. 



The above argument shows that there is not loss of generality to replace

$\Gamma$ by $\Gamma^\prime$ and to assume $y= (y_0, \, 0, \, \cdots 0)$. In order

to  simplify notation, $\Gamma^\prime=\Gamma$. 



In what follows we will always set   $ x=(x_0,\, x^\prime ) $, $x_0\in \R$,

$x^\prime\in \R^n$. We  will also use the convention that $C$ denotes a

constant that can vary from line to line.  With the above notation, 

$$

K_\Gamma (x+iy) = 

\int_{\Gamma^0}e^{i(\zeta_0(x_0+iy_0)+\langle \zeta^\prime, \, x^\prime\rangle)}

d\zeta^\prime\, d\zeta_0.

\tag 1.15

$$

Let   $h(\zeta)\in C^\infty(\R^{n+1})$ be  $\equiv 0$  when $\n\zeta\leq 1$ and 

$\equiv 1$ when

$\n \zeta\ge 2$. Let

$$

L_\Gamma(x+iy)=  \int_{\Gamma^0}e^{i(\zeta_0(x_0+iy_0)+\langle

\zeta^\prime,

\, x^\prime\rangle)}h(\zeta) d\zeta^\prime\, d\zeta_0,

\tag 1.16

$$  

We first show that Theorem $1.2$ follows if we prove   

$$

\N{L_{\Gamma,y}}_{r}  \leq  C(\Gamma)\, 

 |y|^{-(n+1)(1-\frac 1r)}.

\tag 1.17

$$ 

Let 

$$

L_{\epsilon}(x+iy)=  \int_{\Gamma^0}e^{i(\zeta_0(x_0+iy_0)+\langle

\zeta^\prime,

\, x^\prime\rangle)} h(\epsilon^{-1}\zeta)d\zeta^\prime\, d\zeta_0,

\qquad

\epsilon > 0.

$$  

An easy   change of variables shows that 

$  L_{\epsilon}(x+iy)=\epsilon^{n+1}L_\Gamma((x+iy)\epsilon).

$

Then, 

$$

\N{L_{\epsilon,\,y}}_r= \epsilon^{(n+1)(1-\frac 1r)}\N{L_{\Gamma,\,

y\epsilon}}_{r},

\tag 1.18

$$

and, by  $(1.17)$,


$$

\N{L_{\epsilon,\,y}}_r

\leq  C(\Gamma) 

 |y|^{-(n+1)(1-\frac 1r)}.

\tag 1.19


$$ 

The $\{L_{\epsilon,\,y}\}$'s are thus uniformly bounded in

$L^r(\R^{n+1})$ by a constant which  depends only on $y$. Since the unit ball of 

$L^r(\R^{n+1})$ is weakly compact, there exists a subsequence of 

$\{L_{\epsilon,\,y}\}_{\epsilon >0}$ that  converges  to a function $F_y(x)$ in

the weak topology of $L^r(\R^{n+1})$. Then, $F_y\in L^r(\R^{n+1})$, and 

$\N{F_y}_r\leq   C(\Gamma) |y|^{-(n+1)(1-\frac 1r)}$.  Since  

$L_{\epsilon,\,y}(x)\to K_{\Gamma,\,y}(x)$ as

$\epsilon \to 0$, by the Lebesgue dominated convergence theorem, we can conclude that  

$K_{\Gamma,\,y}(x) = F_y(x)$, $K_{\Gamma,\,y}(x)\in L^r(\R^{n+1})$, and 

$\N{K_{\Gamma,\,y}}_r \leq  C(\Gamma) |y|^{-(n+1)(1-\frac 1r)}$.



\medskip

We shall then prove $(1.17)$. Observe first that

$$ 

L_\Gamma (x+iy) = 

\int_1^{+\infty}\int_{\Gamma^0 \cap \{\zeta_0=t\}}

e^{i(t(x_0+iy_0)+\langle \zeta^\prime \, x^\prime\rangle)}\,h(\zeta)

d\zeta^\prime\, dt. 

$$

if we make the change of variables  $\zeta^\prime \to t\zeta^\prime$, 

$$

L_\Gamma (x+iy)= 

\int_1^{+\infty}t^n\int_{\Gamma^0 \cap \{\zeta_0=1\}}

e^{i(t(x_0+iy_0)+t\langle \zeta^\prime \, x^\prime\rangle)}h(t,\,

t\zeta^\prime) d\zeta^\prime\, dt.

\tag 1.20

$$

Let $S^0 = {\Gamma^0}\cap\{\zeta_0 =1\}\ $, and let $\Sigma^0 =  \partial S^0$.

Let  $\chi \in C^\infty_0(\R^{n})$ which is $\equiv 0$ on 

\newline $\{ \zeta \in S^0 : {\text dist}(\zeta,\, \Sigma^0) \ge \frac 12\}$ and is 

$\equiv 1$ on $\{ \zeta \in S^0 : {\text dist}(\zeta,\, \Sigma^0) \leq  \frac 14\}$.   

Then,

$$

\align

L_\Gamma (x+iy) &=\int^{+\infty}_0 \,t^n \,e^{it(x_0+iy_0)}  

\int_{S^0}

\chi(\zeta^\prime)e^{it\langle \zeta^\prime, \, x^\prime\rangle} \, h(t,\,

t\zeta^\prime)

d\zeta^\prime\,dt 

\\

&+ 

\int^{+\infty}_0 \,t^n \,e^{it(x_0+iy_0)}  

\int_{S^0}


(1-\chi(\zeta^\prime))e^{it\langle \zeta^\prime, \, x^\prime\rangle}\,h(t,\,

t\zeta^\prime)

d\zeta^\prime 

\\&= I(x+iy)+I^\prime(x+iy).

\endalign

$$

Without loss of generality, $h(t,\, t\zeta^\prime)= h(t)h(\zeta^\prime)$. 

Let $\dsize \psi(\eta) = 

\int_{S^0} e^{i\langle \zeta^\prime, \, \eta\rangle}h(\zeta^\prime) 

\chi(\zeta^\prime)\, d\zeta^\prime $. Since

$\psi(\eta)$  is the Fourier transform of 

$\chi(\zeta^\prime)h(\zeta^\prime)$, and

$\chi(\zeta^\prime)$ is supported in $S^0$, then 

$ \psi(\eta)\in {\Cal S}(\R^n)$, and   hence

$\n{\d{j}{t^j}\,t^k\,\psi(tx)}\leq C_{N,\,j,\,k}(1+t\n x)^{-N}$ \ 

$ \forall j$, $N$, $k \ge 0$ and $t\ge 1$. With the above notation, 

$$

I(x+iy) = \int^{+\infty}_0 \,t^n \psi (tx^\prime)\,e^{it(x_0+iy_0)}\, h(t)\,dt.  

$$

After integrating by parts,

$$

\align

I(x+iy) &= \frac{i}{x_0+iy_0} 

\int^{+\infty}_0 

e^{it(x_0+iy_0)}\d{}{t}\left(t^n\psi (tx^\prime)\,h(t)\right)\,dt

\\

&=

\frac{i}{x_0+iy_0} \int^{+\infty}_0 e^{it(x_0+iy_0)}\,t^{n-1}\left(t\d{}{ t}

(h(t)\psi (tx^\prime))

+n \psi (tx^\prime)\,h(t)\right)\,dt. 

\endalign

$$ 

Let $\N{I (x_0+iy_0, \, . )}_{L^r(\R^n)}= 

\left( \int_{\R^n}\n{I(x_0+iy_0, \, x^\prime )}^{r}\,dx^\prime\right)^{\frac

1r}$. Recalling that $h(t) \equiv 0$ if $t\leq 1$

and  $ x^\prime\to t\d{}{ t}

(h(t)\psi (tx^\prime)) +n \psi (tx^\prime)\,h(t) \in {\Cal S}(\R^n)$, by 

Minkowsky's inequality,

$$

\N{I(x_0+iy_0, \, . )}_{L^r(\R^n)} \leq  

\frac{1}{\n{x_0+iy_0}} \int^{+\infty}_1\, e^{-ty_0}\,t^{n-1}

\left( \int_{\R^n}\,{{C\,dx^\prime}\over{1+t\n{x^\prime}}^{Nr}} \right)^{\frac 1r}

\,dt,

\tag 1.21$$

Since $(1.21)$ holds for every  $N\ge 1$,  we can take $N > n$. After the change

of  variables $x^\prime\to\frac 1t x^\prime$,  

$$

\align

\N{I(x_0+iy_0, \, . )}_{L^r(\R^n)} &\leq  

\frac{1}{\n{x_0+iy_0}} \int^{+\infty}_1\, e^{-ty_0}\,t^{n-1-\frac nr}

\left( \int_{\R^n}{{C\,dx^\prime}\over{(1+\n{x^\prime})}^{Nr}} 

\right)^{\frac 1r} \,dt 

\\

&\leq

C\frac{1}{\n{x_0+iy_0}} \int^{+\infty}_0 e^{-ty_0}\,t^{n-1 -\frac nr}\,dt.

\endalign

$$

We recall that $r>1$. After the change of variables $t  \to \frac {t}{y_0}$, 

$$

\N{I(x_0+iy_0, \, . )}_{L^r(\R^n)} \leq 

C\frac{y_0^{-n+\frac nr}}{\n{x_0+iy_0}} \int^{+\infty}_0\, e^{-t}\,t^{n-1

-\frac nr}\,dt 

= C\frac{y_0^{-n+\frac nr}}{\n{x_0+iy_0}}. 

$$

Then, 

$$

\align


\N{I(.+iy_0)}_r &= \left(\int_\R \N{I(x_0+iy_0, \, . )}^r_{L^r(\R^n)}

\, dx_0 \right)^{\frac 1r}\\

&\leq C y_0^{-n+\frac nr}

\left( \int_{\R}{{dx_0}\over{(x_0^2 +y_0^2)^{\frac r2}}}

\right)^{\frac 1r} 

= C\,y_0^{-(n+1)(1-\frac 1r)}.

\endalign

$$

We are left with estimating the $L^r$ norm of $I^\prime$. 

We recall that 

$$ 

I^\prime(x+iy) = \int^{+\infty}_0 \,t^n h(t) \,e^{it(x_0+iy_0)}  

\int_{S^0}

(1-\chi(\zeta^\prime))h(\zeta^\prime)e^{it\langle \zeta^\prime, \, 

x^\prime\rangle}\, d\zeta^\prime

$$

Since  $S^0$ is convex, it is also star shaped with respect to the origin. 

In the polar coordinates associated to $S^0$, 

$$

I^\prime(x+iy) =  \int^{+\infty}_0 \!t^nh(t)

\,e^{it(x_0+iy_0)}\!\int^{1}_0\,s^{n-1}\! \int_{\Sigma^0} e^{its\langle \omega, \,

x^\prime\rangle} (1-\chi(s\omega))h(s\omega) d\omega \,ds\,dt.   \tag 

1.22 $$

Without loss of generality, $\chi$  and $h$ depend only on $s$.

By Lemma $1.3$,   

$$\int_{\Sigma^0}

e^{its\langle \omega, \, x^\prime\rangle} \, d\omega = 

b_1(tsx^\prime)\,e^{its\phi_1(x^\prime)} + 

b_2(tsx^\prime)\,e^{its\phi_2(x^\prime)}, 

\tag 1.23 $$

where  $\phi_1$ and $\phi_2$ are homogeneous of degree $1$

and smooth away from the origin, and  $b_1$ and $b_2$

satisfy  $(1.7)$.

If we let


$$ I_j (x_0+iy_0,\,x^\prime) =  \int^{+\infty}_0 \,t^n

h(t)\,e^{it(x_0+iy_0)}\,f_j(tx^\prime)

\,dt \qquad j = 1, \ 2,

$$

with 

$\dsize f_j(\eta) = 

\int^{1}_{0}s^{n-1}(1-\chi(s))b_j(s\eta)e^{is\phi_j(\eta)}ds$, then

$I^\prime(x+iy)$= $I_1(x+iy)+$ $I_2(x+iy)$.





We will need the following technical lemma, 

whose proof will be postponed until the end of the section.

\proclaim{Lemma 1.5}

Let   $\phi(\eta) \in C^\infty(\R^n)$. Suppose that $\phi$ never vanishes on

$S^{n-1}$. Suppose that  $\n{\phi(\eta)} ={\Cal O}(\n \eta)$ for large $\n

\eta$. Let $ B(\eta) \in C^\infty(\R^n) $ be a smooth

function  satisfying:

$$

\n{\d{\alpha}{\eta^\alpha} B(\eta)}  \leq  C(\alpha) 

(1+\n{ \eta}) ^{-m-\n\alpha} \qquad \forall  \alpha \in {\Bbb N}^n,

\tag 1.24

$$

where  $ m$ is a  nonnegative real number. Let $\beta(s) \in C^\infty(0,\,1)$

be a cutoff function which is $\equiv 0$ in a neighborhood of $0$ and is 

$\equiv 1$ in a neighborhood of $1$.

Then, $$

\int^{1}_{0}\beta(s)B(s\eta)e^{is\phi(\eta)}ds = 

h(\eta)e^{i\phi(\eta)},

$$

where  

$$

h(\eta)= -\sum^N_{j=1} 

\strut{\d{j-1}{s^{j-1}}B(s\eta)_{\vert_{s=1}}} + {\Cal O}(\n

\eta^{-m - N-1}).

\tag 1.25

$$

\endproclaim

\noindent

{\it End of the proof of Theorem $1.2$.} We recall that 

$\dsize f_j(\eta) = 

\int^{1}_{0}s^{n-1}(1-\chi(s))b_j(s\eta)e^{is\phi_j(\eta)}ds$, where  

$b_j$ and $\phi_j$  are defined as in  $(1.23)$ and $b_j$ satisfies $(1.7)$. For

every multiindex $\alpha\in {\Bbb N}$,  

$$

\d{\alpha}{\eta^\alpha} f_j(\eta) \leq 

\int^{1}_{0}s^{n-1+|\alpha|}(1-\chi(s))\left|\d{\alpha}{\eta^\alpha}

b_j(s\eta)\right| ds

$$

$$\leq 

C_\alpha 

\int^{1}_{0}s^{n-1+|\alpha|}(1-\chi(s)) (1+|s\eta|)^{-\frac{n-1}{2}-|\alpha|} ds

$$

$$

=

C_\alpha |\eta|^{-\frac{n-1}{2}-|\alpha|}


\int^{1}_{0}(|\eta|s)^{\frac{n-1}{2}+|\alpha|}

(1+|s\eta|)^{-\frac{n-1}{2}-|\alpha|}

s^{\frac{n-1}{2}}(1-\chi(s))

ds \leq C^\prime_\alpha |\eta|^{-\frac{n-1}{2}-|\alpha|}.


$$

Thus,  we can apply  Lemma $1.5$ to $f_j$ with $m=\frac{n-1}{2}$ obtaining  

$$

 f_j(tx^\prime) = a_j(tx^\prime)\,e^{i\phi_j(tx^\prime)},

$$

where $ \ a_j$ is as in  $(1.25)$ with $ m = \frac {n-1}2

$. Then, 

$$ I_j (x_0+iy_0,\, x^\prime) =  \int^{+\infty}_0 \!\!\!t^n h(t)

\,e^{it(x_0+iy_0)}\,f_j(tx^\prime) \,dt = \int^{+\infty}_0 \!\!\!t^n  h(t)

\,a_j(tx^\prime)e^{it\phi_j(x^\prime)+it(x_0+iy_0)}\,dt. $$

If we let $\dsize  \widetilde{\phi_j}(x_0+iy_0, x^\prime)=

\phi_j(x^\prime)+x_0+iy_0$  and  we integrate by parts,

$$ 

\align

 I_j &(x_0+iy_0,\, x^\prime) =\frac{i}{\widetilde \phi_j(x_0+iy_0, x^\prime) }

\int^{+\infty}_0 e^{it\widetilde \phi_j(x_0+iy_0, x^\prime)}\d{}{t}\left(t^nh(t)

a_j(tx^\prime)\right)dt 

\\

&= \frac{i}{\widetilde \phi_j(x_0+iy_0, x^\prime)}\int^{+\infty}_0

e^{it\widetilde \phi_j(x_0+iy_0, \,x^\prime)}

t^{n-1}\left \{ na(tx^\prime)h(t) +t\d{}{t} 

\left(\,h(t)a_j(tx^\prime)\,\right) \right \}dt. 

\tag 1.26


\endalign

$$

Observe that 

$$

\N{I_j (.+iy)}_r = 

\N{\left(\int_\R \n{I_j (x_0+iy_0,\, x^\prime)}^r\,dx_0\right)^{\frac

1r}}_{L^r(\R^n)},

\tag 1.27

$$

where the last norm is computed with respect to $x^\prime$. Let 

$\left(\int_\R \n{I_j (x_0+iy_0,\, x^\prime)}^r\, dx_0\right)^{\frac 1r}=  \N{I_j

(.+iy_0,\, x^\prime)}_{L^r(\R)}$. By H\"older's inequality and $(1.26)$, 

$$

\align

\N{I_j (.+iy_0,\, x^\prime)}_{L^r(\R)}&\leq 

\left(\int_\R  \frac{dx_0}{\n{\widetilde \phi_j(x_0+iy_0, x^\prime)}^r}\right)^{\frac 1r}

\\ 

\times&\sup_{x_0\in \R}\n{\int^{+\infty}_0\!\!\!

e^{it\widetilde \phi_j(x_0+iy_0, x^\prime)}t^{n-1}\left \{ na_j(tx^\prime)h(t)

+t\d{}{t} 

\left(a_j(tx^\prime)h(t)\right) \right

\}dt}.

\tag 1.28

\endalign

$$  

Recalling that $h(t)\equiv 0 $ when $t\leq 1$, by Lemma $1.5$

$\left

\{ na(tx^\prime) +t\d{}{t}  a(tx^\prime) \right\}=$\newline

$ O\left((1+t\n{ x^\prime})^{-\frac{n-1}{2} -1}\right) =

O\left((1+t\n{ x^\prime})^{-\frac{n+1}{2}}\right) $.

Then,

$$

\align

\N{I_j (.+iy_0,\, x^\prime)}_{L^r(\R)}&\leq C

\left(\int_\R  \frac{dx_0}{\n{\widetilde \phi_j(x_0+iy_0, x^\prime)}^r}\right)^{\frac 1r}

\\

 &\times\int^{+\infty}_1 e^{-y_0t}t^{n-1}\frac{1}{

(1+t\n{ x^\prime})^{-\frac{n+1}{2}}}\, dt

\endalign

$$

Since $\widetilde\phi_j(x_0+iy_0, x^\prime)= \phi_j(x^\prime)+x_0+iy_0$, 

$$

\N{\frac1{\widetilde \phi_j(.+iy_0, x^\prime)}}_{L^{r}(\R)} =

\left(\int_\R\frac{dx_0}{((\phi_j(x^\prime)+x_0)^2+ 

y_0^2)^{\frac r2}}\right)^{\frac1r}.

$$

If we let $x_0 = sy_0-\phi_j(x^\prime)$,

$$

\N{\frac1{\widetilde \phi_j(.+iy_0, x^\prime)}}_{L^{r}(\R)} = 

y_0^{-1+\frac 1r}\left(\int_\R\frac{ds}{(s^2+1)^{\frac r2}}\right)^{\frac1r}.

$$

Since $r>1$, the above integral is finite. Consequently, 

$$

\N{\frac1{\widetilde \phi_j(.+iy_0, x^\prime)}}_{L^{r}(\R)} = Cy_0^{-1+\frac 1r}.

\tag 1.29

$$

By $(1.27)$, $(1.28)$ and $(1.29)$, 

$$

\N{I_j (.+iy)}_r \leq C y_0^{-1+\frac 1r}\left( \int_{\R^n}\n{\int^{+\infty}_1

e^{-y_0t}t^{n-1}\frac{1}{ (1+t\n{ x^\prime})^{-\frac{n+1}{2}}}\,

dt}^r\,dx\right)^{\frac1r}.

\tag 1.30

$$

Let $G(y_0)$ be the integral on the right-hand side of $(1.30)$.

By Minkowski's inequality,  

$$

G(y_0)\leq C\int^{+\infty}_0 e^{-y_Ot}t^{n-1}\left (\int_{\R^n}\frac{dx^\prime}{

(1+t\n{ x^\prime})^{-\frac{r(n+1)}{2} }}\right)^{\frac 1{r}}\,dt.

\tag 1.31

$$

If $r > \frac {2n}{n+1}$, the integral with respect to $x^\prime$ is  finite.

After the change of variables $x^\prime \to t^{-1}x^\prime$, 

$$

G(y_0)\leq C\int^{+\infty}_0 e^{-y_0t}t^{n-1-\frac n{r}}, 

$$

and after the change of variables $t \to y_0^{-1}t$, 


$G(y_0)\leq C y_0^{-n+\frac n{r}}$. 

>From the above argument and $(1.31)$ we obtain  

$$

\N{I_j (.+iy) }_{r} 

\leq C y_0^{-(n+1)(1-\frac {1}{r})}.

$$

This concludes the proof of Theorem $1.2$.





\bigskip


\noindent{\it Proof of Lemma 1.5 }

Let $\dsize I(\eta) = 

\int^{1}_{0}\beta(s)B(s\eta)e^{is\phi(\eta)}ds$.

After an integration by parts,

$$

\align

I(\eta) &= \frac{1}{i\phi(\eta)}e^{i\phi(\eta)}B(\eta)

\\ 

-&\frac{1}{i\phi(\eta)}

\int^{1}_{0}e^{is\phi(\eta)}\beta^\prime(s)B(s\eta)ds-

\frac{1}{i\phi(\eta)}\int^{1}_{0}e^{is\phi(\eta)}\beta(s)\d{}{s}B(s\eta)ds. 

\tag 1.32

\endalign

$$

Since $\beta^\prime(s)$ has compact support, and since $\n {\phi(\eta)}= {\Cal


O}(\n \eta)$,  the first integral on the right-hand side of $(1.32)$ is 

${\Cal O}(\n \eta^{-M})$ for all $M\in {\Bbb N}$. Then, 

$$

I(\eta) = \frac{1}{i\phi(\eta)}e^{i\phi(\eta)}B(\eta)-

\frac{1}{i\phi(\eta)}\int_0^1e^{is\phi(\eta)}\beta(s)B_1(s\eta)ds + {\Cal 0}(\n

\eta^{- N}),

$$

where $B_1(\eta)= \d{}{s}B(s\eta) = \sum_{j=1}^n\eta_j\d{}{\eta_j}B(s\eta)$. 

By H\"older's inequality and (1.24), 

$\dsize B_1(\eta)\leq  C|\eta|(1+|\eta s|)^{-m-1}$; since 

$\beta$ is supported away from a neighborhood of the  origin, say a ball of radius

$\delta$, then 

$$

\left|

\int_0^1e^{is\phi(\eta)}\beta(s)B_1(s\eta)ds\right|  \leq  C\int_0^1

\beta(s)(1+|\eta s|)^{-m-1}ds\leq C(1+\delta|\eta|)^{-m-1}. 

$$

Since  $1+\delta|\eta| \ge \delta |\eta|$, this concludes the proof of the lemma

when $N=1$. 

If we iterate the above argument we obtain $(1.25)$.

\bigskip

Proposition $1.1$ easily follows  from  Theorem $1.2$ and  Young

inequality for convolution. As a consequence of 

Proposition $1.1$, Proposition $0.1$ and  M. Riesz interpolation theorem, 

we  have

that  the set ${\cal T}_\Gamma$ of the 

$\left(\frac 1p,\,  \frac 1q\right)$'s for

which

$(0.4)$ holds contains  the open trapezoid  ${\Cal Q}_n$ of vertices

$(1,\,0)$, $(\frac 12 - \frac 1{2n}, \, 0)$, $(0,\, \frac 12 + \frac 1{2n})$,

$(\frac 12,\, \frac 12)$. Perhaps ${\cal T}_\Gamma = {\Cal  Q}_\Gamma$. At 

the moment we can

only prove the following inclusions.  Let  ${\Cal Q}^\prime_n$ be the open

trapezoid  of vertices $(1,\,0)$, 

$(\frac 12 - \frac 1{2n}, \, 0)$, $(0,\, \frac 12 + \frac 1{2n})$,

$(\frac 12 - \frac 1{2n}, \, \frac 12 - \frac 1{2n})$, 

$(\frac 12 + \frac 1{2n}, \, \frac 12 + \frac 1{2n})$.

\proclaim

{Proposition 1.6} Let $\Gamma$ be as in Proposition $1.1$. Let 

${\Cal T}_\Gamma$, 

${\Cal Q}^\prime_n$ and ${\Cal Q}_n$ be as above. Then

$$

{\Cal Q}_\Gamma\subset {\Cal T}_\Gamma \subset {\Cal Q}^\prime_\Gamma

$$

\endproclaim

\bigskip

\noindent

{\it Proof.}  

Let ${\Cal T} $ be  the  closed triangle of vertices $(0,\,0)$, $(1,\,1)$, $(1,

0)$. Recall that  ${\Cal T}_\Gamma \subset 

{\Cal T}$. We  show that $\left(\frac 1p,\, \frac 1p\right)\in {\Cal T}_\Gamma $

if and only if $p=2$. 



Let $\Gamma$ be the  " light cone". Then,

$\Gamma = \Gamma^0$. Let $y= (y_0,\, 0,\, \cdots,\, 0)$.

Suppose that  $\left(\frac 1p,\, \frac 1p\right)\in {\Cal T}_\Gamma $ for some

$p\ge 1$.  Then,  $e^{-y_0 \zeta_0}

\chi_\Gamma(\zeta)$ is a $L^p(\R^{n+1})$ Fourier  multiplier. By  de Leeuw's 

theorem,  (see e.g. [SW]), the characteristic function of the

unit ball of $\R^n$ times  $e^{-y_0}$   is a  $L^p(\R^n)$  Fourier

multiplier, and this is possible if and only if $p=2$,  as  C. Fefferman's

counterexample in [F] shows. 



We shall prove that if  $\left(\frac 1p,\, \frac 1q\right)\in {\Cal T}_\Gamma $,

then $p<\frac{ 2n}{n-1} $ and $q>\frac{ 2n}{n+1}$.  Indeed, let $G\supset \Gamma^0$

be the intersection of a

finite number of half spaces. Let $\overline y\in

\Gamma$ and  let 

$\hat f(\zeta)= e^{-\langle

\overline y,\,

\zeta\rangle}\chi_{G}(\zeta)$, where $\chi_{G }(\zeta)$ is the

characteristic function  of $G$. Since  $f(x)= K_{G^0}(x+i\overline y)$, by


Proposition $0.2$ $f \in L^p(\R^{n+1})$ for every  $p >1$. Since  

$T_\Gamma(f)(x+i \overline y)=\int_{\Gamma^0} e^{-2\langle \overline y,\,

\zeta\rangle+i

\langle x,\, \zeta\rangle}\, d\zeta$ $=$ $K_\Gamma(x+2i\overline y)$,  by

Theorem  $1.2$  $ T_{\Gamma,\,\overline y}(f) \in

L^q(\R^{n+1})$ if and only if $q >\frac{ 2n}{n+1}$. Thus, if  

$\left(\frac 1p,\,

\frac 1q\right)\in {\Cal T}_\Gamma $, then  $q>\frac{ 2n}{n+1}$.  It follows by

duality that  if $\left(\frac 1p,\, \frac 1q\right)\in {\Cal T}_\Gamma $, then 


$p<\frac{ 2n}{n-1} $. This concludes the proof of Proposition $1.6$.

\bigskip

\bigskip

\centerline{REFERENCES}

\bigskip

\item{{\bf [C]}} M.  Christ,\quad {\it Restriction of the Fourier transform to 

submanifolds of low codimension}, Thesis, University of Chicago (1982). 



\item{{\bf [D1]}} L. De Carli,\quad{\it Funzioni generalizzate e problemi non

lineari }, Tesi, Universita' \newline "La Sapienza", Roma, (1993). 



\item{{\bf [D2]}} L. De Carli, \quad {\it $L^p$ estimates for the Cauchy

transform of distributions with respect to convex cones}, Rend.Sem. Mat. Univ.

Padova, {\bf 88} (1992), 35-53. 



\item{\bf{[DI]}}  L. De Carli and A. Iosevich, 

{\it some sharp restriction theorem for homogeneous manifolds}

(1995) (to appear).



\item{{\bf [F]}}  C. Fefferman, \quad {\it The multiplier problem for the ball},

Ann.  Math. (2) {\bf 94},  (1971) 330-336.



\item{{\bf [G]}} A. Greenleaf,\quad {\it Principal curvatures in harmonic analysis},

Ind. Univ. Math. J.  {\bf 30}  (1981), 519-537.



\item{{\bf [H]}} L. H\"ormander,\quad {\it Estimates for translation invariant

operators in $L^p$ spaces}, Acta Math. 104 (1960), 93-140.



\item{\bf [L]} W. Littman,\quad {\it Fourier transforms of surface-carried

measures  and differentiability of surface averages}, Bull. Am. Math. Soc.  {\bf

69} (1963), 766-770.



\item{{\bf [So]}} C.D.Sogge,\quad 

{\it Fourier integrals in classical analysis}, 

Cambridge  University Press,  Cambridge, 1993.



\item{{\bf [SW]}} E. M. Stein , G. Weiss: \quad {\it Introduction to Fourier

Analysis on Euclidean Spaces }, Princeton University Press,  1971




\item{{\bf [V]}} V.S. Vladimirov, \quad {\it Le

distribuzioni nella Fisica Matematica},  MIR, Mosca,  1981

\end

 



