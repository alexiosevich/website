 \documentclass[]{amsart}
\usepackage{amsmath,amsthm,amscd,amssymb}
\usepackage{latexsym}
\numberwithin{equation}{section}

\theoremstyle{plain}
\newtheorem{theorem}{Theorem}[section]
\newtheorem{lemma}[theorem]{Lemma}
\newtheorem{corollary}[theorem]{Corollary}
\newtheorem{proposition}[theorem]{Proposition}
\newtheorem{conjecture}[theorem]{Conjecture}
\newtheorem{criterion}[theorem]{Criterion}
\newtheorem{algorithm}[theorem]{Algorithm}

\theoremstyle{definition}
\newtheorem{definition}[theorem]{Definition}
\newtheorem{condition}[theorem]{Condition}
\newtheorem{problem}[theorem]{Problem}
\newtheorem{example}[theorem]{Example}
\newtheorem{exercise}{Exercise}[section]
\newtheorem{obs}{Observation}

\theoremstyle{remark}
\newtheorem{remark}[theorem]{Remark}
\newtheorem{note}[theorem]{Note}
\newtheorem{notation}[theorem]{Notation}
\newtheorem{claim}[theorem]{Claim}
\newtheorem{summary}[theorem]{Summary}
\newtheorem{acknowledgment}[theorem]{Acknowledgment}
\newtheorem{case[theorem]}{Case}
\newtheorem{conclusion}[theorem]{Conclusion}

\begin{document}

\title{The Erd\H{o}s
distance problem}
\author{Julia
Garibaldi and Alex
Iosevich}

\date{July 31, 2005}

\address{Department of Mathematics, University of Missouri-Columbia,
Columbia Missouri 65211 USA}
\email{iosevich \@
math.missouri.edu}
\thanks{The work was
partly supported by a
grant from the National Science Foundation}
\maketitle
\begin{abstract} This book is born out of our desire to increase our publication list without actually proving anything new. A prospect of seeing our names on a shiny fancy cover is also a plus!
\end{abstract}

\section*{Foreword}

This book is based on the notes were written for the summer program on the Erd\H{o}s
distance problem, held at the University of Missouri, August
1-5, 2005. This was the second year of this program and our plan
continued to be to introduce motivated high school students to
accessible concepts of higher mathematics. Last year's theme was
the Kakeya conjecture in finite fields. This year we concentrated
on one of the most beautiful problems of geometric combinatorics,
the Erd\H{o}s distance conjecture.

Our book is heavily problem oriented. Most of the learning is
meant to be done by doing the exercises interspersed throughout
the lecture notes. Many of these exercises are recently published
results by mathematicians working in the area. In a couple of
places, steps are intentionally left out of proofs and the reader
is then asked to fill them in in the process of working the
exercises. On a number of occasions, solutions to exercises
are used in later chapters in an essential way. Having said that,
let us add that you should not rely solely on exercises in these
notes. Create your own problems and questions! Modify the
lemmas and theorems below, and, whenever possible,
improve them! Mathematics is a highly personal experience and
you will find true fulfillment only when you make the concepts
in these notes your own in some way. Good luck!

\section*{Introduction}

Many theorems in mathematics say, one way or another, that it is
very difficult to arrange mathematical object in such a way that
they do not exhibit some interesting structure. The Erd\H{o}s distance
problem asks for the minimal number of distances determined by a
set  of $N$ points in $\mathbb{R}^d$, $d \ge 2$. More precisely, let
$P$ be a finite subset of $ \mathbb{R}^d$, $d \ge 2$, such that $\#
P=N$. Let
\begin{align}
 \Delta(P)=\{|p-p'|: p,p' \in P\},
\\
\intertext{and}
 |x|=\sqrt{x_1^2+\dots+x_d^2},
\end{align}
 the Euclidean distance.

The Erd\H{o}s distance problem asks for the smallest possible size
of $\Delta(P)$. Let us consider some simple examples. Let
$P=\{(j,0, \dots, 0): j=1,2, \dots, N\}$. Then $\Delta(P)=\{0,1,2,
\dots, N-1\}$. This simple example shows that the best general
result we can hope for is
\begin{equation}
\# \Delta(P) \leq \# P. \end{equation}



This turns out to be too much. Let $P=\mathbb{Z}^d \cap {[0,
N^{\frac{1}{d}}]}^d$, where $N$ is a $d$'th power of an integer.
Then $\Delta(P)=\{|p|: p \in P\}$ (why?) and $\# \Delta(P)=\#
\{{|p|}^2: p \in P\}$. Consider the set of numbers
$p_1^2+p_2^2+\dots+p_d^2$, $p=(p_1, \dots, p_d) \in P$. All these
numbers are positive integers no less than $0$ and no more than
$dN^{\frac{2}{d}}$. Now check that
\begin{equation}
\# \Delta(P) \leq dN^{\frac{2}{d}}+1
\end{equation} follows from this observation.



For dimension 2 the reality is even worse. It turns out (see
Appendix 1) that $\# \Delta(P) \approx N^{\frac{2}{d}}$, if $d
\geq 3$, and $\Delta(P) \approx \frac{N}{\sqrt{\log(N)}}$ if
$d=2$. Here, and throughout the notes, $X \lesssim Y$ means that
there exists a positive constant $C$ such that $X \leq CY$, and $X
\approx Y$ means that $X \lesssim Y$ and $Y \lesssim X$. We take
this notational game a step further and define $X \lessapprox Y$,
with respect to the large parameter $N$, if for every $\epsilon>0$
there exists $C_{\epsilon}>0$ such that $X \leq
C_{\epsilon}N^{\epsilon}Y$.

\begin{proof}[Erd\H{o}s distance conjecture] Let $P$ be a subset of
$\mathbb{R}^d$, $d \ge 2$, such that $\# P=N$. Then
\begin{align}
\# \Delta(P) &\gtrapprox N, \ \text{if} \ d=2, \\ \intertext{and}
\# \Delta(P) &\gtrsim N^{\frac{2}{d}}, \ \text{if} \ d \ge 3.
\end{align}
\end{proof}

The conjecture is nowhere near resolution, but much is known and
we will come very close to the cutting edge of this beautiful
problem in these notes.

%chaper 0 exercises

\begin{exercise} \label{ex0.1}
Define $\Delta_{l^1({\Bbb R}^d)}(P)=\{|p_1-p'_1|+\dots+|p_d-p'_d|:
p,p' \in P\}$. Prove that Erd\H{o}s distance conjecture is false
if $\Delta(P)$ is replaced by $\Delta_{l^1(\mathbb{R}^d)}(P)$.
What should the conjecture say in this context? Can you prove this
conjecture?  Consider the case $d=2$ first.  \end{exercise}

\begin{exercise} \label{ex0.2}
Let $K$ be a convex, centrally symmetric subset of $\mathbb{R}^2$,
contained in the disk of radius $2$ centered at the origin and
containing the disk of radius $1$ centered at the origin. Convex
means that if $x$ and $y$ are in $K$, then the line segment
connecting $x$ and $y$ is contained entirely inside $K$. Centrally
symmetric means that if $x$ is in $K$, then $-x$ is also in $K$.

Let $t={||x||}_K$ denote the number such that $x$ is contained in
$tK$, but is not contained in $(t-\epsilon)K$ for any
$\epsilon>0$. Define $\Delta_K(P)=\{{||p-p'||}_K: p,p' \in P\}$.
If the boundary of $K$ contains a line segment prove that one can
construct a set $P$, with $\# P=N$, such that $\# \Delta_K(P)
\lesssim N^{\frac{1}{d}}$. \end{exercise}

\section{Preliminaries: Cauchy-Schwartz inequality and some simple consequences}

\vskip.125in

In this section we shall follow a procedure often considered nasty, but
the one I hope to convince you to appreciate. We shall work backwards,
discovering concepts as we go along, instead of stating them ahead of
time. Let $a$ and $b$ denote two real numbers. Then
\begin{equation} \label{square}
 {(a-b)}^2 \ge 0. \end{equation}

This statement is so vacuous, you are probably wondering why I am telling
you this. Nevertheless, expland the left hand side of \ref{square}. We get
$$ a^2-2ab+b^2 \ge 0, $$ which implies that
\begin{equation} \label{rectangle} ab \leq \frac{a^2+b^2}{2}. \end{equation}

Now consider
$$ A_N=\sum_{k=1}^N a_k=a_1+\dots+a_N, \ B_N=\sum_{k=1}^N
b_k=b_1+\dots+b_N, $$ where
$a_1, \dots, a_N$, and $b_1, \dots, b_N$ are real numbers. Let
$$ X_N={\left(\sum_{k=1}^N a_k^2 \right)}^{\frac{1}{2}}, \
Y_N={\left(\sum_{k=1}^N b_k^2 \right)}^{\frac{1}{2}}. $$

Our goal is to take advantage of \ref{rectangle}. Let's take a look at
$$ \sum_{k=1}^N a_kb_k=X_N Y_N \sum_{k=1}^N \frac{a_k}{X_N} \cdot
\frac{b_k}{Y_N}$$
\begin{equation} \label{Rectangle} \leq X_NY_N \sum_{k=1}^N
\left[\frac{1}{2}{\left(\frac{a_k}{X_N}\right)}^2+
\frac{1}{2}{\left(\frac{b_k}{Y_N}\right)}^2\right]. \end{equation}

\begin{exercise} Explain using complete English sentences how \ref{rectangle} follows from \ref{Rectangle}. \end{exercise} 

\begin{exercise} \label{prelim2} Explain why if $C$ is a constant, then $\sum_{k=1}^N Ca_k=C\sum_{k=1}^N a_k$. \end{exercise}

\begin{exercise} \label{prelim3} Explain why $\sum_{k=1}^N (a_k+b_k)=\sum_{k=1}^N a_k+\sum_{k=1}^N b_k$. \end{exercise}

We now use \ref{prelim2} and \ref{prelim3} to rewrite $(2.6)$ in the form
$$ X_NY_N \frac{1}{2} \frac{1}{X^2_N} \sum_{k=1} a_k^2+X_NY_N
\frac{1}{2} \frac{1}{Y^2_N}\sum_{k=1}^N b_k^2$$
$$=X_NY_N \frac{1}{2} \frac{1}{X^2_N} X_N^2+X_NY_N
\frac{1}{2} \frac{1}{Y^2_N}Y_N^2$$
$$=\frac{1}{2}X_NY_N+\frac{1}{2}X_NY_N=X_NY_N. $$

Putting everything together, we have shown that
\begin{equation} \label{2rectangle} \sum_{k=1}^N a_k b_k \leq {\left(\sum_{k=1}^N a_k^2
\right)}^{\frac{1}{2}} {\left(\sum_{k=1}^N b_k^2
\right)}^{\frac{1}{2}}. \end{equation}

This is known as the Cauchy-Schwartz inequality.

\begin{exercise} (quite difficult if you do not know calculus) Let
$1<p<\infty$ and define the exponent $p'$ by
the equation $\frac{1}{p}+\frac{1}{p'}=1$. Then
\begin{equation} \label{prectangle}
\sum_{k=1}^N a_k b_k \leq {\left(\sum_{k=1}^N {|a_k|}^p
\right)}^{\frac{1}{p}} {\left(\sum_{k=1}^N {|b_k|}^{p'}
\right)}^{\frac{1}{p'}}. 
\end{equation} \end{exercise}

Observe that \ref{prectangle} reduces to \ref{2rectangle} if $p=2$. Hint: prove that
$ab \leq \frac{a^p}{p}+\frac{b^{p'}}{p'}$ and proceed as in the
case $p=2$. One way to prove this inequality is to set $a^p=e^x$
and $b^{p'}=e^y$ (why are we allowed to do that?). Let
$\frac{1}{p}=t$ and observe that $0 \leq t \leq 1$. We are then
reduced to showing that for any real valued $x,y$ and $t \in
[0,1]$, $e^{tx+(1-t)y} \leq te^x+(1-t)e^y$. Let
$f(t)=e^{tx+(1-t)y}$ and $g(t)=te^x+(1-t)e^y$. Observe that
$f(0)=g(0)=e^y$ and $f(1)=g(1)=e^x$. Can you complete the
argument?



Let's now try to see what
Cauchy-Schwartz (C-S) inequaity is good for. Let $S_N$ be a finite set of
$N$ points in ${\Bbb R}^3=\{(x_1,x_2,x_3): x_j \ \text{is a real
number}\}$, the three-dimensional Euclidean space. Let $x=(x_1,x_2,x_3)
\in {\Bbb R}^3$ and define
$$ \pi_1(x)=(x_2,x_3), \ \pi_2(x)=(x_1,x_3), \ \text{and} \
\pi_3(x)=(x_1,x_2).$$

The question we ask is the following. We are assuming that $\# S_N=N$.
What can we say about the size of $\pi_1(S_N), \pi_2(S_N)$, and
$\pi_3(S_N)$? Before we do anything remotely complicated, let's make up
some silly looking examples and see what we can learn from them.

Let $S_N=\{(0,0,k): k \ \text{integer} \ k=0,1, \dots, N-1\}$. This set
clearly has $N$ elements. What is $\pi_3(S_N)$ in this case. It is
precisely the set $\{(0,0)\}$, a set consisting of one element. However,
$\pi_2(S_N)$ and $\pi_1(S_N)$ are both $\{(0,k): k=0,1,\dots,N-1\}$,
sets consisting of $N$ elements. In summary, one of the projections is
really small and the others are as large as they can be.

Let's be a bit more even handed. Let $S_N=\{(k,l,0): k,l \
\text{integers} \ 1 \leq k \leq \sqrt{N}, 1 \leq l \leq \sqrt{N}\}$, where
$\sqrt{N}$ is an integer. Again $\# S_N=N$. What do projections look
like? Well, $S_N$ is already in the $(x_1,x_2)$-plane, so
$\pi_3(S_N)=\{(k,l): k,l \
\text{integers} \ 1 \leq k \leq \sqrt{N}, 1 \leq l \leq \sqrt{N}\}$. It
follows that $\# \pi_3(S_N)=N$. On the other hand, $\pi_2(S_N)=\{(k,0):
k \ \text{integer} \ 1 \leq k \leq \sqrt{N} \}$, and
$\pi_1(S_N)=\{(l,0): l \ \text{integer} \ 1 \leq l \leq \sqrt{N} \}$,
both containing $\sqrt{N}$ elements. Again we see that it is difficult for
all the projections to be small.

Let's think about our examples so far from a geometric point of view. The
first example is "one-dimensional" since the points all lie on a line.
The second example is "two-dimensional" since the points lie on a plane.
Let's now build a truly "three-dimensional" example with as much symmetry
as possible. Let $S_N=\{(k,l,m): k,l,m \ \text{integers} \ 1 \leq k,l,m
\leq N^{\frac{1}{3}}\}$, where $N^{\frac{1}{3}}$ is an integer. Again,
$\# S_N=N$, as required. The projections this time all look the same. We
have $\pi_1(S_N)=\{(l,m): l,m \ \text{integers} \ 1 \leq l,m \leq
N^{\frac{1}{3}}\}$, a set of size $N^{\frac{2}{3}}$, and the same is true
of $\# \pi_2(S_N)$ and $\# \pi_3(S_N)$.

Let's summarize what happened. In the case when all the projections have
the same size, each projection has $N^{\frac{2}{3}}$ elements. We will
see in a moment that for any $S_N$, one of the projections must of size
at least $N^{\frac{2}{3}}$. We will see here and later in these notes
that C-S inequality is very usefull in showing that the "symmetric" case
is "optimal", whatever that means in a given instance.

To start our investigation we need the following basic definition. Let
$S$ be any set. Define $\chi_S(x)=1$ if $x \in S$ and $0$ otherwise.

\begin{exercise} \label{box.ex}
Let $S_N$ be as above. Then
$$\chi_{S_N}(x) \leq
\chi_{\pi_1(S_N)}(x_2,x_3)\chi_{\pi_2(S_N)}(x_1,x_3)
\chi_{\pi_3(S_N)}(x_1,x_2).$$
\end{exercise}

With exercise \ref{box.ex} in tow, we write
$$ N=\# S_N=\sum_{x} \chi_{S_N}(x) \leq
\sum_{x} \chi_{\pi_1(S_N)}(x_2,x_3)\chi_{\pi_2(S_N)}(x_1,x_3)
\chi_{\pi_3(S_N)}(x_1,x_2)$$
$$=\sum_{x_1,x_2} \chi_{\pi_3(S_N)}(x_1,x_2)
\sum_{x_3} \chi_{\pi_1(S_N)}(x_2,x_3)\chi_{\pi_2(S_N)}(x_1,x_3)$$
$$ \leq {\left(\sum_{x_1,x_2}
\chi^2_{\pi_3(S_N)}(x_1,x_2)\right)}^{\frac{1}{2}}
{\left(\sum_{x_1,x_2} {\left(\sum_{x_3}
\chi_{\pi_1(S_N)}(x_2,x_3)\chi_{\pi_2(S_N)}(x_1,x_3)\right)}^2
\right)}^{\frac{1}{2}}$$
$$=I \times II. $$

Now,
$$ I={\left(\sum_{x_1,x_2}
\chi^2_{\pi_3(S_N)}(x_1,x_2)\right)}^{\frac{1}{2}}={\left(\sum_{x_1,x_2}
\chi_{\pi_3(S_N)}(x_1,x_2)\right)}^{\frac{1}{2}}={(\#
\pi_3(S_N))}^{\frac{1}{2}}. $$

On the other hand,
$$ {II}^2=\sum_{x_1,x_2} {\left(\sum_{x_3}
\chi_{\pi_1(S_N)}(x_2,x_3)\chi_{\pi_2(S_N)}(x_1,x_3)\right)}^2$$
$$=\sum_{x_1,x_2} \sum_{x_3} \sum_{x'_3}
\chi_{\pi_1(S_N)}(x_2,x_3)\chi_{\pi_2(S_N)}(x_1,x_3)
\chi_{\pi_1(S_N)}(x_2,x'_3)\chi_{\pi_2(S_N)}(x_1,x'_3)$$
$$ \leq \sum_{x_1,x_2} \sum_{x_3} \sum_{x'_3}
\chi_{\pi_1(S_N)}(x_2,x_3)\chi_{\pi_2(S_N)}(x_1,x'_3)$$
$$=\sum_{x_2,x_3}\chi_{\pi_1(S_N)}(x_2,x_3)
\sum_{x_1,x'_3}\chi_{\pi_2(S_N)}(x_1,x'_3)=\# \pi_1(S_N) \cdot \#
\pi_2(S_N). $$

Putting everything together, we have proved that
\begin{equation} \label{boxineq} \# S_N \leq \sqrt{\# \pi_1(S_N)}\sqrt{\# \pi_2(S_N)}\sqrt{\#
\pi_3(S_N)}. \end{equation}

\begin{exercise} Verify each step above. Where was C-S inequality
used? Why does $\chi^2_{\pi_j(S_N)}(x)=\chi_{\pi_j(S_N)}(x)$? \end{exercise}

The product of three positive numbers certainly does not exceed the
largest of these numbers raised to the power of three. It follows from
this and \ref{boxineq} that
$$ N=\# S_N \leq \max_{j=1,2,3} {(\# \pi_1(S_N))}^{\frac{3}{2}}. $$

We conclude by raising both sides to the power of $\frac{2}{3}$ that
$$ \# \max_{j=1,2,3} \pi_j(S_N) \ge N^{\frac{2}{3}} $$ as
claimed.

\begin{exercise} Let $\Omega$ be a convex set in ${\Bbb R}^3$.
This means that for any pair of points $x,y \in \Omega$, the line segment
connecting $x$ and $y$ is entirely contained in $\Omega$. Prove that
$vol(\Omega) \leq \sqrt{area(\pi_1(\Omega))} \cdot
\sqrt{area(\pi_2(\Omega))} \cdot \sqrt{area(\pi_3(\Omega))}$. \end{exercise} 

If you can't prove this exactly, can you at least prove using \ref{boxineq}
and its proof that $\max_{j=1,2,3} area(\pi_j(\Omega)) \ge
{(vol(\Omega))}^{\frac{2}{3}}$? This would say that a convex object of
large volume has at least one large coordinate shadow. Using politically
incorrect language this can be restated as saying that if a hyppopatamus
is overweight, there must be a way to place a mirror to make this
obvious... 

\begin{exercise} (Project question) Generalize \ref{boxineq}. What do I
mean, you ask... Replace three dimensions by $d$ dimensions. Replace
projections onto two-dimensional coordinate planes by projections onto
$k$-dimensional coordinate planes, with $1 \leq k \leq d-1$. Finally,
replace the right hand side of \ref{boxineq} by what it should be...
\end{exercise}

\subsection{Incidences and matrices} Consider a set of $n$ lines
and $n$ points in the plane. Define an incidence to be a pair $(p,l)$,
where $p$ is one of the points in our point set, $l$ is one of the lines
in our set of lines, and $p$ lies on $l$. Let $I(n)$ denote the total
number of incidences determined by a given set of $n$ points and a given
set of $n$ lines. In order to avoid needless headaches we assume that
every point in our point set lies on at least one line in our set of
lines, and every line in our line set contains at least one point in our
point set.

How large can $I(n)$ be? Well, it is clear that $I(n) \leq n^2$. This
observation is not terribly valuable, however, since $I(n)$ cannot
possibly be this large! I mean, how can every line contain every point,
and every point lie on every line?! You might retort that maybe, just
maybe, it is possible for about $n/10$ lines to contain about $n/100$
points each, and for each of those points to be contain in about $n/1000$
of those lines. We shall see that nothing like that can happen.

Our main tools in this endeavor are matrices and the C-S inequaity. Recall
that a $N$ by $N$ matrix $A$ is an array with $n$ rows and $n$ columns.
The elements of $A$ are designated by $a_{ij}$, where $i$ determines the
row and $j$ determines the column. Let's define $A$ as follows.
Enumerate the $n$ points in our point set from $1$ to $n$, and do the
same for lines in our set of lines. Let $a_{ij}=1$ if the $i$'th point
lies on the $j$'th line, and $0$ otherwise. Observe that if $j$ and $j'$
are fixed, with $j \not=j'$,
\begin{equation} \label{product} a_{ij} \cdot a_{ij'}=1 \end{equation} for at most one value of $i$. This
is because $a_{ij} \cdot a_{ij'}=1$ if and only if $a_{ij}=1$ and
$a_{ij'}=1$. This means that the $i$'th point is on the $j$'th lines and
also on the $j'$th line. Intersection of two distinct lines is either
empty or consists of exactly one point. It follows that indeed the
equality in \ref{product} can hold for at most one $i$.

We are now ready for action. What is $I(n)$? It is nothing more than the
total number of $1$s in $A$! Since $A$ consists of only $1$s and $0$s,
$$ I(n)=\sum_{i=1}^n \sum_{j=1}^n a_{ij}=\sum_{i=1}^n \left(\sum_{j=1}^n
a_{ij}\right) \cdot 1$$
$$ \leq {\left(\sum_{i=1}^n 1 \right)}^{\frac{1}{2}} {\left( \sum_{i=1}^n
{\left( \sum_{j=1}^n a_{ij} \right)}^2 \right)}^{\frac{1}{2}}=\sqrt{n}
\cdot {\left( \sum_{i=1}^n
{\left( \sum_{j=1}^n a_{ij} \right)}^2 \right)}^{\frac{1}{2}}. $$

Now,
$$ \sum_{i=1}^n
{\left( \sum_{j=1}^n a_{ij} \right)}^2=\sum_{i=1}^n \sum_{j=1}^n
\sum_{j'=1}^n a_{ij}a_{ij'}$$
$$=\sum_{i=1}^n \sum_{1 \leq j,j' \leq n; j \not=j'}
a_{ij}a_{ij'}+\sum_{i=1}^n \sum_{j=1}^n a_{ij}^2=apple+orange. $$

To estimate apple we use \ref{product}. Indeed, since $a_{ij} \cdot a_{ij'}=1$
for at most one $i$,
\begin{equation} \label{apple} apple \leq \# \{(j,j'): 1 \leq j,j' \leq n; j \not=j'\}=n^2-n.\end{equation}

\begin{exercise} Write out the details of the equality on the
right hand side of \ref{apple}. \end{exercise}


On the other hand,
$$ orange \leq \# \{(i,j): 1 \leq i,j \leq n\}=n^2. $$

Putting everything together and using tthe fact that $n^2-n \leq n^2$, we
see that
$$ I(n)=\sum_{i=1}^n \sum_{j=1}^n a_{ij} \leq \sqrt{2} \cdot
n^{\frac{3}{2}}. $$

We conclude that the number of incidences between $n$ points and $n$
lines in the plane is at most $\sqrt{2}n^{\frac{3}{2}}$. Can this
estimate be improved? Sure it can... The sharp answer is $I(n) \leq
Cn^{\frac{4}{3}}$, where $C$ is a fixed positive constant. This is the
celebrated Szemeredi-Trotter incidence theorem and it is sharp in the
sense that one can construct a set of $n$ lines and $n$ points such that
the number of incidences is approximately $n^{\frac{4}{3}}$, up to a
constant. The proof of this result will appear in the second part of
these notes.

\begin{exercise} Show that the estimate $I(n) \leq
Cn^{\frac{3}{2}}$ we just obtained for points and lines in the plane is
best possible for points and lines in ${\Bbb F}_q^2$. Hint: Take as your
point set all the points in ${\Bbb F}_q^2$ and take as your line set all
the lines in ${\Bbb F}_q^2$. \end{exercise}

\begin{exercise} Let $S_N$ be a subset of the plane with $N$
elements. Define $\Delta(S_N)=\{\sqrt{{(x_1-y_1)}^2+{(x_2-y_2)}^2}:
x=(x_1,x_2) \in S_N, y=(y_1,y_2) \in S_N\}$. Use $(2.23)$ to show that
$\# \Delta(S_N) \ge C \sqrt{N}$ for some constant $C$ independent of $N$.

Can you do better? The conjectured asnwer is that $\# \Delta(S_N) \ge
C\frac{N}{\sqrt{\log(N)}}$. The best known result to date is $\#
\Delta(S_N) \ge CN^{\beta}$, where $\beta \approx .86$.

What about higher dimensions? If $S_N \subset {\Bbb R}^d$ of size $N$,
prove that $\# \Delta(S_N) \ge CN^{\frac{1}{d}}$. Can you do better? The
conjectured answer here is $\# \Delta(S_N) \ge CN^{\frac{2}{d}}$ in
dimensions three and higher. Do you see where the exponent $\frac{2}{d}$
is coming from? Hint: Let $S_N=\{n=(n_1, \dots, n_d): n_j \in {\Bbb Z}; 1
\leq n_j \leq N^{\frac{1}{d}} \}$. \end{exercise}

\begin{exercise} Show that the number of incidences between $n$
points and $n$ two-dimensional planes in ${\Bbb R}^3$ can be $n^2$.
Suppose that we further insist that the intersection of any three planes
in our collection contains at most one point. Prove that the number of
incidences is $\leq Cn^{\frac{5}{3}}$.

More generally, prove that if we have $n$ points and $n$
$d-1$-dimensional planes in ${\Bbb R}^d$, then the number of incidences
can be $n^2$. Show that the number of incidences is $\leq
Cn^{2-\frac{1}{d}}$ if we further insist that the intersection of any $d$
planes from our collection intersect at at most one point.
\end{exercise}

\begin{exercise} Prove that $n$ points and $n$ spheres of the
same radius in ${\Bbb R}^d$, $d \ge 4$, can have $n^2$ incidences. Use
the techniques of the chapter that when $d=2$ the number of incidences is
$\leq Cn^{\frac{3}{2}}$. What can you say about the case
$d=3$? \end{exercise}





\section{Erd\H{o}s' original argument}

How does one prove that any set $P$ of size $N$ determines many
distances? Let us start in two dimensions. Chose a point $p_0$ and
draw circles around it that contains at least one point of $P$.
Suppose that we have drawn $t$ circles. If $t$ is big enough then
we are already doing very well. But what if $t$ is happens to be
small? Note that at least one of the $t$ circles must contain at
least $N/t$ points. Draw the East-West line though the center of
that circle. Then at least $N/2t$ are contained in either the
Northern or Southern hemisphere. Without loss of generality
suppose that there are $N/2t$ points in the Northern hemisphere.
Fix the East-most point and draw segments from that point to all
the other points of $P$ in the Northern hemisphere. The length of
these segments are all different, so at least $N/2t$ distances are
thus determined. This proves that

\begin{equation} \label{max1.eqn}
\# \Delta(P) \ge \max \{t, N/2t\}.
\end{equation}

There are several ways to proceed here. One way is to ``guess" the
answer. Since $t<\sqrt{N}$. Then $N/2t>\sqrt{N}/2$, so
either way,
\begin{equation} \label{sqrtn.eqn}
\# \Delta(P) \gtrsim \sqrt{N}.
\end{equation}


A slightly less ``sneaky" approach is to use the fact that
\begin{equation}
\max \{X,Y\} \ge \sqrt{XY} \  \text{(why?)}.
\end{equation}


This transforms $(\ref{max1.eqn})$ into $(\ref{sqrtn.eqn})$. Summarizing, we have just
proved the following.

\begin{theorem}[Erd\H{o}s \cite{Erd}] \label{erdos.thm}
Suppose that $d=2$ and $\# P=N$. Then
$(\ref{sqrtn.eqn})$ holds. \end{theorem}

What about higher dimensions? Let us try the same approach. Choose
a point in $P$ and draw all spheres that contain at least one
point of $P$. As before, let $t$ denote the number of spheres. If
$t$ is large enough, we are done. If not, then one of the spheres
contains at least $N/t$ points. Unfortunately, if $d>2$, we cannot
run the simple minded argument that worked in two dimensions. Or
can we? Notice that if we are working in $\mathbb{R}^d$, the surface
of each sphere is $(d-1)$-dimensional, whatever that means. This
suggests the following approach.

\begin{proof}[Induction Hypothesis] Let $P'$ be a subset of ${\Bbb
R}^k$ , $k \ge 2$, or $S^{k}$, $k \ge 1$. Suppose that $\# P'=N'$.
Then
$$ \# \Delta(P') \gtrsim {(N')}^{\frac{1}{k}}. $$ \end{proof}

In the case of $\mathbb{R}^k$, the induction hypothesis holds if
$k=2$ as we have verified above. Similarly, we have verified the
statement for $S^k$ for $k=1$. We are now ready to complete the
higher dimensional argument.  Then for the dimension $d$ argument
we end up with $t$ $(d-1)$-spheres--one of which must have at
least $N/t$ points on it as in the $d=2$ proof. By induction,
these points determine $\gtrsim
{\left(\frac{N}{t}\right)}^{\frac{1}{d-1}}$ distances. It follows
that
\begin{equation}
\# \Delta(P) \gtrsim \max \left\{t,
{\left(\frac{N}{t}\right)}^{\frac{1}{d-1}} \right\}.
\end{equation}

We now use the fact that
\begin{equation}
\max \{X,Y\} \ge
{(XY^{d-1})}^{\frac{1}{d}} \ \text{(why?)},
\end{equation} which
implies that
\begin{equation} \label{rootd.eqn}
\# \Delta(P) \gtrsim N^{\frac{1}{d}}.
\end{equation}


We just proved the following result.

\begin{theorem} \label{higherdimerdos.thm}
Let $P$ be a subset of $\mathbb{R}^d$, $d \ge
2$, such that $\# P=N$. Then $(\ref{rootd.eqn})$ holds. \end{theorem}

%chapter 1 exercises

\begin{exercise}  \label{ex1.1}
Prove that the minimum of $\max \{t, N/2t\}$ is in fact $\sqrt{N}$.  In other words, show that Erd\H{o}s's method of proof cannot do better than $\# \Delta(P) \gtrsim \sqrt{N}$. \end{exercise}

\begin{exercise} \label{ex1.2}
Let $K$ be a polygon in the plane. Let $\#
P=N$. Prove that $\# \Delta_K(P) \gtrsim \sqrt{N}$. What about
other convex $K$? This problem turns out to be surprisingly difficult.
See a very nice article by Julia Garibaldi. \end{exercise}

\begin{exercise} \label{ex1.3}
We outline an alternate proof of Theorem \ref{erdos.thm}. Let $M_N$ denote the matrix constructed as follows. Fix $t
\in \Delta(P)$ and let the entry $a_{pp'}=1$ if $|p-p'|=t$, and
$0$ otherwise. Observe that for a fixed pair $(p',p'')$, $p'
\not=p''$, $a_{pp'} \cdot a_{pp''}=1$ for at most one value of $p$
(why?). Use this along with the Cauchy-Schwartz inequality to
prove that $\sum_{p,p' \in P} a_{pp'} \lesssim N^{\frac{3}{2}}$.
Conclude that for any $t \in \Delta(P)$, $\# \{(p,p'): |p-p'|=t\}
\lesssim N^{\frac{3}{2}}$. Deduce that $\# \Delta(P) \gtrsim
\sqrt{N}$. Can you make this idea run in higher dimensions?
\end{exercise}

\begin{exercise} \label{ex1.4}
In the proofs of Theorems \ref{erdos.thm} and \ref{higherdimerdos.thm} we
only used spheres centered at a single point. Is there any milage
to be gained from considering, say, two points? Try it.
\end{exercise}

\section{Moser's approach and the Erd\H{o}s integer distance
principle}




Erd\H{o}s' ingenious argument, described in the previous chapter,
relies on spheres centered at a single point, and it stands to
reason that one might gain something out of considering spheres
centered at two points. This point of view was introduced by Moser
in the early 1950s. Before presenting Moser's argument, we will
present the Erd\H{o}s integer distance principle where an idea similar
to Moser's is already present, albeit in a different form and
context.

\begin{proof}[Erd\H{o}s integer distance principle (EIDP), \cite{Erd2}] Let $A$ be an
infinite subset of $\mathbb{R}^d$, $d \ge 2$. Suppose that
$\Delta(A) \subset \mathbb{Z}$. Then $A$ is contained in a line.
\end{proof}

To prove EIDP suppose that $A$ is not contained in a line. Suppose
that $d=2$. Let $a,a',a''$ denote three points of $A$ not lying on
the same line. Let $b$ be any other point of $A$. By assumption,
$|a-b|$ and $|a'-b|$ are both integers, which means that
$|a-b|-|a'-b|$ is also an integer. This means that every point of
$A$ is contained on hyperbolas with focal points at $a$ and $a'$.
(See Appendix 2 for a thorough description of basic theory of
hyperbolas in the plane). How many such hyperbolas are there?
Well, suppose that $|a-a'|=k$, which, by assumption is an integer.
By the triangle inequality, $||a-b|-|a'-b|| \leq |a-a'|=k$. It
follows that there are only $k+1$ different hyperbolas with focal
points at $a$ and $a'$. Similarly, all the points of $A$ are
contained in $l+1$ hyperbolas with focal points at $a'$ and $a''$.
Any hyperbola with focal points at $a$ and $a'$ and a hyperbola
with focal points at $a'$ and $a''$ intersect at at most $4$
points (see Appendix 2 once again). It follows that the number of
points in $A$ cannot exceed $16(k+1)(l+1)$, which is a
contradiction since $A$ is assumed to be infinite. This proves the
two-dimensional case of the Erd\H{o}s integer distance principle. The
higher dimensional argument is outlined in Exercise \ref{ex2.1} below.

The following beautiful extension of the Erd\H{o}s integer distance
principle was proved by Jozsef Solymosi \cite{Soly}.

\begin{theorem} \label{soly.thm}
Suppose that $P$ is a subset of $\mathbb{R}^2$, such that $\Delta(P)
\subset \mathbb{Z}$ and $\# P=N$. Suppose that $P$ is contained in a
disk of radius $R$. Then $R \gtrsim N$. \end{theorem}

The proof of Solymosi's theorem is outlined in Exercise \ref{ex2.5}, and
in Exercise \ref{ex2.6} we ask you to verify that Theorem \ref{soly.thm} would follow
immediately from the Erd\H{o}s distance conjecture.

We are now ready to introduce Moser's idea. Choose points $X$ and
$Y$ in $P$ such that
\begin{equation}
|X-Y| \leq \min \{|p-p'|: p,p' \in P\}.
\end{equation}


Let $O$ be the midpoint of the segment $XY$. Half the points of
$P$ are either above or below the line connecting $X$ and $Y$.
Call this set of points $P'$. Assume without loss of generality
that at least half the points are above the line. Draw half annuli
centered at $O$ of thickness $|X-Y|$ until all the points of $P'$
are covered. Keep only one third of the annuli in such a way that
at least one third of the points of $P'$ are there and such that
if a particular annulus is kept, the next two consecutive annuli
are discarded. (Prove that this can be done and explain why we are
doing this as you read the rest of the argument!). Call the
resulting set of points $P''$. Let $n_j$ denote the number of
points of $P''$ in the $j$th annulus. Let $\mathcal{A}_j$ denote the
intersection of $P''$ with the $j$th annulus. Suppose that
\begin{equation}
\{|p-X|: p \in \mathcal{A}_j\} \cup \{|p-Y|: p \in
\mathcal{A}_j\}=\{d_1,d_2, \dots, d_k\}.
\end{equation}


Let
\begin{align}
A_j&=\{p \in \mathcal{A}_j: |p-X|=d_j\}, \\ \intertext{and}
 B_i&=\{p \in \mathcal{A}_i: |p-Y|=d_i\}.
\end{align}


By construction,
\begin{equation}
A_j=\cup_i \left( A_j \cap B_i \right),
\end{equation}
 since
points of distance $d_j$ from $X$ are of some distance or another
from $Y$.  It follows that
\begin{equation}
\cup_j A_j=\cup_{i,j} \left( A_j \cap B_i \right).
\end{equation}

Now,
\begin{equation}
\#  \cup_j A_j=n_j,
\end{equation}
 while
\begin{equation} \label{maxintersections.eqn}
\# \cup_{i,j} \left( A_j \cap B_i \right) \leq k^2 \max_{i,j} \# \left( A_j \cap B_i \right).
\end{equation}


Now, $A_j$ and $B_i$ are contained on circles of approximately the
same radius centered at different points, so $\max_{i,j} \# (A_j
\cap B_i) \leq 1$. Plugging this into $(\ref{maxintersections.eqn})$ we see that
\begin{equation}
k \ge \sqrt{n_j},
\end{equation}
 from which we deduce that
\begin{equation} \label{moser.eqn}
\# \Delta(P) \ge \# \Delta(P'') \ge \sum_j \sqrt{n_j}.
\end{equation}


We have

\begin{align}
\frac{N}{6} \leq \sum_j n_j&=\sum_j \sqrt{n_j} \cdot
\sqrt{n_j} \leq \sqrt{n_{max}} \cdot \sum_j \sqrt{n_j}, \\
\intertext{where}
n_{max}&=\max_j n_j.
\end{align}


Observe that by the proof of Theorem \ref{erdos.thm},
\begin{equation}
 \# \Delta(P) \ge \# \Delta(P'') \ge n_{max}.
\end{equation}


By $(\ref{moser.eqn})$,
\begin{equation}
\# \Delta(P) \ge \frac{N}{6 \sqrt{n_{max}}}.
\end{equation}

It follows that
\begin{equation}
{(\# \Delta(P))}^2 \cdot \# \Delta(P) \ge
n_{max} \cdot \frac{N^2}{36 n_{max}}=\frac{N^2}{36}.
\end{equation}



Which implies that
\begin{equation}
\# \Delta(P) \ge
\frac{N^{\frac{2}{3}}}{{(36)}^{\frac{1}{3}}},
\end{equation}
and we
have just proved the following theorem.

\begin{theorem}[Moser \cite{Mos}]  \label{moser.thm}
Let $d=2$ and suppose that $\# P=N$. Then $\# \Delta(P) \gtrsim N^{\frac{2}{3}}$.

 \end{theorem}

\bigskip
% chapter 2 exercises

\begin{exercise} \label{ex2.1}
Why did we eliminate $2/3$ of the annuli
in the proof above? Where did we use this in the proof?
\end{exercise}

\begin{exercise} \label{ex2.2}
What does Moser's method yield in higher
dimensions? Can you use the two-dimensional result along with the
induction argument used to prove Theorem \ref{higherdimerdos.thm} instead? Which
approach yields better exponents? \end{exercise}

\begin{exercise} \label{ex2.3}
Let $A$ be an infinite subset of ${\Bbb
R}^d$, $d \ge 2$, with the following property. We assume that
$|a-a'| \ge \frac{1}{100}$ for all $a \not=a' \in A$. We also
assume that for every $m \in \mathbb{Z}^d$, ${[0,1]}^d+m$ contains
exactly one point of $A$. Let $A_q={[0,q]}^d \cap A$. What kind of
a bound can you obtain for $\Delta(A_q)$ using Moser's idea? Why
is this bound better than the one we obtain above?

Take this a step further. Instead of using two points as in
Moser's argument, use $d$ points. How should these points be
arranged? What effect are we trying to achieve? Can you obtain a
better exponent this way? \end{exercise}

\begin{exercise} \label{ex2.4}
Outline of proof of EIDP in higher
dimensions.
\end{exercise}

\begin{exercise} \label{ex2.5}
Deduce Solymosi' s Theorem from the following observation by using ideas from the proof of the EIDP.

\begin{obs} \label{obs1}
For every set of $n$ points in the plane with diameter $\Delta$ and with at most $n/2$ collinear points, there exists two pairs of points $A$,$B$ and $C$,$D$ such that each of the distances $\overline{AB}$ and $\overline{CD}$ are less than $6\Delta/n^{1/2}$.
\end{obs}

Now prove the Observation \ref{obs1}.  {\it Hint:}  Show that there are fewer than $n/2$ points that are not within $6\Delta/n^{1/2}$ of other points.

\end{exercise}


\begin{exercise} \label{ex2.6}
Deduce Solymosi's theorem from the Erd\H{o}s
distance conjecture. \end{exercise}


\section{Incidence theorems and graph theory}


If you are familiar with basic theory of graphs, keep reading. If
not, read Appendix 4 first where basic notions of graph theory are
introduced and proved. We will also make use of some basic
concepts from probability theory. Those are described in Appendix 5
below.

Let $P$ be a finite set of $n$ points in $\mathbb{R}^2$, and let
$L$ be a finite set of $m$ lines. Define an incidence of $P$ and
$L$ to be a pair $(p,l) \in P \times L: p \in l$. Let $I_{P,L}$ denote
the total number of incidences between $P$ and $L$. More precisely,
\begin{equation}
I_{P,L}=\# \{(p,l) \in P \times L: p \in l\}.
\end{equation}



We already proved something about $I_{P,L}$ in
Exercise \ref{ex1.3}, did
we not? Let us think about it for a moment. Let $\delta_{lp}=1$ if
$p \in l$, and $0$ otherwise. Then, by the Cauchy-Schwartz inequality,
\begin{equation}
\begin{split}
 I_{P,L}&=\sum_l \sum_p \delta_{lp} \leq \sqrt{m} {\left( \sum_l
{\left|\sum_p \delta_{lp} \right|}^2 \right)}^{\frac{1}{2}}\\
&=\sqrt{m} {\left( \sum_l \sum_p \delta^2_{lp}+
\sum_l \sum_{p \not=p'} \delta_{lp}\delta_{lp'} \right)}^{\frac{1}{2}}\\
&\leq \sqrt{m} {\left(mn+\sum_l \sum_{p \not=p'} \delta_{lp}\delta_{lp'}
\right)}^{\frac{1}{2}}.
\end{split}
\end{equation}


Now, for each $(p,p') \in P \times P$, $p \not=p'$, there is at most one $l$
such that $\delta_{lp}\delta_{lp'} \not=0$. This is because $\delta_{lp}=1$
means that $p \in l$, and $\delta_{lp'}=1$ means that $p' \in l$. Since two
points uniquely determine a line, the expression $\delta_{lp}\delta_{lp'}$
cannot equal to one for any other $l$. It follows that
\begin{equation}
\sum_l \sum_{p \not=p'} \delta_{lp}\delta_{lp'} \leq \# \{(p,p') \in P \times P:
p \not=p'\}=n(n-1).
\end{equation}



Now it can be shown that the following theorem holds.  Check the details.

%
%If we run this argument after exchanging the roles of $p$ and $l$ (work
%out the details now!), we see that
%\begin{equation}
%I_{P,L} \leq 2n \sqrt{m}.
%\end{equation}

%

%Thus we have proved the following incidence theorem.

\begin{theorem} \label{incidencebound.thm}
Let $P$ be a set of $n$ points in the plane, and let
$L$ be a set of $m$ lines. Then $I_{P,L} \lesssim m \sqrt{n} +  n \sqrt{m}$.

\end{theorem}

As pretty as this result is, it turns out that we can do better.
The following improvement on Theorem \ref{incidencebound.thm} is
due to Szemeredi and Trotter \cite{Sze2}.

\begin{theorem} \label{SzemTrot}
Let $P$ be a set of $n$ points in the plane, and let
$L$ be a set of $m$ lines. Then $I_{P,L} \lesssim n+m+{(nm)}^{\frac{2}{3}}$.
\end{theorem}

We shall deduce this theorem from the following graph theoretic result.

\begin{theorem} \label{crossings}
Let $G$ be a graph with $n$ vertices and $e$
edges. Suppose that $e \ge 4n$. Then \[cr(G) \gtrsim \frac{e^3}{n^2}\].
\end{theorem}

We now prove Theorem \ref{SzemTrot} using Theorem \ref{crossings}.
In order to use Theorem \ref{crossings} we construct the following
graph.  Let the points of $P$ be the vertices of $G$ and let the
line segments connecting points of $P$ on the lines $L$ be the
edges.  You will prove in Exercise \ref{ex3.2} below (not very
difficult) that
\begin{equation} \label{edges.eqn}
 e=I-m.
\end{equation}



There are two possibilities. If $e<4n$, then
\begin{equation} \label{bound1.eqn}
 I<m+4n,
\end{equation}

 which is certainly alright with us.

If $e \ge 4n$, then Theorem \ref{crossings} kicks in and we have
\begin{equation} \label{bound2.eqn}
cr(G) \gtrsim \frac{e^3}{n^2}=\frac{{(I-m)}^3}{n^2}.
\end{equation}



On the other hand, a crossing arises when two edges intersect not at
a vertex. Since edges come from lines and there are $m$ lines,
\begin{equation}
 cr(G) \leq m^2.
\end{equation}

Combining $(\ref{bound1.eqn})$ and $(\ref{bound2.eqn})$, we obtain
the conclusion of Theorem \ref{SzemTrot}.

It remains for us to prove Theorem \ref{crossings}. By Appendix 4,
\begin{equation}
 cr(G) \ge e-3n.
\end{equation}



Choose a random subgraph $H$ of $G$ by keeping each vertex with probability
$p$, a number to be chosen later. It follows that
\[ \mathbb{E}(vertices \ in \ H)=np, \]
\[\mathbb{E}(edges \ in \ H)=ep^2, \]

and
\begin{equation} \label{expect.eqn}
\mathbb{E}(cr(H)) \leq cr(G)p^4,
\end{equation}
where $\mathbb{E}$ denotes the expected value.

By $(\ref{expect.eqn})$ and linearity of expectation,
\begin{equation}
cr(G)p^4 \ge ep^2-3np.
\end{equation}



Choosing $p=\frac{4n}{e}$, as we may, since $e \ge 4n$, we
obtain the conclusion of Theorem \ref{crossings}.

One of the most misused words in mathematics is ``sharp". Nevertheless,
we are about to use it ourselves. We will show that Theorem \ref{SzemTrot} is sharp
in the sense that for any positive integer $n$ and $m$, we can construct
a set $P$ of $n$ points, and a set $L$ of $m$ lines, such that
\begin{equation} \label{SzemTrot.eqn}
I_{P,L} \approx n+m+{(nm)}^{\frac{2}{3}}.
\end{equation}



We shall construct an example in the case $n=m$, but we absolutely
insist that you work out the general case in one of the exercises below. Let
\begin{equation}
P=\{(i,j): 0 \leq i \leq k-1; 0 \leq j \leq 4k^2-1\}.
\end{equation}



Let $L$ be the set consisting of lines given by equations $y=ax+b$,
$0 \leq a \leq 2k-1$, $0 \leq b \leq 2k^2-1$. Thus we have $n$ lines and
$n$ points. Moreover, for $x \in [0,k)$,
\begin{equation}
ax+b<ak+b<4k^2,
\end{equation}

 and it follows that for each $i=0, 1, \dots, k$,
each line of $L$ contains a point of $P$ with
$x$-coordinate equal to $i$. It follows that
\begin{equation}
I_{P,L} \ge k \cdot \# L=\frac{1}{4}n^{\frac{4}{3}}.
\end{equation}

%chapter 3 exercises

\begin{exercise} \label{ex3.1}
Complete the details of the proof of Theorem \ref{incidencebound.thm}.
\end{exercise}

\begin{exercise} \label{ex3.2}
Prove $(\ref{edges.eqn})$ and write out the details.
\end{exercise}

\begin{exercise} \label{ex3.3}
For each $n$ and $m$, construct a set $P$
of $n$ points and a set $L$ of $m$ lines, such that $(\ref{SzemTrot.eqn})$ holds. Use
the argument in the case $n=m$ above as the basis of your construction.
\end{exercise}

\begin{exercise} \label{ex3.4}
Let $P$ be a set of $n$ points in the plane. Let
$L$ be a set of $m$ curves. Let $\alpha_{pp'}$ denote the number of
curves in $L$ that pass through $p$ and $p'$. Let $\beta_{ll'}$ denote
the number of points of $P$ that are contained in both $l$ and $l'$,
Use the proof of Theorem \ref{incidencebound.thm} to show that
\begin{equation}
I_{P,L} \leq n \sqrt{m} {\left(\sum_{p \not=p'} \alpha_{pp'}\right)}
^{\frac{1}{2}} + m \sqrt{n} {\left(\sum_{l \not=l'}
\beta_{ll'}\right)}^{\frac{1}{2}}.
\end{equation}
 \end{exercise}

\begin{exercise} \label{ex3.5.1}
Prove a modified version of Theorem \ref{crossings} which says that if $\alpha$
is the maximum number of edges connecting a pair of vertices in $G$, then
\begin{equation}
cr(G) \gtrsim \frac{e^3}{\alpha n^2}.
\end{equation}
{\it Hint:}  This can be proven by repeatedly using probabilistic arguments similar to those used in the proof of Theorem \ref{crossings}.  First, delete edges independently with probability $1- \frac{1}{k}$ and then delete all the remaining multiple edges--call this resulting graph $G'$.  Calculate the probability $p_e$ that a fixed edge $e$ remains in $G'$.  Now compare the expected number of edges and crossings in $G'$ to the number in the original graph and use Theorem \ref{crossings}.  Finally, use Jensen's inequality which says that $\mathbb{E}[x^a] \geq (\mathbb{E}[x])^a$ for $a \geq 1$.
\end{exercise}


\begin{exercise} \label{ex3.5}
Let $P$ be a set of $n$ points in the plane. Let
$L$ be a set of $m$ curves. Suppose that no more than $\alpha$ curves
in $L$ pass through a pair of points of $P$, and no more than $\beta$ points
of $P$ are contained in the intersection of any two curves in $L$. What should
Theorem \ref{SzemTrot} say under these hypotheses? Do it now because we will use
this result in the next chapter.
{\it Hint:}  Use the result from Exercise \ref{ex3.5.1}.
\end{exercise}


\begin{exercise} \label{ex3.6}
Is the weighted theorem given by Exercise \ref{ex3.5} always
stronger than the one given by Exercise \ref{ex3.4}? Give explicit examples to
support your belief. \end{exercise}

\section{Bisectors enter the game: $n^{\frac{4}{5}}$
plateau is reached}


In this section we shall use graph theory that already bore fruit in
the previous
chapter to improve the Erd\H{o}s exponent from $2/3$ to $4/5$.

Suppose that a set $P$ of $n$ points determined $t$ distinct distances. Draw
a circle centered at each point of $P$ containing at least one other
point of $P$.
By assumption, we have at most $t$ circles around each point and thus the total
number of circles is $nt$. By construction, these circles have
$n(n-1)$ incidences
with the points of $P$. The idea now is to estimate the number of incidences
from above in terms of $n$ and $t$ and then derive the lower bound for $t$.

Delete all circles with at most two points on them. This eliminates at
most $2nt$
incidences, and since we may safely assume that $t$ is much smaller than $n$,
the number of incidences of the remaining circles and the points of
$P$ is still
$\gtrsim n^2$. Form a graph whose vertices are points of $P$ and edges are
circular arcs between the points. This graph $G$ has $\approx n$ vertices,
$\approx n^2$
edges, and the number of crossings is $\lesssim {(nt)}^2$.

Suppose for a moment that we can use Theorem \ref{crossings}. Then
\begin{equation}
\frac{e^3}{n^2} \lesssim cr(G) \lesssim {(nt)}^2,
\end{equation}
 and since
$e \approx n^2$, it would follow that
\begin{equation}
n^4 \lesssim n^2t^2,
\end{equation}
 which would imply the Erd\H{o}s Distance Conjecture.
Unfortunately, life is harder than that since Theorem \ref{crossings} only applies
if there is
at most one edge connecting a pair of vertices. In our case we may assume
that there is at most $2t$ edges connecting a pair of vertices (why? see
Exercise \ref{ex4.1} below). Applying Exercise \ref{ex3.5.1} we see that
\begin{equation}
\frac{e^3}{tn^2} \lesssim cr(G) \lesssim n^2t^2,
\end{equation}
 which
implies that \begin{equation}
 t \gtrsim n^{\frac{2}{3}},
\end{equation}
 the Moser's bound from Chapter 2. All this
for $n^{\frac{2}{3}}$?! We must be able to do better than that! How can we
possibly hope to do that? One way is to study edges of high multiplicity
separately.

We try to take advantage of the following phenomenon. Let $p, p'
\in P$. The centers of all the circles that pass through $p$ and
$p'$ are located on the bisector, $l_{pp'}$,  of the points $p$
and $p'$ in $P$ \footnote{The bisector of $p$ and $p'$ is the set
of points that are equidistant to $p$ and $p'$.  Formally,
$l_{pp'}= \{ z \in \mathbb{R}^2 : |z-p|=|z-p'| \}$.  In the
Euclidean metric this turns out to be the line perpendicular to
$\overline{pp'}$ through their midpoint.  For more general metrics
see Exercise \ref{ex4.2}.}.  Let us consider all the bisectors
with at least $k$ points on them. How many such bisectors are
there? Recall that the Szemeredi-Trotter incidence bound (Theorem
\ref{SzemTrot}) says that the number of incidences between $n$
points and $m$ lines is $\lesssim (n+m+{(nm)}^{\frac{2}{3}})$. Let
$m_k$ denote the number of lines with at least $k$ points. Then
the number of incidences is at least $km_k$. It follows that
\begin{equation}
km_k \lesssim n+m_k+{(nm_k)}^{\frac{2}{3}},
\end{equation}
 and we
conclude that
\begin{equation} \label{richlines}
m_k \lesssim \frac{n}{k}+\frac{n^2}{k^3}.
\end{equation}



This implies that bisectors with at least $k$ points on them have
\begin{equation} \label{richlines2}
\lesssim n+\frac{n^2}{k^2}
\end{equation}
incidences with the points of $P$ (see Exercise \ref{ex4.3}).  


Let $P_k$ denote the set of pairs $(p,p')$ of $P$ connected by
at least $k$ edges. Let $E_k$ denote the set of edges connecting
those pairs. Each edge in $E_k$ connecting a pair $(p,p')$
corresponds to exactly one incidence of $l_{pp'}$ with a point
$p''$ in $P$. However, an incidence of such a $p''$ with some
$l_{pp'}$ corresponds to at most $2t$ edges in $E_k$ since there
at at most $t$ circles centered at $p''$. It follows that
\begin{equation}
 \# E_k \lesssim tn+\frac{tn^2}{k^2}.
\end{equation}
Note that we are almost certainly over counting $E_k$ here since we are removing all possible edges corresponding to incidences--not just those that contribute to high multiplicity.


Now, if we choose $k=c\sqrt{t}$, for an appropriate constant $c$, then
\begin{equation}
 \# E_k \leq \frac{n^2}{2}.
\end{equation}



If we now erase all the edges of $E_k$, there are still more than
$\frac{n^2}{2}$ edges remaining. Applying Exercise \ref{ex3.5.1}
once again, we see that
\begin{equation}
\frac{e^3}{kn^2} \leq cr(G) \leq n^2t^2.
\end{equation}



Since $k \approx \sqrt{t}$ and $e \approx n^2$, it follows that
\begin{equation}
t \gtrsim n^{\frac{4}{5}}.
\end{equation}



We have just proved the following theorem of Szekely \cite{Sze}.

\begin{theorem} \label{szekely.thm}
Let $P$ be a set of $n$ points in the plane. Then
\begin{equation}
\# \Delta(P) \gtrsim n^{\frac{4}{5}}.
\end{equation}
\end{theorem}

%chapter 4 exercises

\begin{exercise} \label{ex4.1}
Explain why there can be at most $2t$ edges connecting
two vertices in the graph $G$ from the above proof.
\end{exercise}

\begin{exercise} \label{ex4.2}
Consider the $l_1$ metric defined in Exercise \ref{ex0.1}.  Try to
figure out what bisectors look like for this metric.
\end{exercise}

\begin{exercise} \label{ex4.3}
Verify Equation \ref{richlines2}.  {\it Hint:}  Define $M_j$ to be the set of lines with between $2^j$ and $2^{j+1}$ points and observe that $m_k$ can be written as a sum of such sets.
\end{exercise}

\section{Arithmetic joins bisectors in the Erd\H{o}s crusade!}


In this chapter we present the Solymosi-Toth beautiful argument
that will get us up to $n^{\frac{6}{7}}$ which opens the door to
further important developments that we sketch in the next chapter.
We start out with the following beautiful observation due to
Jozsef Beck \cite{Beck}.  The proof we give is from \cite{Sol2}.

\begin{lemma} \label{Beck}
Let $P$ be a collection of $n$ points in the plane. Then
one of the following holds:
\begin{enumerate}
\item There exists a line containing  $\approx n$ points of $P$.
\item There exist $ \approx n^2$ different lines each
containing at least two points of $P$.
\end{enumerate}
\end{lemma}


\begin{proof}
Let $L_{u,v}$ be the number of pairs of points of $P$ which
determine a line that goes through at least $u$ but at most $v$
points of $P$.  Equation \ref{richlines} and basic counting
arguments tells us that $L_{u,v} \lesssim \frac{n^2v^2}{u^3} +
\frac{nv^2}{u}$ (see exercise \ref{ex5.3}).  Fix a constant $C$
and consider $L_{C, N/C}$.  Then
\begin{equation}
\begin{split}
L_{C, N/C} &\leq \sum_{i=0}^{\lfloor \log(N) \rfloor} N_{C2^i, C2^{i-1}} \\
&= \sum_{i=0}^{\lfloor \log(N) \rfloor} O\left( \frac{4N^2}{C2^i} + 4CN2^i \right)  \\
&=  O \left( \frac{N^2}{C} \sum_{i=0}^{\lfloor \log(N) \rfloor} 2^{-i} + NC \sum_{i=0}^{\lfloor \log(N) \rfloor} 2^i \right) \\
&= O\left( \frac{N^2}{C} \right).
\end{split}
\end{equation}


In other words, for some $C_o > 0$ we have $L_{C, N/C} \leq C_o
\left( N^2/C \right)$.  Thus for the appropriate choice of $C$ at
least half of the pairs of points determine a line through fewer
than $C$ or at least $C/N$ points.  And consequently at least a
fourth of the pairs go through fewer than $C$ points or a fourth
go through at least $C/N$ points.  In either case we are done.

\end{proof}


Consider a set $P$ of $n$ points and let  $\mathcal{L}$ denote the
set of lines passing through at least two points of $P$.  An
averaging argument (see exercise \ref{ex5.1}) applied to Lemma
\ref{Beck} implies that there exists an absolute constant $c_o$
such that at least $c_on$ points of $P$ are incident to at least
$c_on$ lines of $\mathcal{L}$.  Then let $B$ be the set of such
points, and take some arbitrary point $a \in B$.

Draw in the lines through $a$ that go through points of $P$. There
must be at least $c_on$ such lines.  Choose one point other than
$a$ on each of these lines and draw in the circles around $a$ that
hit those chosen points (deleting those capturing fewer than $3$
points).  On each of these circles break the points in triples,
possibly deleting as many as $2$ from each.  We still have
$\gtrsim n$ points left by our hypotheses (check!).

We call a triple ``bad" if all three bisectors formed from its
points go through at least $k$ points.  And we call the initial
point $a$ from $B$ ``bad" if at least half of its triples are bad.
We would like to choose $k$ such that at least half the points of
$B$ are bad.  Clearly, the smaller $k$ is the ``easier" it is to
get $k$-rich lines and thus more bad points.  However, it will
become clear that we would like $k$ as large as possible.  You
will show in Exercise \ref{ex5.2} that we may take $k=
\frac{c_2n^2}{t^2}$.

Then if we can get the following upper and lower bounds on the number of
incidences $I(L_k, P)$ of $k$-rich lines and bad points we will be done:
\begin{equation}
n^2/t^{2/3} \lesssim I(L_k, P) \lesssim t^4/n^2.
\end{equation}

Finding an upper bound on $I(L_k, P)$ is straight forward.  We
simply apply Equation \ref{richlines} to find a bound on the
number of $k$-rich lines and then use Theorem
\ref{incidencebound.thm} to get that $I(L_k, P) \lesssim n^2/k^2$.
Getting a lower bound on the quantity $I(L_k, p)$ in terms of $n$
and $t$ is somewhat harder.  The following lemma is the key to the
whole proof.

\begin{lemma} \label{lin.lem}
Let $T$ be a set of $N$ triples $(a_i,b_i,c_i)$ of distinct real
numbers such that $a_i<b_i<c_i$ for $i=1, \ldots, N$ and
$c_i<a_{i+1}$ for all but at most $t-1$ of the $i$.  Let $W= \{
\frac{a_i+b_i}{2}, \frac{a_i+c_i}{2},\frac{b_i+c_i}{2}: i=1,
\ldots, N \}$.  Then $|W| \gtrsim \frac{N}{t^{2/3}}$.
\end{lemma}

\begin{proof} Let the range of a triple $(a,b,c) \in T$ be defined
as the interval $[a,c]$. By assumption, the sequence
$(a_1,b_1,c_1,a_2,b_2,c_2, \dots, a_N,b_N,c_N)$ be partitioned
into at most $t$ contiguous monotone increasing subsequences.
Partition the real axis into $N/(2t)$ open intervals so that each
interval fully contains the ranges of $t$ triples. These intervals
are constructed from left to right. Let $x$ denote the right
endpoint of the rightmost interval constructed so far. Discard the
at most $t$ triples whose ranges contain $x$, and move to the
right until you reach a point $y$ that lies to the right of
exactly $t$ new ranges. We add $(x,y)$ as a new open interval and
continue in this manner until all triples are processed.

Let $s$ be one of the open intervals defined in the previous
paragraph. Each triple in $T$ whose range is fully contained in
$s$ contributes three elements to $W \cap s$, and no two triples
of $T$ contribute the same triple to $W \cap s$. It follows that
$|W \cap S| \ge t^{\frac{1}{3}}$, since otherwise the number of
distinct triples of its elements would be smaller than $t$. Since
the number of intervals $s$ is $N/(2t)$, the conclusion of the
lemma follows by the multiplication principle. This completes the
proof of the lemma. \end{proof}

For each point $p \neq a$ in a bad triple, map $p$ to the
orientation of the ray $\overrightarrow{ap}$.  By construction
this map is an injection, and $W$ corresponds to $k$-rich lines.
Therefore the number of $k$-rich lines incident to $a$ is $\gtrsim
n/t^{2/3}$. And since $a$ was an arbitrary element of $B$, we get
that $I(L_k, P) \gtrsim n^2/t^{2/3}$.

The only thing that remains is to show that if we take $k=
\frac{c_2n^2}{t^2}$ then half of the points of $P$ are ``bad".
Construct a multigraph $G$ out of the points that are part of the
triples as in the proof of Theorem \ref{szekely.thm}. Next apply
the result of Exercise \ref{ex3.5.1}. Doing this we
find that we can take $k= \frac{c_2n^2}{t^2}$ and at least
$c_on/2$ points of $B$ will be bad. We leave the details as an
exercise to the reader. See \cite{Sol} for the details.


%chapter 5 exercises

\begin{exercise} \label{ex5.1}
Write up the details of the averaging argument which tells us that
``many'' points go through ``many" lines of $\mathcal{L}$.  {\it Hint:}
recall that we may assume that $t = o(n)$. \end{exercise}

\begin{exercise} \label{ex5.2}
Work out the details showing that we may take $k=
\frac{c_2n^2}{t^2}$ and at least $c_on/2$ points of $B$ will still
be bad.\end{exercise}

\begin{exercise} \label{ex5.3}
Check that Equation \ref{richlines} and basic counting arguments
gives us that $L_{u,v} \lesssim \frac{n^2v^2}{u^3} +
\frac{nv^2}{u}$.\end{exercise}


\begin{exercise} \label{ex5.4}
Find the constants $C$ and $C_o$ in the proof of Theorem
\ref{Beck} and write up the details of why we are done in the case
where at least a fourth of the pairs go through at least $N/C$
points of $P$. \end{exercise}


\appendix

\section{Sums of Squares }




\section{Hyperbolas in the Plane }

The standard equation for a hyperbola in the plane that is centered at the origin and whose foci are $(-c,0)$ and $(c,0)$ is $x^2/a^2-y^2/b^2=1$, where $a^2+b^2=c^c$.  Fixing two points $F_1$ and $F_2$ in the plane a hyperbola can also be described as the set of points $P$ such that  $| |PF_1|-|PF_2|| = 2a$ for some fixed number $a$.  We will tend to use the latter definition.  Check and see how these two definitions are related!

%Include a figure
%what else to include????

\section{The Cauchy-Schwartz Inequality }


Let ${\{a_j\}}_{j=1}^n$ and ${\{b_j\}}_{j=1}^n$ be sequences of real numbers.
Our goal is to prove that
\begin{equation}
\sum_{j=1}^n a_jb_j \leq {\left(\sum_{j=1}^n a_j^2 \right)}^{\frac{1}{2}}
\cdot {\left( \sum_{j=1}^n b_j^2 \right)}^{\frac{1}{2}}.
\end{equation}



Let
\begin{equation}
A={\left(\sum_{j=1}^n a_j^2 \right)}^{\frac{1}{2}}  \  \text{and} \
B={\left(\sum_{j=1}^n b_j^2 \right)}^{\frac{1}{2}} ,
\end{equation}
so it
suffices to prove that
\begin{equation}
\sum_{j=1}^n \frac{a_j}{A} \frac{b_j}{B} \leq 1.
\end{equation}



Since
\begin{equation}
 {\left( \frac{a_j}{A}-\frac{b_j}{B} \right)}^2 \ge 0,
\end{equation}
we
conclude that
\begin{equation}
\frac{a_j}{A} \cdot \frac{b_j}{B} \leq \frac{1}{2}\frac{a_j^2}{A^2}
+\frac{1}{2} \frac{b_j^2}{B^2}.
\end{equation}



It follows that
\begin{equation}
\sum_{j=1}^n \frac{a_j}{A} \frac{b_j}{B} \leq
\frac{1}{2} \sum_{j=1}^n \frac{a_j^2}{A^2}+
\frac{1}{2} \sum_{j=1}^n \frac{b_j^2}{B^2}=\frac{1}{2}+\frac{1}{2}=1.
\end{equation}



Thus we have proved the Cauchy Schwartz inequality:
\begin{theorem} Let $a_j, b_j$ be as above. Then $(6.3.1)$
holds. \end{theorem}



\section{Basic graph theory }

To appear soon!
%Definition of graph, basic theorems (Euler), notation

\section{Basic probability theory  }

Also to appear soon.  We promise.
%basic expectation

\begin{thebibliography}{55}

\bibitem{Agar} P. K. Agarwal, E. Nevo, J. Pach,
R. Pinchasi, M. Sharir, and S. Smorodinsky,
{\it Lenses in arrangements of pseudo-circles
and their applications}, J. ACM, to appear.

\bibitem{Ajt} M. Ajtai, V. Chvatal, M. Newborn,
and E. Szemeredi, {\it Crossing-free subgraphs},
Ann. Discrete Mathematics {\bf 12} (1982) 9--12.

%\bibitem{Aro2} B. Aronov and M. Sharir,
{\it Cutting circles into pseudo-segments and
improved bounds for incidences}, Discrete Comput.
Geom. {\bf 28} (2002), no.~4, 475--490.

%\bibitem{Aru}  G. Arutyunyants and A. Iosevich,
{\it Falconer conjecture, spherical averages and
discrete analogs}, Towards a theory of geometric
graphs, 15--24, Contemp. Math., 342, Amer. Math.
Soc., Providence, RI, 2004.

\bibitem{Beck} J. Beck, {\it On the lattice property
of the plane and some problems of Dirac, Motzkin and
Erd\H{o}s in combinatorial geometry}, Combinatorica
{\bf 3} (1983), no.~3-4, 281--297.

%\bibitem{Beck2} J. Beck and J. Spencer,
{\it Unit distances}, J. Combinatorical Theory
A {\bf 37} (1984), no.~3, 231--238.

%\bibitem{Bou} J. Bourgain,
{\it Hausdorff dimension and distance sets}, Israel J. Math.
{\bf
87} (1994), no.~1-3, 193--201.

%\bibitem{Bou2} J. Bourgain,
{\it On the Erd\H{o}s-Volkmann and Katz-Tao ring conjectures},
Geom. Funct. Anal. {\bf 13} (2003), no.~2, 334--365.

\bibitem{Bra} P. Brass, {\it Erd\H{o}s distance
problems in normed spaces}, Discrete Comput. Geom.
{\bf 17} (1997), no.~1, 111--117.

\bibitem{Chu}F. R. K. Chung, {\it The number of
different distances determined by $n$ points
in the plane}, J. Combin. Theory Ser. A {\bf 36}
(1984), no.~3, 342--354.


\bibitem{Chu2} F. R. K. Chung, E. Szemer\'edi\ and\
W. T. Trotter, {\it The number of different distances
determined by a set of points in the Euclidean plane},
Discrete Comput. Geom. {\bf 7} (1992), no.~1, 1--11.

\bibitem{Cla} K. L. Clarkson, H. Edelsbrunner, L. Guibas,
M. Sharir, and E. Welzl, {\it Combinatorial complexity
bounds for arrangements of curves and spheres}, Discrete
Comput. Geom.
{\bf 5} (1990), no.~2, 99--160.

\bibitem{Erd} P.~Erd\H{o}s,
{\it On sets of distances of $n$ points},
Amer. Math. Monthly {\bf 53}  (1946) 248--250.

\bibitem{Erd2} P.~Erd\H{o}s,
{\it Integral distances},  Bull. Amer. Math. Soc.
{\bf 51}  (1945).

\bibitem{Gar} J. S. Garibaldi, {\it A Lower Bound
for the Erd\H{o}s Distance Problem for Convex Metrics},
preprint.

\bibitem{Ios2} A. Iosevich, {\it Curvature, combinatorics
and the Fourier transform}, Notices Amer. Math. Soc.
{\bf 48} (2001), no.~6, 577--583.

%\bibitem{Ios5} A. Iosevich,
{\it Fourier analysis and geometric combinatorics}, (to appear in
the Birkhauser volume dedicated to the annual Padova lectures in
analysis).

\bibitem{Ios}  A. Iosevich\ and\ I. \L aba,
{\it Distance sets of well-distributed planar sets},
Discrete Comput. Geom. {\bf 31} (2004), no.~2, 243--250.


\bibitem{Ios3} A. Iosevich\ and\ I. \L aba,
{\it K-distance, Falconer conjecture, and

discrete analogs}, preprint.

%\bibitem{Joz} S. Jozsa and E. Szemeredi, {\it The number of unit distances on the plane}, Infinite and finite sets: Coll. Math. Soc. J. Bolyai {\bf 10} (1973), 939--950.

\bibitem{Katz} N. H. Katz and G. Tardos, {\it A new entropy inequality for the Erd\H{o}s distance problem}, Towards a Theory of Geometric Graphs,  (ed. J Pach) Contempory Mathematics {\bf 342} (2004), 119--12.

\bibitem{Kho2} A. G. Khovanski\u\i, {\it A class of systems of transcendental equations}, Dokl. Akad. Nauk SSSR {\bf 255} (1980), no.~4, 804--807.

\bibitem{Kho} A. G. Khovanski\u\i, {\it Fewnomials}, Translated from the Russian by Smilka Zdravkovska,
Amer. Math. Soc., Providence, RI, 1991.


%\bibitem{Kon} S. Konyagin and I. \L aba, {\it Separated sets and the Falconer conjecture for polygonal norms}, preprint.
%
%\bibitem{Kon2} S. Konyagin and I. \L aba, {\it Distance sets of well-distributed planar sets for polygonal norms}, preprint.


%\bibitem{Ios6} S. Hoffman and A. Iosevich, {\it Circular averages and Falconer/Erd\H{o}s distance conjecture in the plane for random metrics}, preprint.


%
%\bibitem{Ios4} A. Iosevich and M. Rudnev, {\it Distinct distances on a sphere}, preprint.

%\bibitem{Ios7} A. Iosevich and M. Rudnev, {\it Endpoint bounds for the non-isotropic Falconer distance problem associated with lattice-like sets}, preprint.


\bibitem{Lei} T. Leighton, {\it Complexity Issues in VLSI, Foundations of Computer Series}, MIT Press, Cambridge, MA, 1983.

\bibitem{MaLi} L. Ma, {\it Bisectors and Voronoi diagrams for convex distance functions}, dissertation (unpublished).

\bibitem{Mos} L. Moser, {\it On the different distances determined by $n$ points}, Amer. Math. Monthly {\bf 59} (1952), 85--91.

\bibitem{Pach} J. Pach and P.K. Agarwal, {Combinatorial Geometry}, Wiley, New York, NY, 1995.

\bibitem{Pach2} J. Pach\ and\ M. Sharir, {\it On the number of incidences of points and curves}, Combin. Probab. Comput. {\bf 7} (1998), no.~1, 121--127.

\bibitem{Sol} J. Solymosi and C. T\'{o}th, {\it Distinct distances in the plane}, Discrete Comput. Geom. {\bf 25} (2001), no.~4, 629--634.

\bibitem{Soly} J. Solymosi, {\it Note on integral distances},
Discrete Comput. Geom.  {\bf 30}  (2003),  no. 2, 337--342.

\bibitem{Sol2} J. Solymosi, G. Tardos\ and\ C. D. T\'oth, \emph{The $k$ most frequent distances in the plane}, Discrete Comput. Geom. \textbf{28} (2002), no.~4, 639--648.

%\bibitem{Sol3} J. Solymosi and V. Vu, {\it Distinct distances in high dimensional homogeneous sets}, Towards a theory of geometric graphs, 259--26, Contemp. Math., 342, Amer. Math. Soc., Providence, RI, (2004).

%\bibitem{Sol4} J. Solymosi and V. Vu, {\it Near optimal bounds for the number of distinct distances in high dimensions}, Combinatorica, to appear.


\bibitem{Sze} L.A.~Sz\'{e}kely, {\it Crossing numbers and hard Erd\H{o}s problems in discrete geometry},
Combin. Probab. Comput. {\bf 6} (1997), no.~3, 353--358.

\bibitem{Sze2} E. Szemer\'edi\ and\ W. T. Trotter, Jr.,
{\it Extremal problems in discrete geometry}, Combinatorica {\bf 3}
(1983), no.~3-4,381--392.

\bibitem{Tar} G.~Tardos, {\it On distinct sums and distinct distances}, Adv. Math. {\bf 180} (2003), no.~1, 275--289.

\end{thebibliography}


\end{document}
