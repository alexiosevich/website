\documentstyle{amsppt}
\pagewidth{5.5in}
\vsize7.0in
\magnification=\magstep1
\widestnumber \key{AAAAAAA}
\topmatter
\author Alex Iosevich
\endauthor
\title Maximal operators associated to families of flat curves in the plane 
\endtitle
\endtopmatter
\document

{\bf Introduction:} Let $C$ denote a smooth curve in the plane. Let  
$M_tf(x)=\int_C f(x-ty)d\sigma(y)$, where $d\sigma(y)$ denotes a 
cutoff function times the Lebesgue
measure on $C$. Let $\Cal{M}f(x)=\sup_{t>0}M_tf(x)$. A question we ask is, 
for what range of the exponents $p$, is the following a priori inequality
satisfied:
$$ {||\Cal{M}f||}_p \leq C_p {||f||}_p, \ \ f\in \Cal{S}. \tag1.1$$

Bourgain showed that if $C$ has non-vanishing curvature, the
inequality (1.1) holds for $p>2$ (see \cite{B}). 
In this paper, we shall consider a situation
when the curvature is allowed to vanish of finite order on a finite set of  
isolated points. We shall need the following definition:

\proclaim{Definition1.1} Let $C:I\rightarrow {\Bbb R}^2$, where $I$ is a 
compact interval in ${\Bbb R}$, and $C$ is smooth. We say that $C$ is 
finite type if $\langle (C(x)-C(x_0)), \mu \rangle$ 
does not vanish of infinite
order for any $x_0\in I$, and any unit vector $\mu$. 
\endproclaim 

We shall also need a more precise definition which would specify the 
order of vanishing at each point. Let $a_0$ denote a point
in the compact interval $I$. We can always find a smooth function $\gamma$,
such that in a small neighborhood of $a_0$, $C(s)=(s,\gamma(s))$, where
$s\in I$. 

\proclaim{Definition1.2} Let $C$ be defined as before. Let 
$C(s)=(s,\gamma(s))$ in a small neighborhood of $a_0$. We say that 
$C$ is finite type $m$ at $a_0$ if ${\gamma}^{(k)}(a_0)=0$ for 
$1\leq k<m$, and ${\gamma}^{(m)}(a_0)\not=0$. \endproclaim 

Our main result is the following:

\proclaim{Theorem 1.1} Let $C$ be a finite type curve which is finite type
$m$ at $a_0$. Let 
$M_tf(x)=\int_C f(x-ty)d\sigma(y)$, where $d\sigma$ is 
the Lebesgue measure on $C$ multiplied by a smooth cutoff function supported
in a sufficiently small neighborhood of $a_0$. 
Let ${\Cal M}f(x)=\sup_{t>0}M_tf(x)$. Then, 
$$ {||{\Cal M}f||}_p \leq C_p{||f||}_p \ \ \text{for} \ \ p>m. \tag1.2$$
\endproclaim

Furthermore, the result is sharp. Let  
$h_p(x)={|x_2|}^{-\frac{1}{p}} {\log(\frac{1}{|x_2|})}^{-1} 
\phi(x)$, where $\phi(x)$ is a non-negative $C^{\infty}_{0}$ function 
supported in the unit ball, such that $\phi \equiv 1$ near the origin. 
It is not hard to check that $h_p\in L^p({\Bbb R}^2)$, if
$p>1$. Locally, $C(s)=(s,\gamma(s))$. Without loss of generality,
let $a_0=0$, and $\gamma(0)=-1$. Then, 
$M_t h_p(x)= \infty$, for $p \leq m$, if
$|x_2|=t$, and $x_2<0$. 

We shall also consider a variable coefficient version of Theorem 1.
The averaging operator $M_t$ that we have considered so far is called
translation invariant, or the constant coefficient operator because it
averages a function over the translates and dilates of a fixed curve. We are
now going to consider an operator which averages a function over
a more general distribution of curves in the plane. We are also going to
consider a more general time dependence. As before, we are going to define
a maximal operator by taking the supremum over the time dependence. In order
to formulate the conditions for $L^p$ boundedness of such operators, a more
sophisticated geometric setup is required. 

Let's consider a curve distribution in $ {\Bbb R}^2$ such that through 
every point $x\in {\Bbb R}^2$ we have a $C^\infty$ curve. 
If we assume that 
this curve distribution is a smooth submanifold for each fixed $t$, 
then we can locally 
express this curve distribution as 
$$ {\Cal D}_t=\{(x,y):y_2=x_2+A(x_1,y_1,x_2,t)\} \ \text{for some smooth} \ 
A(x_1,y_1,x_2,t).$$

Let $S_{x,t}$ denote $ \{ y:(x,y) \in {\Cal D}_t \}$    and consider a 
family of operators
$$ M_t(f)(x)=\int_{S_{x,t}} f(x-y) d{\sigma}_{x,t}(y), \tag1.3$$where    
$d{\sigma}_{x,t}$ is the  
smooth cutoff function times the Lebesgue measure on  $S_{x,t}.$ 

After taking a partial Fourier transform, we can rewrite $M_t(f)(x)$ as 
$$ {(2\pi)}^{-1} \int \int e^{i\theta(A(x_1,y_1,x_2,t)+x_2-y_2)} \psi(x,y) f(y)
d\theta dy, \tag1.4$$ where $\psi(x,y)$ is a cutoff function.

After setting $M_t(f)(x)=(Ff)(x,t)$ and using a standard theorem about orders
of Fourier integral operators (See \cite{So1}), 
we see that the $M_t$'s are Fourier 
integral operators of the order given by 
$$ \text{order(symbol)}+(\frac{1}{2}  
\# \text{oscillating 
vars})-\frac{1}{4} (\# x \ \text{vars}+ \# y \  \text{vars})=0+
\frac{1}{2}-1=-\frac{1}{2}.$$

Let $F_t(x,y)$ denote the kernel of this Fourier integral operator. 
The distribution defined by $(x,y)\rightarrow F(x,y,t)$ is a Lagrangian
distribution of order $-\frac{1}{2}$, the wave front set of which is contained 
in a subset of the cotangent bundle of ${\Bbb R}^2 \times {\Bbb R}^2$ with the
zero section removed (See e.g. \cite{So1}). This subset is called a canonical
relation. The canonical relation is a Lagrangian submanifold of the 
cotangent bundle with respect to the canonical symplectic form 
$dx\wedge d\xi-dy\wedge d\mu$. 

Since the kernel $F_t(x,y)$ is supported in ${\Cal D_t}$, 
it is not hard to see that if we consider a "twisted" canonical relation
$C$ where $(x,\xi,y,\mu)$ is replaced by $(x,\xi,y,-\mu)$, 
$C$ is in fact a conormal bundle of ${\Cal D_t}$ 
denoted by $N^{*}(D_t)$ (See \cite{Trvs}).
This is related to the fact that the Fourier transform
of the smooth density on a hypersurface decays rapidly in every direction 
except the normal directions to the hypersurface (See \cite{Hor1}).

Let ${\Cal C}_t$ denote the canonical relation for a fixed $t$.
If we let $X,Y$ denote the support of $x \rightarrow F_t(x,y)$ and 
$y \rightarrow F_t(x,y)$ respectively, 
${\Cal C}_t$ can be naturally viewed as
a Lagrangian submanifold of the cotangent space of $X \times Y$ with the 
zero section removed. Let ${\pi}_l:{\Cal C_t}\rightarrow T^{*}(X)\backslash 0$  
and ${\pi}_r:{\Cal C_t} \rightarrow T^{*}(Y)\backslash 0$ 
denote the natural projections of ${\Cal C_t}$. 

We say that ${\Cal C}_t$ is a local canonical graph if 
$$ {\Cal C}_t=\{(x,\xi,y,\mu):(y,\mu)=\Xi_t(x,\xi)\},$$  
where $\Xi_t$ is a symplectomorphism for each $t.$ The distribution of curves
is then said to satisfy the rotational curvature condition (See \cite{PhSt1}).
It is not hard to see 
that this condition is equivalent to the condition that both ${\pi}_l$ and
${\pi}_r$ are local diffeomorphisms. 

As we have noted earlier, the canonical relation ${\Cal C_t}$ is the 
conormal bundle of the curve distribution ${\Cal D_t}$. In other words,
the canonical relation is parameterized by the phase function of the
operator (1.4), $\theta \Phi(x,y,t)$, where 
$\Phi(x,y,t)=y_2-x_2-A(x,y_1,t)$.  

The condition that the projection ${\pi}_l$ is a local diffeomorphism 
means that given \newline $(x_0,{\xi}_0) \in T^{*}(X)\backslash 0$ we can find  
$(y,\theta)\in Y\times {\Bbb R}$ such that $\Phi(x_0,y)=0$ and 
$\theta d_x\Phi(x_0,y,t)={\xi}_0$. This means that the Jacobian of the map
$(\theta,y) \rightarrow (\Phi(x,y,t), \theta d_x\Phi(x,y,t))$ is non-zero.
The resulting Jacobian is called the Monge-Ampere determinant: 

$$ J_t(x,y)=Det \left( \matrix
0 & \Phi_{x_1} & \Phi_{x_2}\\
\Phi_{y_1} & \Phi_{x_1y_1} & \Phi_{x_2y_1}\\
\Phi_{y_2} & \Phi_{x_1y_2} & \Phi_{x_2y_2}\\
\endmatrix \right).
\tag1.5$$ 

We are interested in the defining function of the form  
$y_2-x_2-A(x_1,y_1,x_2,t)$. Hence, the
Monge-Ampere determinant becomes $$(1-A_{x_2})A_{x_1y_1}+ 
A_{x_1}A_{x_2y_1}.\tag1.6$$  

Since the Monge-Ampere determinant is symmetric in the $x$ and $y$ variables,
we see that ${\pi}_l$ is a local diffeomorphism if and only if 
${\pi}_r$ is also. 

If each ${\Cal C}_t$ is a local canonical graph, 
we can express the full canonical
relation in the form:
$$ {\Cal C}=\{(x,\xi,y,\mu,t,\tau):(y,\mu)=\Xi_t(x,\xi), 
\ \tau={q}^*(x,t,\xi)\}, 
\tag1.7$$
where ${q}^*(x,t,\xi)$ is homogeneous of degree 1 in $\xi.$ 

Sogge showed that the
rotational curvature condition is not sufficient in two dimensions (see 
\cite{So1}). He showed that the following extra assumption is necessary.
\proclaim{Cone condition} We say that the canonical relation ${\Cal C}$
as in $(1.7)$ satisfies the cone condition if the cone given by the
equation $\tau=q^{*}(x,t,\xi)$, has exactly one non-vanishing principal
curvature.
\endproclaim

We are going to generalize Sogge's result using the following definition
due to Phong and Stein:

\proclaim{Definition 1.3} Let $  {\Sigma}_t=\{ a \in {\Cal C}_t:\pi_l \ 
\text{is not locally 1-1} \} $
where $ \pi_l,\pi_r:
{\Cal C}_t \rightarrow T^* {\Bbb R}^2 \backslash 0 $ are natural 
projections. We say that ${\Cal C}_t$ 
is folding of order $m-2$ if the following 
conditions hold 
\roster \item $ \Sigma_t$ is a submanifold of ${\Cal C}_t$ of 
codimension 1.
\item $det(d{\pi}_l)$ and $det(d{\pi}_r)$ vanish of order $m-2$ along 
$\Sigma_t$\item  $T_a(\Sigma_t)\oplus 
Ker{(d{\pi}_l)}_a =T_a({\Cal C}_t)$ \item 
$T_a(\Sigma_t)\oplus Ker{(d{\pi}_r)}_a=T_a({\Cal C}_t)$
\endroster 
Conditions (3) and (4) are called the transversality conditions.  
\endproclaim

We note that when $m=3$, the above condition is equivalent to the condition
that both ${\pi}_l$ and ${\pi}_r$ are Whitney folds. (See e.g \cite{MlTr},
\cite{Hor1}). 

Let ${\pi}_{X \times Y}$ denote a projection of the canonical relation onto the
$x$ and $y$ variables. Let $V_t={\pi}_{X \times Y}({\Sigma}_t)$. It follows from
the definition that locally $V_t$ can be expressed in the form
$\{(x,y):y_1={\chi}_t(x_1,x_2), y_2=x_2+A(x,y_1,t) \}$ where ${\chi}_t$ is a 
local diffeomorphism for each fixed $(x_2,t)$. In other words, we can find
a smooth function ${\chi}^{*}_t(y_1,x_2)$ such that 
$V_t= \{(x,y):x_1={\chi}^{*}_t(y_1,x_2), y_2=x_2+A(x,y_1,t) \}$ where
${\chi}^{*}_t(y_1,x_2)$ is a local diffeomorphism for each fixed $(x_2,t)$.

$V_t$ can be viewed as a parameterization of the zeroes of the Monge-Ampere
determinant intersected with the curve distribution ${\Cal D_t}$. 
In the translation invariant case, we have 
$y_2=x_2+A(x,y_1,t)$ with $A(x,y_1,t)=\gamma(\frac{y_1-x_1}{t})$, where
$\gamma:{\Bbb R} \rightarrow {\Bbb R}$ is a smooth function satisfying the
finite type condition described earlier. The Monge-Ampere
determinant in this case is just ${\gamma}^{(2)}(\frac{y_1-x_1}{t})$. Hence,
the Monge-Ampere determinant vanishes of order $m-2$ along the diagonal 
$x=y$. Consequently, $V_t$ is just the intersection of ${\Cal D_t}$ 
with the diagonal. 

In general, the situation is more complicated. However, locally a curve
distribution whose canonical relation has a two-sided fold behaves very
much like a translation invariant family. This idea is contained in the
following result:

\proclaim{Lemma 1.1} Suppose that the canonical relation associated to the
curve distribution ${\Cal D_t}$ has a two-sided fold of order $m-2$. Then,
for each fixed $({x}^{\prime},{t}^{\prime})$ the curve given by the equation 
$y_2={x}^{\prime}_2+A({x}^{\prime},y_1,{t}^{\prime})$ is a 
curve of finite type $m$ with a flat point at
$y_1={\chi}_t({x}^{\prime},{t}^{\prime})$. 
\endproclaim  

\demo{Proof} Without loss of generality let 
${\chi}_{t}^{\prime}({x}^{\prime})=0$. We must show that the second partial
derivative of $A({x}^{\prime},y_1,{t}^{\prime})$ with respect to $y_1$ 
vanishes of order $m-2$ at $y_1=0$. 

By our observation in (1.6) the Monge-Ampere determinant is given by
\newline 
$A_{x_1 y_1}-A_{x_2}A_{x_1 y_1}+A_{x_1}A_{x_2 y_1}.$ By assumption and
our observations, the Monge-Ampere determinant vanishes of order $m-2$
along $V_t$ which is given by
$ \{(x,y): y_1={\chi}_t(x_1,x_2), y_2=x_2+A(x,y_1,t) \}$. Hence,
along $V_t$, $\dsize{A_{x_1}=A_{y_1} \frac{\partial {\chi}_t}{x_1}}$. Since 
${\chi}_t$ is a local diffeomorphism for each fixed $(x,t)$, we conclude
that under our assumptions, $A_{y_1 y_1}({x}^{\prime},y_1,{t}^{\prime})$
vanishes of order $m-2$. Hence, the curve obtained by fixing $(x,t)$ is
finite type $m$. \enddemo

We can actually prove a little more. Fix $({x}^{\prime},{t}^{\prime})$ as
before. We can again assume that ${\chi}_{{t}^{\prime}}({x}^{\prime})=0$. 
Then, by Lemma 1.1 and the Malgrange Preparation Theorem
(See \cite{Hor1}),  
there exist ${\delta}_1,{\delta}_2,{\delta}_3>0$,
such that if $|t-{t}^{\prime}|<{\delta}_1, |y_1|<{\delta}_2,$ and
$|x-{x}^{\prime}|<{\delta}_3$, then 
$$ A(x,y_1,t)=g(x,y_1,t)({y_1}^m+a_{m-1}(x,t){y_1}^{m-1}+...+a_0(x,t)), 
\tag1.8$$
where $g(x,{y_1},t)$ is smooth, $g({x}^{\prime},0,{t}^{\prime})\not=0$, 
$a_j$'s are smooth, and
$a_j({x}^{\prime},{t}^{\prime})=0$.   

However, we must conclude that $a_j=0$ for $j>0$. 
If not, let $t^{\prime \prime}$
denote a point in $ \{t:|t-{t}^{\prime}|<{\delta}_1 \}$ such that  
$a_j(t^{\prime \prime})\not=0$. Then the proof of Lemma 1.1 would show that 
the Monge-Ampere determinant at $({x}^{\prime},0,{t}^{\prime \prime})$ vanishes
of order $j-2$. This is a contradiction, since by 
the assumption of Lemma 1.1,
the Monge-Ampere determinant can only vanish of order $m-2$. Hence, we have 
shown that in a small neighborhood of $({x}^{\prime},{t}^{\prime})$, the 
defining function $A(x,y_1,t)$ is of the form 
$$ A(x,y_1,t)=g(x,y_1,t)({y_1}^m+a_0(x,t)), \tag1.9$$ 
where $g(x,y_1,t)$ and $a_0(x,t)$ have the properties described above. 

We can now state the variable coefficient version of Theorem 1.1.
\vskip .125in
\proclaim{Theorem 1.2} Let $M_t$ be as in (1.3). Let 
${\Cal M}(f)(x)=\sup_{t>0}M_t(f)(x)$. Suppose that for each $t$ the canonical
relation is folding of order $ \ m-2 \ $ 
and the cone condition (see \cite{So1}, p.352) is satisfied away 
from $\Sigma$. Then,
$$ {\Cal M}:L^p({\Bbb R}^2) \rightarrow L^p({\Bbb R}^2)\ \text{for} \ 
p>m.\tag1.10$$  
\endproclaim 

\vskip.25in
{\bf Proof of Theorem 1.1}:
\vskip.125in
Our proof will consist of three main steps. First, we will 
decompose each $M_t$ away from the flat point. Then, we will use the method of
stationary phase to express each dyadic operator in terms of the Fourier 
transform of the surface measure on each dyadic piece. We will then use a
stretching argument to expose the behavior of our operator near the flat point.

To complete the proof, we will use a scaling argument and a technical lemma
in order to reduce our problem to the  local smoothing estimates of 
Mockenhaupt, Seeger, and Sogge (\cite{MSSo2}).
In the process, we shall take advantage of the fact that the 
local smoothing arguments in question are valid under small 
smooth perturbations.

We now turn to the details. We wish to use the following stationary phase 
result (See \cite{So2}) which we shall only use for curves in $ {\Bbb R}^2.$
\proclaim{Lemma 1.2} Let $S$ be a smooth hypersurface in  ${\Bbb R}^n$
with non-vanishing 
Gaussian curvature and $d\mu$ a $C^\infty_0$ measure on S. Then,
$$|\widehat{ d\mu}(\xi)| \leq const.(1+|\xi|)^{-\frac{n-1}{2}}.\tag1.11$$
  
Moreover, suppose that $\Gamma \subset {\Bbb R}^n \setminus0$ is the cone consisting
of all $\xi$  which are normal  to some point $x \in S$ belonging to a 
fixed relatively compact neighborhood $\Cal N$ of supp $d\mu$. Then,
$$ \left(\frac{\partial}{\partial\xi}\right)^{\alpha} \widehat{ d\mu}(\xi)=
O(1+|\xi|)^{-N}\   \ \forall N, \text{  if }  \xi   \notin   \Gamma,$$          
$$ \widehat{ d\mu}(\xi)=\sum {}e^{-i\langle x_j,  \xi\rangle} a_j (\xi) \ \text{if}\ \xi \in 
\Gamma,
\tag1.12$$
where the finite sum is taken over all $x_j \in \Cal N$ having $\xi$ as the 
normal and  
$$ \left|\left(\frac{\partial}{\partial\xi}
\right)^{\alpha} 
a_j (\xi)\right| \leq C_\alpha  
(1+|\xi|)^{-\frac{n-1}{2}-|\alpha|}.\tag1.13$$ \endproclaim

As we have noted earlier, if $C$ is finite type $m$ at $a_0$, the curvature at
$a_0$ vanishes of order $m-2$. Hence, in order to take advantage of the above
lemma, we must first decompose each $M_t$ away from the flat point. Without
loss of generality, $a_0=(0,c)$.  

Since our curve is finite type $m$, locally we can write $C(s)=(s,g(s)s^m+c).$ 
With that in mind we define $\rho \in C^\infty_0({\Bbb R})$ such that 
$supp(\rho) \subset (\frac{B}{2},2B) \cup (-2B,-\frac{B}{2})$, $B>0$,  
and $\sum {} \rho (2^k y_1 ) \equiv 1$, where $B$ is chosen to be small
enough so that the interval $(-2B, 2B)$ does not contain any other flat points.


Let 
$M^k_t(f)(x)=\int_C f(x-ty) 
\rho (2^k y_1) d\sigma(y)$. Then,
$M_t(f)=\sum_{k=0}^{\infty} M^k_t(f)$. \newline
Let $\Cal {M}^k(f)(x)=\sup_{t>0} 
M^k_t(f)(x)$. Hence, it would suffice to show that 
$$ \Cal {M}^k :L^p({\Bbb R}^2)\rightarrow L^p({\Bbb R}^2)\ \ \text{with norm}\ 
2^{-k\epsilon(p)
}\ \text{ for some}\ \epsilon(p)>0.\tag1.14$$ 

\vskip .125in
We can now apply Lemma 1.2 to each $M^k_t$ which is defined over a dyadic 
piece of our curve. Simultaneously, we perform a stretching transformation 
$$y_1 \rightarrow 2^k y_1 \ \ \ y_2 \rightarrow 2^{mk} y_2\tag1.15$$
which sends each dyadic piece to the curve of unit length $2^{mk}c$ units up 
the $y_2$-axis.
\vskip .125in
We now apply the lemma to the "stretched" operator $M_t$. Keeping in mind the
Jacobian of the stretching  transformation we get an operator of the form
$$ {\Cal G}_k(f)(x,t)=(2\pi)^{-2}2^{-k}\int_\Gamma e^{i\langle x,\xi\rangle} 
e^{itq_k(\xi)} e^{t2^{mk}c{\xi}_2} \frac{a_k(t\xi)}{(1+t|\xi|)^{\frac{1}{2}}} 
\hat{f}(\xi) 
d\xi, \tag1.16$$     
where $\Gamma$ is a fixed cone away from coordinate axes, $q_k(\xi)$ is 
homogeneous of degree 1, and $a_k(t\xi)$ is a symbol of order $0$.     

\vskip .125in

It suffices to show that 
$$ ||\sup_{t>0}{\Cal G}_k(f)(x,t)||_p \leq 2^{-k\epsilon(p)} C_p ||f||_p \ 
 \text{for some}\ \epsilon(p) >0.\tag1.17$$

We complete the microlocalization of our operator by introducing $\beta \in
C^\infty_0({\Bbb R})$ satisfying 
$$ \text{supp}(\beta) \subset [1/2,2],\ \sum^{\infty}_{-\infty} 
\beta(2^{-j}s)=1,
s>0.$$                                  

Let 
$$ {\Cal G}_{k,j}(f)(x,t)=(2\pi)^{-2}2^{-k}\int_{\Gamma} 
e^{i\langle x,\xi\rangle} 
e^{itq_k(\xi)} e^{it2^{mk}c\xi_2} 
\frac{a_k(t\xi)}{(1+t|\xi|)^{\frac{1}{2}}}      
\beta(|\xi|/2^j) \hat{f}(\xi) d\xi. \tag1.18$$

If we take the supremum over $t$ of the absolute value of the difference 
between 
${\Cal G}_k(f)(x,t)$  and   $\sum^{\infty}_{j=1} {\Cal G}_{k,j}(f)(x,t)$, 
we see that it is  
is dominated by the Hardy--Littlewood Maximal function of $f$. 
Hence, it suffices 
to show that 
$$ ||\sup_{t>0}{\Cal G}_{k,j}(f)(x,t)||_p \leq C_p 2^{-k\epsilon(p)} 
2^{-j\epsilon^{\prime}(p)} ||f||_p\tag1.19$$ $$ \text{for} \  m<p<\infty \ 
\text{and some}\ \epsilon(p), \ \epsilon^{\prime}(p)>0.$$

Since we are dealing with dyadic operators, we can use Littlewood-Paley theory 
(see \cite{So3}) 
to see that the inequality holds iff 
$$ \left \| \sup_{t \in [1,2]} |{\Cal G}_{k,j}(f)(x,t)|  
\right \|_p  \le  C_p 2^{-k\epsilon(p)} 2^{-j\epsilon^{\prime}(p)} 
||f||_p, \ m<p<\infty. \tag1.20$$

In order to complete our argument, we need another technical lemma. 
\proclaim{Lemma 1.3} Suppose that $F$ is $C^1({\Bbb R}).$ Then if $p>1$ and
$ 1/p+1/p^{\prime}=1$, 
$$  \sup_{\lambda} {|F(\lambda)|}^p \leq {|F(0)|}^p + p{\left(\int 
{|F(\lambda)|}^p 
d\lambda\right)  }^{1/p^{\prime}} \times {\left(\int {|F^{\prime}(\lambda)|}^p 
d\lambda\right)}^{1/p}.$$
\endproclaim 

To prove the lemma, we just express $F(\lambda)$ as an integral of its 
derivative using the fundamental theorem of calculus. Then, if we use 
Hoelder's inequality, we get the desired result (see \cite{So2}).

If we let $\rho$ be the cutoff function defined previously 
and apply the lemma, we see that 
${|| \sup_{t\in[1,2]}|\rho(t){\Cal G}_{k,j}(f)(x,t)| \ ||}^p_p$ 
is dominated by 
$$ {\left\|  2^{-k}\int_{\Gamma} e^{i\langle x,\xi\rangle} e^{itq_k(\xi)} 
e^{it2^{mk}c\xi_2}
\frac{a_k(t\xi)}{(1+t|\xi|)^{\frac{1}{2}}} \beta(|\xi|/2^j) \hat{f}(\xi) d\xi
\right\|}^{p-1}_{p}\times \tag1.21$$ 
$$ {\left\| 2^{-k}\! \int_{\Gamma} e^{i\langle x,\xi\rangle} e^{itq_k(\xi)} 
e^{it2^{mk}c\xi_2} \beta(|\xi|/2^j) A_{m,k}(\xi,t) \hat{f}(\xi) d\xi 
\right\|}_p,$$
where
$$A_{m,k}(\xi,t)=  i(q_k(\xi) +2^{mk}c\xi_2)
\frac{a_k(t\xi)}{(1+t|\xi|)^{\frac{1}{2}}}  +
\frac{d}{dt}\frac{a_k(t\xi)}{(1+t|\xi|)^{\frac{1}{2}}},$$ 
where we are taking the $L^p$ \  norm with respect to \ 
$ {\Bbb R}^2 \times$ [1,2].

\vskip .125in

Since $q_k(\xi)\approx |\xi| \approx 2^j$ on the support of $\beta$, 
we can count the orders of the symbols to see that the expression in
$(1.21)$ is dominated by 
$$ C2^{-k(1-\frac{m}{p})} 2^{-j(\frac{1}{2}-\frac{1}{p})} 
{||{\Cal F}_{k,j}f||}_{L^p({\Bbb R}^3)}, \tag1.22$$ where 
$$ {\Cal F}_{k,j}f(x,t)=\rho(t)\int_{\Gamma} 
e^{i\langle x,\xi\rangle} e^{itq_k(\xi)}
e^{it2^{mk}c\xi_2}\beta(|\xi|/2^j)a_k(t,\xi)\hat{f}(\xi)d\xi,\tag1.23$$
and where $a_k(t,\xi)$ is a symbol of order $0$ in $\xi.$  

Since $-(1-\frac{m}{p})<0$ if $p>m$, we can take 
$\epsilon=-1+\frac{m}{p}$. Hence, it suffices to show that 
$$ {||{\Cal F}_{k,j}f||}_{L^p({\Bbb R}^3)} \leq C_{\epsilon^{\prime}} 
2^{j(\frac{1}{2} -\frac{1}{p}-\epsilon^{\prime}(p))} 
{||f||}_{L^p({\Bbb R}^2)}, \ \ m<p<\infty, \tag1.24$$ for some
${\epsilon}^{\prime}>0$. 

In \cite{MSSo1} and \cite{MSSo2}, Mockenhaupt, Seeger, and Sogge proved 
$L^p({\Bbb R}^2) \rightarrow L^p({\Bbb R}^3)$ estimates for the operators
of the form 
$$ P_jf(x,t)=\rho(t) \int e^{i\langle x,\xi \rangle} e^{itq(\xi)} a(t,\xi) 
\hat{f}(\xi) \beta(|\xi|/2^j) d\xi, \tag1.25$$ 
where $a(t,\xi)$ is a symbol of order $0$ in $\xi$, and the Hessian matrix
of $q$ always has rank one. They showed that 
$$ {||P_jf||}_{L^p({\Bbb R}^3)} \leq C_{\delta} 
2^{j(\frac{1}{2}-\frac{1}{p}-\frac{1}{2p}+\delta(p))} {||f||}_{L^p({\Bbb R}^2)}.
\tag 1.26$$ 

Moreover, the proof of \cite{MSSo2} shows that this estimate is valid under
small smooth perturbations. The constants in those estimates depend on only
a finite number of derivatives of the phase function and the symbol. The 
same estimates are still valid for small smooth perturbations of this 
operator with perhaps larger constants.  

In order to take advantage of this fact, we need
to make several observations about the operator ${\Cal F}_{k,j}$. 

The phase function $q_k(\xi)+2^{mk}c{\xi}_2$ is the phase
function of the Fourier transform of the Lebesgue measure on the 
stretched dyadic piece of the
curve $C$. As we have noted, without loss of generality 
$C(s)=(s,g(s)s^m+c),$ where  
$g \in C^\infty({\Bbb R}), g(0) \neq 0,$ and $c$ is a constant. 

If we take a dyadic piece of this curve, and then dilate and stretch, 
we get \newline
$(2^kts,2^{mk}t\{g(s)s^m+c\})$, and setting $u=2^ks,$ we
get $(ut,tg(u/2^k)u^m+2^{mk}ct).$ We notice that as $\ k \rightarrow \infty \ $
this family of curves smoothly approaches a fixed family of curves 
$C^{\prime}(s)=(s,g(0)s^m+c)$. Lemma 1.2 tells us that the phase function
of the Fourier transform of the curve carried measure is given by 
$e^{\langle{x}_j,\xi\rangle}$ where $\xi$ is normal to ${x}_j$. 
If we note that the Gauss
map taking the point on the curve to the normal at that point is smooth as
long as the Gaussian curvature does not vanish, we conclude that 
inside $\Gamma$, $q_k(\xi)+2^{mk}c{\xi}_2$ smoothly converges  
to the phase function $q(\xi)+2^{mk}c{\xi}_2$. In particular, 
$$ |D^{\alpha}_{\xi}q_k(\xi)| \leq C{(1+|\xi|)}^{1-|\alpha|},$$ 
where $C$ does not depend on $k$. 
Moreover, $q(\xi)+c{\xi}_2$ is the phase function of the Fourier transform
of the  Lebesgue measure on the unit length piece of the curve
\newline $C^{\prime}(s)=(s,g(0)s^m+c)$.

It is important to note that the Hessian of $q$ always has rank one.  
In order to see this, we explicitly compute $q(\xi)$ up to a 
multiplicative constant using Lemma 1.2.
After computing the unit normal and taking the dot product, we get
$$ \frac{(m-1)s^m}{{(m^2s^{2m-2}+1)}^{\frac{1}{2}}}. \tag1.27$$

If we note that the normal $\overrightarrow { N}=(ms^{m-1},-1)$, 
then we see that
$$ q(\xi)+c{\xi}_2=const.\frac{{\xi_1}^{\frac{m}{m-1}}}
{{\xi_2}^{\frac{1}{m-1}}}+
c{\xi_2}. \tag1.28$$ 

At this point, a direct computation shows that the Hessian of $q$ has rank
one for $\xi \in \Gamma$, since $\Gamma$ is a cone away from the coordinate
axes. 

We also observe that a similar argument shows that 
$\{a_k(t,\xi)\}$ is contained in a bounded subset of symbols of order $0$. 
More precisely, for $t\in [1,2]$, 
$|D^{\alpha}_{\xi}a_k(t,\xi)|\leq C_{\alpha}{(1+|\xi|)}^{-\alpha}$, where
$C_{\alpha}$ does not depend on $k$. 

Let $a(t,\xi)=a_N(t,\xi)$ for $N$ very large, and let $q(\xi)+c{\xi}_2$ be
the limiting phase function discussed above. Let
$$ {\Cal F}^{*}_{k,j}f(x,t)=\rho(t)\int_{\Gamma} e^{i\langle x,\xi \rangle}
e^{itq(\xi)} e^{it2^{mk}c{\xi}_2} a(t,\xi) \beta(|\xi|/2^{j}) \hat{f}(\xi)
d\xi. \tag1.29$$ 

It is not hard to check that 
$$ {\Cal F}^{*}_{0,j}f(x,t)={\Cal F}^{*}_{k,j}f(x_1,x_2+t2^{mk}c,t).\tag1.30$$

Since the Lebesgue measure is invariant under translations, $(1.30)$ implies
that 
$$ {||{\Cal F}^{*}_{k,j}f||}_{L^p({\Bbb R}^3)}=
{||{\Cal F}^{*}_{0,j}f||}_{L^p({\Bbb R}^3)}.\tag1.31$$ 

The operator ${\Cal F}^{*}_{0,j}$ satisfies the aforementioned local 
smoothing estimates of Mockenhaupt, Seeger and Sogge. More precisely, 
$$ {||{\Cal F}^{*}_{0,j}f||}_{L^p({\Bbb R}^3)} \leq
C_{\delta} 2^{j(\frac{1}{2}-\frac{1}{p}-\frac{1}{2p}+\delta(p))}
{||f||}_{L^p({\Bbb R}^2)}, \ \ \delta(p)>0. \tag1.32$$

The argument above shows that ${\Cal F}_{k,j}$ is a smooth family of
Fourier integral operators which belong to a bounded subset of Fourier
integral operators of order $0$ (see e.g \cite{So2}, Ch.6). 
The proof of the local smoothing estimates (see \cite{MSSo2}) shows that the 
estimates are valid under small perturbations. More precisely, the
proof of the local smoothing estimates combined with the statement $(1.31)$ 
and our observations about the ${\Cal F}_{k,j}$'s, 
imply that the operator ${\Cal F}_{k,j}$, for a large enough $k$,
satisfies the estimate $(1.32)$ with perhaps a larger constant.  

Hence, if we let ${\epsilon}^{\prime}=\frac{1}{2p}-\delta(p)$, we see that
the estimate $(1.24)$ is satisfied. This completes the proof.

\vskip .25in         
{\bf Proof of Theorem 1.2}:
\vskip .125in

Our argument will be based on a scaling argument, Lemma 1.1,
the proof of Theorem 1, and most importantly the local smoothing
estimates of Mockenhaupt, Seeger and Sogge (\cite{MSSo2}). 

Fix $({x}^{\prime},{t}^{\prime})$. Consider a curve given by
$y_2={x_2}^{\prime}+A({x}^{\prime},y_1,t)$. The remarks following the proof of
Lemma 1.1 show that there exists a ${\delta}_1>0$, 
such that for $|t-{t}^{\prime}|<{\delta}_1$,
this curve is finite type $m$ with a flat point at  
$y_1={\chi}_{{t}^{\prime}}({x}^{\prime})$. 

We extend this curve to a translation invariant
family by defining $y_2=x_2+A({x}^{\prime},y_1-x_1+{x_1}^{\prime},t)$. 
Again using the proof of Lemma 1.1 and the discussion that followed,  
we see that for
$|t-{t}^{\prime}|<{\delta}_1$, 
$$ A({x}^{\prime},y_1,t)=g({x}^{\prime},y_1,t)
(({y_1-{\chi}_{{t}^{\prime}}({x}^{\prime}))}^{m}+a_0({x}^{\prime},t)), 
\tag1.33$$
where $a_0({x}^{\prime},{t}^{\prime})=0$, $g({x}^{\prime},y_1,t)\not=0$ when 
$y_1={\chi}_{{t}^{\prime}}({x}^{\prime})$, and 
$t\in \{t:|t-{t}^{\prime}|<{\delta}_1 \} .$ 

We shall first argue that the maximal operator associated to this translation
invariant family satisfies the conclusions of Theorem 1.2. 
As we did in the proof of Theorem 1.1, 
we localize our
operator near the flat point by introducing a cutoff function $\rho$ with 
the same properties as before. We perform a stretching transformation
$$ (y_1-{\chi}_{{t}^{\prime}}({x}^{\prime}))\rightarrow 
2^k(y_1-{\chi}_{{t}^{\prime}}({x}^{\prime})) \ \ 
y_2\rightarrow 2^{mk}y_2. \tag1.34$$  

The limiting operator under this stretching transformation corresponds to
the family of curves given by 
$$y_2=x_2+g({x}^{\prime},{\chi}_{{t}^{\prime}}({x}^{\prime},t)
({(y_1-{\chi}_{{t}^{\prime}}({x}^{\prime}))}^m+a_0({x}^{\prime},t)). \tag1.35$$

The only difference between this family of curves and the ones handled in 
Theorem 1.1 is the $t$ dependence. However, using a Littlewood-Paley argument
as we did in the proof of Theorem 1.1, 
we see that it suffices to take a supremum over
$t\in [1,2]$. Since $g({x}^{\prime},{\chi}_{{t}^{\prime}}({x}^{\prime}),t)$
does not vanish in intersection of this interval with the set
$ \{ t:|t-{t}^{\prime}|<{\delta}_1 \}$, we can treat 
$g({x}^{\prime}, {\chi}_{{t}^{\prime}}({x}^{\prime}),t)$ as our time 
parameter, and the same argument goes through.

We again use the fact that the variable coefficient estimates of Mockenhaupt,
Seeger, and Sogge (See \cite{MSSo2}) are valid under small smooth perturbations.
The estimates only depend on the finite number of derivatives of the phase
function and the symbol of the corresponding Fourier integral operator.
Hence, the estimates which are valid for the limiting operator are also valid
for the sufficiently small perturbation of that operator with perhaps a larger
constant. This is directly related to the fact that the cinematic curvature
condition is stable under smooth changes of coordinates. In other words, if
we consider a family of curves satisfying the cinematic curvature condition,
this family will still satisfy that condition in a new smooth coordinate
system (See \cite{So3}). 

In order to complete the proof, we localize the operator corresponding to the
general family of functions. As before, we use the cutoff function
$\rho$ and we define 
$$ M^{k}_tf(x)=\int e^{i\theta(y_2-x_2-A(x,y_1,t))}       
\rho(y_1-{\chi}_{{t}^{\prime}}({x}^{\prime})) \psi(x,y) f(y)dy. \tag1.36$$

If we perform a stretching transformation sending 
$$ (x_1-{x_1}^{\prime})\rightarrow 2^k(x_1-{x_1}^{\prime}) \ \ 
   (x_2-{x_2}^{\prime})\rightarrow 2^{mk}(x_2-{x_2}^{\prime})$$
$$ (y_1-{\chi}_{{t}^{\prime}}({x}^{\prime}))\rightarrow
2^k(y_1-{\chi}_{{t}^{\prime}}({x}^{\prime})) \ \ 
y_2\rightarrow 2^{mk}y_2, $$
we can use Lemma 1.1 to see that our family of curves smoothly converges  
to the family of translation invariant curves defined above.
Moreover, each $M^{k}_t$ satisfies
the cinematic curvature condition. If we again use the fact that local
smoothing estimates used to analyze the translation invariant family are
valid under small perturbations, we see that our localized operators satisfy
the right estimates. 

More precisely, we can show that if we localize near the point ${x}^{\prime}$
in the plane and take the supremum over $t$ in a sufficiently small 
neighborhood of some ${t}^{\prime}$, the resulting maximal operator maps
$L^p({\Bbb R}^2)\rightarrow L^p({\Bbb R}^2)$ for $p>m$.
As we have noted earlier, we only need to consider 
$t\in[1,2]$ and $x$ in some compact subset of the plane. Hence, using 
partitions of unity and the triangle inequality, we complete the proof. 
\vskip.25in 
\newpage

{\bf Acknowledgements : } This paper represents a part of the author's thesis. I
would like to express my gratitude to my teacher and advisor C.D. Sogge for 
bringing this problem to my attention as well as for his numerous suggestions  
and criticisms.  
I would like to thank the referee for his infinite patience and excellent 
suggestions which greatly improved this paper. I would also like to thank 
Michael Christ and Andreas Seeger for helpful conversations on related 
topics. 
\vskip .25in

\ref \key B  \by J.Bourgain \pages 69-85
\paper Averages in the plane over convex curves and maximal operators
\yr 1986 \vol 47
\jour J.Analyse Math. \endref    
                                                   
\ref \key Falc  \by K.J.Falconer        
\paper The geometry of fractal sets
\yr1985 
\jour Cambridge U. Press, Cambridge  \endref 

\ref \key Hor1  \by L.Hormander 
\paper The Analysis of linear partial differential operators, Vols. 1-4
\yr1983 
\jour Springer-Verlag, Berlin   \endref  

\ref \key Hor2   \by L.Hormander  \pages 93-140
\paper Estimates for translation invariant operators in $L^p$ spaces 
\yr1960 \vol 104
\jour Acta Math.  \endref

\ref \key MlTr   \by R.Melrose and M.Taylor \pages 242-315
\paper Near peak scattering and the corrected Kirchoff approximation for a
convex obstacle
\yr 1985 \vol 55
\jour Adv.Math. 
\endref 

\ref \key MSSo1   \by G.Mockenhaupt, A.Seeger, C.D.Sogge        
\paper Wave front sets, local smoothing and Bourgain's circular maximal theorem
\jour Annals Math. (to appear) \endref

\ref \key MSSo2  \by G.Mockenhaupt, A.Seeger and C.D.Sogge    
\paper Local smoothing of Fourier integral operators and Carleson-Sjolin estimates (Preprint)    
\yr1991 
\endref 

\ref \key Ph   \by D.H.Phong 
\paper Singular integrals and Fourier integral operators \newline
(preprint)
\yr 1992 
\endref

\ref \key PhSt1  \by D.H.Phong and E.M.Stein
\paper Radon transforms and torsion
\yr1991 \vol 4
\jour Duke Math J. \endref

\ref \key PhSt2  \by D.H.Phong and E.M.Stein \pages 99-157 
\paper Hilbert Integrals, singular integrals and Radon transforms I
\yr1986 \vol 157
\jour Acta Math.  \endref

\ref \key So1  \by C.D.Sogge \pages 349-376
\paper Propagation of singularities and maximal functions in the plane 
\yr1991 \vol 104
\jour Invent. Math. \endref

\ref \key So2  \by C.D.Sogge         
\paper Fourier Integrals in Classical Analysis 
\yr1992
\jour Oxford Univ.Press. \endref

\ref \key So3  \by C.D.Sogge
\paper Maximal Operators associated to the hypersurfaces with one non-vanishing
Gaussian curvature (preprint)
\yr 1992
\endref 

\ref  \key SoPan  \by C.D.Sogge and Y.Pan \pages 413-419
\paper Oscillatory integrals associated to folding canonical relations
\yr 1990 \vol 60
\jour Colloquium Mathematicum 
\endref

\ref \key St       \by E.M.Stein \pages 2174-2175
\paper Maximal functions:spherical means
\yr 1986 \vol 73
\jour Proc.Nat.Acad.Sci. 
\endref

\ref \key StWa  \by E.Stein and S.Wainger \pages 1239-1295
\paper Problems in harmonic analysis 
\yr 1978 \vol 84
\jour Bull.A.M.S.
\endref  

\ref \key Trvs   \by F. Treves
\paper Introduction to pseudodifferential and Fourier integral operators
\yr 1980 
\jour Plenum Press
\endref



